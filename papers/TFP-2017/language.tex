\section{The Source Language and Relational Extension}

In this section we present a formal description of a small functional language, taken as a source
for relational conversion. We describe its syntax, typing rules and semantics, and then extend it
with relational constructs. Thus, relational conversion maps a program in the source language into 
the program in extended. We specify the typing rules and semantics for the extension as well.

\subsection{The Source Language}
\label{source_language}

The syntax of our source functional language is shown on Figure~\ref{functional_syntax}. It consists of a lambda calculus, 
enriched with constructors with fixed arities $C^n$, patterns $p$ and pattern-matching constructs. Among constructors
we distinguish two nullary interpreted constructors \lstinline|true| and \lstinline|false|, and we add a boolean equality
operator ``$=$'' and expressions for recursive/non recursive let-bindings. In a pattern matching we only allow shallow
patterns (which is not an essential limitation) and do not allow wildcards (which is important~--- converting 
wildcard pattern matching into relational form would require essentially different projections). This choice of language may 
look quite a restrictive. However, in terms of relational programming the language contains virtually everything we would need. Indeed, from
relational conversion standpoint the standard built-in integer arithmetics, for example, is of no use~--- 
there is simply no way to convert integer expression into relational form, using integer expressions. In order to use relational 
arithmetics one needs to reimplement everything from scratch, using, for example, Peano encoding; but Peano arithmetics can be
easily expressed in the language we present.

\begin{wrapfigure}{r}{0.5\textwidth}
\centering
%\scalebox{0.9}{
$$
\begin{array}{rcl}
   e &=&x\\
     & &\lambda x.e\\
     & &e_1\;e_2\\
     & &C^n(e_1,\dots, e_n)\\
     & &\lstinline|true|\\
     & &\lstinline|false|\\
     & &\lstinline|let $x$ = $e_1$ in $e_2$|\\
     & &\lstinline|let rec $f$ = $\lambda x.e_1$ in $e_2$|\\
     & &e_1\,=\,e_2\\
     & &\lstinline|match $e$ with $\{p_i$ -> $e_i\}$|\\
     & &\\
   p &=&C^n(x_1,\dots,x_n)\\
\end{array}
$$
%}
\caption{The syntax of the source language}
\label{functional_syntax}
\end{wrapfigure}

Our language is equipped with Hidley-Milner type system, and we present the typing rules in conventional syntax-directed form 
on Figure~\ref{functional_typing}. Besides primitive boolean type, type variables and function types our system contains
a number of implicitly defined algebraic datatypes $T^k$, and we stipulate, that each constructor $C^n$ belongs to exactly one
datatype. In rule \textsc{Constr$_T$} we assume, that type $t^C$ has the form $T^k(t_1,\dots,t_k)$, where each of the types
$t_i$ is recovered from the types $t_i^C$ of arguments of constructor $C^n$ and, moreover, these types are agree in the sense of
constructor application. Similarly, in rule \textsc{Match$_T$} the types of all $C_i^{k_i}(x^i_1,\dots,x^i_{k_i})$ are expected
to be equal $t^C$, and $t^{C_i}_j$ is a type of $j$-th argument of constructor $C_i$, used in a pattern.

\setarrow{:}
\newcommand{\typed}[3]{\withenv{#1}{\trans{#2}{}{#3}}}

\begin{figure}
\centering
{\bf Types:}
$$
\begin{array}{rcll}
  \mathcal X &=&\alpha, \beta, \dots                                            &\mbox{\supp{(type variables)}}\\
  \mathcal D &=&T^n,...                                                         &\mbox{\supp{(datatype constructors)}}\\
  \mathcal T &=&\lstinline|bool|\mid\alpha\mid T^k(t_1,\dots,t_k)\mid t_1\to t_2 &\mbox{\supp{(types)}}\\
  \mathcal S &=&\forall\bar{\alpha}.t                                           &\mbox{\supp{(type schemas)}}
\end{array}
$$
{\bf Typing rules:}
\def\arraystretch{0}
\begin{tabular}{p{7cm}p{7cm}}
$$
\typed{\Gamma}{\lstinline|true|,\;\lstinline|false|}{\lstinline|bool|}
\ruleno{Bool$_T$}
$$ 
&
$$
\trule{\typed{\Gamma}{e_1}{t}\;\;\;\;\typed{\Gamma}{e_2}{t}}
      {\typed{\Gamma}{e_1=e_2}{\lstinline|bool|}}
\ruleno{Eq$_T$}
$$
\\
$$
\trule{\typed{\Gamma}{e_i}{t^C_i}}
      {\typed{\Gamma}{C^n(e_1,\dots,e_n)}{t^C}}
\ruleno{Constr$_T$}
$$
&
$$
\typed{\Gamma,x:\forall\bar{\alpha}.t}{x}{t[\bar{\alpha}\gets\bar{t^\prime}]}
\ruleno{Var$_T$}
$$
\\
$$
\trule{\typed{\Gamma}{f}{t_1\to t_2}\;\;\;\;\typed{\Gamma}{e}{t_1}}
      {\typed{\Gamma}{f\;e}{t_2}}
\ruleno{App$_T$}
$$
&
$$
\trule{\typed{\Gamma,\,x:t_1}{f}{t_2}}
      {\typed{\Gamma}{\lambda x.f}{t_1\to t_2}}
\ruleno{Abs$_T$}
$$
\\
\multicolumn{2}{p{14cm}}{
$$
\trule{\typed{\Gamma}{e_1}{t_1}\;\;\;\;\typed{\Gamma,x:\forall\bar{\alpha}.t_1}{e_2}{t}}
      {\typed{\Gamma}{\lstinline|let $\;x\;$ = $\;e_1\;$ in $\;e_2$|}{t}},\;\bar{\alpha}=FV(t_1)\setminus FV(\Gamma)
\ruleno{Let$_T$}
$$}\\
\multicolumn{2}{p{14cm}}{
$$
\trule{\typed{\Gamma,f:t_1}{\lambda x.e_1}{t_1}\;\;\;\;\typed{\Gamma,f:\forall\bar{\alpha}.t_1}{e_2}{t}}
      {\typed{\Gamma}{\lstinline|let rec $\;f\;$ = $\;\lambda x.e_1\;$ in $\;e_2$|}{t}},\;\bar{\alpha}=FV(t_1)\setminus FV(\Gamma)
\ruleno{LetRec$_T$}
$$}\\
\multicolumn{2}{p{14cm}}{
$$
\trule{\typed{\Gamma}{e}{t^C}\;\;\;\;\typed{\Gamma,x^i_1:t^{C_i}_1,\dots,x^i_{k_i}:t^{C_i}_{k_i}}{e_i}{t}}
      {\typed{\Gamma}{\lstinline|match $\;e\;$ with $\;\{C_i^{k_i}(x^i_1,\dots,x^i_{k_i})$ -> $e_i\}$|}{t}}
\ruleno{Match$_T$}
$$}
\end{tabular}
\caption{Typing rules for the source language}
\label{functional_typing}
\end{figure}

We describe the semantics of our language in the form of transition system. The transition relation

\setarrow{\to}
\newcommand{\step}[2]{\trans{\inbr{#1}}{}{\inbr{#2}}}

$$
\step{\mathcal S,\,e}{\mathcal S^\prime,\,e^\prime}
$$

\noindent describes one step of evaluation of expression $e$ with a stack of contexts $\mathcal S$, which results in
a new stack $\mathcal S^\prime$ and a new expression $e^\prime$. A context is an expression with a unique hole; informally speaking, 
a stack of contexts describes a path in the expression being evaluated from the topmost construct to the point, where the evaluation 
currently is taking place. For a context $C$ and an expression $e$ we denote by $C[e]$ a complete expression with no holes, which is 
obtained by plugging $e$ into the unique hole of $C$. From each state $\inbr{C_1:C_2:\dots:C_k,e}$ we can build an 
expression $C_k[\dots[C_2[C_1[e]]\dots]$, which represents an intermediate result of evaluation according to a small-step semantics. 
This form of semantic description originates from Felleisen-style~\cite{Felleisen} approach for small-step semantics, and we've
chosen it since it can be naturally extended for relational case.

Our semantics describes call-by-value left-to-right evaluation; in the rules $\textsc{Beta}$, $\textsc{Mu}$, $\textsc{LetVal}$,
$\textsc{LetRec}$ and $\textsc{MatchVal}$ we perform capture-avoiding substitutions, which respect the names in abstractions and let-bindings,
and in the rules $\textsc{EqTrue}$ and  $\textsc{EqFalse}$ we assume, that values being compared do not contain subvalues of the 
form $\lambda x.e$ or $\mu f\lambda x.e$. Finally, in the rule $\textsc{MatchVal}$ we assume, that at most one pattern matches the 
scrutinee~--- this is an important distinction with usual semantics of pattern matching, when the patterns are examined in a top-down 
manner until the match succeeds.

Finally, for a closed expression $e$ and a value $v$ we write $\sembr{e}=v$, iff 

$$\inbr{\epsilon,e}\to^*\inbr{\epsilon,v}$$

\noindent where $\epsilon$~--- an empty stack, and ``$\to^*$'' is a reflexive-transitive closure of ``$\to$''.

\begin{figure}[t]
\centering
{\bf Values:}
$$
\mathcal V = \lstinline|true|\mid\lstinline|false|\mid C^n(v_1,\dots,v_n)\mid\lambda x.e\mid\mu f\lambda x.e
$$
{\bf Contexts:}
$$
\mathcal C = \Box\;e\mid v\;\Box\mid\lstinline|let $x$ = $\Box$ in $e$|\mid\lstinline|match $\;\Box\;$ with $\{p_i$->$e_i\}$|\mid C^n(\bar{v},\Box,\bar{e})\mid\Box=e\mid v=\Box 
$$
$$
C[e]\mbox{\supp{~--- a context $C$ with an expression $e$ plugged into a hole}}
$$
{\bf Stack of contexts:}
$$
\mathcal S=\epsilon\mid\mathcal C : \mathcal S
$$
{\bf States:}
$$
\inbr{\mathcal S, e}\mbox{\supp{(stack of contexts, expression)}};\;\inbr{\epsilon,e}\mbox{\supp{(initial state)}};\;\inbr{\epsilon,v}\mbox{\supp{(final state)}}
$$
{\bf Transitions:}
\vskip2mm
\bgroup
\def\arraystretch{0}
\begin{tabular}{p{7cm}p{7cm}}
\multicolumn{2}{p{14cm}}{
$$
\step{C:\mathcal S,\, v}{\mathcal S,\, C[v]}\ruleno{Value}
$$}\\
$$
\step{\mathcal S,\, f\;e}{\Box\;e:\mathcal S,\, f}\ruleno{AppL}
$$&
$$
\step{\mathcal S,\, v\;e_2}{v\;\Box:\mathcal S,\, e_2}\ruleno{AppR}
$$\\
$$
\step{\mathcal S,\,e_1=e_2}{\Box=e_2:\mathcal S,\,e_1}\ruleno{EqL}
$$&
$$
\step{\mathcal S,\,v=e}{v=\Box:\mathcal S,\,e}\ruleno{EqR}
$$\\
\multicolumn{2}{p{14cm}}{
$$
\step{C:\mathcal S,\,v=v}{\mathcal S,\,C[\lstinline|true|]}\ruleno{EqTrue}
$$}\\
\multicolumn{2}{p{14cm}}{
$$
\step{C:\mathcal S,\,v_1=v_2}{\mathcal S,\,C[\lstinline|false|]},\;v_1\ne v_2\ruleno{EqFalse}
$$}\\
\multicolumn{2}{p{14cm}}{
$$
\step{\mathcal S,\, (\lambda x.e)\;v}{\mathcal S,\, e[x\gets v]}\ruleno{Beta}
$$}\\
\multicolumn{2}{p{14cm}}{
$$
\step{\mathcal S,\, (\mu f\lambda x.e)\;v}{\mathcal S,\, e[f\gets\mu f\lambda x.e,\, x\gets v]}\ruleno{Mu}
$$}\\
\multicolumn{2}{p{14cm}}{
$$
\step{\mathcal S,\, C^n(v_1,\dots,v_{k-1},e_k,\dots,e_n)}{C^n(v_1,\dots,v_{k-1},\Box,\dots,e_n):\mathcal S,\, e_k}\ruleno{Constr}
$$}\\
\multicolumn{2}{p{14cm}}{
$$
\step{\mathcal S,\, \lstinline|let $\;x\;$ = $\;e_1\;$ in $\;e_2$|}{\lstinline|let $\;x\;$ = $\;\Box\;$ in $\;e_2$|:\mathcal S,\, e_1}\ruleno{Let}
$$}\\
\multicolumn{2}{p{14cm}}{
$$
\step{\mathcal S,\, \lstinline|let $\;x\;$ = $\;v\;$ in $\;e$|}{\mathcal S,\,e[x\gets v]}\ruleno{LetVal}
$$}\\
\multicolumn{2}{p{14cm}}{
$$
\step{\mathcal S,\, \lstinline|let rec $\;f\;$ = $\;\lambda x.e_1\;$ in $\;e_2$|}{\mathcal S,\, e_2[f\gets\mu f\lambda x.e_1]}\ruleno{LetRec}
$$}\\
\multicolumn{2}{p{14cm}}{
$$
\step{\mathcal S,\,\lstinline|match $\;e\;$ with $\;\{p_i$->$e_i\}$|}{\lstinline|match $\;\Box\;$ with $\;\{p_i$->$e_i\}$|:\mathcal S,\, e}\ruleno{Match}
$$}\\
\multicolumn{2}{p{14cm}}{
$$
\step{\mathcal S,\,\lstinline|match $\;C_k^{n_k}(v_1,\dots,v_{n_k})\;$ with $\;\{C_i^{n_i}(x^i_1,\dots,x^i_{n_i})\to e_i\}$|}{\mathcal S,\,e_k[x^k_j\gets v_j]}\ruleno{MatchVal}
$$}
\end{tabular}
\egroup
\caption{Semantics for the source language}
\label{functional_semantics}
\end{figure}

\FloatBarrier

\subsection{Relational Extension}
\label{relational_extension}

The relational extension adds five conventional miniKanren expressions for constructing goals; the syntax is shown on Figure~\ref{relational_syntax}.
Since relational constructs are added to regular functional ones, it becomes possible to construct expressions like \lstinline|fun x.x = (true === fun y.y)| etc.
In order to rule such pathological expressions out we devised an extension for the type system of source language. In fact, this approach follows the
actual implementation for OCaml, where a careful choice of types for representing terms and goals made it possible to reject the majority of non-well-formed
programs at compile-time.

Our extension for the type system introduces one interpreted datatype constructor $\Box^o$ with one data constructor $\uparrow$~--- a polymorphic type and
a constructor for logical terms. In addition we introduce interpreted type of goals $\G$, which is distinct from all other types. The typing rules for the relational 
extension are shown on Figure~\ref{relational_typing}. These rules describe rather expected typing: in unification and disequality constraints only
terms of the same logical type can be used, and conjuction and disjunction can only be taken for goals. Note, in our extension a term can be calculated as
a result of arbitrary expression in initial functional language (as long as this expression has expected logical type), but such ``higher-order'' terms will
never appear as a result of relational conversion, so, in fact, relational extension we describe here defines a richer language, than we actually need.

\begin{wrapfigure}{r}{0.5\textwidth}
\centering
$$
\begin{array}{rl}
   e\mathrel{{+}{=}}&\lstinline|fresh ($x$) $\;e$|\\
                    &e_1\equiv e_2\\
                    &e_1\not\equiv e_2\\
                    &e_1\vee e_2\\
                    &e_1\wedge e_2
\end{array}
$$
\caption{Syntax of the relational extension}
\label{relational_syntax}
\end{wrapfigure}

\setarrow{:}
\begin{figure}[t]
\centering
{\bf Types:}
$$
\begin{array}{rcl}
 \mathcal L &=               &\alpha^o \mid\lstinline|bool|^o\mid (T^n(l_1,\dots,l_n))^o\\
 \mathcal T &\mathrel{{+}{=}}&\G
\end{array}
$$
{\bf Typing rules:}
\def\arraystretch{0}
\begin{tabular}{p{7cm}p{7cm}}
\multicolumn{2}{p{14cm}}{
$$
\trule{\typed{\Gamma,x:l}{e}{\G}}
      {\typed{\Gamma}{\lstinline|fresh ($x$) $\;e$|}{\G}}
\ruleno{Fresh$_T$}
$$}\\
$$
\trule{\typed{\Gamma}{e_1}{l}\;\;\;\;\typed{\Gamma}{e_2}{l}}
      {\typed{\Gamma}{e_1\equiv e_2}{\G}}
\ruleno{Unify$_T$}
$$&
$$
\trule{\typed{\Gamma}{e_1}{l}\;\;\;\;\typed{\Gamma}{e_2}{l}}
      {\typed{\Gamma}{e_1\not\equiv e_2}{\G}}
\ruleno{Disequality$_T$}
$$\\
$$
\trule{\typed{\Gamma}{e_1}{\G}\;\;\;\;\typed{\Gamma}{e_2}{\G}}
      {\typed{\Gamma}{e_1\wedge e_2}{\G}}
\ruleno{Conjunction$_T$}
$$&
$$
\trule{\typed{\Gamma}{e_1}{\G}\;\;\;\;\typed{\Gamma}{e_2}{\G}}
      {\typed{\Gamma}{e_1\vee e_2}{\G}}
\ruleno{Disjunction$_T$}
$$
\end{tabular}
\caption{Typing rules for the relational extension}
\label{relational_typing}
\end{figure}

\setarrow{\leadsto}
\def\arraystretch{0}
\begin{figure}[t]
\centering
{\bf Semantic variables:}
\begin{gather*}
\mathfrak S = \mathfrak s_1, \mathfrak s_2, \dots\\[-2mm]
\Sigma, \Sigma^\prime\dots \subset 2^{\mathcal S}\;\mbox{\supp{(sets of allocated semantics variables)}}\\[-1mm]
\inbr{\Sigma^\prime, \mathfrak s}\gets\lstinline|new|\;\Sigma,\;\Sigma^\prime=\Sigma\cup\{\mathfrak s\},\;{\mathfrak s}\notin\Sigma\;\mbox{\supp{(allocation of a new semantic variable)}}
\end{gather*}
{\bf Values:}
$$
\mathcal V \mathrel{{+}{=}} \lstinline|success|\mid\mathfrak s
$$
{\bf Contexts:}
$$
\mathcal C \mathrel{{+}{=}}\Box\equiv e\mid v\equiv\Box\mid\Box\not\equiv e\mid v\not\equiv\Box\mid\Box\wedge e\mid e\wedge\Box
$$
{\bf States:}
\begin{gather*}
\inbr{\Sigma,\mathcal S,e,\sigma}\mbox{\supp{(set of allocated semantic variables, stack of contexts, expression, logical state)}}\\[-1mm]
\inbr{\emptyset,\epsilon,e,\iota}\mbox{\supp{(initial state)}}
\end{gather*}
{\bf Transitions:}
{\def\arraystretch{0}
\begin{tabular}{p{14cm}}
$$
\step{\Sigma,\,\mathcal S,\,\lstinline|fresh($x$) $\;e$|,\,\sigma}{\Sigma^\prime,\,\mathcal S,\,e[x\gets\mathfrak s],\,\sigma},\,\inbr{\Sigma^\prime,\mathfrak s}\gets\lstinline|new|\;\Sigma\ruleno{Fresh}
$$\\
$$
\step{\Sigma,\,\mathcal S,\,e_1\equiv e_2,\,\sigma}{\Sigma,\,\Box\equiv e_2:\mathcal S,\,e_1,\,\sigma}\ruleno{UnifyL}
$$\\
$$
\step{\Sigma,\,\mathcal S,\,v\equiv e,\,\sigma}{\Sigma,\,v\equiv\Box:\mathcal S,\,e,\,\sigma}\ruleno{UnifyR}
$$\\
$$
\step{\Sigma,\,\mathcal S,\,v_1\equiv v_2,\,\sigma}{\Sigma,\,\mathcal S,\,\lstinline|success|,\,\sigma^\prime},\,{\bf unify}\,(\sigma,\,v_1,\,v_2)=\sigma^\prime\ruleno{Unify}
$$\\
$$
\step{\Sigma,\,\mathcal S,\,e_1\not\equiv e_2,\,\sigma}{\Sigma,\,\Box\not\equiv e_2:\mathcal S,\,e_1,\,\sigma}\ruleno{DisEqL}
$$\\
$$
\step{\Sigma,\,\mathcal S,\,v\not\equiv e,\,\sigma}{\Sigma,\,v\not\equiv\Box:\mathcal S,\,e,\,\sigma}\ruleno{DisEqR}
$$\\
$$
\step{\Sigma,\,\mathcal S,\,v_1\not\equiv v_2,\,\sigma}{\Sigma,\,\mathcal S,\,\lstinline|success|,\,\sigma^\prime},\,{\bf diseq}\,(\sigma,\,v_1,\,v_2)=\sigma^\prime\ruleno{DisEq}
$$\\
$$
\step{\Sigma,\,\mathcal S,\,e_1\vee e_2,\,\sigma}{\Sigma,\,\mathcal S,\,e_1,\,\sigma}\ruleno{DisjL}
$$\\
$$
\step{\Sigma,\,\mathcal S,\,e_1\vee e_2,\,\sigma}{\Sigma,\,\mathcal S,\,e_2,\,\sigma}\ruleno{DisjR}
$$\\
$$
\step{\Sigma,\,\mathcal S,\,e_1\wedge e_2,\,\sigma}{\Sigma,\,\Box\wedge e_2:\mathcal S,\,e_1,\,\sigma}\ruleno{ConjStartL}
$$\\
$$
\step{\Sigma,\,\mathcal S,\,e_1\wedge e_2,\,\sigma}{\Sigma,\,e_1\wedge\Box:\mathcal S,\,e_2,\,\sigma}\ruleno{ConjStartR}
$$\\
$$
\step{\Sigma,\,\mathcal S,\,\lstinline|success|\wedge e,\,\sigma}{\Sigma,\,\mathcal S,\,e,\,\sigma}\ruleno{ConjL}
$$\\
$$
\step{\Sigma,\,\mathcal S,\,e\wedge\lstinline|success|,\,\sigma}{\Sigma,\,\mathcal S,\,e,\,\sigma}\ruleno{ConjR}
$$
\end{tabular}}
\caption{Semantics for the relational extension}
\label{relational_semantics}
\end{figure}

The semantics of extended language is shown on Figure~\ref{relational_semantics}. First, the state is extended: besides the stack of contexts and
current expression it now contains a set of used \emph{semantic variables} $\Sigma$ and a \emph{logical state} $\sigma$. 
Semantic variables are allocated and substituted for syntactic logic variable occurrences when \lstinline|fresh| expression is evaluated 
(see rule \textsc{Fresh}). Logical states are affected when unification or disequality constraint is evaluated; we explain them
in details below. All existing rules for the initial language are considered rewritten to propagate newly added components of states unchanged.
Then, we modify the substitution to respect names, bounded in \lstinline|fresh| as well. 
Next, we consider two new kinds of values: a semantic variable and a special value \lstinline|success|. The former is a result of evaluation for
a free logic variable, the latter~--- the result of evaluation for a succeeded goal.

We also extend the definition of context to handle new kinds of expressions. In unification and disequality constraint the terms are evaluated left-to right.
Conjunction and disjunction, however, evaluate nondeterministically: in disjunction only one subgoal is chosen (see rules \textsc{DisjL} and \textsc{DisjR}),
a conjunction can evaluate either left, or right subgoal first (see rules \textsc{ConjStartL} and \textsc{ConjStartR}). When chosen subgoal is evaluated
to a value \lstinline|success|, the other subgoal starts its evaluation (rules \textsc{ConjL} and \textsc{ConjR}).
We have chosen a nondeterministic variant for the semantics since different existing miniKanren implementations use (a little bit) different search, and we do 
not want to depend on implementation details. An opposite side of this solution is that for a concrete program and a concrete miniKanren implementation 
the result of evaluation might not coincide with that, prescribed by the semantics: in concrete implementation a program can diverge, while
nondeterministic semantics may still define a certain scenario to complete with a result. We argue, that in this case it will always be possible to
rewrite a program or/and interpreter to converge according to that scenario.

Finally, we describe the structure of a logical state and the implementation of unification and disequality constraint. The development is mainly based on the existing implementation~\cite{CKanren} and standard approaches for implementing unification~\cite{Unification}. We therefore assume the familiarity of the reader with the following notions:

\begin{itemize}
  \item substitution ($\theta$);
  \item application of substitution $\theta$ to a term $t$ ($t\,\theta$);
  \item composition of substitutions ($\theta\theta^\prime$);
  \item most general unifier of two terms ($mgu\,(t_1, t_2)$).
\end{itemize} 

As it can be seen from the semantics and typing rules, a unification (or disequality constraint) can only
be applied to equally-typed logical values, and we consider substitutions to be partial functions from
semantic variables ($\mathfrak S$) to logical values.

A logical state contains two components

$$
\sigma=(\theta,\Theta^-)
$$

\noindent where $\theta$ is a substitution, $\Theta^-$~--- a set of substitutions describing disequality constraints, 
which can potentially be violated. The initial state contains undefined substitution and empty set:

$$
\iota=(\bot,\emptyset)
$$

The effect of unification is described by the following primitive:

$$
{\bf unify}\,(\sigma,\,t_1,\,t_2)={\bf unify}\,((\theta,\Theta^-),\,t_1,\,t_2)
$$

First, it calculates the most general unifier for the terms under consideration w.r.t. current substitution:

$$
\rho=mgu\,(t_1\,\theta,t_2\,\theta)
$$

If there is no such $\rho$ the unification fails, and the evaluation terminates unsuccessfully. Otherwise,
$\rho$ has to be checked against disequality constraints, represented by $\Theta^-$ (if $\Theta^-$ is empty, the
check succeeds immediately).

Being a substitution, $\rho$ at the same time can be considered as the following unification problem: we can try to unify a pair of terms 

$$
\begin{array}{rcl}
t_l&=&(\mathfrak s_1,\dots,\mathfrak s_k)\\
t_r&=&(\rho(\mathfrak s_1),\dots,\rho(\mathfrak s_k))
\end{array}
$$

\noindent where $\{\mathfrak s_i\}=dom\,(\rho)$. We pick every substitution $\theta^-\in\Theta^-$ and calculate 
$mgu\,(t_l\,\theta^-,t_r\,\theta^-)$. There are three possible outcomes:

\begin{enumerate}
\item The unification fails. This means, that disequality constraint, represented by $\theta^-$, can no
longer be violated. We remove $\theta^-$ from $\Theta^-$ and continue with the next disequality constraint.
\item The unification succeeds with an empty substitution. This means, that the
disequality constraint, represented by $\theta^-$, is violated. The check stops, and the whole top-level 
unification fails.
\item The unification succeeds with a non-empty substitution $\theta^{\prime-}$. This means, that in order not to 
voilate the disequality constraint, represented by $\theta^-$, $\theta^{\prime-}$ has to be respected. We replace
$\theta^-$ with $\theta^{\prime-}$ in $\Theta^-$ and continue with the next disequality constraint.
\end{enumerate}

If the disequality check succeeds, by the end we have a modified set $\Theta^{\prime-}$, and we assume

$$
{\bf unify}\,((\theta,\Theta^-),\,t_1,\,t_2)=(\theta\rho,\Theta^{\prime-})
$$

The evaluation of disequality constraint is performed in a similar manner using the primitive

$$
{\bf diseq}\,(\sigma,\,t_1,\,t_2)={\bf diseq}\,((\theta,\Theta^-),\,t_1,\,t_2)
$$

First, an $mgu\,(t_1\,\theta,t_2\,\theta)$ is calculated. Again, there are three
possible cases:
\FloatBarrier

\begin{enumerate}
\item The unification fails. This means, that disequality constraint succeeds.
\item The unification succeeds with an empty substitution. This means, that disequality
constraint fails.
\item The unification succeeds with a non-empty substitution $\theta^{\prime-}$. This means, that 
this substitution describes the disequality constraint, which have to be respected in
the future, so we add it to $\Theta^-$. 
\end{enumerate}

If disequality constraint succeeds, we obtain (potentially) modified set $\Theta^{\prime-}$, and we
assume

$$
{\bf diseq}\,((\theta,\Theta^-),\,t_1,\,t_2)=(\theta,\Theta^{\prime-})
$$

Finally, for a closed goal $g$ and a logical state $\sigma$ we define $\sembr{g}^r=\sigma$, iff

$$
\inbr{\emptyset,\epsilon,g,\iota}\leadsto^*\inbr{\Sigma,\epsilon,\lstinline|success|,\sigma}\mbox{ for some $\Sigma$}
$$
 
\noindent where ``$\leadsto^*$'' is a reflexive-transitive closure of ``$\leadsto$''. 

%\subsection{Adequacy of Relational Semantics}




