\documentclass{llncs}

\usepackage{makeidx}
\usepackage{amssymb}
\usepackage{listings}
\usepackage{indentfirst}
\usepackage{verbatim}
\usepackage{amsmath, amssymb}
\usepackage{graphicx}
\usepackage{xcolor}
\usepackage{url}
\usepackage{stmaryrd}
\usepackage{xspace}

\sloppy

\def\transarrow{\xrightarrow}
\newcommand{\setarrow}[1]{\def\transarrow{#1}}

\newcommand{\trule}[2]{\frac{#1}{#2}}
\newcommand{\crule}[3]{\frac{#1}{#2},\;{#3}}
\newcommand{\withenv}[2]{{#1}\vdash{#2}}
\newcommand{\trans}[3]{{#1}\transarrow{#2}{#3}}
\newcommand{\ctrans}[4]{{#1}\transarrow{#2}{#3},\;{#4}}
\newcommand{\llang}[1]{\mbox{\lstinline[mathescape]|#1|}}
\newcommand{\pair}[2]{\inbr{{#1}\mid{#2}}}
\newcommand{\inbr}[1]{\left<{#1}\right>}
\newcommand{\highlight}[1]{\color{red}{#1}}
\newcommand{\ruleno}[1]{\eqno[\scriptsize\textsc{#1}]}
\newcommand{\inmath}[1]{\mbox{$#1$}}
\newcommand{\lfp}[1]{fix_{#1}}
\newcommand{\gfp}[1]{Fix_{#1}}
\newcommand{\vsep}{\vspace{-2mm}}
\newcommand{\supp}[1]{\scriptsize{#1}}
\newcommand{\G}{\mathfrak G}
\newcommand{\sembr}[1]{\llbracket{#1}\rrbracket}
\newcommand{\cd}[1]{\texttt{#1}}
\newcommand{\miniKanren}{\texttt{miniKanren}\xspace}

\lstdefinelanguage{ocanren}{
keywords={fresh, let, in, match, with, when, class, type,
object, method, of, rec, repeat, until, while, not, do, done, as, val, inherit,
new, module, sig, deriving, datatype, struct, if, then, else, open, private, virtual, include, success, failure},
sensitive=true,
commentstyle=\small\itshape\ttfamily,
keywordstyle=\ttfamily\underbar,
identifierstyle=\ttfamily,
basewidth={0.5em,0.5em},
columns=fixed,
fontadjust=true,
literate={->}{{$\to\;\;$}}3 {===}{{$\equiv$}}3 {=/=}{{$\not\equiv$}}3 {|>}{{$\triangleright$}}3,
morecomment=[s]{(*}{*)}
}

\lstset{
mathescape=true,
basicstyle=\small,
identifierstyle=\ttfamily,
keywordstyle=\bfseries,
commentstyle=\scriptsize\rmfamily,
basewidth={0.5em,0.5em},
fontadjust=true,
language=ocanren
}

\usepackage{letltxmacro}
\newcommand*{\SavedLstInline}{}
\LetLtxMacro\SavedLstInline\lstinline
\DeclareRobustCommand*{\lstinline}{%
  \ifmmode
    \let\SavedBGroup\bgroup
    \def\bgroup{%
      \let\bgroup\SavedBGroup
      \hbox\bgroup
    }%
  \fi
  \SavedLstInline
}

\begin{document}

\mainmatter

\title{Relational Conversion of Typed High-Order Programs}

\author{
  Petr Lozov\inst{1} \and Andrei Vyatkin\inst{2} \and Dmitry Boulytchev\inst{3}
}

\institute{
St.Petersburg State University\\
Universitetski pr., 28, 198504, St.Petersburg, Russia\\
%JetBrains Research\\
\email{lozov.peter@gmail.com}
\and
St.Petersburg State University\\
Universitetski pr., 28, 198504, St.Petersburg, Russia\\
\email{dewshick@gmail.com}
\and
St.Petersburg State University\\
Universitetski pr., 28, 198504, St.Petersburg, Russia\\
%JetBrains Research\\
\email{dboulytchev@math.spbu.ru}
}

\maketitle

\begin{abstract}
We address the problem of transforming high-order typed functional programs into relational form. 
In this form a program can be run in various ``directions'' with various arguments left free, making it possible to
aquire different behaviors from a single specification. We specify the syntax, typing rules and semantics for the source language
as well as its relational extension, describe the conversion and prove its correctness both in terms of typing and dynamic
semantics. We also discuss the limitations of our approach, present the implementation of the conversion for the subset of ML and 
evaluate it for a number of realistic examples.
\end{abstract}

\section{Introduction}
\label{intro}

Relational programming is an attractive technique, based on the idea of constructing programs as relations.
Many logic programming languages, such as Prolog, Mercury\footnote{\url{https://mercurylang.org}}, 
or Curry\footnote{\url{http://www-ps.informatik.uni-kiel.de/currywiki}} to some extent can be considered as relational. 
In this paper we, specifically, focus on \miniKanren~\cite{TRS}. 

\miniKanren\footnote{\url{http://minikanren.org}} was originally designed as a small relational DSL, embedded in Scheme/Racket. 
Being rather a minimalistic language, which can be implemented with just a few data structures and combinators, \miniKanren found 
its way in dozens of host languages, including Scala, Haskell and Standard ML. The advantage of this approach is the flexibility in
combining functional and logic features; in addition \miniKanren possesses some quite appealing features (complete search, purity, 
declarativity, etc).

With relational approach, it becomes possible to give a simple and elegant solution for the problems, otherwise
considered as tricky, tough, tedious, or boring. For example, relational interpreters can be used to derive
\emph{quines}~--- programs, which reduce to itself, as well as \emph{twines} or \emph{trines} (a pairs or triples of
programs, reducing to each other)~\cite{Untagged}; a straightforward relational description of
simply typed lambda calculus~\cite{Lambda} inference rules works both as type inferencer and inhabitation problem solver~\cite{WillThesis};
relational list sorting can be used to generate all permutations~\cite{ocanren}, etc. 

On the other hand, writing relational specification can sometimes be a tricky and error-prone task. Fortunately, many 
specifications can be written systematically by ``generalizing'' a certain functional program. From the very beginning 
the conversion from functional to relational form was considered as an element of relational programming thesaurus~\cite{TRS}. However,
the traditional approach~--- \emph{unnesting}~--- works only for first-order programs, where none of the arguments of
function under conversion can in turn be functions.

We present a generalized form of relational conversion, which can be applied to high-order terms as well. It turns out, however, that
in order to perform the conversion the term has to be properly typed. In the paper we study the relational conversion for a small ML-like 
language (essentialy, a certain subset of OCaml), equipped with Hindley-Milner type system with let-polymorphism~\cite{Types}. 
We start from retelling the syntax, typing rules and operational semantics, and then extend the source language with conventional set of 
relational constructs. This set corresponds to the existing typed embedding of \miniKanren into OCaml~\cite{ocanren}. We then present typing rules and 
develop operational semantics for this relational extension; to our knowledge this is the first attempt to specify formal semantics for
\miniKanren. Next, we develop formal rules for relational conversion and prove, that these rules respect both typing and
semantics. We establish two type-related restrictions in order for the conversion to be possible and correct, namely:

\begin{itemize}
\item all used-defined types in the source program have to be fully-polymorphic (i.e. all parameters of all their constructors must be type variables).
\item the polymorphism in the relational extension is considered as implicitly bounded~--- we allow type parameters to be instantiated only
by ground types (i.e. non-function types or types, parameterized only by ground types).
\end{itemize}

Finally, we describe the implementation of relational convertor and evaluate it on the number of realistic programs.

\section{\miniKanren in a Nutshell}

In the context of this paper we will use a certain concrete implementation of \cd{miniKanren}~--- a shallow DSL for Objective 
Caml\footnote{https://github.com/dboulytchev/ocanren}, called \cd{ocanren}.



In order to use relational constructs used-defined types as a rule need to be injected into logic domain.

\begin{lstlisting}
type $\alpha$ list = Nil | Cons of $\alpha$ * $\alpha$ list
\end{lstlisting}



\section{The Source Language}

\section{Relational Fragment}

\section{Relational Conversion}

\section{Conversion Correctness}


\section{Evaluation}


\section{Conclusion}

\begin{figure}
\centering
{\bf Supplementary syntax categories:}
$$
\begin{array}{rcll}
  \mathcal C &=&\lstinline|True|,\,\lstinline|False|,\,C^n,\dots                &\mbox{\supp(constructors with arity)}\\
  \mathcal X &=&x,\,y,\,z,\,\dots                                               &\mbox{\supp(variables)}\\
  \mathcal P &=&C^n\,(x_1,\,\dots,\,x_n)                                         &\mbox{\supp(shallow patterns)}
\end{array}
$$
{\bf Expressions:}
$$
\begin{array}{rcll}
  \mathcal E &=&x                                                               &\mbox{\supp(variable occurrence)}\\
             & &\lambda x.e                                                     &\mbox{\supp(abstraction)}\\
             & &e_1\;e_2                                                        &\mbox{\supp(application)}\\ 
             & &C^n(e_1,\dots, e_n)                                             &\mbox{\supp(constructor application)}\\
             & &\lstinline|let $x$ = $e_1$ in $e_2$|                            &\mbox{\supp(let-binding)}\\
             & &\lstinline|let rec $f$ = $\lambda x.e_1$ in $e_2$|              &\mbox{\supp(recursive let-binding)}\\
             & &e_1\,=\,e_2                                                     &\mbox{\supp(equality test)}\\
             & &\lstinline|match $e$ with $\{p_i$ -> $e_i\}$| &\mbox{\supp(pattern matching)}
\end{array}
$$
\caption{Syntax of the source language}
\label{functional_syntax}
\end{figure}

\setarrow{:}
\newcommand{\typed}[3]{\withenv{#1}{\trans{#2}{}{#3}}}

\begin{figure}
\centering
{\bf Types:}
$$
\begin{array}{rcll}
  \mathcal X &=&\alpha, \beta, \dots                                            &\mbox{\supp{(type variables)}}\\
  \mathcal D &=&T^n,...                                                         &\mbox{\supp{(datatype constructors)}}\\
  \mathcal T &=&\lstinline|bool|\mid\alpha\mid T^n(t_1,\dots,t_n)\mid t_1\to t_2 &\mbox{\supp{(types)}}\\
  \mathcal S &=&\forall\bar{\alpha}.t                                           &\mbox{\supp{(type schemas)}}
\end{array}
$$
{\bf Typing rules:}
\begin{tabular}{p{7cm}p{7cm}}
$$
\typed{\Gamma}{\lstinline|True|,\;\lstinline|False|}{\lstinline|bool|}
\ruleno{Bool$_T$}
$$ 
&
$$
\trule{\typed{\Gamma}{e_1}{t}\;\;\;\;\typed{\Gamma}{e_2}{t}}
      {\typed{\Gamma}{e_1=e_2}{\lstinline|bool|}}
\ruleno{Eq$_T$}
$$
\\
$$
\trule{\typed{\Gamma}{e_i}{t^C_i}}
      {\typed{\Gamma}{C^n(e_1,\dots,e_n)}{t^C}}
\ruleno{Constr$_T$}
$$
&
$$
\typed{\Gamma,x:\forall\bar{\alpha}.t}{x}{t[\bar{\alpha}\gets\bar{t^\prime}]}
\ruleno{Var$_T$}
$$
\\
$$
\trule{\typed{\Gamma}{f}{t_1\to t_2}\;\;\;\;\typed{\Gamma}{e}{t_1}}
      {\typed{\Gamma}{f\;e}{t_2}}
\ruleno{App$_T$}
$$
&
$$
\trule{\typed{\Gamma,\,x:t_1}{f}{t_2}}
      {\typed{\Gamma}{\lambda x.f}{t_1\to t_2}}
\ruleno{Abs$_T$}
$$
\\
\multicolumn{2}{p{14cm}}{
$$
\trule{\typed{\Gamma}{e_1}{t_1}\;\;\;\;\typed{\Gamma,x:\forall\bar{\alpha}.t_1}{e_2}{t}}
      {\typed{\Gamma}{\lstinline|let $\;x\;$ = $\;e_1\;$ in $\;e_2$|}{t}},\;\bar{\alpha}=FV(t_1)\setminus FV(\Gamma)
\ruleno{Let$_T$}
$$}\\
\multicolumn{2}{p{14cm}}{
$$
\trule{\typed{\Gamma,f:t_1}{\lambda x.e_1}{t_1}\;\;\;\;\typed{\Gamma,f:\forall\bar{\alpha}.t_1}{e_2}{t}}
      {\typed{\Gamma}{\lstinline|let rec $\;f\;$ = $\;\lambda x.e_1\;$ in $\;e_2$|}{t}},\;\bar{\alpha}=FV(t_1)\setminus FV(\Gamma)
\ruleno{LetRec$_T$}
$$}\\
\multicolumn{2}{p{14cm}}{
$$
\trule{\typed{\Gamma}{e}{t^C}\;\;\;\;\typed{\Gamma,x^i_1:t^{C_i}_1,\dots,x^i_{k_i}:t^{C_i}_{k_i}}{e_i}{t}}
      {\typed{\Gamma}{\lstinline|match $\;e\;$ with $\;\{C_i^{k_i}(x^i_1,\dots,x^i_{k_i})$ -> $e_i\}$|}{t}}
\ruleno{Match$_T$}
$$}
\end{tabular}
\caption{Typing rules for the source language}
\label{functional_typing}
\end{figure}

\setarrow{\to}
\newcommand{\step}[2]{\trans{\inbr{#1}}{}{\inbr{#2}}}

\begin{figure}
\centering
{\bf Values:}
$$
\mathcal V = \lstinline|True|\mid\lstinline|False|\mid C^n(v_1,\dots,v_n)\mid\lambda x.e\mid\mu f\lambda x.e
$$
{\bf Contexts:}
$$
\mathcal C = \Box\;e\mid v\;\Box\mid\lstinline|let $x$ = $\Box$ in $e$|\mid\lstinline|match $\;\Box\;$ with $\{p_i$->$e_i\}$|\mid C^n(\bar{v},\Box,\bar{e})\mid\Box=e\mid v=\Box 
$$
$$
C[e]\mbox{\supp{~--- a context $C$ with an expression $e$ plugged into a hole}}
$$
{\bf Stack of contexts:}
$$
\mathcal S=\epsilon\mid\mathcal C : \mathcal S
$$
{\bf States:}
$$
\inbr{\mathcal S, e}\mbox{\supp{(stack of contexts, expression)}};\;\inbr{\epsilon,e}\mbox{\supp{(initial state)}};\;\inbr{\epsilon,v}\mbox{\supp{(final state)}}
$$
{\bf Transitions:}
\vskip2mm
\bgroup
\def\arraystretch{0}
\begin{tabular}{p{7cm}p{7cm}}
\multicolumn{2}{p{14cm}}{
$$
\step{C:\mathcal S,\, v}{\mathcal S,\, C[v]}\ruleno{Value}
$$}\\
$$
\step{\mathcal S,\, f\;e}{\Box\;e:\mathcal S,\, f}\ruleno{AppL}
$$&
$$
\step{\mathcal S,\, v\;e_2}{v\;\Box:\mathcal S,\, e_2}\ruleno{AppR}
$$\\
$$
\step{\mathcal S,\,e_1=e_2}{\Box=e_2:\mathcal S,\,e_1}\ruleno{EqL}
$$&
$$
\step{\mathcal S,\,v=e}{v=\Box:\mathcal S,\,e}\ruleno{EqR}
$$\\
\multicolumn{2}{p{14cm}}{
$$
\step{C:\mathcal S,\,v=v}{\mathcal S,\,C[\lstinline|True|]}\ruleno{EqTrue}
$$}\\
\multicolumn{2}{p{14cm}}{
$$
\step{C:\mathcal S,\,v_1=v_2}{\mathcal S,\,C[\lstinline|False|]},\;v_1\ne v_2\ruleno{EqFalse}
$$}\\
\multicolumn{2}{p{14cm}}{
$$
\step{\mathcal S,\, (\lambda x.e)\;v}{\mathcal S,\, e[x\gets v]}\ruleno{Beta}
$$}\\
\multicolumn{2}{p{14cm}}{
$$
\step{\mathcal S,\, (\mu f\lambda x.e)\;v}{\mathcal S,\, e[f\gets\mu f\lambda x.e,\, x\gets v]}\ruleno{Mu}
$$}\\
\multicolumn{2}{p{14cm}}{
$$
\step{\mathcal S,\, C^n(v_1,\dots,v_{k-1},e_k,\dots,e_n)}{C^n(v_1,\dots,v_{k-1},\Box,\dots,e_n):\mathcal S,\, e_k}\ruleno{Constr}
$$}\\
\multicolumn{2}{p{14cm}}{
$$
\step{\mathcal S,\, \lstinline|let $\;x\;$ = $\;e_1\;$ in $\;e_2$|}{\lstinline|let $\;x\;$ = $\;\Box\;$ in $\;e_2$|:\mathcal S,\, e_1}\ruleno{Let}
$$}\\
\multicolumn{2}{p{14cm}}{
$$
\step{\mathcal S,\, \lstinline|let $\;x\;$ = $\;v\;$ in $\;e$|}{\mathcal S,\,e[x\gets v]}\ruleno{LetVal}
$$}\\
\multicolumn{2}{p{14cm}}{
$$
\step{\mathcal S,\, \lstinline|let rec $\;f\;$ = $\;\lambda x.e_1\;$ in $\;e_2$|}{\mathcal S,\, e_2[f\gets\mu f\lambda x.e_1]}\ruleno{LetRec}
$$}\\
\multicolumn{2}{p{14cm}}{
$$
\step{\mathcal S,\,\lstinline|match $\;e\;$ with $\;\{p_i$->$e_i\}$|}{\lstinline|match $\;\Box\;$ with $\;\{p_i$->$e_i\}$|:\mathcal S,\, e}\ruleno{Match}
$$}\\
\multicolumn{2}{p{14cm}}{
$$
\step{\mathcal S,\,\lstinline|match $\;C_k^{n_k}(v_1,\dots,v_{n_k})\;$ with $\;\{C_i^{n_i}(x^i_1,\dots,x^i_{n_i})\to e_i\}$|}{\mathcal S,\,e_k[x^k_j\gets v_j]}\ruleno{MatchVal}
$$}
\end{tabular}
\egroup
\caption{Semantics for the source language}
\label{functional_semantics}
\end{figure}

\begin{figure}
\centering
$$
\begin{array}{rcll}
  \mathcal E &\mathrel{{+}{=}}&\lstinline|fresh ($x$) $\;e$| &\mbox{\supp{(fresh logical variable binder)}}\\
             &                &e_1\equiv e_2                 &\mbox{\supp{(unification)}}                   \\
             &                &e_1\not\equiv e_2             &\mbox{\supp{(disequality constraint)}}        \\
             &                &e_1\vee e_2                   &\mbox{\supp{(disjunction)}}                   \\
             &                &e_1\wedge e_2                 &\mbox{\supp{(conjunction)}}
\end{array}
$$
\caption{Syntax of the relational extension}
\label{relational_syntax}
\end{figure}

\setarrow{:}
\begin{figure}
\centering
{\bf Types:}
$$
\begin{array}{rcll}
 \mathcal L &=               &\;\uparrow\!\alpha \mid\;\uparrow\!\lstinline|bool|\mid\;\uparrow\!T^n(l_1,\dots,l_n)&\mbox{\supp{(type of logical terms)}}\\
 \mathcal T &\mathrel{{+}{=}}& \G                                                                            &\mbox{\supp{(type of logical goals)}}
\end{array}
$$
{\bf Typing rules:}
\begin{tabular}{p{7cm}p{7cm}}
\multicolumn{2}{p{14cm}}{
$$
\trule{\typed{\Gamma,x:l}{e}{\G}}
      {\typed{\Gamma}{\lstinline|fresh ($x$) $\;e$|}{\G}}
\ruleno{Fresh$_T$}
$$}\\
$$
\trule{\typed{\Gamma}{e_1}{l}\;\;\;\;\typed{\Gamma}{e_2}{l}}
      {\typed{\Gamma}{e_1\equiv e_2}{\G}}
\ruleno{Unify$_T$}
$$&
$$
\trule{\typed{\Gamma}{e_1}{l}\;\;\;\;\typed{\Gamma}{e_2}{l}}
      {\typed{\Gamma}{e_1\not\equiv e_2}{\G}}
\ruleno{Disequality$_T$}
$$\\
$$
\trule{\typed{\Gamma}{e_1}{\G}\;\;\;\;\typed{\Gamma}{e_2}{\G}}
      {\typed{\Gamma}{e_1\wedge e_2}{\G}}
\ruleno{Conjunction$_T$}
$$&
$$
\trule{\typed{\Gamma}{e_1}{\G}\;\;\;\;\typed{\Gamma}{e_2}{\G}}
      {\typed{\Gamma}{e_1\vee e_2}{\G}}
\ruleno{Disjunction$_T$}
$$
\end{tabular}
\caption{Typing rules for the relational extension}
\label{relational_typing}
\end{figure}

\setarrow{\to}
\begin{figure}
\centering
{\bf Semantic variables:}
\begin{gather*}
\mathfrak S = \mathfrak s_1, \mathfrak s_2, \dots\\
\Sigma, \Sigma^\prime\dots \subset 2^{\mathcal S}\;\mbox{\supp{(sets of allocated semantics variables)}}\\
\inbr{\Sigma^\prime, \mathfrak s}\gets\lstinline|new|\;\Sigma,\;\Sigma^\prime=\Sigma\cup\{\mathfrak s\}\;\mbox{\supp{(allocation of a new semantic variable)}}
\end{gather*}
{\bf Values:}
$$
\mathcal V \mathrel{{+}{=}} \lstinline|success|\mid\mathfrak s
$$
{\bf Contexts:}
$$
\mathcal C \mathrel{{+}{=}}\Box\equiv e\mid v\equiv\Box\mid\Box\not\equiv e\mid v\not\equiv\Box\mid\Box\wedge e\mid e\wedge\Box
$$
{\bf States:}
\begin{gather*}
\inbr{\Sigma,\mathcal S,e,\sigma}\mbox{\supp{(set of allocated semantic variables, stack of contexts, expression, logical state)}}\\
\inbr{\emptyset,\epsilon,e,\iota}\mbox{\supp{(initial state)}}\\
\inbr{\_,\epsilon,\lstinline|success|,\sigma}\mbox{\supp{(final state)}}
\end{gather*}
{\bf Transitions:}
\vskip2mm
\bgroup
\def\arraystretch{0}
\begin{tabular}{p{14cm}}
$$
\step{\Sigma,\,\mathcal S,\,\lstinline|fresh($x$) $\;e$|,\,\sigma}{\Sigma^\prime,\,\mathcal S,\,e[x\gets\mathfrak s],\,\sigma},\,\inbr{\Sigma^\prime,\mathfrak s}\gets\lstinline|new|\;\Sigma\ruleno{Fresh}
$$\\
$$
\step{\Sigma,\,\mathcal S,\,e_1\equiv e_2,\,\sigma}{\Sigma,\,\Box\equiv e_2:\mathcal S,\,e_1,\,\sigma}\ruleno{UnifyL}
$$\\
$$
\step{\Sigma,\,\mathcal S,\,v\equiv e,\,\sigma}{\Sigma,\,v\equiv\Box:\mathcal S,\,e,\,\sigma}\ruleno{UnifyR}
$$\\
$$
\step{\Sigma,\,\mathcal S,\,v_1\equiv v_2,\,\sigma}{\Sigma,\,\mathcal S,\,\lstinline|success|,\,\sigma^\prime},\,{\bf unify}\,(\sigma,\,v_1,\,v_2)=\sigma^\prime\ruleno{Unify}
$$\\
$$
\step{\Sigma,\,\mathcal S,\,e_1\not\equiv e_2,\,\sigma}{\Sigma,\,\Box\not\equiv e_2:\mathcal S,\,e_1,\,\sigma}\ruleno{DisEqL}
$$\\
$$
\step{\Sigma,\,\mathcal S,\,v\not\equiv e,\,\sigma}{\Sigma,\,v\not\equiv\Box:\mathcal S,\,e,\,\sigma}\ruleno{DisEqR}
$$\\
$$
\step{\Sigma,\,\mathcal S,\,v_1\not\equiv v_2,\,\sigma}{\Sigma,\,\mathcal S,\,\lstinline|success|,\,\sigma^\prime},\,{\bf diseq}\,(\sigma,\,v_1,\,v_2)=\sigma^\prime\ruleno{DisEq}
$$\\
$$
\step{\Sigma,\,\mathcal S,\,e_1\vee e_2,\,\sigma}{\Sigma,\,\mathcal S,\,e_1,\,\sigma}\ruleno{DisjL}
$$\\
$$
\step{\Sigma,\,\mathcal S,\,e_1\vee e_2,\,\sigma}{\Sigma,\,\mathcal S,\,e_2,\,\sigma}\ruleno{DisjR}
$$\\
$$
\step{\Sigma,\,\mathcal S,\,e_1\wedge e_2,\,\sigma}{\Sigma,\,\Box\wedge e_2:\mathcal S,\,e_1,\,\sigma}\ruleno{ConjStartL}
$$\\
$$
\step{\Sigma,\,\mathcal S,\,e_1\wedge e_2,\,\sigma}{\Sigma,\,e_1\wedge\Box:\mathcal S,\,e_2,\,\sigma}\ruleno{ConjStartR}
$$\\
$$
\step{\Sigma,\,\mathcal S,\,\lstinline|success|\wedge e,\,\sigma}{\Sigma,\,\mathcal S,\,e,\,\sigma}\ruleno{ConjL}
$$\\
$$
\step{\Sigma,\,\mathcal S,\,e\wedge\lstinline|success|,\,\sigma}{\Sigma,\,\mathcal S,\,e,\,\sigma}\ruleno{ConjR}
$$
\end{tabular}
\egroup
\caption{Semantics for the relational extension}
\label{relational_semantics}
\end{figure}

\begin{figure}
\centering
{\bf Terms:}
$$
\mathfrak T = \mathfrak s\mid\;\uparrow\!C^n(\mathfrak t_1,\dots,\mathfrak t_n)
$$
{\bf Substitution:}
\begin{gather*}
\theta:\mathfrak S\to\mathfrak T\;\mbox{\supp{(a partial mapping from semantic variables to terms)}}\\
\forall\theta\;\forall\mathfrak s,\mathfrak s^\prime\in dom(\theta)\;:\;\theta(\mathfrak s)\not\ni\mathfrak s^\prime\\
\mbox{\supp{Substitution application:}}\;\mathfrak t\theta=\left\{\begin{array}{rcl}
                           \theta\;\mathfrak s&,&\mathfrak s\in dom(\theta)\\
                           \uparrow\!C^n(\mathfrak t_1\theta,\dots,\mathfrak t_n\theta)&,&\mathfrak t=\uparrow\!C^n(\mathfrak t_1,\dots,\mathfrak t_n)\\
                           \mathfrak t&,&\mbox{\supp{otherwise}}
                         \end{array}
                  \right.\\
\phi\;\theta=\lambda\mathfrak s\,.\,(\theta\;\mathfrak s)\phi\;\mbox{\supp{(substitution composition)}}
\end{gather*}
{\bf Logical state:}
$$
\sigma=\inbr{\theta, \{\zeta_1,\dots,\zeta_k\}}\;\mbox{\supp{(a pair of a substitution and a set of substitutions)}}
$$
{\bf Unification:}
$$
{\bf unify}\,(\inbr{\theta, \{\zeta_1,\dots,\zeta_k\}}, \mathfrak t_1,\mathfrak t_2)
$$
{\bf Disequality constraint:}
$$
{\bf diseq}\,(\inbr{\theta, \{\zeta_1,\dots,\zeta_k\}}, \mathfrak t_1,\mathfrak t_2)
$$
\caption{Logical states and transitions}
\end{figure}


\begin{figure}
\centering
{\bf Ground types:}
$$
\mathcal G=\alpha\mid\lstinline|bool|\mid T^n(g_1,\dots,g_n)
$$
{\bf Type conversion:}
$$
\begin{array}{rcl}
\left[g\right]              &=&g\to\G\\
\left[t_1\to t_2\right]     &=&\left[t_1\right]\to\left[t_2\right]\\
\left[\forall\alpha.t\right]&=&\forall\alpha.\left[t\right]
\end{array}
$$
{\bf Term conversion:}
\vskip2mm
\bgroup
\def\arraystretch{0.2}
\begin{tabular}{p{7cm}p{7cm}}
\multicolumn{2}{p{14cm}}{$$
\sembr{x} = x\ruleno{Var$_{RC}$}
$$}\\
$$
\sembr{\lambda x.e}=\lambda x.\sembr{e}\ruleno{Abs$_{RC}$}
$$&
$$
\trule{e : \_\to\_}
      {\sembr{f\;e}=\sembr{f}\;\sembr{e}}\ruleno{App$_{RC}$}
$$\\
$$
\sembr{\lstinline|True|}=\lambda q\,.\,q\;\equiv\;\uparrow\!\lstinline|True|\ruleno{True$_{RC}$}
$$&
$$
\sembr{\lstinline|False|}=\lambda q\,.\,q\;\equiv\;\uparrow\!\lstinline|False|\ruleno{False$_{RC}$}
$$\\
\multicolumn{2}{p{14cm}}{$$
\sembr{\lstinline|let $\;x\;$ = $\;e_1\;$ in $\;e_2$|}=\lstinline|let $\;x\;$ = $\;\sembr{e_1}\;$ in $\;\sembr{e_2}$|\ruleno{Let$_{RC}$}
$$}\\
\multicolumn{2}{p{14cm}}{$$
\sembr{\lstinline|let rec $\;f\;$ = $\;e_1\;$ in $\;e_2$|}=\lstinline|let rec $\;f\;$ = $\;\sembr{e_1}\;$ in $\;\sembr{e_2}$|\ruleno{LetRec$_{RC}$}
$$}\\
\multicolumn{2}{p{14cm}}{$$
\trule{e : g\;\;\;\;f : g\to t_1\to\dots\to t_n\to g_0}
      {\sembr{f\;e}=\lambda y_1\dots y_nq\,.\,\lstinline|fresh ($q^\prime$) $\;(\sembr{e}\;q^\prime)\wedge(\sembr{f}\,(\equiv q^\prime)\,y_1\dots y_n\,q)$|}\ruleno{AppG$_{RC}$}
$$}\\
\multicolumn{2}{p{14cm}}{$$
\sembr{C^n(x_1,\dots,x_n)}=\lambda q\,.\,\lstinline|fresh ($q_1\dots q_n$)|\;(\bigwedge\sembr{x_i}\;q_i)\wedge(q\;\equiv\;\uparrow\!C^n(q_1,\dots,q_n))\ruleno{Constr$_{RC}$}
$$}\\
\multicolumn{2}{p{14cm}}{$$
\trule{\lstinline|match $\;e\;$ with $\;\{C^{n_i}_i(x^i_1,\dots,x^i_{n_i})\;$->$e_i\}$|:t_1\to\dots\to t_r\to g}
      {
       \begin{array}{ccc}
                        \multicolumn{3}{l}{\sembr{\lstinline|match $\;e\;$ with $\;\{C^{n_i}_i(x^i_1,\dots,x^i_{n_i})\;$->$e_i\}$|}=}\\ 
            \phantom{X}&\multicolumn{2}{l}{
                           \begin{array}{l}
                              \lambda q_1\dots q_rq\,.\,\lstinline|fresh ($s$)|\\
                              \phantom{XX}(\sembr{e}\;s)\wedge\\
                              \phantom{XX}\bigvee\lstinline|fresh ($s^i_1\dots s^i_{n_i}$)|\\
                              \phantom{XXXX}(s\;\equiv\;\uparrow\!C^{n_i}_i(s^i_1,\dots,s^i_{n_i}))\wedge\\
                              \phantom{XXXX}(\lambda x^i_1,\dots,x^i_{n_i}\,.\,\sembr{e_i}\;q_1\dots q_r\;q)(\equiv\;s^i_1)\dots(\equiv\;s^i_{n_i})
                           \end{array}}                        
       \end{array}
      }\ruleno{Match$_{RC}$}
$$}\\
\multicolumn{2}{p{14cm}}{$$
\begin{array}{ccc}
                \multicolumn{3}{l}{\sembr{e_1=e_2}=}\\
    \phantom{X}&\multicolumn{2}{l}{
       \begin{array}{l}
        \lambda q\,.\,\lstinline|fresh ($q_1\;q_2$)|\\
        \phantom{XX}(\sembr{e_1}\;q_1)\wedge\\
        \phantom{XX}(\sembr{e_2}\;q_2)\wedge\\
        \phantom{XX}((q_1\;\equiv\;q_2\wedge q\;\equiv\;\uparrow\!\lstinline|True|)\vee(q_1\;\not\equiv\;q_2\wedge q\;\equiv\;\uparrow\!\lstinline|False|))
       \end{array}
    }
\end{array}\ruleno{Eq$_{RC}$}
$$}
\end{tabular}
\egroup
\caption{Relational conversion rules}
\end{figure}


\begin{thebibliography}{99}
\bibitem{TRS}
Daniel P. Friedman, William E.Byrd, Oleg Kiselyov. The Reasoned Schemer. The MIT
Press, 2005.

\bibitem{MicroKanren}
Jason Hemann, Daniel P. Friedman. $\mu$Kanren: A Minimal Core for Relational Programming //
Proceedings of the 2013 Workshop on Scheme and Functional Programming (Scheme '13).

\bibitem{CKanren}
Claire E. Alvis, Jeremiah J. Willcock, Kyle M. Carter, William E. Byrd, Daniel P. Friedman.
cKanren: miniKanren with Constraints //
Proceedings of the 2011 Workshop on Scheme and Functional Programming (Scheme '11).

\bibitem{Untagged}
William E. Byrd, Eric Holk, Daniel P. Friedman.
miniKanren, Live and Untagged: Quine Generation via Relational Interpreters (Programming Pearl) //
Proceedings of the 2012 Workshop on Scheme and Functional Programming (Scheme '12).

%\bibitem{Implicits}
%Leo White, Fr\'ed\'eric Bour, Jeremy Yallop.
%Modular Implicits // Workshop on ML, 2014, arXiv:1512.01438.

%\bibitem{Unparsing}
%Olivier Danvy.
%Functional Unparsing // Journal of Functional Programming, Vol.~8, Issue~6, November 1998.

%\bibitem{DoWeNeed}
%Daniel Fridlender, Mia Indrika.
%Do we need dependent types? // Journal of Functional Programming, Vol.~10, Issue~4, July 2000.

\bibitem{DGP}
Jeremy Gibbons. Datatype-generic Programming //
Proceedings of the 2006 International Conference on Datatype-generic Programming.

%\bibitem{Deriving}
%Jeremy Yallop.
%Practical Generic Programming in OCaml // Proceedings of 2007 Workshop on ML.

%\bibitem{InstantGenerics}
%Manuel M. T. Chakravarty, Gabriel C. Ditu, Roman Leshchinskiy.
%Instant Generics: Fast and Easy. \url{http://www.cse.unsw.edu.au/~chak/papers/CDL09.html}, 2009.

%\bibitem{ALaCarte}
%Wouter Swierstra. Data Types \'a la Carte  // Journal of Functional Programming, Vol.~18, Issue~4, 2008.

\bibitem{Kumar}
Ramana Kumar. Mechanising Aspects of miniKanren in HOL. Bachelor Thesis, The Australian National University, 2010.

\bibitem{Unification}
Franz Baader, Wayne Snyder. Unification theory. In John Alan Robinsonand Andrei Voronkov, editors,
Handbook of Automated Reasoning. Elsevier and MIT Press, 2001.

%\bibitem{triangular}
%David C Bender, Lindsey Kuper, William E Byrd, Daniel P Friedman.
%Efficient Representations for Triangular Substitutions: a Comparison in miniKanren. Indiana University, 2009.

\bibitem{HKinded}
Jeremy Yallop, Leo White. Lightweight Higher-Kinded Polymorphism. FLOPS 2014.

\bibitem{Lambda}
Henk Barendregt. Lambda Calculi with Types, Handbook of Logic in Computer Science (Vol.~2), 1992.

\bibitem{WillThesis}
William E. Byrd. Relational Programming in miniKanren: Techniques, Applications, and Implementations. PhD Thesis,
Indiana University, Bloomington, IN, September 30, 2009.

\bibitem{ocanren}
Dmitry Kosarev, Dmitry Boulytchev. Typed Embedding of a Relational Language in OCaml // International Workshop on ML, 2016.

\bibitem{Types}
Benjamin Pierce. Types and Programming Languages. MIT Press, 2002.
\end{thebibliography}

\end{document}

