\section{Relational Programming in \miniKanren}
\label{ocanren}

In the context of this paper we will use a certain concrete implementation of \miniKanren~--- a shallow DSL for Objective 
Caml\footnote{https://github.com/dboulytchev/ocanren}, called \ocanren~\cite{ocanren}. \ocanren corresponds to \miniKanren with
disequality constraints~\cite{CKanren}, and (modulo typing) follows the original implementation. Here we describe the external view 
on \ocanren, giving only intuitive meaning of its constructs; the formal description will be presented in Section~\ref{relational_extension}.
We also use a simplified syntax, which is a little bit different from the concrete syntax in actual implementation, but assumed to
be easier to read.

The central notion of \miniKanren is \emph{goal}; in \ocanren a goal can be arbitrary expression of reserved goal type, which we denote $\G$.
There are only five syntactic forms of goals (denoted below as $g, g_1, g_2$, etc.):

\begin{itemize}
  \item conjunction $g_1\wedge g_2$;
  \item disjunction $g_1\vee g_2$;
  \item fresh variable introduction $\lstinline|fresh ($x$) $\;g$|$;
  \item unification $t_1\;\equiv\;t_2$;
  \item disequality constraint $t_1\;\not\equiv\;t_2$.
\end{itemize}

Two last forms of goals constitute a basis for goal construction; here $t_1$ and $t_2$ are \emph{terms}. In \ocanren a term is arbitrary expression of polymorphic logic type $\alpha^o$. The postfix notation $\Box^o$ is a traditional way to denote relational enities, and we will use it for types as well\footnote{In the real implementation the terms have more complex two-parametric type, which encodes tagging, needed to be performed when the results of relational program are returned into the functional word; these details, however, are irrelevant to the objectives of the paper, and we stick with the simplified version.}.

The simplest expression of logic type is a variable, bound in \lstinline|fresh|. Another example is a primitive value, \emph{injected} into the logic domain with a built-in primitive ``$\uparrow$'', such as $\uparrow\!3$ (of type \lstinline|int$^o$|) or \lstinline|$\uparrow$true| (of type \lstinline|bool$^o$|). With these simplest forms some relational programs can already be written:

\begin{lstlisting}
   val is_zero$^o$ : int$^o$ -> bool$^o$ -> $\G$
   let is_zero$^o$ n b = 
     (n === ^0 /\ b === ^true) |||
     (n =/= ^0 /\ b === ^false)
\end{lstlisting}

Function \lstinline|is_zero$^o$| implements a binary relation between integers and booleans; when called with specific arguments, it returns a goal, which can be executed or combined with another goals. In former case, a stream of \emph{answers} is returned. An element of the stream contains the description of certain constraints for logical variables, which have to be respected in order for the relation to hold. We denote running primitive ``$\leadsto$'', so

\begin{lstlisting}
   iz_zero$^o$ ^0 ^true $\leadsto$ [()]
\end{lstlisting}

\noindent returns a stream with one ``empty'' element since no additional constraints are discovered. 
Running 

\begin{lstlisting}
   iz_zero$^o$ q ^true $\leadsto$ [q$\binds$0]
\end{lstlisting}

\noindent (where \lstinline|q| is a fresh logical variable) returns a stream, which represents the single expected answer. Goal

\begin{lstlisting}
   iz_zero$^o$ q ^false $\leadsto$ [q$\binds\free{0}$ (=/= 0)]
\end{lstlisting}

\noindent provides a stream, which can be interpreted as ``\lstinline|q| can be arbitrary integer as long as it is different from zero'' ($\free{0}$, $\free{1}$, etc. denote free logical variables). Finally, the goal

\begin{lstlisting}
   iz_zero$^o$ q p $\leadsto$ [(q$\binds$0, p$\binds$true); (q$\binds\free{0}$ (=/= 0), p$\binds$false)]
\end{lstlisting}

\noindent returns a stream with two elements, which represents all possible pairs in the relation.

Other types (pairs, lists, user-defined algebraic datatypes, etc.) can be used in relational specifications as well, being injected by the same primitive. For example, expression \lstinline{^(1, "abc")} has type \lstinline{(int * string)$^o$}, \lstinline{^[1; 2; 3]}~--- type \lstinline{(int list)$^o$}, etc. The subtle part is that (since the unification only works for logical types) the placement of ``$^o$'' determines the granularity of unification. Indeed, a logical variable can only be placed where logical type is expected. Thus, in unification one can use a value of type \lstinline{(int * int)$^o$} as \emph{a whole}, but in order to control the \emph{contents} of the pair relationally the type \lstinline{(int$^o$, int$^o$)$^o$} is required. This makes it impossible to reuse some built-in or standard types in relational code~--- for example, predefined list type is not flexible enough, since it does not allow the tail of the list to be logical. Instead, logical list type has to be introduced:

\begin{lstlisting}
   type $\alpha$ llist = Nil | Cons of $\alpha^o$ * ($\alpha$ llist)$^o$
\end{lstlisting}

With logical list type, we can implement relations for lists:

\begin{lstlisting}
   val append : $\alpha$ llist$^o$ -> $\alpha$ llist$^o$ -> $\alpha$ llist$^o$ -> $\G$
   let rec append$^o$ x y xy =
     (x === ^Nil /\ xy === y) |||
     (fresh (h t ty)
        x  === ^(Cons (h, t)) /\
        xy === ^(Cons (h, ty)) /\
        append$^o$ t y ty
     ) 
\end{lstlisting}

Here we defined relational list concatenation \lstinline{append$^o$}, a canonical example in the field. This definition takes three logical lists \lstinline{x}, \lstinline{y} and \lstinline{xy} as arguments, and constructs a goal, which, being ran, finds all bindings for free variables in the lists (if any), trying to satisfy the desired relation. The search is described in terms of case analysis and recursion:

\begin{enumerate}
\item If the first list is empty, then the second and the third lists must be equal.
\item Otherwise, the first list can be split into a head and a tail, and two fresh variables \lstinline{h} and \lstinline{t} are needed to denote them. We also need a fresh variable \lstinline{ty} to denote the list, such, that appending \lstinline{y} to \lstinline{t} equals \lstinline{ty}. To ensure this property we use a recursive call to \lstinline{append$^o$}. Finally, we acquire the final result by consing \lstinline{h} and \lstinline{ty}. 
\end{enumerate}

As it can be seen from the type, relational concatenation is polymorphic, like its functional counterpart. However, the query

\begin{lstlisting}
   appendo$^o$ ^(Cons (^$\lambda$x.x, ^Nil)) q ^(Cons (^$\lambda$y.y, ^Nil))  
\end{lstlisting}

\noindent ends with a run-time error due to inability to unify closures. This is a fundamental limitation in original \miniKanren as well, as it deals only with first-order syntactic unification~\cite{Unification}. This example demostrates, that, unlike pure OCaml, the typing in OCanren is somewhat weak. In order to restore the strong typing some of type variables have to be bounded to range over only non-functional types. The lack of direct support for bounded polymorphism~\cite{cardelli} in OCaml makes this step problematic. Our experience, however, shows, that in practice this deficiency rarely gets in the way. In the following development we assume, that in polymorphic types some type variables may be implicitly bounded to only non-function types, and these boundings are respected in all instantiations of those type variables.

Finally, we describe the unnesting technique~\cite{TRS}, which was described as a method for manually transform
a functional program into relational form. As an example we take list concatenation:

\begin{lstlisting}
   let rec append x y =
     match x with
     | Nil -> y
     | Cons (h, t) -> Cons (h, append t y)
\end{lstlisting}

The unnesting introduces a new name for each nested function call; in our case there is only one nested call, namely,
the recursive call to \lstinline|append| itself, so the unnested function looks like

\begin{lstlisting}
   let rec append x y =
     match x with 
     | Nil -> y
     | Cons (h, t) -> 
        let ty = append t y in
        Cons (h, ty)
\end{lstlisting}

Now the conversion is straightforward: the pattern-matching construct is transformed into a disjunction, and new names,
introduced in pattern bindings and unnestings, are transformed into \lstinline|fresh| variables:

\begin{lstlisting}
   let rec append$^o$ x y xy =
     (t === ^Nil /\ xy === y) |||
     (fresh (h t ty)
        (x  === ^Cons (h, t)) |||
        (xy === ^Cons (h, ty)) |||
        (append$^o$ t y ty)
     )
\end{lstlisting}

However, not every definition can be converted to a relational form by unnesting. Being local and purely syntactic
transformation, unnesting can not handle, for example, partial applications. In order to apply unnesting all 
definitions has to be transformed into $\eta$-long form. We stress, that our relational conversion, described in 
Section~\ref{conversion}, is essentially different from unnesting (in particular, we do not use $\eta$-long transformation).








