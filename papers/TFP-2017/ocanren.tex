\section{Relational Programming in \miniKanren}
\label{ocanren}

In the context of this paper we will use a certain concrete implementation of \cd{miniKanren}~--- a shallow DSL for Objective 
Caml\footnote{https://github.com/dboulytchev/ocanren}, called \ocanren~\cite{ocanren}. \ocanren corresponds to \miniKanren with
disequality constraints~\cite{CKanren}, and (modulo types) follows the original implementation. Here we describe the external view 
on \ocanren, giving only intuitive meaning of its constructs; the formal semantics will be presented in section~\ref{relational_extension}.
We also use a simplified syntax, which is different from the concrete syntax in actual implementation, but assumed to
be easier to read.

The central notion of \miniKanren is \emph{goal}; in \ocanren a goal can be arbitrary expression of reserved goal type, which we denote $\G$.
There are only five syntactic forms of goals (below $g, g_1, g_2$ denote arbitrary goals):

\begin{itemize}
  \item conjunction: $g_1\wedge g_2$;
  \item disjunction: $g_1\vee g_2$;
  \item fresh variable introduction: $\lstinline|fresh ($x$) $\;g$|$;
  \item unification: $t_1\;\equiv\;t_2$;
  \item disequality constraint: $t_1\;\not\equiv\;t_2$.
\end{itemize}

Two last forms constitute a basis for goal construction; here $t_1$ and $t_2$ are \emph{terms}. In \ocanren a term is
arbitrary expression of polymorphic \emph{logic type}\footnote{In actual 
implementation the terms have more complex two-parametric type, which encodes tagging, needed to be performed when the results of
relational program are returned into the functional word; these details, however, are irrelevant to the objectives of the paper, and we stick with the 
simplified version.} $\uparrow\!\alpha$. The simplest expression of logic type is a variable, bound in \lstinline|fresh|. Another examples are primitive values, \emph{injected} 
into the logic domain with a built-in primitive ``$\uparrow$'', such as $\uparrow\!3$ (of type \lstinline|$\uparrow$int|) or \lstinline|$\uparrow$True| 
(of type \lstinline|$\uparrow$bool|). With these simplest forms some relational programs can already be written:

\begin{lstlisting}
   val is_zero : ^int -> ^bool -> $\G$
   let is_zero n b = 
     (n === ^0 /\ b === ^True) \/
     (n =/= ^0 /\ b === ^False)
\end{lstlisting}

Function \lstinline|is_zero| implements a binary relation between integers and booleans; when called with specific arguments, it
returns a goal, which can be executed or combined with another goals. In former case, a stream of \emph{answers} is returned. 
An element of the stream contains the description of certain constraints for logical variables, which have to be respected in
order for the relation to hold. 

For example, running the goal 

\begin{lstlisting}
   iz_zero ^0 ^True
\end{lstlisting}

\noindent returns a stream \lstinline|[()]| with one ``empty'' element since no additional constraints are discovered. 
Running 

\begin{lstlisting}
   iz_zero q ^True
\end{lstlisting}

\noindent (where \lstinline|q| is a fresh logical variable) returns a stream \lstinline|[q=0]|, which represents the single
answer. Goal

\begin{lstlisting}
   iz_zero q ^False
\end{lstlisting}

\noindent provides a stream \lstinline|[q=_.0 (=/= 0)]|, which can be interpreted as ``\lstinline|q| can be arbitrary
integer as long as it is different from zero''. 
Finally, the goal

\begin{lstlisting}
   iz_zero q p
\end{lstlisting}

\noindent returns a stream with two elements \lstinline|[(q=0, p=True); (q=_.0 (=/= 0), p=False)]|, which represents all possible
pairs in the relation.

As more interesting example consider the following definition\footnote{From now on we will maintain the relational naming convention
by adding ``$^o$'' to the end of every relational definition name.}:

\begin{lstlisting}
   val eq$^o$ : ^$\alpha$ -> ^$\alpha$ -> ^bool -> $\G$
   let eq$^o$ x y b = (x ===  y /\ b === ^True) \/ (x =/= y /\ b === ^False)
\end{lstlisting}

The interesting part here is polymorphism: both \lstinline|eq$^o$ ^3 q ^True| and \lstinline|eq$^o$ ^False q ^True| would work, 
giving as results streams \lstinline|[q=3]| and \lstinline|[q=False]| respectively. However, in the case of relational definition
the polymorphism is implicitly \emph{bound}~\cite{cardelli}: despite the type, the application \lstinline|eq$^o$ ($\lambda$x.x) ($\lambda$y.y) q|
would result in a run-time error due to inability to unify closures. This is a fundamental limitation in original \miniKanren as well, as it
deals only with first-order syntactic unification~\cite{Unification}. The lack of direct support for bounded polymorphism in Objective Caml
makes relational specifications in \ocanren somewhat weakly typed, so we express the requirement to unify only non-functional types in the form 
of convention. A bit later we formulate this requirement in more precise form.

Values of user-defined types have to be injected into the logic domain as well. Since we want to be 
able to put a logical variable in arbitrary position inside a data structure, it is desirable to keep type definitions
\emph{fully polymorphic}. In other words, instead of the following (very common) definition for, say, lists

\begin{lstlisting}
   type $\alpha$ list = Nil | Cons of $\alpha$ * $\alpha$ list
\end{lstlisting}

\noindent we suggest to use more abstract fully polymorphic type

\begin{lstlisting}
   type ($\alpha$, $\beta$) alist = Nil | Cons of $\alpha$ * $\beta$
\end{lstlisting}

This form of definition can be used to acquire both the original type

\begin{lstlisting}
   type $\alpha$ list = ($\alpha$, $\alpha$ list) alist
\end{lstlisting}

\noindent and its logical counterpart

\begin{lstlisting}
   type $\alpha$ llist = ^((^$\alpha$, $\alpha$ llist) alist)
\end{lstlisting}

Then, the conversion between these two can be expressed using datatype-generic~\cite{DGP} function
\lstinline|fmap|, which can be derived using existing frameworks~\cite{Deriving}:

\begin{lstlisting}
   val ($\uparrow_{\mbox{\cd{list}}}$) : $\alpha$ list -> $\alpha$ llist
   let rec ($\uparrow_{\mbox{\cd{list}}}$) l = ^(fmap$_{\mbox{\cd{llist}}}$ (^) ($\uparrow_{\mbox{\cd{list}}}$) l)
\end{lstlisting}

We consider fully polymorphic types to be a very attractive pattern since they allow to import/export
data structures to/from logic domain in a systematic manner. Moreover, with fully polymorphic types the non-functional
unification convention can be expressed in more straightforward form: we assume, that polymorphic types never
instantiated for functional types (or types, which instantiated for functional types, etc.)

Finally, we describe the unnesting technique, used for relational conversion of first-order functions. As an example we take
list concatenation:

\begin{lstlisting}
   let rec append x y =
     match x with
     | Nil -> y
     | Cons (h, t) -> Cons (h, append t y)
\end{lstlisting}

The unnesting technique introduces a new name for each nested function call; in our case there is only one nested call, namely,
the recursive call to \lstinline|append| itself, so the unnested function looks like

\begin{lstlisting}
   let rec append x y =
     match x with 
     | Nil -> y
     | Cons (h, t) -> 
       let ty = append t y in
       h :: ty
\end{lstlisting}

Now the conversion is straightforward: the pattern-matching construct is transformed into a disjunction, and new names, 
introduced in via bindings in patterns and unnestings, are transformed into \lstinline|fresh| variables:

\begin{lstlisting}
   let rec append$^o$ x y xy =
     (t === ^Nil /\ xy === y) \/
     (fresh (h t ty)
        (x  === ^Cons (h, t)) /\
        (xy === ^Cons (h, ty)) /\
        (append$^o$ t y ty)
     )
\end{lstlisting}

Now, consider the following very natural definition of the function, which builds a composition for the
list of functions:

\begin{lstlisting}
   let rec compose l =
     let ($\circ$) f g = $\lambda$x.f (g x) in
     match l with
     | Nil         -> id
     | Cons (h, t) -> h $\circ$ compose t
\end{lstlisting}

This definition can not be converted to the relational form directly by unnesting, since lambda is used as a result. 
In order to apply unnesting the definition has to be converted into $\eta$-long form:

\begin{lstlisting}
   let rec compose l x =
     let ($\circ$) f g x = f (g x) in  
     match l with
     | Nil         -> id x
     | Cons (h, t) -> h (compose t x)
\end{lstlisting}

This transformation, however, ruins the flavor of functional programming; it would be very unreasonable to encourage this
style of programming in order for relational conversion to work.

In our approach $\eta$-expansion is blended with the conversion algorithm and performed on-demend invisible for the end-user.








