\section{Implementation and Application}
\label{evaluation}

We implemented relational conversion for the subset of OCaml language, using the infrastructure of the 
original compiler. In its current form, the converter takes the whole file and converts every definition into relational form, 
but in future, we consider to implement a more flexible approach, when only some definitions are converted, being attributed
for this purpose in some way. Our converter rewrites the original abstract syntax tree, annotated with the types, inferred by the compiler, 
into relational form, using the set of combinators from \ocanren. Note, the semantics of OCaml is different from the 
semantics of source language we presented in Section~\ref{source_language}: in OCaml, the order of reductions in
application and binary operators is unspecified (unlike left-to-right in our case), pattern-matching in
OCaml is performed in a top-down manner (and, thus, there can be more than one pattern matching the scrutinee), etc. We, therefore, trust an end user to apply relational conversion only to programs, for which these differences play
no role.

Our preliminary evaluation discovered two problems. First, the converter used to generate a lot of abstractions, 
many of which could be applied immediately. We additionally implemented an optimization pass, which performs 
administrative reductions where possible. This optimization greatly improves the quality of converted 
programs in terms of both readability and performance. Next, in our initial implementation, too many values were
functionalized and, as a result, massively recalculated with essential performance degradation. We improved
the implementation by identifying the important specific case and handling it with a little different 
transformation.

\begin{figure}[t]
  \begin{subfigure}[t]{0.4\textwidth}
    \begin{lstlisting}[basicstyle=\small]
val append : $\alpha$ list ->
             $\alpha$ list ->
             $\alpha$ list
let rec append = fun a.fun b.
  match a with
  | Nil         -> b
  | Cons (h, t) ->
      Cons (h, append t b)
    \end{lstlisting}
    \vskip1.5cm
    \caption{}
  \end{subfigure}
  \begin{subfigure}[t]{0.6\textwidth}    
    \begin{lstlisting}[basicstyle=\small]
val append$^o$ : (($\alpha$ llist)$^o$ -> $\G$) ->
              (($\alpha$ llist)$^o$ -> $\G$) -> 
              ($\alpha$ llist)$^o$ -> $\G$
let rec append$^o$ a b q1 =
  fresh (q2) 
    (a q2) /\
    (((q2 === ^Nil) /\ (b q1)) |||
      (fresh (q3 q4)
        (q2 === ^(Cons (q3, q4))) /\
        (fresh (q6 q7)
          (q6 === q3) /\             
          (q1 === ^(Cons (q6, q7))) /\
          (append$^o$ (=== q4) b q7))))
    \end{lstlisting}
    \vskip-4mm
    \caption{}
  \end{subfigure}
  \vskip5mm
  \caption{An example of relational conversion}
  \label{relational_conversion_example}
\end{figure}

As the first example of the conversion we consider the implementation of concatenation function for lists (see Fig.~\ref{relational_conversion_example}a).
In Section~\ref{ocanren}, we already saw the canonical version of relational concatenation. The result of relational conversion, however, is slightly different
(see Fig.~\ref{relational_conversion_example}b). The main difference comes from the functionalization of primitive values: while conventional \lstinline|append$^o$| operates
on logical lists, the converted variant uses a goal-returning functions. Thus, the conventional \lstinline{append$^o$} for arguments
\lstinline|x|, \lstinline|y| and \lstinline|q| can be expressed using the converted one as \lstinline{append$^o$ (===x) (===y) q}.

\begin{comment}
With this observation in mind, it is rather easy to understand the rest of the code. First, the value of the first argument is
associated with the fresh variable \lstinline|q2|, and case analysis is performed. If the first argument is an empty list, then the 
value of the third argument is associated with the value of the second (by applying \lstinline|b| to \lstinline|q1|). Otherwise, the first
argument is split into the head \lstinline|q3| and the tail \lstinline|q4|, the tail is functionalized and passed to the recursive
call. The result \lstinline|q7| is consed with the head \lstinline|q3| of the original list and unified with the result \lstinline|q1|. One extra
unification \lstinline|q6 === q3| is an artifact of conversion implementation.
\end{comment}

In the next subsections, we consider more elaborated and interesting examples. From now on, we refrain from presenting the complete source and
converted code and consider only the signatures and some interesting queries. 

\subsection{Higher-order Lambda Interpreter}

As we mentioned in Section~\ref{intro}, one of the important application domains for miniKanren is the implementation of relational interpreters~\cite{WillThesis,unified,Untagged}. 
Writing relational interpreter, as a rule, amounts to a careful rewriting of functional implementation in miniKanren. In this regard, obtaining a relational
interpreter automatically from a functional specification looks a natural idea.

In our case, we generalize this idea a little bit: we build a relational interpreter for a family of languages~--- essentially, the lambda calculus with
various reduction orders. The construction of this interpreter was inspired by Felleisen-style semantic description~\cite{Felleisen}. 
Our interpreter takes as its first argument a function, which decomposes a term, passed as a second argument, into a redex and a context (if possible). 
After the decomposition, the interpreter performs beta-reduction on the redex and reconstructs the term by plugging the result back into the context. 
These steps are repeated until the decomposition is no longer possible (or infinitely). This approach brings us a few benefits: first, various reduction orders can be 
expressed by changing only the decomposition function, and next, we demonstrate the applicability of our technique for a higher-order case.

The signatures of relevant functions are

\begin{lstlisting}[basicstyle=\small]
   val eval : (term -> split) -> term -> term
   val call_by_name  : term -> split
   val call_by_value : term -> split
   val normal_order  : term -> split
\end{lstlisting}

\noindent where \lstinline|term| and \lstinline|split| are the types of the terms (in de Bruijn form) and context-term pairs respectively; \lstinline|eval| 
is a higher-order interpreter, all other functions define corresponding reduction orders. Relational counterparts for these definitions, provided 
by the conversion, are shown below:

\begin{lstlisting}[basicstyle=\small]
   val eval$^o$ : ((term$^o$ -> $\G$) -> split$^o$ -> $\G$) -> (term$^o$ -> $\G$) -> 
               term$^o$ -> $\G$
   val call_by_name$^o$   : (term$^o$ -> $\G$) -> split$^o$ -> $\G$
   val call_by_value$^o$ : (term$^o$ -> $\G$) -> split$^o$ -> $\G$
   val normal_order$^o$   : (term$^o$ -> $\G$) -> split$^o$ -> $\G$
\end{lstlisting}

Note, due to the compositionality of the conversion, the type of functions, representing reduction orders, still corresponds to the type of the first 
argument of the interpreter.

The interpreter, constructed by our tool, can be run in a forward direction (for readability purposes, we use here a symbolic quoted representation of
the terms instead of concrete datatype constructor-based):

\begin{lstlisting}[basicstyle=\small]
   eval$^o$ normal_order$^o$ (=== `(fun $\dbi{0}$) $\dbi{1}$`) q $\leadsto$ [q $\binds$ `$\dbi{1}$`]   
   eval$^o$ call_by_name$^o$ (=== `$\dbi{0}$ ((fun $\dbi{0}$) $\dbi{1}$)`) q $\leadsto$ [q $\binds$ `$\dbi{0}$ ((fun $\dbi{0}$) $\dbi{1}$)`]   
   eval$^o$ call_by_value$^o$ (=== `$\dbi{0}$ ((fun $\dbi{0}$) $\dbi{1}$)`) q $\leadsto$ [q $\binds$ `$\dbi{0}$ $\dbi{1}$`] 
\end{lstlisting}

As it is expected from relational interpreter, it equally can be run in the opposite direction, returning for a term a (potentially infinite) stream
of terms, reducing to it:

\begin{lstlisting}[basicstyle=\small]
   eval$^o$ normal_order$^o$ (=== q) (`fun $\dbi{0}$`) $\leadsto$ [
       q $\binds$ `fun $\dbi{0}$`; 
       q $\binds$ `(fun $\dbi{0}$) (fun $\dbi{0}$)`; 
       q $\binds$ `fun ((fun $\dbi{1}$) $\framebox{0}$)`; 
       q $\binds$ `(fun $\dbi{0}$) ((fun $\dbi{0}$) (fun $\dbi{0}$))`; ...] 
   eval$^o$ call_by_name$^o$ (=== q) (`fun $\dbi{0}$`) $\leadsto$ [
       q $\binds$ `fun $\dbi{0}$`; 
       q $\binds$ `(fun $\dbi{0}$) (fun $\dbi{0}$)`; 
       q $\binds$ `(fun $\dbi{0}$) ((fun $\dbi{0}$) (fun $\dbi{0}$))`; 
       q $\binds$ `(fun fun $\;\dbi{0}$) $\dbi{0}$`; ...] 
\end{lstlisting}

This interpreter can be extended to the subset of Scheme, with which the quines/twines/thrines benchmarks~\cite{Untagged} can be
reproduced.

\subsection{Hindley-Milner Type Inference}

Our next example is the type inference for Hindley-Milner type system~\cite{Types}. Interestingly enough, that while typing rules for
STLC can be directly expressed in relational terms, providing the solutions for type inference, type checking, and type inhabitation
problems at the same time, for not so different Hindley-Milner system with let-polymorphism, the problem becomes much
harder. The most robust existing relational solution requires the extension of miniKanren with nominal constructs~\cite{alphaKanren}, while
the correctness of other implementations in conventional miniKanren is still a matter of discussion~\cite{WillOnHM}.

On the other hand, in terms of functional programming, this task is rather a textbook exercise. We implemented a simple version of
syntax-directed type inference and converted it into the relational form; the signatures for the original and converted
implementations are shown below:

\begin{lstlisting}[basicstyle=\small]
   val type_inference  : term -> typ
   val type_inference$^o$ : (term$^o$ -> $\G$) -> typ$^o$ -> $\G$
\end{lstlisting}

For this example, we use a conventional representation of terms with named variables. In a forward direction, our relational implementation works, 
as expected, as a type inferencer~--- given a term it infers its type:

\begin{lstlisting}[basicstyle=\small]
   type_inference$^o$ (=== `$\lambda x \to x$`) q $\leadsto$ [q $\binds$ `$a\to a$`]
\end{lstlisting}

In a reverse direction, relational type inferencer is capable of finding the inhabitants of a specified type:

\begin{lstlisting}[basicstyle=\small]
   type_inference$^o$ (=== q) '$a$' $\leadsto$ $\bot$
   type_inference$^o$ (=== q) '$a$ -> $a$' $\leadsto$ [
       q $\binds$ 'fun $\framebox{0}$ -> $\framebox{0}$'; 
       q $\binds$ 'fun $\framebox{0}$ -> (fun $\framebox{1}$ -> $\framebox{1}$) $\framebox{0}$'; 
       q $\binds$ 'fun $\framebox{0}$ -> let $\framebox{1}$ = $\framebox{2}$ in $\framebox{0}$' ($\framebox{0}$ =/= $\;\;\framebox{1}$);
       q $\binds$ '(fun $\framebox{0}$ -> $\framebox{0}$) (fun $\framebox{1}$ -> $\framebox{1}$)'; ...]
\end{lstlisting}

Note, the first query diverges, providing no results (which is rather expected since the type is un-inhabited). This is a long-time known phenomenon of
miniKanren~--- the search can diverge, when no answers exist; relational specifications, which always stop in this case, are called
\emph{refutationally complete}~\cite{WillThesis}. Given example demonstrates, that our derived relational specification is not refutationally
complete, which is not a rarity in the relational world; making it refutationally complete is a separate task.

It may appear at first glance, that using relational Hindley-Milner inferencer for solving inhabitance problem is superfluous, since the inhabitance for
Hindley-Milner is equivalent to inhabitance for STLC. However, with relational inferencer we may solve some problems, which are distinct from
both pure inference and pure inhabitance:

\begin{lstlisting}[basicstyle=\small]
   type_inference$^o$ (=== `let f = $\Box$ in f (fun x -> f x)`) `$a\to a$` $\leadsto$ 
     [$\Box$ $\binds$ `fun $\framebox{0}$ -> $\framebox{0}$`; ...]
\end{lstlisting}

In this query, we supplied an \emph{incomplete} term with a hole ($\Box$) and some type, and as a result, we've got a term to plug into the hole 
in order for the complete term to have that type. Note, the term we've got as a result cannot be typed in STLC, since the variable \lstinline|f|
is applied there twice with different types of arguments.

A final observation: we do not claim to completely solve the problem of relational implementation of Hindley-Milner type system. Even though our converted
relational implementation behaves as expected, it still not ideal~--- indeed, in functional implementation we had to implement unification on types, 
which does not make use of built-in unification in miniKanren and, to some extent, doubles the work. We, therefore, do not consider this approach as an ideal solution.

\subsection{miniKanren with Disequality Constraints}

Our final example is an implementation of miniKanren in miniKanren. Although there already exist a few similar implementations, written directly in miniKanren, 
our version is different, since it supports disequality constraints. We consider this as an important distinction~--- first, the presence of disequality
constraints makes the language much more expressible, and next, implementing disequality constraints directly in miniKanren is a very tedious
and error-prone task. On the other hand, providing relationally converted version amounts only to repeating a well-known and rather compact original 
implementation~\cite{CKanren}.

The signatures for functional and relational miniKanren implementations are as follows:

\begin{lstlisting}[basicstyle=\small]
   val mk  : goal -> substitution list
   val mk$^o$ : (goal$^o$ -> $\G$) -> (substitution llist)$^o$ -> $\G$
\end{lstlisting}

Here \lstinline|goal| stands for the type, representing the goals, \lstinline|substitution|~--- for the type of substitutions. Again, 
our relational miniKanren interpreter works in both directions. As a more interesting query, we consider the following:

\begin{lstlisting}[basicstyle=\small]
   mk$^o$
     (=== 
        'let rec add a b c =
           ((a === Z) $\wedge$ (b ===  c)) $\vee$
           (fresh (a$_0$ c$_0$) (a ===$\;$ S a$_0$) $\wedge$ $\Box$ $\wedge$ (add a$_0$ b c$_0$))
        in fresh (x y z) (add x y z)') ([[x='1'; y='1'; z='2']])  $\leadsto$   
     [$\Box$ $\binds$ `c ===$\;$ S c$_0$`; ...]
\end{lstlisting}

Here we specified an incomplete relational program (specifically, a relational addition of numbers in Peano form). The hole ($\Box$) replaces
one of the branches, and expected substitution describes the results of addition. Our relational interpreter, converted from functional
implementation, turned out to be capable of finding the correct subgoal~--- ``\lstinline|c === S c$_0$|''~--- to be placed into the hole.


