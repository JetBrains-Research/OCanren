\section{Introduction}
\label{intro}

Relational programming is an attractive technique, based on the idea of constructing programs as relations.
Many logic programming languages, such as Prolog, Mercury\footnote{\url{https://mercurylang.org}}, 
or Curry\footnote{\url{http://www-ps.informatik.uni-kiel.de/currywiki}} to some extent can be considered as relational. 
In this paper we, specifically, focus on \miniKanren~\cite{TRS}. 

\miniKanren\footnote{\url{http://minikanren.org}} was originally designed as a small relational DSL, embedded in Scheme/Racket. 
Being rather a minimalistic language, which can be implemented with just a few data structures and combinators, \miniKanren found 
its way in dozens of host languages, including Scala, Haskell and Standard ML. The advantage of this approach is the flexibility in
combining functional and logic features; in addition \miniKanren possesses some quite appealing features (complete search, purity, 
declarativity, etc).

With relational approach, it becomes possible to give simple and elegant solutions for the problems, otherwise
considered as tricky, tough, tedious, or boring. For example, relational interpreters can be used to derive
\emph{quines}~--- programs, which reduce to itself, as well as \emph{twines} or \emph{trines} (a pairs or triples of
programs, reducing to each other)~\cite{Untagged}; a straightforward relational description of
simply typed lambda calculus~\cite{Lambda} inference rules works both as type inferencer and inhabitation problem solver~\cite{WillThesis};
relational list sorting can be used to generate all permutations~\cite{ocanren}, etc. 

On the other hand, writing relational specification can sometimes be a tricky and error-prone task. Fortunately, many 
specifications can be written systematically by ``generalizing'' a certain functional program. From the very beginning 
the conversion from functional to relational form was considered as an element of relational programming thesaurus~\cite{TRS}. However,
the traditional approach~--- \emph{unnesting}~--- was formulated for untyped case and worked only for specifically written
programs.

We present a generalized form of relational conversion, which can be applied to typed terms in general form. We study the relational conversion 
for a small ML-like language (essentialy, a certain subset of Objective Caml), equipped with Hindley-Milner type system with let-polymorphism~\cite{Types}. 
We start from retelling the syntax, typing rules and operational semantics, and then extend the source language with conventional set of 
relational constructs. This set corresponds to the existing typed embedding of \miniKanren into OCaml~\cite{ocanren}. We then present typing rules and 
develop operational semantics for this relational extension; to our knowledge this is the first attempt to specify formal semantics for
\miniKanren. Next, we develop formal rules for relational conversion and prove, that these rules respect both typing and
semantics. We establish two type-related restrictions in order for the conversion to be possible and correct, namely:

\begin{itemize}
\item all used-defined types in the source program have to be fully-polymorphic (i.e. all parameters of all their constructors must be type variables).
\item polymorphism in the relational extension is considered as implicitly bounded~--- we allow type parameters to be instantiated only
by ground types (i.e. non-function types or types, parameterized only by ground types).
\end{itemize}

Finally, we describe the implementation of relational convertor and evaluate it on the number of realistic programs.
