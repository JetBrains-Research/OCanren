\documentclass[final,20pt]{beamer}

\usepackage[orientation=portrait,size=a0,scale=1.4]{beamerposter}
\usepackage{wrapfig}
\usepackage[english]{babel}
\usepackage{listings}
\usepackage{verbatimbox}
\usepackage{filecontents}
\usepackage{hhline}
\usepackage{tikz}
\usepackage{comment}
\usepackage{mathtools}
\usepackage{ragged2e}
\usetikzlibrary{backgrounds,tikzmark}

%% \usepackage[backend=bibtex,style=numeric,maxnames=2,firstinits=true]{biblatex}

\usetheme{confposter} % Use the confposter theme supplied with this template

%% \setbeamercolor{titlelike}         {bg=jblue,fg=white}
\setbeamercolor{block title example}{fg=white,bg=dgreen!70}
\setbeamercolor{block body example}{fg=black,bg=dgreen!10}
\setbeamercolor{block title}{fg=white,bg=jblue!70} % Colors of the block titles
\setbeamercolor{block body}{fg=black,bg=white} % Colors of the body of blocks
\setbeamercolor{block alerted title}{fg=white,bg=dblue!70} % Colors of the highlighted block titles
\setbeamercolor{block alerted body}{fg=black,bg=dblue!10} % Colors of the body of highlighted blocks
%\setbeamercolor{block example title}{fg=black,bg=dblue!70} % Colors of the example block titles
%\setbeamercolor{block example body}{fg=black,bg=dblue!10} % Colors of the body of example blocks
% Many more colors are available for use in beamerthemeconfposter.sty

% Define the column widths and overall poster size
% Define the column widths and overall poster size
% To set effective sepwid, onecolwid and twocolwid values, first choose how many columns you want and how much separation you want between columns
% In this template, the separation width chosen is 0.024 of the paper width and a 4-column layout
% onecolwid should therefore be (1-(# of columns+1)*sepwid)/# of columns e.g. (1-(4+1)*0.024)/4 = 0.22
% Set twocolwid to be (2*onecolwid)+sepwid = 0.464
% Set threecolwid to be (3*onecolwid)+2*sepwid = 0.708

\newlength{\sepwid}
\newlength{\onecolwid}
\newlength{\twocolwid}
\newlength{\threecolwid}

%% \setlength{\paperwidth}{48in} % A0 width: 46.8in
%% \setlength{\paperheight}{36in} % A0 height: 33.1in

\setlength{\sepwid}{0.03\paperwidth} % Separation width (white space) between columns
\setlength{\onecolwid}{0.455\paperwidth} % Width of one column
\setlength{\twocolwid}{0.920\paperwidth} % Width of two columns
\setlength{\threecolwid}{0.920\paperwidth} % Width of three columns
\setlength{\topmargin}{-0.5in} % Reduce the top margin size

%% \input{sizeDefsVert}            

\usepackage{graphicx}
\usepackage{booktabs}
\usepackage{wasysym}

%% \bibliography{../references,../proceedings}

%% \newcommand{\rulehskip}{\hskip 1.5em}
\newcommand{\rulevspace}{\vspace{1em}}

\newcommand{\pvfill}{\pause\vfill}

% \mathchardef\mhyphen="2D

\theoremstyle{definition}

\newtheorem{example}{Example}[section]
\newtheorem{definition}{Definition}
\newtheorem{lemma}{Lemma}
\newtheorem{remark}{Remark}
\newtheorem{theorem}{Theorem}

\newenvironment{subproof}[1][\proofname]{%
  \renewcommand{\qedsymbol}{$\blacksquare$}%
  \begin{proof}[#1]%
}{%
  \end{proof}%
}

% \newtheorem{prop}{Proposition}

%% \counterwithin{lemma}{section}

\newcommand{\textdef}[1]{\textit{#1}}

\newcommand{\imm}{{\textrm IMM}~}

% inline code 
\newcommand{\code}[1]{\texttt{#1}}

% tuple with angle brackets
\newcommand{\tup}[1]{\langle #1 \rangle}

% semantics brackets
\newcommand{\sem}[1]{\llbracket #1 \rrbracket}

% equality by definition
\newcommand{\defeq}{\triangleq}

% function arrow
\newcommand{\fun}{\rightarrow}

% partial function arrow
\newcommand{\pfun}{\rightharpoonup}

% some math sets
\newcommand{\N}{{\mathbb{N}}}
\newcommand{\Q}{{\mathbb{Q}}}

% domain/codomain notation
\newcommand{\dom}[1]{\textit{dom}{({#1})}}
\newcommand{\codom}[1]{\textit{codom}{({#1})}}

\newcommand{\isground}[1]{\textit{is\_ground}({#1})}

\newcommand{\mgu}{\textit{mgu}}

\newcommand{\vars}[1]{\textit{Vars}({#1})}

\newcommand{\sapp}[2]{{#2}{#1}}
\newcommand{\subs}{\sqsubseteq}

% some logical notation
%\newcommand{\implies}{{\Rightarrow}}
%\newcommand{\iff}{{\Leftrightarrow}}

% check-mark and cross-mark
\newcommand{\cmark}{\text{\color{green!60!black}\ding{51}}}
\newcommand{\xmark}{\text{\color{red!60!black}\ding{55}}}

%% axiom labels

\newcounter{mylabelcounter}

\makeatletter
\newcommand{\labelAxiom}[2]{%
\hfill{\normalfont\textsc{(#1)}}\refstepcounter{mylabelcounter}
\immediate\write\@auxout{%
  \string\newlabel{#2}{{\unexpanded{\normalfont\textsc{#1}}}{\thepage}{{\unexpanded{\normalfont\textsc{#1}}}}{mylabelcounter.\number\value{mylabelcounter}}{}}
}%
}
\makeatother

%% warning

\colorlet{colorWARNING}{yellow!90!black}

% \newcommand{\warning}[1]{{\color{colorWARNING}\texttt{WARNING}}: #1}
% \newcommand{\app}[1]{{\color{blue}\textbf{ANTON: #1}}}
% \newcommand{\note}[1]{{\color{cyan}\textbf{EVG: #1}}}

\newcommand\ExecScaleFactor{1}

\newcommand{\todo}[1]{{\color{red}\textbf{TODO: #1}}}

%% OCanren's listings

\lstdefinelanguage{ocanren}{
    keywords={fresh, let, in, match, with, when, class, type,
    object, method, of, rec, repeat, until, while, not, do, done, as, val, inherit,
    new, module, sig, deriving, datatype, struct, if, then, else, open, private, virtual, include, success, failure,
    true, false},
    sensitive=true,
    commentstyle=\small\itshape\ttfamily,
    identifierstyle=\ttfamily,
    keywordstyle=\bfseries,
    basewidth={0.5em,0.5em},
    columns=fixed,
    fontadjust=true,
    abovecaptionskip=\bigskipamount,
    literate={->}{{$\to$}}3 {===}{{$\equiv$}}1 {=/=}{{$\not\equiv$}}1 {|>}{{$\triangleright$}}3  {/\\}{{$\wedge$}}2 {\\/}{{$\vee$}}2 {^}{{$\uparrow$}}1 {'}{{$^\prime$}}1 {~}{{$\neg$}}1 {=>}{{$\Rightarrow$}}2, 
    morecomment=[s]{(*}{*)}
}

\lstset{
    mathescape=true,
    %basicstyle=\small,
    commentstyle=\scriptsize\rmfamily,
    language=ocanren,
    captionpos=b,
    % escapeinside={(*}{*)},
}

\newcolumntype{H}{>{\collectcell\lstinline}l<{\endcollectcell}}
%% \input{../abbrevmap}
%% \lstset{escapeinside=||,basicstyle=\large}
%% \input{posterDefs}

\lstdefinelanguage{ocanren}{
keywords={fresh, let, in, match, with, when, class, type,
object, method, of, rec, while, not, do, done, as, val, inherit,
new, module, sig, deriving, datatype, struct, if, then, else, open, private, virtual, include, success, failure},
sensitive=true,
commentstyle=\small\itshape\ttfamily,
keywordstyle=\ttfamily\underbar,
identifierstyle=\ttfamily,
basewidth={0.5em,0.5em},
columns=fixed,
fontadjust=true,
literate={fun}{{$\lambda$}}1 {->}{{$\to$}}3 {===}{{$\equiv$}}1 {=/=}{{$\not\equiv$}}1 {|>}{{$\triangleright$}}3 {/\\}{{$\wedge$}}2 {\\/}{{$\vee$}}2 {^}{{$\uparrow$}}1,
morecomment=[s]{(*}{*)},
escapechar=~
}

\lstset{
mathescape=true,
%basicstyle=\small,
identifierstyle=\ttfamily,
keywordstyle=\bfseries,
commentstyle=\scriptsize\rmfamily,
basewidth={0.5em,0.5em},
fontadjust=true,
language=ocanren
}

%% \lstset{escapeinside=||}        

%% \setbeamertemplate{itemize item}{\color{\dbluecolor}$\blacktriangleright$}
%% \setbeamertemplate{itemize subitem}{\color{\bluecolor}$\blacktriangleright$}

\newcommand*{\code}[1]{\texttt{#1}}
\newcommand{\ocanren}[1]{\mbox{\lstinline|#1|}}
\newcommand{\Xrightarrow}[1]{\xrightarrow{\phantom{x}#1\phantom{x}}}

\begin{filecontents*}{appendo.ml}
let rec append$^o$ x y xy =
  (x === [] /\ y === xy) \/
  (fresh (h t ty) 
    (x === h :: t) /\
    (append$^o$ t y ty) /\
    (xy === h :: ty)
  )
\end{filecontents*}

\begin{filecontents*}{appendorun.ml}
(fun q -> append$^o$ [1;2] [3] q) $\leadsto^*$ {q=[1;2;3]}
(fun q r -> append$^o$ q r [1;2;3]) $\leadsto^*$ $\bot$
\end{filecontents*}

\begin{filecontents*}{sorto.ml}
let rec sort$^o$ xs ys =
  (xs === [] /\ ys === []) \/
  (fresh (s xst yst) 
    (ys === s :: yst) /\
    (smallest$^o$ xs s xst) /\  (*1*)
    (sort$^o$ xst yst)~\phantom{$\wedge$}~         (*2*)
  )
\end{filecontents*}

\begin{filecontents*}{permo.ml}
let perm$^o$ xs ys =
  fresh (ts) (sort$^o$ xs ts) /\ (sort$^o$ ys ts)    
\end{filecontents*}

\begin{filecontents*}{divo.ml}
let rec div$^o$ n m q r =
  (r === n /\ [] === q /\ plus$^o$ r m n /\ lt$^o$ r m) \/
  ([1] === q /\ eql$^o$ n m /\ plus$^o$ r m n /\ lt$^o$ r m) \/
  ((ltl$^o$ m n) /\ (lt$^o$ r m) /\ (pos$^o$ q) /\
   (fresh (nh nl qh ql qlm qlmr rr rh) (
     (split$^o$ n r nl nh) /\
     (split$^o$ q r ql qh) /\
     ((([] === nh) /\ ([] === qh) /\ 
       (minus$^o$ nl r qlm) /\ (mult$^o$ ql m qlm)) \/
       ((pos$^o$ nh) /\ (mult$^o$ ql m qlm) /\
       (plus$^o$ qlm r qlmr) /\
       (minus$^o$ qlmr nl rr) /\
       (split$^o$ rr r [] rh) /\
       (div$^o$ nh m qh rh))
     )
   ))
  )
\end{filecontents*}

\begin{filecontents*}{divoeasy.ml}
let div$^o$ n m q r =
  fresh (mq) 
    (mult$^o$ m q mq) /\ 
    (plus$^o$ mq r n) /\ 
    (lt$^o$ r m)    
\end{filecontents*}


%----------------------------------------------------------------------------------------
%	TITLE SECTION 
%----------------------------------------------------------------------------------------

\title[Improving Refutational Completeness of Relational Search]{Improving Refutational Completeness of Relational Search}

\author
{\textbf{Dmitri Rozplokhas}\inst{1}}% \quad Dmitri Boulytchev\inst{2}}

\institute{
  \inst{1}%
  Saint Petersburg Academic University, JetBrains Research, Russia% \quad
%  \inst{2}%
%  Saint Petersburg State University, JetBrains, Russia
}

%% \vspace*{.5cm}
%% {\normalsize
%% GitHub: \textbf{anlun/OperationalSemanticsC11}
%% \quad 
%% E-mail: \textbf{a.podkopaev@2009.spbu.ru}
%% }
%% \vspace*{-1cm}

%----------------------------------------------------------------------------------------
\sloppy
\begin{document}

%% \addtobeamertemplate{block end}{}{\vspace*{1ex}} % White space under blocks
%% \addtobeamertemplate{block alerted end}{}{\vspace*{1ex}} % White space under highlighted (alert) blocks

\setlength{\belowcaptionskip}{2ex} % White space under figures
\setlength\belowdisplayshortskip{2ex} % White space under equations

\begin{frame}[t] % The whole poster is enclosed in one beamer frame

\begin{columns}[t]
  \begin{column}{\sepwid}\end{column} % Empty spacer column

  \begin{column}{\onecolwid} % The first column
  
    \begin{block}{Relational Programming}
      \vskip7mm
      \begin{itemize}
        \justifying
        \item From programs as \emph{functions} to programs as \emph{relations}.
        \item Evaluation in various directions (e.g. evaluating the arguments from a result value).
        \item Elegant solutions by duality for some non-trivial problems:
          \bigskip

          \begin{center}
          \textbf{\textcolor{blue}{
          \begin{tabular}{rcl}
             sorting        & $\leftrightarrow$ & permutations\\
             type inference & $\leftrightarrow$ & type inhabitation\\
             interpretation & $\leftrightarrow$ & program synthesis
          \end{tabular}}}
          \end{center}

      %\textbf{\textcolor{blue}{Sorting}} for \textbf{\textcolor{blue}{generation of permutations}} \\
      %\textbf{\textcolor{blue}{Typechecker}} for \textbf{\textcolor{blue}{inhabitation problem}} \\
      %\textbf{\textcolor{blue}{Interpreter}} for \textbf{\textcolor{blue}{generation of quines}}
      \end{itemize}      
   
    %\begin{block}{MiniKanren}
      MiniKanren [Friedman, Byrd, Kiselyov, 2005]: % is a family of relational EDSLs. Relational constructs:
      %Host languages include:
      %\begin{itemize}
      %  \item Scheme (original implementation)
      %  \item Closure
      %  \item Haskell
        %% \item Go
      %  \item OCaml (implementation we use)
      %\end{itemize}
      %Specification constructors:

      \begin{itemize}
        \item unification ($\equiv$) of two terms;
        \item conjunction ($\wedge$) and disjunction ($\vee$);
        \item fresh variable introduction;
        \item plus some host language (Scheme/Racket, Clojure, Haskell, \textbf{OCaml}).
      \end{itemize}
      
      \begin{exampleblock}{Relational List Concatenation}
        \lstinputlisting{appendo.ml}
      \end{exampleblock}
      \vskip7mm
      Running relational programs:

      \bigskip
      \begin{center}
      \lstinline|(fun q -> append$^o$ [1; 2] [3] q) $\;\leadsto^*\;$ \{q=[1; 2; 3]\}|
      \end{center}
      \bigskip
      %\lstinputlisting{appendorun.ml}
    \end{block}
    
    \begin{block}{Refutational Completeness}
      \vskip7mm

      MiniKanren search:
      \begin{itemize}
         \justifying
         \item is complete (capable of finding all answers); 
         \item can diverge, when no answers exist (or the number of 
         requested answers exceeds the number of existing ones):
         \begin{center}
            \lstinline|(fun q r -> append$^o\;$ q r [1;2;3]) $\;\leadsto^*\;$ $\bot$|
         \end{center}
      \end{itemize}

      Relational specification is \textbf{refutationally complete} [Byrd, 2009], if any query terminates when 
      no answers left. Many important specifications are refutationally \emph{incomplete}.
      \bigskip
      \end{block}

      \begin{alertblock}{Goal}
        Improve the search to make more specifications refutationally complete.
      \end{alertblock}
   
      \begin{block}{Approach}
      \vskip7mm
      One of the reasons for refutational incompleteness is the non-commutativity of conjunction:

      \begin{figure}
        %\large{Information passing}\\
        \begin{tikzpicture}[scale=0.7]
          \node[text width=4cm] at (0,0) {\large $g_1 \wedge g_2$};
          \draw [->, line width=9, blue] (-7,0) -- (-3.5,0);
          \draw [->, line width=9, blue] (3.5,0) -- (7,0); 
          \draw [->, line width=9, blue] (-2,1) .. controls (-1,3) and (1,3) .. (2,1);
          \draw [->, line width=9, blue] (2,-1) .. controls (1,-3) and (-1,-3) .. (-2,-1);
          \draw [line width=9, red] (-0.75,-3.25) -- (0.75,-1.75); 
          \draw [line width=9, red] (-0.75,-1.75) -- (0.75,-3.25);
        \end{tikzpicture}
      %\caption{Information flow}
      \end{figure}
      If $g_1$ diverges, then $g_1 \wedge g_2$ diverges even when $g_2$ fails. 
      \bigskip
 
      \textbf{Idea:} swap the constituents of a conjunction as soon as the divergence of the first
      one is detected dynamically, and repeat the search.

      %\bigskip
      %\bigskip
      
      
      %Shift of recursive call to the end makes {\ttfamily append$^o$} RC.
      %It doesn't work in more complex cases.
    \end{block}
    
    %\begin{block}{Possible solutions}
     % \begin{enumerate}
     %   \item Advanced technics of writing specifications, \\ such as bounding the sizes of terms
     %   \item Simulation of commutative conjunction
     %   \item Reordering of conjuncts during execution
     %   \item[\textcolor{red}{$\uparrow$}] \textcolor{red}{ \textbf{our approach} }
     % \end{enumerate}
    %\end{block}
  
  \end{column}

  \begin{column}{\sepwid}\end{column} % Empty spacer column

  \begin{column}{\onecolwid} % The second column
   
    \begin{block}{Implementation and Evaluation}
    
    \vskip5mm
     
     Improved search was implemented for deeply embedded version of MiniKanren for OCaml. A few
existing specificaitons were evaluated:
     \bigskip

     \begin{exampleblock}{Relational Sorting/Permutations}
        \justifying
        \lstinputlisting{sorto.ml}
        \bigskip
        %For termination we need

        %\begin{itemize}
        %  \item order $1 \to 2$ for forward evaluation
        %  \item order $2 \to 1$ for reverse evaluation
        %\end{itemize}
 
       %\bigskip

        %Calculating all permutations via sorting:
        %\bigskip
        \lstinputlisting{permo.ml}
        \bigskip
        $\cal{NB}:$ requires evaluation of \lstinline|sort$^o$| in \emph{both} directions simultaneously.

        \bigskip
        \justifying
        \textbf{Conventional search:}
        \begin{itemize}
           \justifying
           \item diverges, when \emph{all} permututations are requested for 
                 any positive list length (regardless the direction of evaluation);
           \item unreasonably slow for list length $>3$. 
        \end{itemize}
        \textbf{Improved search:}
        \begin{itemize}
           \justifying
           \item returns all permutations for all directions of evaluation;
           \item works in a reasonable time for all reasonable list lengths (7, 8, ...)
        \end{itemize}
	 \end{exampleblock}
     %Extended search reconstructs (experimentally) \\ the way of information propagation for each call.

     %Multidirectional calls (like in {\ttfamily perm$^o$}) converge.
      \bigskip
      
     %Extention allows to write RC specifications naively.

      \begin{exampleblock}{Binary Division with Reminder}
        \justifying
        Definition-based natural implementation:

        %For the division with remainder 

        %\[ n = m \cdot q + r, \quad r < m \]
 
        %for binary representation of numbers the following definition-based implementation seems to be natural:
        \bigskip
        \lstinputlisting{divoeasy.ml}
        \bigskip

        Does not work well in all directions with conventional search.

        \bigskip

        Fix: use tricky and sophisticated solution [Kiselyov, Byrd, Friedman, Shan, 2008]:
 
        \lstinputlisting{divo.ml}

        ... or improved search.        
      \end{exampleblock}
    \end{block}

  \end{column}

  \begin{column}{\sepwid}\end{column} % Empty spacer column

  \begin{column}{\onecolwid} % The third column
  
    \begin{block}{Formal Development}
      \vskip5mm      
      \begin{center} 
         \textbf{Big-step operational semantics:}

	 $$\Gamma, \iota \vdash \sigma \Xrightarrow{\ocanren{$g$}} S$$

         \begin{itemize}
           \item $\Gamma$~--- functional environment
           \item $\iota$~--- valuation of variables
           \item $\sigma$~--- substitution
           \item $g$~--- goal
           \item $S$~--- (multi) set of answers
         \end{itemize}
$$
\def\arraystretch{3.3}
\begin{array}{c}
       %\Gamma,\,\iota \vdash (\sigma,\,\delta) \Xrightarrow{\ocanren{$t_1\;$ === $\;t_2$}} 
       %  \left\{
       %     \def\arraystretch{1}
       %     \begin{array}{c}
       %        \{(\sigma\ \mbox{\bf{mgu}}\,(t_1 \iota \sigma,\,t_2 \iota \sigma),\,\delta)\}\\
       %        \emptyset,\;\mbox{if no mgu exists}
       %     \end{array}
       %  \right. \\
      
       \Gamma,\,\iota \vdash \sigma \Xrightarrow{\ocanren{$t_1\;$ === $\;t_2$}} 
         \left\{
            \def\arraystretch{1}
            \begin{array}{l}
               \{\sigma\ \mbox{mgu}\,(t_1 \iota \sigma,\,t_2 \iota \sigma)\}\\
               \emptyset,\;\mbox{if no m.g.u. exists}
            \end{array}
         \right. \\
      
      %\dfrac{ \Gamma,\,\iota[x \leftarrow s] \vdash (\sigma,\,\delta \cup \{s\}) \Xrightarrow{\ocanren{$g$}} S }{ \Gamma, \iota \vdash (\sigma,\,\delta) \Xrightarrow{\ocanren{fresh ($x$) $\;g$}} S },\;s \not\in \delta\\ 
      
\dfrac{ \Gamma,\,\iota\,[x \leftarrow s] \vdash \sigma \Xrightarrow{\ocanren{$g$}} S }{ \Gamma, \iota \vdash \sigma \Xrightarrow{\ocanren{fresh ($x$) $\;g$}} S },\;s\;\mbox{is fresh}\\ 
      
%       \dfrac{\Gamma,\,\iota[x_i\gets t_i\iota\sigma] \vdash (\epsilon,\,\delta) \Xrightarrow{\ocanren{$g$}} \{ (\sigma_j,\,\delta_j) \}}
%             {\Gamma,\,\iota \vdash (\sigma,\,\delta) \Xrightarrow{\ocanren{$r^n \ t_1 \ \dots \ t_n$}} \{ (\sigma \sigma_j,\,\delta_j)\} },\,\Gamma\,(r^n) = \lambda\bar{x_i}.g \\

       \dfrac{\Gamma,\,\iota\,[x_i\gets t_i\iota\sigma] \vdash \epsilon \Xrightarrow{\ocanren{$g$}} \{\sigma_j\}}
             {\Gamma,\,\iota \vdash \sigma \Xrightarrow{\ocanren{$r^n \ t_1 \ \dots \ t_n$}} \{ \sigma \sigma_j\} },\,\Gamma\,(r^n) = \lambda x_1\dots x_n.g \\
      
%       \dfrac{ \Gamma,\, \iota \vdash (\sigma,\, \delta) \Xrightarrow{ \ocanren{$g_1$} } S_1,\; 
%       \Gamma,\, \iota \vdash (\sigma,\, \delta) \Xrightarrow{ \ocanren{$g_2$} } S_2}{ \Gamma,\, \iota \vdash (\sigma,\, \delta) \Xrightarrow{ \ocanren{$g_1\;\vee\;g_2$} } S_1 \cup S_2 } \\

      \dfrac{ \Gamma,\, \iota \vdash \sigma \Xrightarrow{ \ocanren{$g_1$} } S_1,\;\;\; 
       \Gamma,\, \iota \vdash \sigma \Xrightarrow{ \ocanren{$g_2$} } S_2}{ \Gamma,\, \iota \vdash \sigma \Xrightarrow{ \ocanren{$g_1\;\vee\;g_2$} } S_1 \uplus S_2 } \\
      
%       \dfrac{ \Gamma,\, \iota \vdash (\sigma,\, \delta) \Xrightarrow{ \ocanren{$g_1$} } \{ (\sigma_i,\, \delta_i)\},\;
%       \Gamma,\, \iota \vdash (\sigma_i,\, \delta_i) \Xrightarrow{ \ocanren{$g_2$} } S_i }{ \Gamma,\,\iota \vdash (\sigma,\, \delta) \Xrightarrow{ \ocanren{$g_1 \wedge g_2$} } \cup_i S_i }

\dfrac{ \Gamma,\, \iota \vdash \sigma \Xrightarrow{ \ocanren{$g_1$} } \{ \sigma_i\},\;\;\;
       \Gamma,\, \iota \vdash \sigma_i \Xrightarrow{ \ocanren{$g_2$} } S_i }{ \Gamma,\,\iota \vdash \sigma \Xrightarrow{ \ocanren{$g_1 \wedge g_2$} } \uplus\,S_i } 
\end{array}
$$
\vskip5mm
      $\cal{NB}:$ describes only terminating runs. 
\end{center}
\vskip1.5cm

\begin{center}      
        \textbf{Divergence test:}
\end{center}
\vskip3mm

        \underline{Theorem} If we have a derivation

        $$
          \begin{array}{c}
            \Gamma, \iota_2 \vdash \sigma_2 \Xrightarrow{\ocanren{$r^n\;b_1 \dots b_n$}} \\
            \ldots \\
            \Gamma, \iota_1 \vdash \sigma_1 \Xrightarrow{\ocanren{$r^n\;a_1 \dots a_n$}}
          \end{array}
        $$

        where for some substitution $\tau$

        \[ a_i \iota_1 \sigma_1 = (b_i \iota_2 \sigma_2) \tau \]

        then there is no $S$, such that

        \[ \Gamma, \iota_1 \vdash \sigma_1 \Xrightarrow{\ocanren{$r^n\;a_1 \dots a_n$}} S \]

\vskip1cm
\begin{center}
     \textbf{Improved Semantics}
      
     $$\Gamma,\, \iota,\, \dots \vdash \sigma \xRightarrow{\ocanren{$\phantom{x}g\phantom{x}$}} S$$

     (too hairy to be presented here)
\end{center}
\vskip1cm
\underline{Theorem} For arbitrary $\Gamma$, $\sigma$, $\iota$, $g$, and $S$ if

     $$
       \Gamma,\,\iota\vdash \sigma\Xrightarrow{\ocanren{$g$}} S
     $$

     then

     $$
       \Gamma,\,\iota,\,\dots\vdash\sigma\xRightarrow{\ocanren{$\phantom{x}a\phantom{x}$}} S
     $$

    \end{block}
    

%    \begin{block}{Implementation details}
%      \begin{itemize}
%        \item Deep embedding required
%        \item $\mathcal{O}(n^2)$ orders are checked in worst case, \\ but optimal can't be missed
%        \item Extended search is $10$ times slower \\ on terminating programs at the moment
%      \end{itemize}
%    \end{block}
    
%    \begin{block}{Acknowledgements}
%      This work is sponsored by JetBrains Research.
%    \end{block}

  \end{column}
  
  \begin{column}{\sepwid}\end{column} % Empty spacer column
\end{columns}

\end{frame} % End of the enclosing frame

\end{document}
