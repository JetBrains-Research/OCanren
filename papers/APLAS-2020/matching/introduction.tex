\section{Introduction}
\label{sec:intro}

Algebraic data types (ADT) are an important tool in functional programming which deliver a way to represent flexible and easy to manipulate data structures.
To inspect the contents of an ADT's values a generic construct~--- \emph{pattern matching}~--- is used. The importance of pattern matching efficient
implementation stimulated the development of various advanced techniques which provide good results in practice. The objective of our work is to use these
results as a baseline for a case study of relational synthesis\footnote{We have to note that this term is overloaded and can be used to refer to completely
different approaches than we utilize.}~--- an approach for program synthesis based on application of relational programming~\cite{TRS,WillThesis}, and,
in particular, relational interpreters~\cite{unified} and relational conversion~\cite{conversion}. Relational programming can be considered as a specific form
of constraint logic programming centered around \textsc{miniKanren}, a combinator-based DSL, implemented for a number of host languages.\footnote{\url{minikanren.org}}
Unlike \textsc{Prolog}, which employs a deterministic depth-first search, \textsc{miniKanren} advocates a completely declarative approach, in which a user is not
allowed to rely on a concrete search discipline, which means, that the specifications, written in \textsc{miniKanren}, are understood much more symmetrically.
The distinctive feature of \textsc{miniKanren} is complete \emph{interleaving search}~\cite{search}. The basic constraint is unification with occurs check, although
advanced implementations support other primitive constructs, such as disequality or finite-domain constraints~\cite{CKanren}. Syntactically, \textsc{miniKanren} is mutually
convertible to \textsc{Prolog}, but, unlike latter, makes use of explicit logical connectives (conjunction and disjunction), existential quantification and unification.

A distinctive application of relational programming is implementing \emph{relational interpreters}~\cite{Untagged}. Unlike conventional interpreters, which for a program and
input value produce output, relational interpreters can operate in various directions: for example, they are capable of computing an input value for a given
program and a given output, or even synthesize a program for a given pairs of input-output values. The latter case forms a basis for program synthesis~\cite{unified,eigen}.
Our approach is based on relational representation of the source language pattern matching semantics on the one hand, and
the semantics of the intermediate-level implementation language on the other. We formulate the condition necessary for a correct and complete implementation of pattern matching and use it to
construct a top-level goal which represents a search procedure for all correct and complete implementations. We also present a number of techniques which make it possible to come up with an
\emph{optimal} solution as well as optimizations to improve the performance of the search. Similarly to many other prior works we use the size of the synthesized code, which can be measured
statically, to distinguish better programs. Our implementation\footnote{\url{https://github.com/kakadu/pat-match}} makes use of \textsc{OCanren}\footnote{\url{https://github.com/JetBrains-Research/OCanren}}~--- a typed implementation of \textsc{miniKanren} for \textsc{OCaml}~\cite{OCanren}, and \textsc{noCanren}\footnote{\url{https://github.com/Lozov-Petr/noCanren}}~--- a converter from the subset
of plain \textsc{OCaml} into \textsc{OCanren}~\cite{conversion}. On an initial  evaluation, performed for a set of benchmarks taken from other papers, showed our synthesizer performing well.
However, being aware of some pitfalls of our approach, we came up with a set of counterexamples on which it did not provide any results in observable time, so we do not consider the problem
completely solved. We also started to work on mechanized formalization\footnote{\url{https://github.com/dboulytchev/Coq-matching-workout}}, written in \textsc{Coq}~\cite{Coq}, to
make the justification of our approach more solid and easier to verify, but this formalization is not yet complete. 

 

\begin{comment}
We apply relational programming techniques to the problem of synthesizing efficient implementation for a pattern matching construct.
Although in principle pattern matching can be implemented in a trivial way, the result suffers from inefficiency in terms of both
performance and code size. Thus, in implementing functional languages alternative, more elaborate  approaches are widely used.
However, as there are multiple kinds and flavors of pattern matching constructs, these approaches have to be specifically developed
and justified for each concrete inhabitant of the pattern matching ``zoo''. We formulate the pattern matching synthesis problem in
declarative terms and apply relational programming, a specific form of constraint logic programming, to develop a 
develop optimizations which improve the efficiency of the synthesis and guarantee the
optimality of the result. 
\end{comment}
