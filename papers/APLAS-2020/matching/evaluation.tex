\newcommand{\mywrap}[1]{\centering{{\tiny\textbf{#1}}}}
\newcommand{\myhack}[2]{
  \multicolumn{1}{m{#1}|}{
    \rotatebox{0}{\parbox[c][1.5cm]{#1}{\mywrap {#2}}}
  }}
\newcommand{\myhackd}[2]{
  \multicolumn{1}{m{#1}||}{
    \rotatebox{0}{\parbox[c][1.5cm]{#1}{\mywrap {#2}}}
  }
%  \multirow{1}{*}{\parbox[c][2cm]{#1}{\rotatebox[origin=c]{90}{\mywrap {#2}}}}
}
\newcolumntype{C}{ >{\centering\arraybackslash} m{4cm} }

\begin{table}[h]
\caption{The results of synthesis evaluation}\label{fig:eval}
\centering
\begin{tabular}{|l||c|c|c||c||c|c||c|}
  \hline  
%  \parbox[b][.5cm]{3cm}{\textbf{Patterns}} & 
  \mywrap{\phantom{PP1} Patterns} & 
  \myhack{1.2cm}{Number of samples} &
  \myhack{1.2cm}{1st answer size } &
  \myhackd{1.2cm}{1st answer time (ms)} &
  \myhackd{1.2cm}{Answers found} &
  \myhack{1.2cm}{Optimal answer size} &
  \myhackd{1.2cm}{Optimal answer time (ms)} &
  \myhack{1.2cm}{Total search time (ms)} 
    \\ \hline

    \begin{lstlisting}[basicstyle=\scriptsize,belowskip=-1.5em]

 A
 B
 C
 $$
    \end{lstlisting}&3&2&1&2&1&2 & 2\\
        \hline
    \begin{lstlisting}[basicstyle=\scriptsize,belowskip=-1.5em]

 true
 false
 $$
    \end{lstlisting} &2&1&$<$1&1& N/A & N/A & $<$1  \\
        \hline
%        3&
        \begin{lstlisting}[basicstyle=\scriptsize,belowskip=-1.5em]
          
 (true, _)
 (_, true)
 (false, false)
 $$         
    \end{lstlisting} &4&2&5&1& N/A & N/A & 11\\
        \hline
%        4&
     \begin{lstlisting}[basicstyle=\scriptsize,belowskip=-1.5em]

 (_, false, true)
 (false, true, _)
 (_, _, false)
 (_, _, true)
 $$      
    \end{lstlisting} &8&6&$\sim$1000&3&4&$\sim$2100& $\sim$2300\\
        \hline
%        5&
%     \begin{lstlisting}[basicstyle=\scriptsize]
%([], _)
%(_, [])
%(_ :: _, _ :: _)
%    \end{lstlisting} &100&10&4&5&1&6\\
%        \hline
%        6&
        \begin{lstlisting}[basicstyle=\scriptsize,belowskip=-1.5em]
          
 (Succ _, Succ _)
 (Zero, _)
 (_, Zero)
 $$         
    \end{lstlisting} &4&2&30&1& N/A & N/A &$\sim$50  \\
        \hline
        \begin{lstlisting}[basicstyle=\scriptsize,belowskip=-1.5em]
          
 (Nil, _)
 (_, Nil)
 (Nil2, _)
 (_, Nil2)
 (Cons (_,_),Cons(_,_))
 $$      
    \end{lstlisting}  &9&5&45&1& N/A & N/A  &   $\sim$800       \\ 
      \hline
      \begin{lstlisting}[basicstyle=\scriptsize,belowskip=-1.5em]
        
 (_, _, (Ldi _)::_)
 (_, _, (Push _)::_)
 $$
      \end{lstlisting} &5&3&11&1& N/A  & N/A & $\sim$ 30    \\
        \hline      
        \begin{lstlisting}[basicstyle=\scriptsize,belowskip=-1.5em]
          
 (_, _, (Ldi _)::_)
 (_, _, (Push _)::_)
 (Int(_), _, (IOp _)::_)
 $$
        \end{lstlisting}
     &20&7&$\sim$1700&3&5&   $\sim$1800   &   $\sim$11000       \\ \hline
\end{tabular}
\end{table}
\FloatBarrier

\section{Evaluation}
\label{sec:eval}

%In the table~\ref{fig:eval} we summarize benchmarking results for our synthesis algorithm.

We performed an evaluation of the pattern matching synthesizer on a number of benchmarks.
The majority of benchmarks were prepared manually; we didn't use any specific benchmark sets mentioned in literature~\cite{Scott2000WhenDM} yet.
The evaluation was performed on a desktop computer with Intel Core i7-4790K CPU @ 4.00GHz processor and 8GB of memory,
\textsc{OCanren} was compiled with \mbox{ocaml-4.07.1+fp+flambda}. All benchmarks were executed in the native mode ten times,
then average monotonic clock time was taken. The results of the evaluation are shown in Table~\ref{fig:eval}.

In the table the double bar separates input data from output. %Inputs are: 
The patterns used for synthesis form the input of synthesis algorithm.
%and the requested number of answers. 
Outputs are: the size of generated complete samples set, the size of the first answer, the running time before receiving the
first answer, the total number of programs synthesized, the size of the optimal (last) answer and the total time of the synthesis. The information about a last answer will be omitted, if the synthesizer finds only a single answer. After discovering of the last answer, the synthesizer may spend some time to check that no smaller answer exists.
%The time between the start and the end of synthesis is shown in the last column. % of Table.~\ref{fig:eval}.
In all benchmarks the structural constraint is being checked every 100 unifications and all answers were requested.

We also give an example of synthesized program for the 4th benchmark, which was taken from~\cite{maranget2008}. We used this benchmark in the Section~\ref{sec:related} as the first example (see Fig.~\ref{fig:match-example}).

Our algorithm starts the
synthesis and in about 1 second discovers the first answer, which is equivalent to the solution on Fig.~\ref{fig:matching-example2} and
consists of 6 \lstinline|switch| expressions. In about half a second it synthesizes a better answer with 5 \lstinline|switch| expressions:

\begin{lstlisting}
switch $\bullet[0]$ with
| true -> (switch $\bullet[2]$ with  
           | true -> (switch $\bullet[1]$ with  | true -> 4 | _ -> 1)
           | _ -> 3)
| _ -> (switch $\bullet[1]$ with  
        | true -> 2
        | _ -> (switch $\bullet[2]$ with  | true -> 1 | _ -> 3))
\end{lstlisting}

And after about half a second it synthesizes the optimal answer of size 4. Then it searches for an answer which would have less than 4 switch expressions
for some time, fails to find one and finishes the synthesis. 
The time between the start and the end of synthesis is shown in the last column of Table.~\ref{fig:eval}.

Our approach currently does not work fast for large benchmarks. On Fig.~\ref{fig:pcf} we cite an example extracted from a bytecode
machine for PCF~\cite{maranget2008,Plotkin1977LCFCA}. For such a complex examples (in terms of type definition complexity and the number and size of patterns)
both the size of the search space and the number of samples is too large for our approach to work so far.

The last two benchmarks were constructed by reducing number of types (Fig.~\ref{fig:pcf-reduced}) and clauses in PCF example.% (fig.~\ref{fig:pcf}).

\begin{figure}%[H]
\centering
\begin{tabular}{c} % tabular only for centering
\begin{lstlisting}
type code = Push | Ldi of int | IOp of int | Int of int 
type stack_item = Val of code | Env of int | Code of int 
type scrutinee = code * stack_item list * code list
\end{lstlisting}
\end{tabular}
\caption{Reduced types of PCF example }
\label{fig:pcf-reduced}
\end{figure}

\begin{figure}%[H]
\centering
\begin{tabular}{c} % tabular only for centering
\begin{lstlisting}
let rec run  s c e =
  match s,c,e with
  | (_,_,Ldi i::_) -> 1
  | (_,_,Push::_)  -> 2
  | (Int _,Val (Int _)::_,IOp _::_) -> 3
  | (Int _,_,Test (_,_)::c) -> 4
  | (Int _,_,Test (_,_)::c) -> 5
  | (_,_,Extend::_) -> 6
  | (_,_,Search _::_) -> 7
  | (_,_,Pushenv::_) -> 8
  | (_,Env e::s,Popenv::_) -> 9
  | (_,_,Mkclos cc::_) -> 10
  | (_,_,Mkclosrec _::_) -> 11
  | (Clo (_,_), Val _::_, Apply::_) -> 12
  | (_,(Code _::Env _::_),[]) -> 13
  | (_,[],[]) -> 14
\end{lstlisting}
\end{tabular}
\caption{An example from a bytecode machine for PCF}
\label{fig:pcf}
\end{figure}

