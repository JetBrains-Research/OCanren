\section{Related Works}
\label{sec:related}

Pattern matching can be considered as a generalization of conventional conditional control-flow construct ``\lstinline|if .. then .. else|'' and in principle
can be decomposed into a nested hierarchy of those; from this standpoint the problem of pattern matching implementation can be considered trivial. However,
some decompositions are obviously better than others. We repeat here an example from~\cite{maranget2008} to demonstrate this difference (see Fig.~\ref{fig:match-example}).
Here we match a triple of boolean values $x$, $y$, and $z$ against four patterns (Fig.~\ref{fig:matching-example1}; we use \textsc{OCaml}~\cite{ocaml} as
reference language). The na\"{i}ve implementation of this example is shown on Fig.~\ref{fig:matching-example2}; however if we decide to match $y$ first the result
becomes much better (Fig.~\ref{fig:matching-example3}).

\begin{figure}[t]
\begin{subfigure}[t]{0.25\linewidth}
\centering
\begin{lstlisting}
match x,y,z with
| _, F, T -> 1
| F, T, _ -> 2
| _, _, F -> 3
| _, _, T -> 4
\end{lstlisting}
\vskip18.5mm
\caption{Pattern matching}
\label{fig:matching-example1}
\end{subfigure}
\hspace{0.5cm}
\begin{subfigure}[t]{0.3\linewidth}
\centering
\begin{lstlisting}
if x then
  if y then
    if z then 4 else 3
  else
    if z then 1 else 3
else
  if y then 2
  else
    if z then 1 else 3
\end{lstlisting}
\caption{A correct but non-optimal implementation}
\label{fig:matching-example2}
\end{subfigure}
\hspace{1.0cm}
\begin{subfigure}[t]{0.3\linewidth}
\centering
\begin{lstlisting}
if y then
  if x then
    if z then 4 else 3
  else 2
else
  if z then 1 else 3
\end{lstlisting}
\vskip13.5mm
\caption{Optimal implementation}
\label{fig:matching-example3}
\end{subfigure}
\caption{Pattern matching implementation example} 
\label{fig:match-example}
\end{figure}

\begin{comment}
\begin{figure}[ht]
\begin{minipage}[b]{0.3\linewidth}
\centering
\label{fig:figure1}
\end{minipage}
\hspace{0.5cm}
\begin{minipage}[b]{0.3\linewidth}
\centering
\begin{lstlisting}
switch x with 
| true -> 
    switch y with 
    | true -> 
       switch z with 
       | true -> 4
       | _ -> 3
    | _ -> 
      switch z with 
      | true -> 1
      | _ -> 3 
| _ -> 
   switch y with 
   | true -> 2 
   | _ -> if z then 1 else 3
\end{lstlisting}
\end{minipage}
\hspace{0.5cm}
\begin{minipage}[b]{0.3\linewidth}
\centering
\end{minipage}
\end{figure}
\end{comment}


%clasification 1
%Although semantics of pattern matching can be given as a sequence of srutinee's sub expression comparisons (Figure~\ref{fig:matchpatts}) effective compilers don't follow
%this approach. 

The quality of a pattern matching implementation can be measured in various ways. One can either optimise the run time cost by minimizing the amount of checks performed, or the static
cost by minimizing the size of the generated code. \emph{Decision trees}
%~\cite{?} 
% I removed cite because 
% 1) can't find a good citation for it not in the field of compilers
% 2) folks in other papers don't refernece them. Only backtracking automata are begin referenced
are considered suitable for the first criterion as they check every subexpression no more than once.
However, minimizing the size of a decision tree is known to be NP-hard~\cite{baudinet1985tree}, and as a rule various heuristics, using, for example,
the number of nodes, the length of the longest path and the average length of all paths are applied during compilation. In \cite{Scott2000WhenDM} the results of experimental
evaluation of nine heuristics for Standard ML of New Jersey are reported.

For minimizing the static cost \emph{backtracking automata} can be used since they admit a compact representation but in some cases can perform repeated checks.

%clasification 2
There is a certain difference in dealing with pattern matching in strict and non-strict languages. For strict languages checking sub-expressions of the scrutinee in any order is allowed.
The pattern matching implementation for strict languages can operate in \emph{direct} or \emph{indirect} styles. In the direct style the construction of an implementation is done explicitly. In indirect the construction of implementation requires some post-processing, which can vary from easy simplifications to complicated supercompilation
techniques~\cite{sestoft1996}. The main drawback of indirect approach is that the size of intermediate data structures can be exponentially large.

For non-strict languages pattern matching should evaluate only those sub-expressions which are necessary for performing pattern matching. If not done carefully pattern matching can
change the termination behavior of the program. In general non-strict languages put more constraints on pattern matching and thus admit a smaller  set of heuristics. 
A few approaches for checking sub-expressions in lazy languages have been proposed. In~\cite{augustsson1985} a simple left-to-right order of subexpression checking was proposed
with a proof that this particular order doesn't affect termination. The backtracking automaton being built takes a form of a DAG to reduce the code size. A few refinements have been added in~\cite{wadler1987}
as a part of textbook~\cite{peytonjones1987} on the implementation of lazy functional languages. The approach from this book is used in the current version of GHC~\cite{marlow2012the}.
%~\footnote{The Glasgow Haskel Compiler, \url{http://www.haskell.org/ghc/}}
\cite{laville1991} models values in lazy languages using \emph{partial terms}, although it doesn't scale to types with infinite sets of constructors (like integers). The approach doesn't
test all subexpressions from left to right as does~\cite{augustsson1985} but aims to  avoid performing unnecessary checks by constructing \emph{lazy automaton}. Pattern matching for
lazy languages has been compiled also to decision trees~\cite{maranget1992} and later into \emph{decision DAGs} which in some cases allows the compiler to make the code
smaller~\cite{maranget1994}.

%about automata
The inefficiency of backtracking automata have been
improved in~\cite{maranget2001}. The approach utilizes a matrix representation for pattern matching. It splits the current matrix according to constructors in the
first column and reduces the task to compiling matrices with fewer rows. The technique is indirect; in the end a few optimizations are performed by introducing
special \emph{exit} nodes to the compiled representation. %No preprocessing is required for this scheme: or-pattern receives a special treatment during compilation process.
The approach from this paper is used in the current implementation of the \textsc{OCaml} compiler.

The previous approach uses the first column to split the matrix. In~\cite{maranget2008} the \emph{necessity} heuristic has been introduced which recommends which column should be
used to perform the split. Good decision trees which are constructed in this work can perform better in corner cases than~\cite{maranget2001}, but for practical use the
difference is insignificant.

While existing approaches deliver appropriate solutions for certain forms of pattern matching constructs, they have to be extended in an \emph{ad hoc} manner each time
the syntax and semantics of pattern matching construct changes. For example, besides a simple conventional form of pattern matching there are a number of extensions:
guards (first appeared in KRC language~\cite{turner2013}), disjunctive patterns~\cite{ocaml}, non-linear patterns~\cite{mcbride1969symbol}, active patterns~\cite{activepatterns} and pattern matching for polymorphic variants~\cite{Garrigue98} 
%and generalized algebraic datatypes~\cite{?}) 
which require a separate customized algorithms to be developed.

\begin{comment}
There are a few different approaches for compiling pattern matching. GHC is using influential paper~\cite{Jones1987}, OCaml is currently based on~\cite{maranget2001} although a work~\cite{maranget2008} can slightly improve effectiveness of generated code. 

Although semantics of pattern matching can be given as a sequence of scrutinee's sub expression comparisons (Fig.~\ref{fig:matchpatts}) effective compilers don't follow this approach. One can either optimise run time cost by minimizing amount of checks performed or static cost by minimizing the size of generated code. \emph{Decision trees} are good for the first criteria, because they check every subexpression not more than once. \emph{Backtracking automata} are rather compact but in some cases can perform repeated checks.


Minimizing the size of decision tree is  NP-hard (\cite{baudinet1985tree}, without proof) and usually heuristics are applied during compilation, for example: count of nodes, length of the longest path, average length of all paths. The paper~\cite{Scott2000WhenDM} performs experimental evaluation of 9  heuristics on the base of for Standard ML of New Jersey.


The matching compilers for strict languages can work in \emph{direct} or \emph{indirect} styles. The first ones return effective code immediately. In the second style to construct final answer some post processing is required. It can vary from easy simplifications to complicated supercompilation techniques~\cite{sestoft1996}. The main drawback of indirect style is that size of intermediate data structures can be exponentially large.

For strict languages checking  sub expressions of scrutinee in any order is allowed. For lazy languages pattern matching should evaluate only these sub expressions which are necessary for performing pattern matching. If not careful pattern matching can change termination behavior of the program.  In general lazy languages setup more constraints on pattern matching and because of that allow lesser set of heuristics.

A few approaches for checking sub expressions in lazy languages has been proposed~\cite{augustsson1985,laville1991}. \cite{laville1991} models values in lazy languages using \emph{partial terms}, although this approach doesn't scale to types with infinite constructor sets (like integers). In  the \cite{suarez1993} the similar approach is extended by special treatment of overlapping patterns. Pattern matching has been compiled to decision trees~\cite{maranget1992} and later \cite{maranget1992} into \emph{decision DAGs} that allow in some cases to make code smaller.

The first works on compilation to backtracking automaton are~\cite{augustsson1985,wadler1987}. 

The inefficiency of backtracking automaton has been improved in~\cite{maranget2001}. The approach utilizes matrix representation for pattern matching. It splits current matrix  according to constructors in the first column and reduces the task to compiling matrices with less rows. The technique is indirect, in the end a few optimizations are performed by introducing special \emph{exit} nodes to the compiled representation.
No preprocessing is required for this scheme: or-pattern receive a special treatment during compilation process.
 The approach from this paper is used in current implementation of OCaml compiler.

Previous approach uses first column to split the matrix. In~\cite{maranget2008} has been introduced \emph{necessity} heuristic that recommends which column should be used to perform split. Good decision trees that are constructed in this work can perform better in corner cases than~\cite{maranget2001} but for practical cases the difference is insignificant.

To summarize, compilers can try to optimize pattern matching either for guaranteed code speed or for guaranteed code size. There are distinct techniques to minimize drawbacks of both approaches.
\end{comment}
