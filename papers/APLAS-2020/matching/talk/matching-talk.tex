\def\pgfsysdriver{pgfsys-dvipdfm.def} % required for tress inside
\documentclass[aspectratio=169
  , xcolor={svgnames}
  , hyperref={ colorlinks,citecolor=Blue
             , linkcolor=DarkRed,urlcolor=DarkBlue}
  , russian
  ]{beamer}
\usetheme{CambridgeUS}
\usefonttheme{professionalfonts}

\usepackage{xcolor}
\definecolor{YellowGreen} {HTML}{B5C28C}
\definecolor{ForestGreen} {HTML}{009B55}
\definecolor{MyBackground}{HTML}{F0EDAA}


\usepackage{xltxtra} % load xunicode
\usepackage{polyglossia}
\setmainlanguage{english}

\let\cyrillicfonttt\monofamily

%\newfontfamily\dejaVuSansMono{DejaVu Sans Mono}
% https://github.com/vjpr/monaco-bold/raw/master/MonacoB/MonacoB.otf
%\newfontfamily\monacoB{MonacoB}

\usepackage{listings}
\lstdefinelanguage{ocanren}{
keywords={run, conde, fresh, let, in, match, with, when, class, type,
object, method, of, rec, repeat, until, while, not, do, done, as, val, inherit,
new, module, sig, deriving, datatype, struct, if, then, else, open, private, virtual, include, success, failure,
true, false},
sensitive=true,
commentstyle=\small\itshape\ttfamily,
keywordstyle=\ttfamily\textbf,
identifierstyle=\ttfamily,
basewidth={0.5em,0.5em},
columns=fixed,
mathescape=true,
fontadjust=true,
literate={fun}{{$\lambda$}}1 {->}{{$\to$}}3 {===}{{$\equiv$}}1 {=/=}{{$\not\equiv$}}1 {|>}{{$\triangleright$}}3 {\\/}{{$\vee$}}2 {/\\}{{$\wedge$}}2 {^}{{$\uparrow$}}1,
morecomment=[s]{(*}{*)}
}

\lstset{
language=ocanren
}
  
%%%%%%%%%%%%%%%%%%%%%%%%%%%%%%%%%%%%%%%%%%%%
\setmainfont[
 Ligatures=TeX,
 Extension=.otf,
 BoldFont=cmunbx,
 ItalicFont=cmunti,
 BoldItalicFont=cmunbi,
]{cmunrm}
% С засечками (для заголовков)
\setsansfont[
 Ligatures=TeX,
 Extension=.otf,
 BoldFont=cmunsx,
 ItalicFont=cmunsi,
]{cmunss}


\usefonttheme{professionalfonts}
\usepackage{times}
\usepackage{tikz}
\usetikzlibrary{cd}
% \usepackage{tikz-cd}
\usepackage{amsmath}
%\DeclareMathOperator{->}{\rightarrow}
\newcommand\iso{\ensuremath{\cong}}
\usepackage{verbatim}
\usepackage{graphicx}
\usetikzlibrary{arrows,shapes}

\usepackage{fontawesome}
% \newfontfamily{\FA}{Font Awesome 5 Free} % some glyphs missing
\expandafter\def\csname faicon@facebook\endcsname{{\FA\symbol{"F09A}}}
\def\faQuestionSign{{\FA\symbol{"F059}}}
\def\faQuestion{{\FA\symbol{"F128}}}
\def\faExclamation{{\FA\symbol{"F12A}}}
\def\faUploadAlt{{\FA\symbol{"F093}}}
\def\faLemon{{\FA\symbol{"F094}}}
\def\faPhone{{\FA\symbol{"F095}}}
\def\faCheckEmpty{{\FA\symbol{"F096}}}
\def\faBookmarkEmpty{{\FA\symbol{"F097}}}

\newcommand{\faGood}{\textcolor{ForestGreen}{\faThumbsUp}}
\newcommand{\faBad}{\textcolor{red}{\faThumbsODown}}
\newcommand{\faWrong}{\textcolor{red}{\faTimes}}
\newcommand{\faMaybe}{\textcolor{blue}{\faQuestion}}
\newcommand{\faCheckGreen}{\textcolor{ForestGreen}{\faCheck}}

\usepackage{soul} % for \st that strikes through
\usepackage[normalem]{ulem} % \sout


\usepackage{tabulary}

% sudo aptget install ttf-mscorefonts-installer
\defaultfontfeatures{Ligatures={TeX}}
\setmainfont{Times New Roman}
\setsansfont{CMU Sans Serif}

\setmonofont[Scale=1.0,
    BoldFont=lmmonolt10-bold.otf,
    ItalicFont=lmmono10-italic.otf,
    BoldItalicFont=lmmonoproplt10-boldoblique.otf
]{lmmono9-regular.otf}

%\usepackage[cache=true]{minted}

\author[]{\underline{Dmitry Kosarev} \textsuperscript{1,2} \and Dmitry Boulytchev\inst{1,2}}
\institute[]{\textsuperscript{1} JetBrains Research \and \inst{2} Saint-Petersburg University}

\addtobeamertemplate{title page}{}{
  \begin{center}{\tiny Build date: \today}\end{center}
}

\usepackage{inconsolata}
\setmonofont[Scale=1.0,
    BoldFont=lmmonolt10-bold.otf,
    ItalicFont=lmmono10-italic.otf,
    BoldItalicFont=lmmonoproplt10-boldoblique.otf
]{lmmono9-regular.otf}

\newfontfamily{\monott}{Latin Modern Mono}

\usepackage{listings}
\lstdefinelanguage{ocanren}{
keywords={run, conde, fresh, let, in, match, with, when, class, type,
object, method, of, rec, repeat, until, while, not, do, done, as, val, inherit,
new, module, sig, deriving, datatype, struct, if, then, else, open, private, virtual, include, success, failure,
true, false},
sensitive=true,
commentstyle=\small\itshape\ttfamily
,keywordstyle=\fontfamily{monott}\selectfont \textbf
,identifierstyle=\fontfamily{monott}\selectfont%\ttfamily
,basewidth={0.5em,0.5em},
columns=fixed,
mathescape=true,
fontadjust=true,
literate={fun}{{$\lambda$}}1 {->}{{$\to$}}3 {===}{{$\equiv$}}1 {=/=}{{$\not\equiv$}}1 {|>}{{$\triangleright$}}3 {\\/}{{$\vee$}}2 {/\\}{{$\wedge$}}2 {^}{{$\uparrow$}}1,
morecomment=[s]{(*}{*)}
}

\setmonofont[Mapping=tex-text]{CMU Typewriter Text}
\lstdefinelanguage{ocaml}{
keywords={@type, function, fun, let, in, match, with, when, class, type,
nonrec, object, method, of, rec, repeat, until, while, not, do, done, as, val, inherit, and,
new, module, sig, deriving, datatype, struct, if, then, else, open, private, virtual, include, success, failure,
lazy, assert, true, false, end},
sensitive=true,
commentstyle=\small\itshape\ttfamily,
keywordstyle=\ttfamily\bfseries, %\underbar,
identifierstyle=\ttfamily,
basewidth={0.5em,0.5em},
columns=fixed,
fontadjust=true,
literate={->}{{$\to$}}3 {===}{{$\equiv$}}1 {=/=}{{$\not\equiv$}}1 {|>}{{$\triangleright$}}3 {\\/}{{$\vee$}}2 {/\\}{{$\wedge$}}2 {>=}{{$\ge$}}1 {<=}{{$\le$}} 1,
morecomment=[s]{(*}{*)}
}

\usepackage{fontawesome}
%\newcommand{\faGood}{\textcolor{ForestGreen}{\faThumbsUp}}
%\newcommand{\faBad}{\textcolor{red}{\faThumbsODown}}

\usepackage{subcaption}
\usepackage{tikz}
\usepackage{tikz-qtree}
\usetikzlibrary{trees}
\usepackage[edges]{forest}
\forestset{.style={
  forked edges,
    for tree={    grow'=0,    draw,    align=c,    font=\sffamily,
        rounded corners  }
  }}

%\newcommand{\lstquot}[1]{``\lstinline{#1}''}
%\newcommand{\sembr}[1]{\llbracket{#1}\rrbracket}
%\newcommand\false{$f\!alse$}
%\newcommand\myif{i\!f}


\def\transarrow{\xrightarrow}
\newcommand{\setarrow}[1]{\def\transarrow{#1}}

\def\padding{\phantom{X}}
\newcommand{\setpadding}[1]{\def\padding{#1}}

\def\subarrow{}
\newcommand{\setsubarrow}[1]{\def\subarrow{#1}}

%\newcommand{\trule}[2]{\dfrac{#1}{#2}}
%\newcommand{\crule}[3]{\dfrac{#1}{#2},\;{#3}}
%\newcommand{\withenv}[2]{{#1}\vdash{#2}}
\newcommand{\trans}[3]{{#1}\transarrow{\padding{\textstyle #2}\padding}\subarrow{#3}}
%\newcommand{\ctrans}[4]{{#1}\transarrow{\padding#2\padding}\subarrow{#3},\;{#4}}
%\newcommand{\llang}[1]{\mbox{\lstinline[mathescape]|#1|}}
%\newcommand{\pair}[2]{\inbr{{#1}\mid{#2}}}
\newcommand{\inbr}[1]{\left<{#1}\right>}
%\newcommand{\highlight}[1]{\color{red}{#1}}
%\newcommand{\ruleno}[1]{\eqno[\scriptsize\textsc{#1}]}
\newcommand{\ruleno}[1]{\mbox{[\textsc{#1}]}}
\newcommand{\rulename}[1]{\textsc{#1}}
\newcommand{\inmath}[1]{\mbox{$#1$}}
%\newcommand{\lfp}[1]{fix_{#1}}
%\newcommand{\gfp}[1]{Fix_{#1}}
\newcommand{\vsep}{\vspace{-2mm}}
\newcommand{\supp}[1]{\scriptsize{#1}}
\newcommand{\sembr}[1]{\llbracket{#1}\rrbracket}
\newcommand{\cd}[1]{\texttt{#1}}
%\newcommand{\free}[1]{\boxed{#1}}
%\newcommand{\binds}{\;\mapsto\;}
\newcommand{\dbi}[1]{\mbox{\bf{#1}}}
\newcommand{\sv}[1]{\mbox{\textbf{#1}}}
%\newcommand{\bnd}[2]{{#1}\mkern-9mu\binds\mkern-9mu{#2}}
%\newcommand{\meta}[1]{{\mathcal{#1}}}
%\newcommand{\dom}[1]{\mathtt{dom}\;{#1}}
%\newcommand{\primi}[2]{\mathbf{#1}\;{#2}}
%\renewcommand{\dom}[1]{\mathcal{D}om\,({#1})}
%\newcommand{\ran}[1]{\mathcal{VR}an\,({#1})}
%\newcommand{\fv}[1]{\mathcal{FV}\,({#1})}
%\newcommand{\tr}[1]{\mathcal{T}r_{#1}}
%\newcommand{\diseq}{\not\equiv}
%\newcommand{\reprfunset}{\mathcal{R}}
%\newcommand{\reprfun}{\mathfrak{f}}
%\newcommand{\cstore}{\Omega}
%\newcommand{\cstoreinit}{\cstore_\epsilon^{init}}
%\newcommand{\csadd}[3]{add(#1, #2 \diseq #3)}  %{#1 + [#2 \diseq #3]}
%\newcommand{\csupdate}[2]{update(#1, #2)}  %{#1 \cdot #2}
\newcommand{\primi}[1]{\ensuremath{\mathbf{#1}}}
\newcommand{\sem}[1]{\llbracket #1 \rrbracket}
\newcommand{\ir}{\ensuremath{\mathcal{S}}}
\usepackage{tikz}
\newcommand*\circled[1]{\tikz[baseline=(char.base)]{
    \node[shape=circle,draw,inner sep=1pt] (char) {#1};}}

%\let\emptyset\varnothing
%\let\eps\varepsilon

% for fancy table
%\newcommand{\lheadl}[2]{\multicolumn{1}{|>{\centering\arraybackslash}m{#1}|}{{#2}}}
%\newcommand{\head}[2]{\multicolumn{1}{>{\centering\arraybackslash}m{#1}}{\textbf{\small #2}}}
%\newcommand{\headll}[2]{\multicolumn{1}{>{\centering\arraybackslash}m{#1}||}{\textbf{\small #2}}}
%\newcommand{\lheadll}[2]{\multicolumn{1}{|>{\centering\arraybackslash}m{#1}||}{\textbf{\small #2}}}
%\newcommand{\headl}[2]{\multicolumn{1}{>{\centering\arraybackslash}m{#1}|}{\textbf{\small #2}}}
%\usepackage{longtable}
%\newcommand{\nodata}{}
%\newcommand{\tablenotemark}[1]{#1}

\newcommand{\contributions}{
\begin{enumerate}
\item[I] Synthesis via combination of two relational interpreters
\item[II] Replacing $\forall$ quantifier by a finite set of examples
\item[III] Branch and bound optimization of synthesis using \emph{structural constraints}
\end{enumerate}
}

\lstdefinelanguage{ocamllambda}{
keywords={catch, switch, default, case, with, failwith, exit, true, false, ::},
sensitive=true,
commentstyle=\small\itshape\ttfamily,
keywordstyle=\ttfamily\textbf,
identifierstyle=\ttfamily,
basewidth={0.5em,0.5em},
columns=fixed,
mathescape=true,
fontadjust=true,
literate={->}{{$\to$}}3 {===}{{$\equiv$}}1  {Scru}{{$\bullet$}}2,
morecomment=[s]{(*}{*)}
}

%%%%%%%%%%%%%%%%%%%%%%%%%%%%%%%%%%%%%%%%

\title{Relational Synthesis for Pattern Matching}

\date{APLAS, 2020}

%\newcommand{\verbatimfont}[1]{\def\verbatim@font{#1}}
%\usepackage{verbatimbox}

\begin{document}
{
\begin{frame}[fragile]
  \begin{tabular}{p{5.5cm} p{5.5cm}}
   \begin{center}
%      \includegraphics[height=1.5cm]{pictures/jetbrainsResearch.pdf}
    \end{center}
    &
    \begin{center}
   %   \includegraphics[height=1.5cm]{pictures/SPbGU_Logo.png}
    \end{center}
  \end{tabular}
  \titlepage
\end{frame}
}

%\maketitle

% For every picture that defines or uses external nodes, you'll have to
% apply the 'remember picture' style. To avoid some typing, we'll apply
% the style to all pictures.
\tikzstyle{every picture}+=[remember picture] 

% By default all math in TikZ nodes are set in inline mode. Change this to
% displaystyle so that we don't get small fractions.
\everymath{\displaystyle}
%\begin{comment}
% Uncomment these lines for an automatically generated outline.
\begin{frame}{Pattern Matching}
Efficient compilation of pattern matching is essential for functional programming
\vspace{1cm}

%\includegraphics{pictures/SPbGU_Logo.png}

Two main approaches to compile:
\begin{itemize}
\item decision diagrams
\begin{itemize}
\item guaranteed amount of checks performed (speed)
\end{itemize}
\item backtracking automaton
\begin{itemize}
\item guaranteed code size
\end{itemize}\pause
\item \textbf{synthesis} (our approach)
\end{itemize}

\end{frame}

\defverbatim[colored]{\matchA}{
\begin{lstlisting}
match x,y,z with
| _,F,T -> 1
| F,T,_ -> 2
| _,_,F -> 3
| _,_,T -> 4
\end{lstlisting}
}

\defverbatim[colored]{\matchB}{
\begin{lstlisting}
if x then
  if y then
    if z then 4 else 3
  else
    if z then 1 else 3
else
  if y then 2
  else
    if z then 1 else 3
\end{lstlisting}
}

\defverbatim[colored]{\matchC}{
\begin{lstlisting}
if y then
  if x then
    if z then 4 else 3
  else 2
else
  if z then 1 else 3
\end{lstlisting}
}

\defverbatim[colored]{\forestB}{
\begin{forest} 
[\texttt{x}%, tikz={\draw[{Latex}-, thick] (.north) --++ (0,1);}
    [\texttt{y}
          [\texttt{z}
                [4] 
                [3] 
            ]   
         [\texttt{z}
               [1] 
               [3] 
           ]    
    ]   
    [\texttt{y}
        [2] 
        [\texttt{z}
            [1] 
            [3] 
        ]   
    ]   
] 
\end{forest}
}

\defverbatim[colored]{\forestC}{
\begin{forest} 
[\texttt{y}%, tikz={\draw[{Latex}-, thick] (.north) --++ (0,1);}
    [\texttt{x}
          [\texttt{z}
                [4] 
                [3] 
            ]   
         [2]    
    ]
    [\texttt{z}
               [1] 
               [3] 
           ]
]
\end{forest}
}



\begin{frame}[fragile]{An Example of Pattern Matching}
\framesubtitle{And It's Two Compiled Representations}
%\begin{figure}[t]
%\begin{subfigure}[c]{0.25\linewidth}
%\begin{minipage}[t]{0.25\linewidth}
%\matchA
%\end{minipage}
%\caption{Pattern matching}
%\end{subfigure}
%\hspace{0.5cm}
%\begin{subfigure}[c]{0.32\linewidth}
%\begin{lstlisting}
%if x then
%  if y then
%    if z then 4 else 3
%  else
%    if z then 1 else 3
%else
%  if y then 2
%  else
%    if z then 1 else 3
%\end{lstlisting}
%\caption{Correct but semi-optimal compilation}
%\end{subfigure}
%\hspace{0.5cm}
%\begin{subfigure}[c]{0.3\linewidth}
%\centering
%\begin{lstlisting}
%if y then
%  if x then
%    if z then 4 else 3
%  else 2
%else
%  if z then 1 else 3
%\end{lstlisting}
%\caption{Optimal compilation}
%\end{subfigure}
%\caption{Pattern matching compilation can be non-trivial (example from~\cite{maranget2008}).}\label{fig:match-example}
%\end{figure}
\begin{figure}
\begin{subfigure}[c][6cm][b]{0.25\linewidth}
\begin{minipage}{8cm}
\matchA
\end{minipage}
\caption{Pattern matching\\ \phantom{ }}
\end{subfigure}
\hspace{0.5cm}
\begin{subfigure}[c][6cm][b]{0.32\linewidth}
\only<1>{\begin{minipage}{8cm}
\matchB
\end{minipage}
\caption{Correct but semi-optimal compilation}}
\only<2>{\begin{minipage}{8cm}
\matchC
\end{minipage}
\caption{Optimal compilation\\ \phantom{ }}}

\end{subfigure}
\hspace{0.5cm}
\begin{subfigure}[c][6cm][b]{0.3\linewidth}
\only<1>{\begin{minipage}{8cm}
\vspace{2em}
\forestB
\end{minipage}}
\only<2>{\begin{minipage}{8cm}
\vspace{2em}
\forestC
\end{minipage}}
\caption{Decision tree\\ \phantom{ }}
\end{subfigure}
\caption{Pattern matching compilation can be non-trivial (example from~\cite{maranget2008}).}\label{fig:match-example}
\end{figure}
\note{This is an example from Luc Maranget's paper.
 
Pattern matchin is on the left. We can start testing from $x$, and then $y$ and $z$, it will give us a program with 6 $if$ constructions. But we can start testing from $y$ and if it is $false$ we can not test $x$. It will give us a program with four tests.}
\end{frame}

\begin{frame}{Contributions}
\Large
\contributions
\end{frame}



\begin{frame}[fragile]{Pattern Matching Syntax}
\begin{figure}[ht]
\begin{subfigure}[t]{0.4\linewidth}
\uncover<2->{
\[
match^o: \mathcal{V}\times \mathcal{P}^* \times  \mathbb{N}
\]}
\begin{center}
Pattern matching language
\end{center}
\[
 \begin{array}{rcll}
    \mathcal{C} & = & \{ C_1^{k_1}, \dots, C_n^{k_n} \}\\
    \mathcal{V} & = & \mathcal{C}\,\mathcal{V}^*\\  
    \mathcal{P} & = & \_ \mid \mathcal{C}\,\mathcal{P}^*
 \end{array}
\]
\[
 \begin{array}{c}
\trans{\inbr{v;\,p_1,\dots,p_k}}{}{i}\\
1\leqslant i\leqslant k+1
 \end{array}
\]
\end{subfigure}
\hspace{0.5cm}
\begin{subfigure}[t]{0.5\linewidth}
\uncover<2->{
\[
eval^o_{\mathcal S}: \mathcal{V}\times \mathcal{S} \times  \mathbb{N}
\]}
\begin{center}
Language \ir{} of intermediate representation
\end{center}
\[
\begin{array}{rcl}
  \mathcal M & = & \bullet \\
  & \mid  & \mathcal M\,[\mathbb{N}] \\
  \ir & = & \primi{return}\,\mathbb{N} \\
  & \mid  & \primi{switch}\;\mathcal{M}\;\primi{with}\; [\mathcal{C}\; \primi{\rightarrow}\; \ir]^*\;\primi{otherwise}\;\ir
\end{array}
\]
\end{subfigure}
\end{figure}
\vspace{0.5cm}
Things that are omitted for simplicity: guards, pattern variables

\end{frame}

\begin{frame}{Synthesis Algorithm}
\[
\forall v \quad \forall  (1\leqslant\!i\leqslant \!k\!+\!1) \quad
(match^o\,v\,\,p_1,\dots,p_k\,\,i) \;\; \Leftrightarrow \;
eval^o_{\mathcal S}\,v\,\circled{?}\,i
\]

\begin{itemize}
\item $match^o\,v\,\,p_1,\dots,p_k\,\,i$ -- a declarative interpreter of pattern matching, for every  possible scrutinee $v$ returns a single branch $i$ 
\item $\circled{?}$ is a program to be synthesized
\item $eval^o_{\mathcal S}\,v\,\circled{?}\,i$ -- a declarative interpreter of $\mathcal S$, ensures that synthesized program $\circled{?}$ on example $v$ returns a correct branch index $i$
\end{itemize}
\vspace{1cm}
\begin{itemize}
\item[\faGood] It's easy to implement $match^o$ and  $eval^o$ (even for extension of pattern matching)
\item[\faBad]  No universal quantifier in miniKanren with disequality constraints
\end{itemize}

\end{frame}


\begin{frame}{Eliminating $\forall$ by Generating a Finite Set of Examples}
For every concrete pattern matching we know:
\begin{itemize}
\item the type of scrutinee
\item all its patterns 
\begin{itemize}
\item maximal depth of these patterns
\end{itemize}
\end{itemize}

\begin{alertblock}{\textbf{Idea}}
Let's generate all inhabitants of the scrutinee's type until certain depth and use them as examples
\end{alertblock}
\vspace{1cm}

\begin{itemize}
\item[\faBad] The number of examples is  worst case exponential
\end{itemize}
\end{frame}

\defverbatim[colored]{\exampleAFullSet}{
\begin{lstlisting}{ocaml}
match (s : unit list) with 
| _ :: _ :: _ -> 1
| _           -> 2
\end{lstlisting}
}


\begin{frame}{Example 1: Three Examples is a Complete Set}
\centering
\begin{minipage}{.381\textwidth}
\begin{minipage}{.381\textwidth}
\exampleAFullSet
\end{minipage}
\end{minipage}

\begin{figure}
\begin{subfigure}[t]{0.3\linewidth}
\begin{subfigure}[b][4cm][t]{0.2\linewidth}
\begin{tikzpicture}[sibling distance=3em,
  every node/.style = {shape=rectangle, rounded corners, minimum height=.5cm,
    draw, align=center, font=\ttfamily
    ,top color=white, bottom color=blue!20}]]
  \node {Cons}
    child { node {\phantom{1}\_\phantom{1}} }
    child { node {Cons}
      child { node  {\phantom{1}\_\phantom{1}} }
      child { node {\phantom{1}\_\phantom{1}} } 
    };
\end{tikzpicture}
\end{subfigure}
\hspace{2cm}
\begin{subfigure}[b][4cm][t]{0.1\linewidth}
\begin{tikzpicture}[sibling distance=3em,
  every node/.style = {shape=rectangle, rounded corners, minimum height=.5cm,
    draw, align=center, font=\ttfamily
    ,top color=white, bottom color=blue!20}]]
  \node {\phantom{1}\_\phantom{1}};
\end{tikzpicture}
\end{subfigure}
\caption{Patterns}
\end{subfigure}
\hspace{1.5cm}
\begin{subfigure}[t]{0.5\linewidth}
\begin{subfigure}[b][4cm][t]{0.2\linewidth}
\begin{tikzpicture}[sibling distance=3em,
  every node/.style = {shape=rectangle, rounded corners, minimum height=.5cm,
    draw, align=center, font=\ttfamily
    ,top color=white, bottom color=green!20}]]
  \node {Cons}
    child { node {()} }
    child { node {Cons}
      child { node {()} }
      child { node { Nil } } 
    };
\end{tikzpicture}
\end{subfigure}
\hspace{1cm}
\begin{subfigure}[b][4cm][t]{0.15\linewidth}
\begin{tikzpicture}[sibling distance=3em,
  every node/.style = {shape=rectangle, rounded corners, minimum height=.5cm,
    draw, align=center, font=\ttfamily
    ,top color=white, bottom color=green!20}]]
  \node {Cons}
    child { node {()} }
    child { node { Nil} } ;
\end{tikzpicture}
\end{subfigure}
\hspace{1cm}
\begin{subfigure}[b][4cm][t]{0.15\linewidth}
\begin{tikzpicture}[sibling distance=3em,
  every node/.style = {shape=rectangle, rounded corners, minimum height=.5cm,
    draw, align=center, font=\ttfamily
    , top color=white, bottom color=green!20}]]
  \node {Nil};
\end{tikzpicture}
\end{subfigure}
\caption{Three examples}
\end{subfigure}
\end{figure}
\end{frame}

\defverbatim[colored]{\exampleNotAFullSet}{
\begin{lstlisting}{ocaml}
match (s : unit list) with 
| []  -> 1
| _   -> 2
\end{lstlisting}
}

\begin{frame}{Example 2: Not a Complete Set of Examples}
\begin{minipage}{0.4\linewidth}
\begin{minipage}{0.4\linewidth}
\exampleNotAFullSet
\end{minipage}
\end{minipage}
\begin{minipage}{0.55\linewidth}
Two patterns:\\
\begin{tikzpicture}[sibling distance=3em,
  every node/.style = {shape=rectangle, rounded corners, minimum height=.5cm,
    draw, align=center, font=\ttfamily
    ,top color=white, bottom color=blue!20}]]
  \node { Nil };
\end{tikzpicture}
\begin{tikzpicture}[sibling distance=3em,
  every node/.style = {shape=rectangle, rounded corners, minimum height=.5cm,
    draw, align=center, font=\ttfamily
    ,top color=white, bottom color=blue!20}]]
  \node {\phantom{1}\_\phantom{1}};
\end{tikzpicture}
\vspace{1em}

The set of examples bound by depth 1 is a singleton set\\
\begin{tikzpicture}[sibling distance=2em,
  every node/.style = {shape=rectangle, rounded corners, minimum height=.5cm,
    draw, align=center, font=\ttfamily
    , top color=white, bottom color=green!20}]]
  \node {Nil};
\end{tikzpicture}
\vspace{1em}

This program behave well on the set of examples \\
\[
 \begin{array}{ll}
    \primi{switch} \dots \primi{with}   & \\
    |\ \text{Nil} \rightarrow 1  & \\  
    |\ \primi{otherwise} \rightarrow  1 & 
 \end{array}
\]
but is wrong 
\end{minipage}
\end{frame}


\begin{frame}{Current Algorithm for Examples Generation}
\begin{figure}
\begin{subfigure}[b]{0.75\linewidth}
Algorithm briefly:
\begin{itemize}
\item Evaluate the depth of patterns $h$
\item Synthesize all inhabitants, but
\item on depth $h+1$ use a single predefined per type inhabitant 
\end{itemize}
\vspace{1cm}

\begin{itemize}
\item[\faBad] The number of examples is  worst case exponential
\end{itemize}
\end{subfigure}
\hspace{.5cm}
\begin{subfigure}[b]{0.45\linewidth}
\end{subfigure}
\end{figure}

\end{frame}

\defverbatim[colored]{\improvementA}{
\begin{lstlisting}{ocaml}
match (s : bool * bool * bool) with 
| (_,_,F) -> 1
| (_,_,T) -> 2
\end{lstlisting}
}

\begin{frame}{Improvement: Reducing Examples Set}
An example:
\improvementA
\vspace{1em}
We can statically determine that 

\begin{itemize}
\item There are $2^3$ inhabitants of type \lstinline=bool * bool * bool=
\item No need to  check that scrutinee is a triple
\item No need to look at the 1st and  the 2nd subtrees of a triple (only the 3rd matters)
\end{itemize}
\vspace{.5cm}


So, we can run the synthesis only on two examples $\{(\mathcal{B},\mathcal{B},\text{T}),(\mathcal{B},\mathcal{B},\text{F})\}$
where $\mathcal{B}$ is any value of type \lstinline=bool=
if we \textbf{forbid the investigation of  some subtrees} of examples
\end{frame}

\begin{frame}[fragile]{Improvement: skipping the equivalent programs}
Realtional interpreter of language \ir{} could enumerate equivalaent programs 
\begin{center}
  \begin{minipage}[t]{0.2\linewidth}
  \begin{center}
  \begin{lstlisting}[language=ocamllambda,gobble=2]
  switch Scru with 
  true -> ...
  false -> ...
  \end{lstlisting}
  \end{center}
  \end{minipage}\hspace{.5cm}
  \begin{minipage}[t]{0.2\linewidth}
  \huge
  \begin{center}
  \[
  \Leftrightarrow
  \]
  \end{center}
  \end{minipage}\hspace{.5cm}
  \begin{minipage}[t]{0.2\linewidth}
  \begin{lstlisting}[language=ocamllambda,gobble=2]
  switch Scru with 
  false -> ...
  true -> ...
  \end{lstlisting}
  \end{minipage}
\end{center}
Which is solved by 
\begin{itemize}
\item Establishing order of \primi{switch}'s branches using type information (the names of constructors)
\item It ``broke'' interpreter a little bit, but synthesis has become faster 
\end{itemize}

\end{frame}


\begin{frame}{Branch and Bound Optimization}
The synthesis answer is stored in variable $\circled{?}$ and 
during the search we \emph{only} extend it with new $\primi{switch}$es \\

\begin{alertblock}{\textbf{Idea}}
If current estimate is longer than the shortest answer already found -- cut this branch
\end{alertblock}

Requires a modified miniKanren's \lstinline=run= primitive: for every answer found we 
\begin{itemize}
\item evaluate it's size
\item update already found minimum
\end{itemize}

Pruning search branches is implemented using a new primitive called  \emph{structural constraint}
\end{frame}

\begin{frame}{Structural constraint}
A new primitive called \emph{structural constraint}:
\begin{itemize}
\item Takes a logic value and performs reification in a current state
\item Takes a predicate which inspects the reified value 
\item If there are too many branches, stops the search ($failure^o$)
\item Otherwise continues search without changing the state ($success^o$)
\end{itemize}
\vspace{1em}
Subtleties:
\begin{itemize}
\item Used to count the size of current estimate
\item May take or may not take into account disequality constraints
\item Can be used to implement $absent^o$ or similar constraints
\item Deals with reified values $\Rightarrow$ can be slow
\end{itemize}
\end{frame}



\begin{frame}{Minimization Criterion for Synthesized Program}
\begin{figure}
\begin{subfigure}[b]{0.3\linewidth}
\[
\begin{array}{l}
\primi{switch}\;\mathcal{M}\;\primi{with}\; \\
\mathcal{C}_1\; \primi{\rightarrow}\; \ir_1\\
\dots \\
\mathcal{C}_n\; \primi{\rightarrow}\; \ir_n\\
\primi{otherwise}\;\ir\\
\end{array}\\
\]
\[
\Updownarrow
\]
\[
\begin{array}{l}
\primi{if}\; \mathcal{M} = \mathcal{C}_1\;\primi{then}\; \ir_1\\
\dots \\
\primi{else if}\; \mathcal{M} = \mathcal{C}_n\; \primi{then}\;  \ir_n\\
\primi{else}\;\ir\\
\end{array}\\
\]
\vspace{1cm}

\end{subfigure}
\hspace{1cm}
\begin{subfigure}[b]{0.6\linewidth}
A single switch with $n$ branches roughly can be encoded as $n$ \texttt{if}s\\

Let's say that the size of
\begin{itemize}
\item  $\primi{switch}$ is the number of branches
\item $\primi{return}$ is 0
\item the whole program in $\ir$ is the sum of sizes of its $\primi{switch}$es 
\end{itemize}
Our minimization criterion: reduce the size of synthesized program\\

But there can be other criteria:
\begin{itemize}
\item The  depth 
\item The branching factor
\item etc.
\end{itemize}
\end{subfigure}
\end{figure}
\end{frame}


%%%%%%%%%%%%%%%%%%%%%%%%%%%%%%%%%%%%%%%%%%%%%%%%
\begin{frame}{Implementation}
\OCanren{} --- typed embedding of \miniKanren{} to \OCaml{}.
\vspace{2em}

\noCanren{}~\cite{RelConversion} generates relational code for \OCanren{}~\cite{OCanren,OCanrenWeb} from \textsc{ML}-like language
\vspace{1em}

Project repo~\cite{Repo}
\end{frame}
%%%%%%%%%%%%%%%%%%%%%%%%%%%%%%%%%%%%%%%%%%%%%%%%

\defverbatim[colored]{\synthInputA}{
\begin{lstlisting}
match $\bullet$ with
(_, false, true) -> 1
(false, true, _) -> 2
(_, _, false) -> 3
(_, _, true) -> 4
\end{lstlisting}
}
\defverbatim[colored]{\synthResultA}{
\begin{lstlisting}[language=ocamllambda]
switch $\bullet$[0] with  
| true -> 
    (switch $\bullet$[1] with  
    | true -> 
        (switch $\bullet$[2] with true -> 4 | _ -> 3)
    | _ -> 
        (switch $\bullet$[2] with true -> 1 | _ -> 3))  
| _ -> 
    (switch $\bullet$[1] with  
    | true -> 2   
    | _ -> 
         (switch $\bullet$[2] with true -> 1 | _ -> 3))
\end{lstlisting}
}

\defverbatim[colored]{\synthResultB}{
\begin{lstlisting}[language=ocamllambda]
switch Scru[0] with  
| true -> 
    (switch Scru[2] with  
    | true -> 
       (switch Scru[1] with true -> 4 | _ -> 1)
    | _ -> 3 )  
| _ -> 
    (switch Scru[1] with  
    | true -> 2   
    | _ -> (switch Scru[2] with  true -> 1 | _ -> 3))
\end{lstlisting}
}

\defverbatim[colored]{\synthResultC}{
\begin{lstlisting}[language=ocamllambda]
switch $\bullet$[1] with  
| true -> 
    (switch $\bullet$[0] with  
    | true -> 
        (switch $\bullet$[2] with true -> 4 | _ -> 3)
    | _ -> 2) 
| _ -> 
    (switch $\bullet$[2] with true -> 1 | _ -> 3)
\end{lstlisting}
}




%%%%%%%%%%%%%%%%%%%%%%%%%%%%%%%%%%%%%%%%%%%%%%%%

\begin{frame}{Evaluation: matching three boolean values}
\begin{minipage}[c][5cm][t]{0.32\linewidth}
\begin{minipage}{0.35\linewidth}
\synthInputA
\end{minipage}
\end{minipage}
\begin{minipage}{0.63\linewidth}
\begin{onlyenv}<1>
Answer of size 6 (in 1.6 seconds)

\begin{minipage}{0.63\linewidth}
\synthResultA
\end{minipage}
\end{onlyenv}
\begin{onlyenv}<2>
Answer of size 5 (in +0.4 seconds)

\begin{minipage}{0.63\linewidth}
\synthResultB
\end{minipage}
\end{onlyenv}
\begin{onlyenv}<3>
Answer of size 4 (in +0.7 seconds)

\begin{minipage}{0.63\linewidth}
\synthResultC
\end{minipage}
\end{onlyenv}
\end{minipage}
\end{frame}

%%%%%%%%%%%%%%%%%%%%%%%%%%%%%%%%%%%%%%%%%%%%%%%%

\defverbatim[colored]{\plotkinBig}{
\begin{lstlisting}[basicstyle=\small,language=ocaml]
match a,s,c with
| (_,_,Ldi i::_) -> 1
| (_,_,Push::_)  -> 2
| (Int _,Val (Int _)::_,IOp _::_) -> 3
| (Int _,_,Test (_,_)::c) -> 4
| (Int _,_,Test (_,_)::c) -> 5
| (_,_,Extend::_) -> 6
| (_,_,Search _::_) -> 7
| (_,_,Pushenv::_) -> 8
| (_,Env e::s,Popenv::_) -> 9
| (_,_,Mkclos cc::_) -> 10
| (_,_,Mkclosrec _::_) -> 11
| (Clo (_,_), Val _::_, Apply::_) -> 12
| (_,(Code _::Env _::_),[]) -> 13
| (_,[],[]) -> 14
\end{lstlisting}
}

\begin{frame}{Performance problem (1/2)}
\begin{figure}
\begin{subfigure}[t]{0.5\linewidth}\vspace{0em}
\begin{minipage}{0.5\linewidth}
\plotkinBig
\end{minipage}
\end{subfigure}
\hspace{1cm}
\begin{subfigure}[t]{0.35\linewidth}\vspace{0em}
PCF (mini-ML) interpreter from G.Plotkin's paper, 1977
\vspace{4em}

Doesn't currently work because the types are too large (11102 examples  generated).
\end{subfigure}
\end{figure}
\end{frame}

\begin{frame}[fragile]{Performance problem (2/2)}
\begin{figure}
\begin{subfigure}[b][6cm][b]{0.45\linewidth}
\begin{lstlisting}[basicstyle=\small]
type code = 
| Push 
| Ldi of int 
| IOp of int 
| Int of int 
type prog = code list 
type item = 
| Val of code 
| Env of int 
| Code of int
type stack = item list 

match (code, stack,prog) with
| (_, _, (Ldi _)::_) -> 1
| (_, _, (Push _)::_) -> 2
\end{lstlisting}
\begin{onlyenv}<1>
\vspace{4mm}
\end{onlyenv}
\begin{onlyenv}<2>
\begin{lstlisting}[basicstyle=\small,aboveskip=-0.5em]
| (Int _, _, (IOp _)::_) -> 3
\end{lstlisting}
\end{onlyenv}
\end{subfigure}
\hspace{.5cm}
\begin{subfigure}[b]{0.45\linewidth}
\begin{overlayarea}{7cm}{6cm}
Example is reduces by 
\begin{itemize}
\item types 
\item clauses
\end{itemize}
For two clauses we require 5 examples \\\vspace{1em}

\begin{onlyenv}<2->
For three clauses and the same types we require 20 examples
\begin{itemize}
\item in 1,5s it gives the  1st answer of size 7
\item in the next half a second it gives the 2nd and the 3rd (last) answers of sizes 6 and 5
\item in the end it in 10s proves that no shorter answers exist
\end{itemize}
%\begin{lstlisting}[basicstyle=\small,aboveskip=-0.5em]
%| (Int _, _, (IOp _)::_) -> 3
%\end{lstlisting}
\end{onlyenv}
\end{overlayarea}
\vspace{1cm}
\end{subfigure}
\end{figure}

\end{frame}

\begin{frame}{Conclusions and Future Work}
Contributions: \contributions
\vspace{1em}

An approach shows correct results on small problems...
 but performs not fast enough\\
\vspace{1em}
 
Areas for possible improvements:
\begin{itemize}
\item Reification for structural constraints can be made faster (lazy computations)
\item Use finite-domain constraints instead of disequality constraints 
\item Memoization is not used (because of presence of disequality constraints)
\item More clever way to generate less examples
\item Another variation of \ir{} language
\end{itemize}
\vspace{1em}\pause
\begin{center}
{\Huge Thanks!}
\end{center}

\end{frame}


\begin{frame}%[t, allowframebreaks]
\frametitle{References}
\bibliographystyle{amsalpha}
\bibliography{references}
\vspace{1cm}
\end{frame}


\end{document}
