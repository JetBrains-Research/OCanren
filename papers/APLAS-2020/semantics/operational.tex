\section{Operational Semantics}
\label{operational}

In this section we describe the operational semantics of \textsc{miniKanren}, which corresponds to the known
implementations with interleaving search. The semantics is given in the form of a labeled transition system (LTS)~\cite{LTS}. From now on we
assume the set of semantic variables to be linearly ordered ($\mathcal{A}=\{\alpha_1,\alpha_2,\dots\}$).

We introduce the notion of substitution

\[
  \sigma : \mathcal{A}\to\mathcal{T_A}
\]

\noindent as a (partial) mapping from semantic variables to terms over the set of semantic variables. We denote $\Sigma$ the
set of all substitutions, $\dom{\sigma}$~--- the domain for a substitution $\sigma$,
$\ran{\sigma}=\bigcup_{\alpha\in\mathcal{D}om\,(\sigma)}\fv{\sigma\,(\alpha)}$~--- its range (the set of all free variables in the image).

The \emph{non-terminal states} in the transition system have the following shape:

\[
S = \mathcal{G}\times\Sigma\times\mathbb{N}\mid S\oplus S \mid S \otimes \mathcal{G}
\]

As we will see later, an evaluation of a goal is separated into elementary steps, and these steps are performed interchangeably for different subgoals. 
Thus, a state has a tree-like structure with intermediate nodes corresponding to partially-evaluated conjunctions (``$\otimes$'') or
disjunctions (``$\oplus$''). A leaf in the form $\inbr{g, \sigma, n}$ determines a goal in a context, where $g$ is a goal, $\sigma$ is a substitution accumulated so far,
and $n$ is a natural number, which corresponds to a number of semantic variables used to this point. For a conjunction node, its right child is always a goal since
it cannot be evaluated unless some result is provided by the left conjunct.

The full set of states also include one separate terminal state (denoted by $\diamond$), which symbolizes the end of the evaluation.

\[
\hat{S} = \diamond \mid S
\]

We will operate with the well-formed states only, which are defined as follows.

\begin{definition}
  Well-formedness condition for extended states:
  
  \begin{itemize}
  \item $\diamond$ is well-formed;
  \item $\inbr{g, \sigma, n}$ is well-formed iff $\fv{g}\cup\dom{\sigma}\cup\ran{\sigma}\subseteq\{\alpha_1,\dots,\alpha_n\}$;
  \item $s_1\oplus s_2$ is well-formed iff $s_1$ and $s_2$ are well-formed;
  \item $s\otimes g$ is well-formed iff $s$ is well-formed and for all leaf triplets $\inbr{\_,\_,n}$ in $s$ it is true that $\fv{g}\subseteq\{\alpha_1,\dots,\alpha_n\}$.
  \end{itemize}
  
\end{definition}

Informally the well-formedness restricts the set of states to those in which all goals use only allocated variables.

Finally, we define the set of labels:

\[
L = \step \mid \Sigma\times \mathbb{N}
\]

The label ``$\step$'' is used to mark those steps which do not provide an answer; otherwise, a transition is labeled by a pair of a substitution and a number of allocated
variables. The substitution is one of the answers, and the number is threaded through the derivation to keep track of allocated variables.

\begin{figure*}[t]
  \renewcommand{\arraystretch}{1.6}
  \[
  \begin{array}{cr}
    \inbr{t_1 \equiv t_2, \sigma, n} \xrightarrow{\step} \Diamond , \, \, \nexists\; mgu\,(t_1 \sigma, t_2 \sigma) &\ruleno{UnifyFail} \\
    \inbr{t_1 \equiv t_2, \sigma, n} \xrightarrow{(mgu\,(t_1 \sigma, t_2 \sigma) \circ \sigma,\, n)} \Diamond & \ruleno{UnifySuccess} \\
    \inbr{g_1 \lor g_2, \sigma, n} \xrightarrow{\step} \inbr{g_1, \sigma, n} \oplus \inbr{g_2, \sigma, n} & \ruleno{Disj} \\
    \inbr{g_1 \land g_2, \sigma, n} \xrightarrow{\step} \inbr{ g_1, \sigma, n} \otimes g_2 & \ruleno{Conj} \\
    \inbr{\mbox{\lstinline|fresh|}\, x\, .\, g, \sigma, n} \xrightarrow{\step} \inbr{g\,[\bigslant{\alpha_{n + 1}}{x}], \sigma, n + 1} & \ruleno{Fresh} \\
    \dfrac{R_i^{k_i}=\lambda\,x_1\dots x_{k_i}\,.\,g}{\inbr{R_i^{k_i}\,(t_1,\dots,t_{k_i}),\sigma,n} \xrightarrow{\step} \inbr{g\,[\bigslant{t_1}{x_1}\dots\bigslant{t_{k_i}}{x_{k_i}}], \sigma, n}} & \ruleno{Invoke}\\
    \dfrac{s_1 \xrightarrow{\step} \Diamond}{(s_1 \oplus s_2) \xrightarrow{\step} s_2} & \ruleno{SumStop}\\
    \dfrac{s_1 \xrightarrow{r} \Diamond}{(s_1 \oplus s_2) \xrightarrow{r} s_2} & \ruleno{SumStopAns}\\
    \dfrac{s \xrightarrow{\step} \Diamond}{(s \otimes g) \xrightarrow{\step} \Diamond} &\ruleno{ProdStop}\\
    \dfrac{s \xrightarrow{(\sigma, n)} \Diamond}{(s \otimes g) \xrightarrow{\step} \inbr{g, \sigma, n}}  & \ruleno{ProdStopAns}\\
    \dfrac{s_1 \xrightarrow{\step} s'_1}{(s_1 \oplus s_2) \xrightarrow{\step} (s_2 \oplus s'_1)} &\ruleno{SumStep}\\
    \dfrac{s_1 \xrightarrow{r} s'_1}{(s_1 \oplus s_2) \xrightarrow{r} (s_2 \oplus s'_1)} &\ruleno{SumStepAns}\\
    \dfrac{s \xrightarrow{\step} s'}{(s \otimes g) \xrightarrow{\step} (s' \otimes g)} &\ruleno{ProdStep}\\
    \dfrac{s \xrightarrow{(\sigma, n)} s'}{(s \otimes g) \xrightarrow{\step} (\inbr{g, \sigma, n} \oplus (s' \otimes g))} & \ruleno{ProdStepAns} 
  \end{array}
  \]
  \caption{Operational semantics of interleaving search}
  \label{lts}
\end{figure*}

The transition rules are shown in Fig.~\ref{lts}. The first two rules specify the semantics of unification. If two terms are not unifiable under the current substitution
$\sigma$ then the evaluation stops with no answer; otherwise, it stops with the most general unifier applied to a current substitution as an answer.

The next two rules describe the steps performed when disjunction or conjunction is encountered on the top level of the current goal. For disjunction, it schedules both goals (using ``$\oplus$'') for
evaluating in the same context as the parent state, for conjunction~--- schedules the left goal and postpones the right one (using ``$\otimes$'').

The rule for ``\lstinline|fresh|'' substitutes bound syntactic variable with a newly allocated semantic one and proceeds with the goal.

The rule for relation invocation finds a corresponding definition, substitutes its formal parameters with the actual ones, and proceeds with the body.

The rest of the rules specify the steps performed during the evaluation of two remaining types of the states~--- conjunction and disjunction. In all cases, the left state
is evaluated first. If its evaluation stops, the disjunction evaluation proceeds with the right state, propagating the label (\textsc{SumStop} and \textsc{SumStep}), and the conjunction schedules the right goal for evaluation in the context of the returned answer (\textsc{ProdStopAns}) or stops if there is no answer (\textsc{ProdStop}).

The last four rules describe \emph{interleaving}, which occurs when the evaluation of the left state suspends with some residual state (with or without an answer). In the case of disjunction
the answer (if any) is propagated, and the constituents of the disjunction are swapped (\textsc{SumStep}, \textsc{SumStepAns}). In the case of conjunction, if the evaluation step in
the left conjunct did not provide any answer, the evaluation is continued in the same order since there is still no information to proceed with the evaluation of the right
conjunct (\textsc{ProdStep}); if there is some answer, then the disjunction of the right conjunct in the context of the answer and the remaining conjunction is
scheduled for evaluation (\textsc{ProdStepAns}).

The introduced transition system is completely deterministic: there is exactly one transition from any non-terminal state.
There is, however, some freedom in choosing the order of evaluation for conjunction and
disjunction states. For example, instead of evaluating the left substate first, we could choose to evaluate the right one, etc.
\begin{comment}
In each concrete case, we would
end up with a different (but still deterministic) system that would prescribe different semantics to a concrete goal.
\end{comment}
This choice reflects the inherent non-deterministic nature of search in relational (and, more generally, logical) programming.
Although we could introduce this ambiguity into the semantics (by replacing specific rules for disjunctions and conjunctions evaluation with some conditions on it), we want an operational semantics that would be easy to present and easy to employ to describe existing language extensions (already described for a specific implementation of interleaving search), so we instead base the semantics on one canonical search strategy.
At the same time, as long as deterministic search procedures are sound and complete, we can consider them ``equivalent''.\footnote{There still can be differences in observable behavior of concrete goals under different sound and complete search strategies.
For example, a goal can be refutationally complete~\cite{WillThesis} under one strategy and non-complete under another.}

It is easy to prove that transitions preserve well-formedness of states.

\begin{lemma}{(Well-formedness preservation)}
\label{lem:well_formedness_preservation}
For any transition $s \xrightarrow{l} \hat{s}$, if $s$ is well-formed then $\hat{s}$ is also well-formed.
\end{lemma}

A derivation sequence for a certain state determines a \emph{trace}~--- a finite or infinite sequence of answers. The trace corresponds to the stream of answers
in the reference \textsc{miniKanren} implementations. We denote a set of answers in the trace for state $\hat{s}$ by $\tr{\hat{s}}$.

We can relate sets of answers for the partially evaluated conjunction and disjunction with sets of answers for their constituents by the two following lemmas.

\begin{lemma}
\label{lem:sum_answers}
For any non-terminal states $s_1$ and $s_2$, $\tr{s_1 \oplus s_2} = \tr{s_1} \cup \tr{s_2}$.
\end{lemma}

\begin{lemma}
\label{lem:prod_answers}
For any non-terminal state $s$ and goal $g$,  \mbox{$\tr{s \otimes g} \supseteq \bigcup_{(\sigma, n) \in \tr{s}} \tr{\inbr{g, \sigma, n}}$}.
\end{lemma}

These two lemmas constitute the exact conditions on definition of these operators that we will use to prove the completeness of an operational semantics.

We also can easily describe the criterion of termination for disjunctions.

\begin{lemma}
\label{lem:disj_termination}
For any goals $g_1$ and $g_2$, sunbstitution $\sigma$, and number $n$, the trace from the state $\inbr{g_1 \vee g_2, \sigma, n}$ is finite iff the traces from both $\inbr{g_1, \sigma, n}$ and $\inbr{g_2, \sigma, n}$ are finite.
\end{lemma}

These simple statements already allow us to prove two important properties of interleaving search as corollaries: the ``fairness'' of disjunction~--- the fact that the trace for disjunction contains all the answers from both streams for disjuncts~--- and the ``commutativity'' of disjunctions~--- the fact that swapping two disjuncts (at the top level) does not change the termination of the goal evaluation. 
