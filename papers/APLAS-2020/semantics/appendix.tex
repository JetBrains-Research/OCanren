\clearpage

\section{Details of Proofs}
\label{appendix_proofs}

\subsection{Proof of Lemma~\ref{lem:den_sem_change_var}}
\label{appendix_den_sem_change_var_proof}

The proof goes by structural induction, all cases naturally use inductive hypotheses, except for the case of fresh variable introduction. We examine this case in more detail.

We have a nested fresh variable introduction $\mbox{\lstinline|fresh|}\,y\,.\,g'$ as the goal $g$. $g'$ may contain syntactic variables $x$ and $y$ (the case when $g$ does not contain syntactic variable $x$ is trivial and we have to consider it separately). By the statement of the lemma and the definition of denotational semantics we have ${\mathfrak f'_1} \in \sembr{g'\,[\alpha_1/x]\,[\alpha_3/y]}$, where $\alpha_3$ is some fresh semantic variable, and ${\mathfrak f'_1}$ is some representing function which may differ from ${\mathfrak f_1}$ only on variable $\alpha_3$. Variable $\alpha_3$ may not be equal to $\alpha_1$, but may be equal to $\alpha_2$ and further in proof we sometimes have to consider these two cases separetly, because they are essentially different. We should find some fresh variable $\alpha_4$ and representing function ${\mathfrak f'_2} \in \sembr{g'\,[\alpha_2/x]\,[\alpha_4/y]}$ that may differ from ${\mathfrak f_2}$ only on variable $\alpha_4$.

Among different options we considered, the easiest choice for a variable $\alpha_4$ turned out to be the freshest possible variable (one that is not equal to any of the mentioned above and that does not occur in the goal $g'$). Then we need to move from the function ${\mathfrak f'_1}$ to the function ${\mathfrak f'_2}$ by applying the induction hypothesis twice (since now we change two substituted variables simultaneously). To do this we need to carefully change the values of the function on a few suitable points. It turned out, that we have to change the values on all four mentioned variables. First, we build an intermediate function ${\mathfrak f'_{1.5}} = {\mathfrak f'_1}[\alpha_3\gets {\mathfrak f_2}(\alpha_3)][\alpha_4\gets {\mathfrak f'_1}(\alpha_3)]$ that should belong to an intermediate semantics $\sembr{g'\,[\alpha_1/x]\,[\alpha_4/y]}$. Then we build a final function ${\mathfrak f'_2} = {\mathfrak f'_{1.5}}[\alpha_1\gets {\mathfrak f_2}(\alpha_1)][\alpha_2\gets {\mathfrak f'_{1.5}}(\alpha_1)]$ on top of it. For both transitions, we carefully check that all conditions on two functions from the statement of the lemma (and, therefore, from the induction hypothesis) are satisfied by examination of values of the functions on all mentioned variables.

\subsection{Proof of Generalized Completeness}
\label{appendix_gen_completeness_proof}

The proof goes by induction on the level of bounded denotational semantics, then by structural induction on the goal.

\begin{itemize}
\item In case of unification $t_1\equiv t_2$ we need to show that any representing function that unifies terms $t_1$ and $t_2$ and is an extension of $\sigma$ also is an extension of $mgu\,(t_1 \sigma, t_2 \sigma) \circ \sigma$. This requires some observations about representing functions, most importantly the fact that for any representing function $f$, unifying two terms, there exists a unifying substitution for these terms, for which $f$ is an extension.
\item In case of disjunction we apply lemma~\ref{lem:sum_answers} to induction hypotheses.
\item In case of conjunction we apply lemma~\ref{lem:prod_answers} to induction hypotheses. Note that we use two nested inductions instead of one induction on evidence because of this case: we need to apply an induction hypothesis to different representing functions (which may have different values on non-allocated free variables), and otherwise the induction hypotheses would not be flexible enough.
\item In the case of fresh variable introduction, we can not simply use induction hypothesis, because it has a goal with an arbitrary fresh variable substituted (see definition of denotational semantics on Fig.~\ref{denotational_semantics_of_goals}), but we need to relate two semantics of the goal with exactly the first non-allocated variable substituted (from transition in operational semantics, see Fig.~\ref{lts}). To overcome this inconsistency, we apply lemma~\ref{lem:den_sem_change_var} about changing substituted fresh variable. When we do this, we lose the equality of values of representing functions on the first non-allocated variable. It is exactly the place where the proof of the strong version of the completeness theorem (without the restriction of domains of representing functions on allocated variables) fails (as explained in the section~\ref{equivalence}, the strong version of the completeness theorem does not hold).
\item In case of relational invocation we simply use the induction hypothesis about the previous level.
\end{itemize}

\section{\textsc{Coq} Specification}
\label{appendix_coq}

The \textsc{Coq} specification consists of three parts: specification of preliminary notions that we use in the semantics, specification of the language and semantics for interleaving search, specification of the language extended with cut and semantics for SLD resolution with cut (which repeats the previous part with only a few modifications).

Preliminary notions are notions from unification theory and definition of streams.

From unification theory we need (finitary) terms, substitutions, and operations on them (application to a term, composition). The most important notion we have to formalize here is the most general unifier. As it was described in the section~\ref{specification}, we define it as an inductive relation.

\begin{lstlisting}[language=Coq]
  Inductive mgu : term -> term -> option subst -> Set := ...
\end{lstlisting}

It uses an auxiliary function that finds the leftmost mismatch of the two terms (if there is any).

Then we prove by well-founded induction that this relation is a function and it satisfies the defining properties of MGU:

\begin{lstlisting}[language=Coq]
  Lemma mgu_result_$\texttt{exists}$ : forall t1 t2, {r & mgu t1 t2 r}.
  Lemma mgu_result_unique : forall t1 t2 r r',
      mgu t1 t2 r -> mgu t1 t2 r' ->r = r'.
  Definition unifier (s : subst) (t1 t2 : term) : Prop :=
    apply_subst s t1 = apply_subst s t2.
  Lemma mgu_unifies: forall t1 t2 s,
    mgu t1 t2 (Some s) -> unifier s t1 t2.
  Definition more_general (m s : subst) : Prop :=
    exists (s' : subst),
    forall (t : term),
      apply_subst s t = apply_subst s' (apply_subst m t).
  Lemma mgu_most_general : forall t1 t2 m,
      mgu t1 t2 (Some m) ->
      forall (s : subst), unifier s t1 t2 -> more_general m s.
  Lemma mgu_non_unifiable : forall t1 t2,
      mgu t1 t2 None -> forall s,  ~ (unifier s t1 t2).
\end{lstlisting}

For this well-founded induction we use the number of free variables in argument terms as a well-founded order on pairs of terms:

\begin{lstlisting}[language=Coq]
  Definition terms := term * term.
  Definition fvOrder (t : terms) :=
    length (union (fv_term (fst t)) (fv_term (snd t))).
  Definition fvOrderRel (t p : terms) :=
    fvOrder t < fvOrder p.
  Lemma fvOrder_wf : well_founded fvOrderRel.
\end{lstlisting}

Possibly infinite streams is a coinductive data type:

\begin{lstlisting}[language=Coq]
  Context {A : Set}.
  CoInductive stream : Set :=
  | Nil : stream
  | Cons : A -> stream -> stream.
\end{lstlisting}

However, some of its properties we are working with make sense only when defined inductively:

\begin{lstlisting}[language=Coq]
  Inductive in_stream : A -> stream -> Prop := ...
  Inductive finite : stream -> Prop := ...
\end{lstlisting}

For the equality of streams we need to define a new coinductive proposition instead of using the standard syntactic equality in order for coinductive proofs to work~\cite{CPDT}:

\begin{lstlisting}[language=Coq]
  CoInductive equal_streams : stream -> stream -> Prop :=
  | eqsNil  :  equal_streams Nil Nil
  | eqsCons : forall h1 h2 t1 t2,
                       h1 = h2 ->
                       equal_streams t1 t2 ->
                       equal_streams (Cons h1 t1) (Cons h2 t2).
\end{lstlisting}

Here we also define a relation for one-by-one interleaving of streams which we will use to describe the evaluation of dijunstions in the operational semantics:

\begin{lstlisting}[language=Coq]
  CoInductive interleave : stream -> stream -> stream -> Prop :=
  | interNil : forall s s', equal_streams s s' -> interleave Nil s s'
  | interCons : forall h t s rs,
                interleave s t rs ->
                interleave (Cons h t) s (Cons h rs).
\end{lstlisting}

This allows us to prove the expected properties of interleaving in a more general setting of arbitrary streams.

\begin{lstlisting}[language=Coq]
  Lemma interleave_in : forall s1 s2 s,
    interleave s1 s2 s ->
    forall x, in_stream x s <-> in_stream x s1 \/ in_stream x s2.
  Lemma interleave_finite : forall s1 s2 s,
    interleave s1 s2 s ->
    (finite s <-> finite s1 /\ finite s2).
\end{lstlisting}

The syntax of the language can be formalized in \textsc{Coq} in a natural way via inductive data types. Thanks to the higher-order syntax we need to work explicitly only with semantic variables. We use type ``\lstinline[language=Coq]{name}'' for all named entities (variables, constructors, and relations), and we define it simply as natural numbers.

As it was noted in the section~\ref{specification} all terms in our specification have arities either zero or two:

\begin{lstlisting}[language=Coq] 
   Inductive term : Set :=
   | Var : name -> term
   | Cst : name -> term
   | Con : name -> term -> term -> term.
\end{lstlisting}

And all relations have exactly one argument:

\begin{lstlisting}[language=Coq]
   Definition rel : Set := term -> goal.
\end{lstlisting}

We introduce one additional auxilary type of goals~--- \emph{failure}~--- for deliberately unsuccessful computation (empty stream). As a result, the definition of goals looks as follows:

\begin{lstlisting}[language=Coq] 
   Inductive goal : Set :=
   | Fail   : goal
   | Unify  : term -> term -> goal
   | Disj   : goal -> goal -> goal
   | Conj   : goal -> goal -> goal
   | Fresh  : (name -> goal) -> goal
   | Invoke : name -> term -> goal.
\end{lstlisting}

Note the use of HOAS for fresh variable introduction.

We define sets of free variables for terms naturally as \textsc{Coq} sets:

\begin{lstlisting}[language=Coq,mathescape=true] 
   Fixpoint fv_term (t : term) : $\mbox{\small{set}}$ name :=
     ...
\end{lstlisting}

And we use them to define ground terms as a subset type:

\begin{lstlisting}[language=Coq]
   Definition ground_term : Set :=
     {t : term | fv_term t = empty_set name}.
\end{lstlisting}

However, for goals it was more convenient to define a set of free variables as a proposition:

\begin{lstlisting}[language=Coq]
   Inductive is_fv_of_goal (n : name) : goal -> Prop :=
     ...
\end{lstlisting}

We use it to state our two explicit requirements on the syntax representation: the consistency of bindings and the absence of unbound variables in the definitions of relations. The second one is easy to describe:

\begin{lstlisting}[language=Coq] 
  Definition closed_goal_in_context 
    (c : list name) (g : goal) : Prop :=
      forall n, is_fv_of_goal n g -> In n c.
  Definition closed_rel (r : rel) : Prop :=
    forall (arg : term),
    closed_goal_in_context (fv_term arg) (r arg).
\end{lstlisting}

The consistency is based on variable renaming. It turned out to be rather non-trivial to define regular variable renaming for goals in higher-order syntax, but for our purposes
a weaker version, which deals only with non-free variables, is sufficient:

\begin{lstlisting}[language=Coq]
   Inductive renaming (old_x : name) (new_x : name) :
       goal -> goal -> Prop :=
   ...
   | rFreshNFV : forall fg,
                 (~is_fv_of_goal old_x (Fresh fg)) ->
                 renaming old_x new_x (Fresh fg) (Fresh fg)
   | rFreshFV : forall fg rfg,
                (is_fv_of_goal old_x (Fresh fg)) ->
                (forall y, (~is_fv_of_goal y (Fresh fg)) ->
                      renaming old_x new_x (fg y) (rfg y)) ->
                renaming old_x new_x (Fresh fg) (Fresh rfg)
   ...
\end{lstlisting}

The consistency for goals and relations then can be defined as follows:

\begin{lstlisting}[language=Coq]
   Definition consistent_binding (b : name -> goal) : Prop :=
     forall x y, (~ is_fv_of_goal x (Fresh b)) ->
           renaming x y (b x) (b y).
   Inductive consistent_goal : goal -> Prop :=
     ...
   Definition consistent_function
     (f : term -> goal) : Prop :=
     forall a1 a2 t, renaming a1 a2 (f t)
                     (f (apply_subst [(a1, Var a2)] t)).
   Definition consistent_rel (r : rel) : Prop :=
     forall (arg : term), consistent_goal (r arg) /\
                     consistent_function r.
\end{lstlisting}

In the snippet above the ``\lstinline[language=Coq]{consistent_goal}'' property inductively ensures that all bindings occurring
in the goal are consistent and ``\lstinline[language=Coq]{apply_subst [(a1, Var a2)] t}'' in ``\lstinline[language=Coq]{consistent_function}''
definition renames a variable \lstinline[language=Coq]{a1} into in \lstinline[language=Coq]{a2} term \lstinline[language=Coq]{t}.

We set an arbitrary environment (a map from a relational symbol to a definition of relation with described requirements) to use further throughout the formalization.
Failure goals allow us to define it as a total function:

\begin{lstlisting}[language=Coq]
   Definition def : Set := 
     {r : rel | closed_rel r /\ consistent_rel r}.
   Definition env : Set := name -> def.
   Axiom Prog : env.
\end{lstlisting}

To formalize denotational semantics in \textsc{Coq} we can define representing functions simply as \textsc{Coq} functions:

\begin{lstlisting}[language=Coq]
   Definition repr_fun : Set := name -> ground_term.
\end{lstlisting}

We define the semantics via the inductive proposition ``\lstinline|in_denotational_sem_goal|'' (with notation ``\lstinline[mathescape=true]{[| $\bullet$ , $\bullet$ |]}'')
such that

\[
\forall g,\mathfrak{f}\;:\;\mbox{\lstinline|in_denotational_sem_goal|}\;g\;\mathfrak{f}\Longleftrightarrow\mathfrak{f}\in\sembr{g}
\]

The head of the definition is as follows:

\begin{lstlisting}[language=Coq,morekeywords={where,at,level}]
  Reserved Notation "[| g , f |]" (at level 0).
  Inductive in_denotational_sem_goal :
    goal -> repr_fun -> Prop :=
    ...
  where "[| g , f |]" := (in_denotational_sem_goal g f).
\end{lstlisting}

and the body just goes through the cases shown in Fig.~\ref{denotational_semantics_of_goals}.

We also need to explicitly define the bounded version of denotational semantics described in the section~\ref{equivalence}:

\begin{lstlisting}[language=Coq,morekeywords={where,at,level}]
  Reserved Notation "[| n | g , f |]" (at level 0).
  Inductive in_denotational_sem_lev_goal :
    nat -> goal -> repr_fun -> Prop :=
    ...
  | dslgInvoke : forall l r t f,
      [| l  | proj1_sig (Prog r) t , f |] ->
      [| S l | Invoke r t , f |]
  where "[| n | g , f |]" :=
    (in_denotational_sem_lev_goal n g f).
\end{lstlisting}

Recall that the environment ``\lstinline[language=Coq]|Prog|'' maps every relational symbol to the definition of relation,
which is a pair of a function from terms to goals and a proof that it is closed and consistent.
So ``\lstinline[language=Coq]|(proj1_sig (Prog r) t)|'' here simply takes the body of the corresponding relation.

The lemma relating bounded and unbounded denotational semantics in \textsc{Coq} looks as follows:

\begin{lstlisting}[language=Coq] 
  Lemma in_denotational_sem_some_lev:
    forall (g : goal) (f : repr_fun),
      [| g , f |] -> exists l, [| l | g , f |].
\end{lstlisting}

We prove two important properties of the denotational semantics: lemma~\ref{lem:den_sem_change_var} that states that the choice of a subsituted fresh variable doesn't matter, and the closedness condition (lemma~\ref{lem:closedness_condition}). The second proof is by a straightforward structural induction, the structure of the first one is described in the section~\ref{appendix_den_sem_change_var_proof}.

For operational semantics, the described transition relation can be encoded naturally as an inductively defined proposition (here ``\lstinline|nt_state|''
stands for a non-terminal state and ``\lstinline|state|''~--- for arbitrary one):

\begin{lstlisting}[language=Coq]
  Inductive eval_step :
    nt_state -> label -> state -> Set := ...
\end{lstlisting}

We state the fact that our system is deterministic through the existence and uniqueness of a transition for every state:

\begin{lstlisting}[language=Coq]
  Lemma eval_step_$\texttt{exists}$ : forall (nst : nt_state),
    {l : label & {st : state & eval_step nst l st}}.
  Lemma eval_step_unique : forall (nst : nt_state),
      (l1 l2 : label)
      (st1 st2 : state'),
    eval_step nst l1 st1 ->
    eval_step nst l2 st2 ->
    l1 = l2 /\ st1 = st2.
\end{lstlisting}

Then we define a trace coinductively as a stream of labels in transition steps and prove that there exists a unique trace from any extended state:

\begin{lstlisting}[language=Coq]
  Definition trace : Set := $@$stream label.
  CoInductive op_sem : state -> trace -> Set :=
  | osStop : op_sem Stop Nil
  | osState : forall nst l st t,
      eval_step nst l st ->
      op_sem st t ->
      op_sem (State nst) (Cons l t).
  Lemma op_sem_$\texttt{exists}$ (st : state):
    {t : trace & op_sem st t}.
  Lemma op_sem_unique:
    forall st t1 t2,
      op_sem st t1 ->
      op_sem st t2 ->
      equal_streams t1 t2.
\end{lstlisting}

And we prove the important properties of operational semantics (lemmas~\ref{lem:well_formedness_preservation}, \ref{lem:sum_answers}, \ref{lem:prod_answers} and \ref{lem:disj_termination}).

Finally, we prove both soundness and completeness of the operational semantics of the interleaving search w.r.t. the denotational one.
The statements of the theorems are as follows:

\newpage \begin{lstlisting}[language=Coq]]
  Theorem search_correctness:
    forall (g : goal) (k : nat) (f : repr_fun) (t : trace),
      closed_goal_in_context (first_nats k) g ->
      op_sem (State (Leaf g empty_subst k)) t ->
      {| t , f |} ->
      [| g , f |].
  Theorem search_completeness:
    forall (g : goal) (k : nat) (f : repr_fun) (t : trace),
      consistent_goal g ->
      closed_goal_in_context (first_nats k) g ->
      op_sem (State (Leaf g empty_subst k)) t ->
      [| g , f |] ->
      exists (f' : repr_fun),
        {| t , f' |} /\
        forall (x : var), In x (first_nats k) -> f x = f' x.
\end{lstlisting}

Note that we need the consistency requirement only for completeness, not for soundness.

The proof of soundness is by a straightforward structural induction, it is divided between lemmas~\ref{lem:answer_soundness} and \ref{lem:next_state_soundness}. The proof of completeness is described in the section~\ref{appendix_gen_completeness_proof}.

The specification of semantics for SLD resolution with cut simply repeats the just described specification of semantics for interleaving search with minimal modification. Specifically, we introduce one additional type of goal (\lstinline|Cut|), in denotational semantics cuts correspond to the whole domain of representing functions, in operational semantics we distinguish two kinds of \lstinline|Sum| state, introduce possible cut signal into the definition of transitions and add rules described in the section~\ref{appendix_cut} to the transition system. 

\begin{lstlisting}
  Inductive cutting_mark : Set :=
  | StopCutting : cutting_mark
  | KeepCutting : cutting_mark.
  Inductive nt_state : Set :=
  ...
  | Sum  : cutting_mark -> nt_state -> nt_state -> nt_state
  ...
  Inductive cut_signal : Set :=
  | NoCutting  : cut_signal
  | YesCutting : cut_signal.
  Inductive eval_step_SLD :
     nt_state -> label -> cut_signal -> state -> Set := ...
\end{lstlisting}

We then repeat the soundness proof for these semantics with minor changes (and skip the completeness proof, since it does not hold for SLD resolution).

\section{SLD Resolution with Cut}
\label{appendix_cut}

\begin{figure*}[t]
\[
  \begin{array}{cr}
    \dfrac{s_1 \xrightarrow{\step} \Diamond}{(s_1 \circledast s_2) \xrightarrow{\step} s_2} & \ruleno{AstStop}\\
    \dfrac{s_1 \xrightarrow{r} \Diamond}{(s_1 \circledast s_2) \xrightarrow{r} s_2} & \ruleno{AstStopAns}\\
    \dfrac{s_1 \xrightarrow{\step} s'_1}{(s_1 \circledast s_2) \xrightarrow{\step} (s_2 \circledast s'_1)} &\ruleno{AstStep}\\
    \dfrac{s_1 \xrightarrow{r} s'_1}{(s_1 \circledast s_2) \xrightarrow{r} (s_2 \circledast s'_1)} &\ruleno{AstStepAns}\\
  \end{array}
\]
\caption{Rules for ``$\circledast$'' evaluation}
\label{asterisk-rules}
\end{figure*}

\begin{figure*}[t]
\[
\begin{array}{crc|ccr}
  \dfrac{s_1 \xrightarrow{\step}_c \Diamond}{(s_1 \oplus s_2) \xrightarrow{\step} \Diamond} & \ruleno{SumStopC}& & &
  \dfrac{s_1 \xrightarrow{r}_c \Diamond}{(s_1 \oplus s_2) \xrightarrow{r} \Diamond} & \ruleno{SumStopAnsC}\\
  \dfrac{s_1 \xrightarrow{\step}_c s'_1}{(s_1 \oplus s_2) \xrightarrow{\step} s'_1} &\ruleno{SumStepC}& & &
  \dfrac{s_1 \xrightarrow{r}_c s'_1}{(s_1 \oplus s_2) \xrightarrow{r} s'_1} &\ruleno{SumStepAnsC}\\
  \dfrac{s_1 \xrightarrow{\step}_c \Diamond}{(s_1 \circledast s_2) \xrightarrow{\step}_c \Diamond} & \ruleno{AstStopC}& & &
  \dfrac{s_1 \xrightarrow{r}_c \Diamond}{(s_1 \circledast s_2) \xrightarrow{r}_c \Diamond} & \ruleno{AstStopAnsC}\\
  \dfrac{s_1 \xrightarrow{\step}_c s'_1}{(s_1 \circledast s_2) \xrightarrow{\step}_c s'_1} &\ruleno{AstStepC}& & &
  \dfrac{s_1 \xrightarrow{r}_c s'_1}{(s_1 \circledast s_2) \xrightarrow{r}_c s'_1} &\ruleno{AstStepAnsC}\\
  \dfrac{s \xrightarrow{\step}_c \Diamond}{(s \otimes g) \xrightarrow{\step}_c \Diamond} & \ruleno{ProdStopC}& & &
  \dfrac{s \xrightarrow{(\sigma, n)}_c \Diamond}{(s \otimes g) \xrightarrow{\step}_c \inbr{g, \sigma, n}} & \ruleno{ProdStopAnsC}\\
  \dfrac{s \xrightarrow{\step}_c s'}{(s \otimes g) \xrightarrow{\step}_c (s' \otimes g)} &\ruleno{ProdStepC}& & &
  \dfrac{s \xrightarrow{(\sigma, n)}_c s'}{(s \otimes g) \xrightarrow{\step}_c (\inbr{g, \sigma, n} \circledast (s' \otimes g))} &\ruleno{ProdStepAnsC}
\end{array}
\]
\caption{Cut signal propagation rules}
\label{cut-signal-propagation}
\end{figure*}
%\FloatBarrier

The semantics for SLD resolution with cut is built upon our usual semantics for SLD resolution described in section~\ref{sld} using a few additional constructs and rules for them.

We introduce one additional type of goal --- ``cut''. In denotational semantics, we interpret cut as success (thus, denotationally we treat all cuts as green).

Operationally, we modify SLD semantics in such a way that a cut cuts all other branches of all enclosing nodes, marked with ``$\oplus$'', up to
the moment when the evaluation of the disjunct, containing the cut, was started. It is easy to see that this node will always
be the nearest ``$\oplus$'', derived from the disjunction. Unfortunately, in the tree other ``$\oplus$'' nodes can
appear due to the evaluation of ``$\otimes$'' nodes, thus we need a way to distinguish these two sorts of ``$\oplus$''. We introduce a separate kind of state ``$\circledast$'' for the sums of goals created during the evaluation of ``$\otimes$'' nodes.

Therefore, in the semantics the rule \textsc{[ProdStepAns]} is replaced with the following.

\[
\begin{array}{cr}
  \dfrac{s \xrightarrow{(\sigma, n)} s'}{(s \otimes g) \xrightarrow{\circ} (\inbr{g, \sigma, n} \circledast (s' \otimes g))} & \ruleno{ProdStepAns} 
\end{array}
\]
\vskip3mm

The rules for ``$\circledast$'' evaluation mirror those for ``$\oplus$'' (see Fig.~\ref{asterisk-rules}).

To perform the cutting of the branches, we introduce a separate kind of transitions to propagate the signal for cutting (denoted by $\xrightarrow{l}_c$). The signal is risen when a ``cut'' construct is encountered.

\[
\begin{array}{cr}
  \inbr{!, \sigma, n} \xrightarrow{(\sigma, n)}_c \Diamond &\ruleno{Cut} 
\end{array}
\]
\vskip3mm

When the signal is being propagated through ``$\oplus$'' and ``$\circledast$'' nodes, their right branches are cut out, and for ``$\circledast$'' the
propagation continues; in the case of ``$\otimes$'' nodes the signal is simply propagated (see Fig.~\ref{cut-signal-propagation}).