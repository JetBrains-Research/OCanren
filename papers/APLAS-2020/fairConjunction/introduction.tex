\section{Introduction}

%Declarative programming is a programming paradigm, which allows to express the logic of a computation without describing its control flow. One of the declarative languages
%is relational language \mk~\cite{fair:TheReasonedSchemer}. This minimalistic language has only few operators. This minimalistic language consists of only a few operators,
%and its minimal implementation~\cite{fair:micro} contains only 54 code lines of Racket. Despite its modest size, \mk is Turing complete and allows to express complex
%computations in the form of compact declarative specifications. For instance, quine generator~\cite{fair:quines}, proof checker and theorem prover~\cite{fair:teorem-prover},
%interpreter to perform example-based program synthesis~\cite{fair:seven} has declarative solution of \mk.

\mk~\cite{fair:TheReasonedSchemer,fair:micro} is known for its capability to express solutions for complex problems~\cite{fair:quines,fair:theorem-prover,fair:seven}
in the form of compact declarative specifications. However, \emph{conjunction} in \mk has somewhat imperative flavor. The evaluation of conjunction is asymmetrical
and the order of conjuncts affects not only the performance of the program but even the convergence. As a result, the order of conjuncts determines control flow in
a relational program. We may call this directed behavior of conjunction \textit{unfair}.

% Проблема несправедливого поведения конъюнкции уже рассматривалась с нескольких разных точек зрения. Одним из аспектов несправедливого поведения конъюнкции является управления приоритетами вычисления независимых ветвей, которые порождает конъюнкция. В оригинальном языке больший приоритет отдается более ранней ветке. Однако существует подход, который позволяет сбалансировать время исполнения. Это делает поведение конъюнкции более справедливым, но чувствительность к порядку конъюнктов сохраняется. Также конъюнкция становится более честной, если детектировать расхождение конъюнктов. Данный подход во время исполнения может обнаружить расхождение конъюкта. В этом случае данные, которые были получены при исполнении конъюнкта стираются. После чего происходит перестановка конъюнктов и продолжается исполнение программы. Данный подход оказывается эффективен на практике. Недостатком является консервативная перестановка конъюнктов, которпя не использует информацию, полученную при вычислении конъюнкта до перестановки.
The problem of unfair behavior of conjunction has already been examined from several different points of view. One aspect of the unfair behavior of conjunction is to prioritize the
evaluation of the independent branches which conjunction generates. In the original \mk a higher priority is given to an earlier branch. However, there is an approach~\cite{fair:towardsAM}
which allows one to balance the evaluation time of different branches. This makes the conjunction behavior fairer, but the order of the conjuncts still affects both performance and convergence.
Also, the conjunction can be made fairer if we are capable of detecting the divergence in its conjuncts. The approach~\cite{fair:DivTest} detects the divergence at run time and
performs switching of conjuncts. In this case, the data that was received during the evaluation of the diverging conjunct is erased. The approach turned out to be efficient in practice.
However, the conservative rearrangement of the conjuncts does not use the information obtained when evaluating the conjunct before the rearrangement. There are also examples for
this approach where the order of the conjuncts affects convergence.

The contribution of this paper is a more declarative approach to the evaluation of relational programs of \mk. This approach executes the conjuncts alternately, choosing a more optimal execution order.
The fair conjunction that we propose is comparable in efficiency to the classic unfair one, but the order of the conjuncts weakly affects both efficiency and convergence. Our approach also 
demonstrates a more convergent behavior: we present some examples where classical conjunction diverges for any order of conjuncts while the fair conjunction converges.

The paper is organized as follows. In Section~\ref{sec:directed} we discuss the advantages and drawbacks of the classical directed conjunction. Section~\ref{sec:directed-semantics} contains a
description of \mk syntax as well as operational semantics based on the unfolding operation. In Section~\ref{sec:naive} we present the semantics of a naive fair conjunction,
and in Section~\ref{sec:structural} we extend these semantics by controlling the conjunction order based on structural recursion. Section~\ref{sec:eval} is devoted to the evaluation
and performance comparison on different semantics in the form of interpreters. The final section concludes.
