\begin{figure*}[t]
\[
\begin{array}{cccll}
  &\mathcal{C} & = & \{C_i^{k_i}\} & \mbox{constructors with arities} \\
  &\mathcal{T}_X & = & X \cup \{C_i^{k_i} (t_1, \dots, t_{k_i}) \mid t_j\in\mathcal{T}_X\} & \mbox{terms over the set of variables $X$} \\
  &\mathcal{D} & = & \mathcal{T}_\emptyset & \mbox{ground terms}\\
  &\mathcal{X} & = & \{ x, y, z, \dots \} & \mbox{syntactic variables} \\
  &\mathcal{A} & = & \{ \alpha, \beta, \gamma, \dots \} & \mbox{semantic variables} \\
  &\mathcal{R} & = & \{ R_i^{k_i}\} &\mbox{relational symbols with arities} \\
  &\mathcal{G} & = & \mathcal{T_X}\equiv\mathcal{T_X}   &  \mbox{unification} \\
  &            &   & \mathcal{G}\wedge\mathcal{G}     & \mbox{conjunction} \\
  &            &   & \mathcal{G}\vee\mathcal{G}       &\mbox{disjunction} \\
  &            &   & \mbox{\lstinline|fresh|}\;\mathcal{X}\;.\;\mathcal{G} & \mbox{fresh variable introduction} \\
  &            &   & R_i^{k_i} (t_1,\dots,t_{k_i}),\;t_j\in\mathcal{T_X} & \mbox{relational symbol invocation} \\
  &\mathcal{S} & = & \{R_i^{k_i} = \lambda\;x_1^i\dots x_{k_i}^i\,.\, g_i;\}\; g & \mbox{specification}
\end{array}
\]
\caption{The syntax of the source language}
\label{syntax}
\end{figure*}

\begin{figure}[t]
%\centering
\[
\begin{array}{rcl}
  \mathcal{FV}\,(x)&=&\{x\}\\
  \mathcal{FV}\,(C_i^{k_i}\,(t_1,\dots,t_{k_i}))&=&\bigcup\mathcal{FV}\,(t_i)\\
  \mathcal{FV}\,(t_1\equiv t_2)&=&\mathcal{FV}\,(t_1)\cup\mathcal{FV}\,(t_2)\\
  \mathcal{FV}\,(g_1\wedge g_2)&=&\mathcal{FV}\,(g_1)\cup\mathcal{FV}\,(g_2)\\
  \mathcal{FV}\,(g_1\vee g_2)&=&\mathcal{FV}\,(g_1)\cup\mathcal{FV}\,(g_2)\\
  \mathcal{FV}\,(\mbox{\lstinline|fresh|}\;x\;.\;g)&=&\mathcal{FV}\,(g)\setminus\{x\}\\
  \mathcal{FV}\,(R_i^{k_i}\,(t_1,\dots,t_{k_i}))&=&\bigcup\mathcal{FV}\,(t_i)
\end{array}
\]
\caption{Free variables in terms and goals}
\label{free}
\end{figure}

\section{The Language}
\label{language}
 
In this section we introduce the syntax of the language we use throughout the paper, describe the informal semantics and give some examples.

The syntax of the language is shown in Figure~\ref{syntax}. First, we fix a set of constructors $\mathcal{C}$ with known arities and consider
a set of terms $\mathcal{T}_X$ with constructors as functional symbols and variables from $X$. We parameterize this set with an alphabet of
variables since in the semantic description we will need \emph{two} kinds of variables. The first kind, \emph{syntactic} variables, is denoted
by $\mathcal{X}$. The second kind, \emph{semantic} or \emph{logic} variables, is denoted by $\mathcal{A}$.
We also consider an alphabet of \emph{relational symbols} $\mathcal{R}$ which are used to name relational definitions.
The central syntactic category in the language is \emph{goal}. In our case, there are five types of goals: \emph{unification} of terms,
conjunction and disjunction of goals, fresh variable introduction, and invocation of some relational definition. Thus, unification is used
as a constraint, and multiple constraints can be combined using conjunction, disjunction, and recursion. \orangenote{\textbf{Переместить к примеру}} {\color{orange} For the sake of brevity we
abbreviate immediately nested ``\lstinline|fresh|'' constructs into the one, writing ``\lstinline|fresh $x$ $y$ $\dots$ . $g$|'' instead of
``\lstinline|fresh $x$ . fresh $y$ . $\dots$ $g$|''.} The final syntactic category is a \emph{specification} $\mathcal{S}$. It consists of a set
of relational definitions and a top-level goal. A top-level goal represents a search procedure which returns a stream of substitutions for
the free variables of the goal. The definition for a set of free variables for both terms and goals is given in Figure~\ref{free}; as ``\lstinline|fresh|''
is the sole binding construct the definition is rather trivial. The language we defined is first-order, as goals can not be passed as parameters,
returned or constructed at runtime.

We now informally describe how relational search works. As we said, a goal represents a search procedure. This procedure takes a \emph{state} as input and returns a
stream of states; a state (among other information) contains a substitution that maps semantic variables into the terms over semantic variables. Then five types of
scenarios are possible (depending on the type of the goal):

\begin{itemize}
\item Unification ``\lstinline|$t_1$ === $t_2$|'' unifies terms $t_1$ and $t_2$ in the context of the substitution in the current state. If terms are unifiable,
  then their MGU is integrated into the substitution, and a one-element stream is returned; otherwise the result is an empty stream.
\item Conjunction ``\lstinline|$g_1$ /\ $g_2$|'' applies $g_1$ to the current state and then applies $g_2$ to each element of the result, concatenating
  the streams.
\item Disjunction ``\lstinline|$g_1$ \/ $g_2$|'' applies both its goals to the current state independently and then concatenates the results.
\item Fresh construct ``\lstinline|fresh $x$ . $g$|'' allocates a new semantic variable $\alpha$, substitutes all free occurrences of $x$ in $g$ with $\alpha$, and
  runs the goal.
\item Invocation ``$\lstinline|$R_i^{k_i}$ ($t_1$,...,$t_{k_i}$)|$'' finds a definition for the relational symbol \mbox{$R_i^{k_i}=\lambda x_1\dots x_{k_i}\,.\,g_i$}, substitutes
  all free occurrences of a formal parameter $x_j$ in $g_i$ with term $t_j$ (for all $j$) and runs the goal in the current state.
\end{itemize}

We stipulate that the top-level goal is preceded by an implicit ``\lstinline|fresh|'' construct, which binds all its free variables, and that the final substitutions
for these variables constitute the result of the goal evaluation.

Conjunction and disjunction form a monadic~\cite{Monads} interface with conjunction playing role of ``\lstinline|bind|'' and disjunction~--- of ``\lstinline|mplus|''.
In this description, we swept a lot of important details under the carpet~--- for example, in actual implementations the components of disjunction are not evaluated in
isolation, but both disjuncts are being evaluated incrementally with the control passing from one disjunct to another (\emph{interleaving})~\cite{Search};
the evaluation of some goals can be additionally deferred (via so-called ``\emph{inverse-$\eta$-delay}'')\cite{MicroKanren}; instead of streams
the implementation can be based on ``ferns''~\cite{BottomAvoiding} to defer divergent computations, etc. In the following sections, we present
a complete formal description of relational semantics which resolves these uncertainties in a conventional way.

As an example consider the following specification:

\begin{lstlisting}
  append$^o$ = fun x y xy .
    ((x === Nil) /\ (xy === y)) \/
    (fresh h t ty .
       (x  === Cons (h, t))  /\
       (xy === Cons (h, ty)) /\
       (append$^o$ t y ty)
    );
  revers$^o$ = fun x y .
    ((x === Nil) /\ (y === Nil)) \/
    (fresh h t t' .
       (x === Cons (h, t)) /\
       (append$^o$ t' (Cons (h, Nil)) y) /\
       (revers$^o$ t t') 
    );
  revers$^o$ x x
\end{lstlisting}

Here we defined\footnote{We respect here a conventional tradition for \textsc{miniKanren} programming to superscript all relational names with ``$^o$''.}
two relational symbols~--- ``\lstinline|append$^o$|'' and ``\lstinline|revers$^o$|'',~--- and specified a top-level goal ``\lstinline|revers$^o$ x x|''.
The symbol ``\lstinline|append$^o$|'' defines a relational concatenation of lists~--- it takes three arguments and performs a case analysis on the first one. \orangenote{\textbf{Можно убрать подробности:}, performing a recursive call on the tail in the case where it is not empty. The definition resembles the usual one from functional programming.}{ \color{red} If the
first argument is an empty list (``\lstinline|Nil|''), then the second and the third arguments are unified. Otherwise, the first argument is deconstructed into a head ``\lstinline|h|''
and a tail ``\lstinline|t|'', and the tail is concatenated with the second argument using a recursive call to ``\lstinline|append$^o$|'' and additional variable ``\lstinline|ty|'', which
represents the concatenation of ``\lstinline|t|'' and ``\lstinline|y|''. Finally, we unify ``\lstinline|Cons (h, ty)|'' with ``\lstinline|xy|'' to form a final constraint.} Similarly,
``\lstinline|revers$^o$|'' defines relational list reversing. The top-level goal represents a search procedure for all lists ``\lstinline|x|'', which are stable under reversing, i.e.
palindromes. Running it results in an infinite stream of substitutions:

\begin{lstlisting}
   $\alpha\;\mapsto\;$ Nil
   $\alpha\;\mapsto\;$ Cons ($\beta_0$, Nil)
   $\alpha\;\mapsto\;$ Cons ($\beta_0$, Cons ($\beta_0$, Nil))
   $\alpha\;\mapsto\;$ Cons ($\beta_0$, Cons ($\beta_1$, Cons ($\beta_0$, Nil)))
   $\dots$
\end{lstlisting}

where ``$\alpha$''~--- a \emph{semantic} variable, corresponding to ``\lstinline|x|'', ``$\beta_i$''~--- free semantics variables.

The syntax described above can be formalized in \textsc{Coq} in a natural way using inductive data types. We have made a few non-essential simplifications and modifications for the sake of convenience.
Specifically, we restrict the arities of constructors to be either zero or two:

\begin{lstlisting}[language=Coq] 
   Inductive term : Set :=
   | Var : name -> term
   | Cst : name -> term
   | Con : name -> term -> term -> term.
\end{lstlisting}

Here ``\lstinline[language=Coq]{name}'' is a type for all named entities (variables, constructors and relations), for which we simply take natural numbers:

\begin{lstlisting}[language=Coq]
  Definition name : Set := nat.
\end{lstlisting}

Similarly, we restrict relations to always have exactly one argument:

\begin{lstlisting}[language=Coq]
   Definition rel : Set := term -> goal.
\end{lstlisting}

These restrictions do not make the language less expressive in any way since we can always represent a sequence of terms as a list using constructors \lstinline|Nil$^0$| and \lstinline|Cons$^2$|.

We also introduce one additional type of goals~--- \emph{failure}~--- for deliberately unsuccessful computation (empty stream). As a result, the definition of goals looks as follows:

\begin{lstlisting}[language=Coq] 
   Inductive goal : Set :=
   | Fail   : goal
   | Unify  : term -> term -> goal
   | Disj   : goal -> goal -> goal
   | Conj   : goal -> goal -> goal
   | Fresh  : (name -> goal) -> goal
   | Invoke : name -> term -> goal.
\end{lstlisting}

{ \color{red}

We define sets of free variables for terms naturally as \textsc{Coq} sets: 

\begin{lstlisting}[language=Coq,mathescape=true] 
   Fixpoint fv_term (t : term) : $\mbox{\small{set}}$ name :=
     ...
\end{lstlisting}

And we use them to define ground terms as a subset type:

\begin{lstlisting}[language=Coq]
   Definition ground_term : Set :=
     {t : term | fv_term t = empty_set name}.
\end{lstlisting}

However, for goals it is more convenient to define a set of free variables as a proposition:

\begin{lstlisting}[language=Coq]
   Inductive is_fv_of_goal (n : name) : goal -> Prop :=
     ...
\end{lstlisting}

}

Note that in our formalization we use higher-order abstract syntax for variable binding~\cite{HOAS}, therefore we work explicitly only with semantic variables.
We preferred it to the first-order syntax because it gives us the ability to use substitution and the induction principle provided by \textsc{Coq}.
On the other hand, we need to explicitly specify some requirements on the syntax representation which are trivially fulfilled in the first-order case.

First, we need a requirement that the definitions of relations do not contain unbound variables:

\begin{lstlisting}[language=Coq] 
  Definition closed_goal_in_context 
    (c : list name) (g : goal) : Prop :=
      forall n, is_fv_of_goal n g -> In n c.
  Definition closed_rel (r : rel) : Prop :=
    forall (arg : term),
    closed_goal_in_context (fv_term arg) (r arg).
\end{lstlisting}

The second requirement is that all bindings have to be ``consistent'', i.e. if we instantiate a higher-order ``\lstinline|fresh|''
construct for different variables the results will be the same up to some renaming (provided that both those variables are not free in
the body of the binder). \orangenote{Можно перенести в аппендикс} { \color{orange} It turned out to be rather non-trivial to define regular variable renaming for goals in higher-order syntax, but for our purposes
a weaker version which deals with only non-free variables is sufficient:

\begin{lstlisting}[language=Coq]
   Inductive renaming (old_x : name) (new_x : name) :
     goal -> goal -> Prop :=
   ...
   | rFreshNFV : forall fg,
                 (~is_fv_of_goal old_x (Fresh fg)) ->
                 renaming old_x new_x (Fresh fg) (Fresh fg)
   | rFreshFV : forall fg rfg,
                (is_fv_of_goal old_x (Fresh fg)) ->
                (forall y, (~is_fv_of_goal y (Fresh fg)) ->
                      renaming old_x new_x (fg y) (rfg y)) ->
                renaming old_x new_x (Fresh fg) (Fresh rfg)
   ...
\end{lstlisting}

The consistency for goals and relations then can be defined as follows:

\begin{lstlisting}[language=Coq]
   Definition consistent_binding (b : name -> goal) : Prop :=
     forall x y, (~ is_fv_of_goal x (Fresh b)) ->
           renaming x y (b x) (b y).
   Inductive consistent_goal : goal -> Prop :=
     ...
   Definition consistent_function
     (f : term -> goal) : Prop :=
     forall a1 a2 t, renaming a1 a2 (f t)
                     (f (apply_subst [(a1, Var a2)] t)).
   Definition consistent_rel (r : rel) : Prop :=
     forall (arg : term), consistent_goal (r arg) /\
                     consistent_function r.
\end{lstlisting}

In the snippet above the ``\lstinline[language=Coq]{consistent_goal}'' property inductively ensures that all bindings occurring
in the goal are consistent and ``\lstinline[language=Coq]{apply_subst [(a1, Var a2)] t}'' in ``\lstinline[language=Coq]{consistent_function}''
definition renames a variable \lstinline[language=Coq]{a1} into in \lstinline[language=Coq]{a2} term \lstinline[language=Coq]{t}.

}

We can now set an arbitrary environment (a map from a relational symbol to a definition of relation with described constraints) to use further throughout the formalization.
Failure goals allow us to define it as a total function:

\begin{lstlisting}[language=Coq]
   Definition def : Set := 
     {r : rel | closed_rel r /\ consistent_rel r}.
   Definition env : Set := name -> def.
   Variable Prog : env.
\end{lstlisting}
