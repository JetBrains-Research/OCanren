\section{Related Works}
\label{sec:relworks}

There is a predictable difficulty in implementing \miniKanren for a strongly typed language.
Designed in the metaprogramming-friendly and dynamically typed realm of Scheme/Racket, the original
\miniKanren implementation pays very little attention to what has a significant importance in (specifically)
ML or Haskell. In particular, one of the capstone constructs of \miniKanren~--- unification~--- has to work for
different data structures, which may have types different beyond parametricity.

There are a few ways to overcome this problem. The first one is simply to follow the untyped paradigm and
provide unification for some concrete type, rich enough to represent any reasonable data structures.
Some Haskell \miniKanren libraries\footnote{\url{https://github.com/JaimieMurdock/HK}, \url{https://github.com/rntz/ukanren}}
as well as the previous OCaml implementation\footnote{\url{https://github.com/lightyang/minikanren-ocaml}} take this approach.
As a result, the original implementation can be retold with all its elegance; the relational specifications, however,
become weakly typed. A similar approach was taken in early works on embedding Prolog into Haskell~\cite{PrologInHaskell}.

Another approach is to utilize \emph{ad hoc} polymorphism and provide a type-specific unification for each ``interesting'' type.
Some \miniKanren implementations, such as Molog\footnote{\url{https://github.com/acfoltzer/Molog}} and
MiniKanrenT\footnote{\url{https://github.com/jvranish/MiniKanrenT}}, both for Haskell, can be mentioned as examples.
While preserving strong typing, this approach requires a lot of ``boilerplate''
code to be written, so some automation --- for example, using Template Haskell\footnote{\url{https://wiki.haskell.org/Template_Haskell}} ---
is desirable. In~\cite{TypedLogicalVariables} a separate type class was introduced to both perform the unification
and detect free logical variables in end-user data structures. The requirement for end user to provide a way to represent
logical variables in custom data structures looks superfluous for us since these logical variables would require proper
handling in the rest of the code outside the logical programming subsystem.

There is, actually, another potential approach, but we do not know if anybody has tried
it: implementing unification for a generic representation of types as sum-of-products and fixpoints of
functors~\cite{InstantGenerics, ALaCarte}. Thus, unification would work for any type for which a representation
is provided. We believe that implementing this representation would require less boilerplate code to be written.

As follows from this exposition, a typed embedding of \miniKanren in OCaml can be done with
a combination of datatype-generic programming~\cite{DGP} and \emph{ad hoc} polymorphism. There are 
a number of generic frameworks for OCaml (for example,~\cite{Deriving}). On the other hand, the support
for \emph{ad hoc} polymorphism in OCaml is weak; there is nothing comparable in power to Haskell
type classes, and even though sometimes the object-oriented layer of the language can be used to mimic
desirable behavior, the result, as a rule, is far from satisfactory. Existing proposals for \emph{ad hoc} polymorphism (for example,
modular implicits~\cite{Implicits}) require patching the compiler, which we want to avoid. Therefore, we 
take a different approach, implementing polymorphic unification once and for all logical types~--- a purely \emph{ad hoc} 
approach, since the features which would provide a less \emph{ad hoc} solution are not yet well integrated into the language. To deal
with user-defined types in the relational subsystem, we propose to use their logical representations (see Section~\ref{sec:injection}), 
which free an end user from the burden of maintaining logical variables, and we use generic programming to build conversions from and to logical
representations in a systematic manner.



