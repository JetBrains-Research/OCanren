\section{Reification and Top-Level Primitives}
\label{reification}

Since we do not make states accessible on the user level explicitly, we provide
a set of top-level combinators, which should be used to surround relational code
and perform reification in a transparent manner. Note, from a pragmatic
standpoint only variables supplied as arguments for the top-level goal have
to be reified (the original \miniKanren implementation follows the same convention).

The top-level primitive in our implementation is \lstinline{run}, which takes three
arguments. The exact type of \lstinline{run} is rather complex and non-instructive,
so we prefer to describe the typical form of its application:

\begin{lstlisting}[mathescape=true]
   run $\overline{n}$ (fun $l_1\dots l_n$ -> $\;\;G$) (fun $a_1\dots a_n$ -> $\;\;H$)
\end{lstlisting}

Here $\overline{n}$ stands for a \emph{numeral}, which describes the number of
parameters for two other arguments of \lstinline{run}, \mbox{$l_1\dots l_n$}~---
free logical variables, $G$~--- a goal (which can make use of \mbox{$l_1\dots l_n$}),
\mbox{$a_1\dots a_n$}~--- reified answers for \mbox{$l_1\dots l_n$}, respectively, and,
finally, $H$~--- a \emph{handler} (which can make use of \mbox{$a_1\dots a_n$}).

The types of \mbox{$l_1\dots l_n$} are inferred from $G$ and always have a form

\begin{lstlisting}
   $\{\alpha,\;[\beta]\}$
\end{lstlisting}

since the types of variables can be constrained only in unification or disequality constraints.

The types of \mbox{$a_1\dots a_n$} are inferred from types of \mbox{$l_1\dots l_n$} and
have the form

\begin{lstlisting}
   $(\alpha,\;\beta)$ reified stream
\end{lstlisting}

where the type \lstinline{reified}, in turn, is

\begin{lstlisting}
   type ($\alpha$, $\beta$) reified = $<\;$prj : $\alpha$; reify : (helper -> $\{\alpha,\;\beta\}$ -> $\beta$) -> $\beta>$
\end{lstlisting}

Two methods of this type can be used to perform two different styles of reificationt: first, a value without
free variables can be returned as is (method \lstinline{prj} does this: it checks that in the value of
interest no free variables occur, and raises an exception otherwise). If the value contains free
variables, it has to be properly injected into the logic domain~--- this is what \lstinline{reify} stands
for. It takes as an argument a type-specific tagging function, constructed using generic
primitives described in the previous section.

In other words a user-defined handler takes streams of reified answers for all variables supplied to the top-level
goal. All streams $a_i$ contains coherent elements, so they all have the same length and $n$-th elements of all
streams correspond to the $n$-th answer, produced by the goal $G$.

There are a few predefined numerals for one, two, etc. arguments (called, by tradition,
\lstinline{q}, \lstinline{qr}, \lstinline{qrs} etc.), and a successor function, which
can be applied to existing numeral to increment the number of expected arguments. The
implementation technique generally follows~\cite{Unparsing, DoWeNeed}.
