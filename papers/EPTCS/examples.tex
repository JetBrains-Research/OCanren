\section{Examples}
\label{sec:examples}

In this section we present some examples of a relational specification, written with the aid of our library.
Besides \miniKanren combinators themselves, our implementation contains two syntax extensions~--- one
for \lstinline{fresh} construct and another for \emph{inverse-$\eta$-delay}~\cite{MicroKanren}, which is
sometimes necessary to delay recursive calls in order to prevent infinite looping. In addition, we included a
small relational library of data structures like lists, numbers, booleans, etc. This library is written
completely on the user level using techniques described in Section~\ref{sec:injection} with no utilization
of any unsafe features. The examples given below illustrate the usage of all these elements as well.

\subsection{List Concatenation and Reversing}

List concatenation and reversing are usually the first relational programs considered, and we do not wish
to deviate from this tradition. We've already considered the implementation of \lstinline{append$^o$} in
original \miniKanren in Section~\ref{sec:demo}. In our case, the implementation looks familiar:

\begin{lstlisting}
   let rec append$^o$ x y xy =
     (x === nil ()) &&& (y === xy) |||
     (fresh (h t)
       (x === h % t) &&&
       (fresh (ty)
         (h % ty === xy) &&& (append$^o$ t y ty)
       )
     )

   let rec revers$^o$ a b =
     conde [
       (a === nil ()) &&& (b === nil ());
       (fresh (h t)
         (a === h % t) &&&
         (fresh (a')
           (append$^o$ a' !< h b) &&& (revers$^o$ t a')
         )
       )
     ]
\end{lstlisting}

Here we make use of our implementation of relational lists, which provides convenient shortcuts for
standard functional primitives:

\begin{itemize}
  \item ``\lstinline{nil ()}'' corresponds to ``\lstinline{[]}'';
  \item ``\lstinline{h % t}'' corresponds to ``\lstinline{h::t}'';
  \item ``\lstinline{a %< (b %< (c !< d))}'' corresponds to ``\lstinline{[a; b; c; d]}''.
\end{itemize}

In our implementation the basic \miniKanren primitive ``\lstinline{conde}'' is implemented as a
disjunction of a list of goals, not as a built-in syntax construct. We also make use of explicit
conjunction and disjunction infix operators instead of nested bracketed structures which, we
believe, would look too foreign here.

\subsection{Relational Sorting and Permutations}

For the next example we take list sorting; specifically, we present a sorting for lists of natural numbers
in Peano form since our library already contains built-in support for them. However, our example can be
easily extended for arbitrary (but linearly ordered) types.

List sorting can be implemented in \miniKanren in a variety of ways~--- virtually any existing algorithm can
be rewritten relationally. We, however, try to be as declarative as possible to demonstrate the
advantages of the relational approach. From this standpoint, we can claim that the sorted version of an empty list is an
empty list, and the sorted version of a non-empty list is its smallest element, concatenated with a sorted
version of the list containing all its remaining elements.

The following snippet literally implements this definition:

\begin{lstlisting}
   let rec sort$^o$ x y =
     conde [
       (x === nil ()) &&& (y === nil ());
       fresh (s xs xs')
         (y === s % xs')
         (sort$^o$ xs xs')
         (smallest$^o$ x s xs)
     ]
\end{lstlisting}

The meaning of the expression ``\lstinline{smallest$^o$ x s xs}'' is ``\lstinline{s} is the smallest element of a (non-empty) list \lstinline{x}, and \lstinline{xs} is the
list of all its remaining elements''. Now, \lstinline{smallest$^o$} can be implemented using a case analysis (note, ``\lstinline{l}'' here is a non-empty list):

\begin{lstlisting}
   let rec smallest$^o$ l s l' =
     conde [
       (l === s % nil ()) &&& (l' === nil ());
       fresh (h t s' t' max)
         (l' === max % t') &&&
         (l === h % t) &&&
         (minmax$^o$ h s' s max) &&&
         (smallest$^o$ t s' t')
     ]
\end{lstlisting}

Finally, we implement a relational minimum-maximum calculation
primitive:

\begin{lstlisting}
   let minmax$^o$ a b min max =
     conde [
       (min === a) &&& (max === b) &&& (le$^o$ a b);
       (max === a) &&& (min === b) &&& (gt$^o$ a b)
     ]
\end{lstlisting}

Here ``\lstinline{le$^o$}'' and ``\lstinline{gt$^o$}'' are built-in comparison goals for natural numbers in Peano form.

Having relational \lstinline{sort$^o$}, we can implement sorting for regular integer lists:

\begin{lstlisting}
   let sort l =
     run q (sort$^o$ (inj_nat_list l))
           (fun qs -> from_nat_list ((Stream.hd qs)#prj) )
\end{lstlisting}

Here \lstinline{Stream.hd} is a function which takes a head from a lazy stream of answers,
\lstinline{inj_nat_list}~--- injection from regular integer lists into logical lists of logical Peano numbers,
\lstinline{from_nat_list}~--- projection from lists of Peano numbers to lists of integers.

It is interesting, that since \lstinline{sort$^o$} is
relational, it can be used to calculate the list of all \emph{permutations}
for a given list. Indeed, sorting each permutation results in the same list.
So, the problem of finding all permutations can be relationally reformulated into
the problem of finding all lists which are converted by sorting into the given one:

\begin{lstlisting}
let perm l = map (fun a -> from_nat_list a#prj)
  (run q (fun q -> fresh (r)
                     (sort$^o$ (inj_nat_list l) r)
                     (sort$^o$ q r)
         )
         (Stream.take ~n:(fact @@ length l)))
\end{lstlisting}

Note, for sorting the original list we used exactly the same primitive. Note also,
we requested exactly \lstinline{fact @@ length l} answers; requesting more
would result in an infinite search for non-existing answers.

\subsection{Type Inference for STLC}

Our final example is a type inference for Simply Typed Lambda Calculus~\cite{Lambda}. The problem and
solution themselves are rather textbook examples again~\cite{TRS, WillThesis}; however, with this example
we show once again the utilization of generic programming techniques we described in Section~\ref{sec:injection}.
As a supplementary generic programming library here we used object-oriented generic transformers\footnote{\url{https://github.com/dboulytchev/GT}};
we presume, however, that any other framework could equally be used.

We first describe the type of lambda terms and their logic representation:

\begin{lstlisting}
   module Term =
     struct
       module T =
         struct
           @type ('varname, 'self) t =
           | V   of 'varname
           | App of 'self    * 'self
           | Abs of 'varname * 'self
           with gmap

           let fmap f g x = gmap(t) f g x
         end

     include T
     include FMap2(T)

     let v   s   = inj @@ distrib @@ V s
     let app x y = inj @@ distrib @@ App (x, y)
     let abs x y = inj @@ distrib @@ Abs (x, y)
   end
\end{lstlisting}

Now we have to repeat the work for the type of simple types:

\begin{lstlisting}
   module Type =
     struct
       module T =
         struct
           @type ('a, 'b) t =
           | P   of 'a
           | Arr of 'b * 'b
           with gmap

          let fmap f g x = gmap(t) f g x
        end

       include T
       include FMap2(T)

       let p   s   = inj @@ distrib @@ P s
       let arr x y = inj @@ distrib @@ Arr (x, y)
     end
\end{lstlisting}

Note, the ``relational'' part is trivial, boilerplate and short (and could even be generated
using a more advanced framework).

The relational type inferencer itself rather resembles the original implementation. The only
difference (besides the syntax) is that instead of data constructors we use their logic
counterparts:

\begin{lstlisting}
   let rec lookup$^o$ a g t =
     fresh (a' t' tl)
       (g === (inj_pair a' t') % tl) &&&
       (conde [
         (a' === a) &&& (t' === t);
         (a' =/= a) &&& (lookup$^o$ a tl t)
       ])

   let infer$^o$ expr typ =
     let rec infer$^o$ gamma expr typ =
       conde [
         fresh (x)
           (expr === v x) &&&
           (lookupo x gamma typ);
         fresh (m n t)
           (expr === app m n) &&&
           (infer$^o$ gamma m (arr t typ)) &&&
           (infer$^o$ gamma n t);
         fresh (x l t t')
           (expr === abs x l) &&&
           (typ  === arr t t') &&&
           (infer$^o$ ((inj_pair x t) % gamma) l t')
       ]
     in
     infer$^o$ (nil()) expr typ
\end{lstlisting}
