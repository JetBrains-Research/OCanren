\section{Streams, States, and Goals}
\label{sec:goals}

This section describes a top-level framework for our implementation. Despite it contains
nothing more, then a reiteration of the original implementation~\cite{MicroKanren, CKanren} 
in OCaml, we still need some notions to be properly established.

\miniKanren is organized as a set of combinators, designated to describe a search for
the solution of a certain \emph{goal}. The search itself is implemented using a
backtracking lazy stream monad:

\begin{lstlisting}
   type 'a stream

   val mplus : 'a stream -> 'a stream -> 'a stream
   val bind  : 'a stream -> ('a -> 'b stream) -> 'b stream
\end{lstlisting}

Monadic primitives describe the shape of the search, and their implementations may 
vary in concrete \miniKanren versions.

An essential component of the implementation is a bundle of the following types

\begin{lstlisting}
   type env         = $\dots$
   type subst       = $\dots$
   type constraints = $\dots$

   type state = env * subst * constraints
\end{lstlisting}

Type \lstinline{state} describes a point in a lazily constructed search tree: type \lstinline{env} corresponds 
to an \emph{environment}, which contains some supplementary information (in particular, environment is needed to
correctly allocate fresh variables), type \lstinline{subst} describes a substitution, which keeps current bindings 
for some logical variables, and, finally, type \lstinline{constraints} represents disequality constraints, 
which have to be respected. In a simplest case \lstinline{env} is just a counter for the number of next free
variable, \lstinline{subst} is a map-like structure and \lstinline{constraints} is a list of substitutions. In our
case environment contains some extra information to make it possible to identify variables in constant time.

The next cornerstone is the \emph{goal} type, which is considered as a transformer of a state into 
lazy stream of state:

\begin{lstlisting}
   type goal = state -> state stream
\end{lstlisting}

In terms of search, a goal nondeterministically performs one step of the search: for a given
node in a search tree it produces its immediate descendants.

Next to last, there is a number of predefined combinators:

\begin{lstlisting}
   val (&&&)      : goal -> goal -> goal
   val (|||)      : goal -> goal -> goal
   val call_fresh : ($v$ -> goal) -> goal
   ....
\end{lstlisting}

Conjunction \lstinline{&&&} combines the results of its argument goals using \lstinline{bind}, 
disjunction \lstinline{|||} concatenates the results using \lstinline{mplus}, abstraction
primitive \lstinline{call_fresh} takes an abstracted goal and applies it to a freshly created
variable. Type $v$ in latest case designated the type for a fresh variable, which we leave
abstract for now. These conbinators serves as bricks for implementation of conventional 
\miniKanren constructions and syntax extensions (\lstinline{conde}, \lstinline{fresh} etc.)

Finally, there are two primitive goal constructors:

\begin{lstlisting}
   val (===) : $t$ -> $t$ -> goal
   val (=/=) : $t$ -> $t$ -> goal
\end{lstlisting}

The first one is unification, while the other~--- disequality constraint. Here we, again, left 
the type of terms $t$ abstract; it will be instantiated later.

In the implementation of \miniKanren both these goals are implemented using unification~\cite{CKanren}; this
is true for us as well. However, due to a drastic difference in host languages the implementation of
efficient polymorphic unification itself leads to a number of tricks with typing and data representation,
non-existing in the original version.

