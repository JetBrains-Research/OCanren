\section{Term Representation and Injection}
\label{sec:injection}

Polymorphic unification, considered in the previous section, works for the values of any type under the assumption that we
are capable of identifying logical variables. The latter depends on the term representation. In the original
implementation all terms are represented as a conventional S-expressions, while logical variables (in a simplest case)~--- as one-element vectors; it's an end user responsibility to respect this convention and refrain from operating
with vectors as a user data.

In our case we want to preserve both strong typing and type inference. Since we have chosen to use polymorphic
unification, it is undesirable to represent logical variables of different types differently (while technically
possible, it would compromise the lightweight approach we used so far). This means that terms with logical
variables have to be typed differently from user-defined data~--- otherwise it would be possible to use
terms in contexts where logical variables are not handled properly. At the same time we do not want term types
to be completely different from user-defined types --- for example, we would like to reuse user-defined constructors, etc.
This considerations boil down to the idea of \emph{logical representation} for a user-defined type. Informally,
a logical representation for the type $t$ is a type $\rho_t$ with a couple of conversion functions:

$$
\begin{array}{rcl}
   \uparrow  \;: t \to \rho_t & - & \mbox{injection}\\
   \downarrow\;: \rho_t \to t & - & \mbox{projection}
\end{array}
$$

The type $\rho_t$ repeats the structure of $t$, but can contain logic variables. So, the injection is total,
while the projection is partial.

It is important to design representations as instances of some generic scheme (otherwise, \miniKanren combinators
could not be properly typed). In addition it is desirable to provide a generic way to build
injection/projection pair in a uniform manner (and, even better, automatically) to lift the burden of
their implementation off the end user shoulders and improve the reliability of the solution. Finally,
the representation must provide a way to detect logic variable occurrences.


%Unification, considered in Section~\ref{sec:unification}, works for values of any types. However, it
%is too generic to be used directly. As long as we use it to unify closed terms, it's OK (but rather meaningless);
%however, for terms with free variables the situation becomes more complicated. Indeed, as we've seen in the
%previous section, a variable $x^\tau$ must have a runtime representation of some value of type $\tau$. As different
%types may not have any values in common, if we wish to unify arbitrary types, we would need to specify the
%ground type for each logical variable explicitly, which would throw our implementation into a non-polymorphic
%realm with no type inference.

In this part we consider two approaches to implementing logical representations.  The first is rather easy to
develop and implement; unfortunately, the implementation demonstrates a poor performance for a number of
important applications. In order to fix this deficiency, we develop a more elaborate technique which
nevertheless reuses some components from the previous one. In Section~\ref{sec:evaluation}
we present the results of performance evaluation for both approaches.

\subsection{Tagged Logical Values}

The first natural solution is to use tagging for representing logical representations.
We introduce the following polymorphic type $[\alpha]$\footnote{In concrete syntax called ``$\alpha\;$\lstinline{logic}''}, which
corresponds to a logical representation of the type $\alpha$:

\begin{lstlisting}
   type $[\alpha]$ = Var of int | Value of $\alpha$
\end{lstlisting}

Informally speaking, any value of type $[\alpha]$ is either a value of type $\alpha$, or a free
logic variable. Note, the constructors of this type cannot be disclosed to an end user, since the only possible way to create a logic variable
should still be by using the ``\lstinline{fresh}'' construct; thus the logic type is abstract in the interface.
Now, we may redefine the signature of abstraction, unification and disequality primitives in the
following manner

\begin{lstlisting}
   val call_fresh : ($[\alpha]$ -> goal) -> goal

   val (===)      : $[\alpha]$ -> $[\alpha]$ -> goal
   val (=/=)      : $[\alpha]$ -> $[\alpha]$ -> goal
\end{lstlisting}

Both unification and disequality constraint would still use the same polymorphic unification; their external, visible type,
however, is restricted to logical types only.

Apart from variables, other logical values can be obtained by injection; conversely, sometimes a logical value can be projected to
a regular one. We supply two basic functions\footnote{In concrete syntax called ``\lstinline{inj}'' and ``\lstinline{prj}''.}
for these purposes

\begin{lstlisting}[mathescape=true]
   val ($\uparrow_\forall$) : $\alpha$ -> $[\alpha]$
   val ($\downarrow_\forall$) : $[\alpha]$ -> $\alpha$

   let ($\uparrow_\forall$) x = Value x
   let ($\downarrow_\forall$) = function Value x -> x | _ -> failwith $\mbox{``not a value''}$
\end{lstlisting}

\noindent which can be used to perform a \emph{shallow} injection/projection. As expected, the injection is total, while the projection is partial.

The shallow pair works well for primitive types; to implement injection/projection for arbitrary types we exploit the
idea of representing regular types as fixed points of functors~\cite{ALaCarte}. For our purposes it is desirable to make
the functors fully polymorphic~--- thus a type, in which we can place a logical variable into arbitrary position,
can be easily manufactured. In addition this approach makes it possible to refactor the existing code to use relational
programming with only minor changes.

To illustrate this approach, we consider an iconic example~--- the list type. Let us have a conventional definition
for a regular polymorphic list in OCaml:

\begin{lstlisting}
   type $\alpha$ list = Nil | Cons of $\alpha$ * $\alpha$ list
\end{lstlisting}

For this type we can only place a logical variable in the position of a list element, but not of the tail, since the tail
always has the type \lstinline{$\alpha$ list}, fixed in the definition of constructor \lstinline{Cons}. In order to create
a full-fledged logical representation, we first have to abstract the type into a fully-polymorphic functor:

\begin{lstlisting}
   type ($\alpha$, $\beta$) $\mathcal L$ = Nil | Cons of $\alpha$ * $\beta$
\end{lstlisting}

Now, the original type can be expressed as

\begin{lstlisting}
   type $\alpha$ list = ($\alpha$, $\alpha$ list) $\mathcal L$
\end{lstlisting}

\noindent and its logical representation~--- as

\begin{lstlisting}
   type $\alpha$ list$^o$ = $[$($[\alpha]$, $\alpha$ list$^o$) $\mathcal L]$
\end{lstlisting}

Moreover, with the aid of conventional functor-specific mapping function

\begin{lstlisting}
   val fmap$_{\mathcal L}$ : ($\alpha$ -> $\alpha^\prime$) -> ($\beta$ -> $\beta^\prime$) -> ($\alpha$, $\beta$) $\mathcal L$ -> ($\alpha^\prime$, $\beta^\prime$) $\mathcal L$
\end{lstlisting}

\noindent both the injection and the projection functions can be implemented:

\begin{lstlisting}[mathescape=true]
   let rec $\uparrow_{\mbox{\texttt{list}}}$ l = $\uparrow_{\forall}$(fmap$_{\mathcal L}$ ($\uparrow_\forall$) $\uparrow_{\mbox{\texttt{list}}}$ l)
   let rec $\downarrow_{\mbox{\texttt{list}}}$ l = fmap$_{\mathcal L}$ ($\downarrow_\forall$) $\downarrow_{\mbox{\texttt{list}}}$ ($\downarrow_\forall$ l)
\end{lstlisting}

As functor-specific mapping functions can be easily written or, better, derived automatically using a number of existing frameworks for
generic programming for OCaml, one can easily provide injection/projection pair for user-defined data types.

We now can address the problem of variable identification during polymorphic unification. As we do not know the types, we cannot discriminate logical
variables by their tags only and, thus, cannot simply use pattern matching. In our implementation we perform a variable test
as follows:

\begin{itemize}
\item in an environment, we additionally keep some unique boxed value~--- the \emph{anchor}~--- created by \lstinline{run} at the moment of initial
state generation; the anchor is inherited unchanged in all derived environments during the search session;
\item we change the logic type definition into

\begin{lstlisting}
   type $[\alpha]$ = Var of int * anchor | Value of $\alpha$
\end{lstlisting}

\noindent making it possible to save in each variable the anchor, inherited from the environment, in which the variable was created;

\item inside the unification, in order to check if we are dealing with a variable, we test the conjunction of the following properties:

  \begin{enumerate}
    \item the scrutinee is boxed;
    \item the scrutinee's tag and layout correspond to those for variables (i.e. the values, created with the constructor \lstinline{Var} of
type \lstinline{[$\alpha$]});
    \item the scrutinee's anchor and the current environment's anchor have equal addresses.
  \end{enumerate}
\end{itemize}

Taking into account that the state type is abstract at the user level, we guarantee that only those variables which were
created during the current run session would pass the test, since the pointer to the anchor is unique among all pointers to a boxed value
and could not be disclosed anywhere but in the variable-creation primitive.

The only thing to describe now is the implementation of the reification stage. The reification is represented by the following
function:

\begin{lstlisting}
   val reify : state -> $[\alpha]$ -> $[\alpha]$
\end{lstlisting}

This function takes a state and a logic value and recursively substitutes all logic variables in that value w.r.t.
the substitution in the state until no occurrences of bound variables are left. Since in our implementation the type of a substitution is
not polymorphic, \lstinline{reify} is also implemented in an unsafe manner. However, it is easy to see that \lstinline{reify}
does not produce ill-typed terms. Indeed, all original types of variables are preserved in a substitution; unification does not
change unified terms, so all terms bound in a substitution are well-typed. Hence, \lstinline{reify} always substitutes
some subterms in a well-typed term with other terms of the corresponding types, which preserves the well-typedness.

In addition to performing substitutions, \lstinline{reify} also reifies disequality constraints. Constraint reification
attaches to each free variable in a reified term a list of reified terms, describing the disequality constraints for that
free variable. Note, disequality can be established only for equally-typed terms, which justifies the type-safety of reification.
Note also, additional care has to be taken to avoid infinite looping, since reification of answers and constraints are
mutually recursive, and the reification of a variable can be potentially invoked from itself due to a chain of disequality
constraints. In the following example

\begin{lstlisting}
   let foo q =
      fresh (r s)
        (q === $\uparrow_{\forall}$ (Some r)) &&&
        (r =/= s) &&&
        (s =/= r)
\end{lstlisting}

\noindent the answer for the variable $q$ will contain a disequality constraint for the variable $r$; the reification of $r$ will in turn
lead to the reification of its own constraint, this time the variable $s$; finally, the reification of $s$ will again invoke the
reification of $r$, etc.

After the reification, the content of a logical value can be inspected via the following function:

\begin{lstlisting}
   val destruct : $[\alpha]$ -> [`Var of int * $[\alpha]$ list | `Value of $\alpha$]
\end{lstlisting}

Constructor \lstinline{`Var} corresponds to a free variable with unique integer identifier and a list of terms,
representing all disequality constraints for this variable.

\subsection{Tagless Logical Values and Type Bookkeeping}

The solution presented in the previous subsection suffers from the following deficiency: in order to perform unification,
we inject terms into the logic domain, making them as twice as large. As a result, this implementation loses to the original one in
terms of performance in many important applications, which compromises the very idea of using OCaml as a host language.

Here we develop an advanced version, which eliminates this penalty. As a first step, let's try to eliminate the tagging with
a drastic measure:

\begin{lstlisting}
   type $[\alpha]$ = $\alpha$
\end{lstlisting}

What consequences would this have? Of course, we would not be able to create logical variables in a conventional way. However,
we still could have a separate type of variables

\begin{lstlisting}
   type var = Var of int * anchor
\end{lstlisting}

\noindent and use \emph{the same} variable test procedure. As the type $[\alpha]$ is abstract, this modification does not change the interface.
As we reuse the variable test, polymorphic unification can continue to work \emph{almost} correctly. The problem is that
now it can introduce the occurrences of free logic variables in non-logical, tagless, data structures. These free logic variables
do not get in the way of unification itself (since it can handle them properly, thanks to the variable test), but they can not
be disclosed to the outer world as is.

Our idea is to use this generally unsound representation for all internal actions, and perform tagging only during the reification
stage. However, this scenario raises the following question: what would the type of \lstinline{reify} be? It can not be simply

\begin{lstlisting}
   val reify : state -> $[\alpha]$ -> $[\alpha]$
\end{lstlisting}

anymore since $[\alpha]$ now equals $\alpha$. We \emph{want}, however, it be something like

\begin{lstlisting}
   val reify : state -> $[\alpha]$ -> $(\mbox{``tagged'' } [\alpha])$
\end{lstlisting}

If $\alpha$ is not a parametric type, we can simply test if the value is a variable, and if yes, tag it with the constructor \lstinline{Var};
we tag it with \lstinline{Value} otherwise, and we're done. This trick, however, would not work for parametric types. Consider, for example,
the reification of a value of type \lstinline{$[[$int$]$ list$]$}. The (hypothetical) approach being described would return a value of
type \lstinline{$(\mbox{``tagged'' }[[$int$]$ list$])$}, i.e. tagged only on the top level; we need to repeat the procedure
recursively. In other words, we need the following (meta) type for the reification primitive:

\begin{lstlisting}
   val reify : state -> $[\alpha]$ -> $\mbox{(``tagged''} [\beta])$
\end{lstlisting}

\noindent where $\beta$ is the result of tagging $\alpha$.

These considerations can be boiled down to the following concrete implementation.

First, we roll back to the initial definition of $[\alpha]$~--- it will play the role of our ``tagged'' type.
We introduce a new, two-parameter type\footnote{In concrete syntax called ``$(\alpha,\;\beta)\;$\lstinline{injected}''.}

\begin{lstlisting}
   type $\{\alpha,\;\beta\}$ = $\alpha$
\end{lstlisting}

Of course, this type is kept abstract at the end-user level. Informally speaking, the type $\{\alpha,\;\beta\}$ designates the
injection of a tagless type $\alpha$ into a tagged type $\beta$; the value itself is kept in the tagless form, but
the tagged type can be used during the reify stage as a constraint, which would allow us to reify a tagless
representation only to a feasible tagged one. In other words, we record the injection steps using the second
type parameter of the type ``\{,\}'', performing the bookkeeping on the type level rather than on the value level.

We introduce the following primitives for the type $\{\alpha,\;\beta\}$:

\begin{lstlisting}
   val lift : $\alpha$ -> $\{\alpha,\;\alpha\}$
   val inj  : $\{\alpha,\;\beta\}$ -> $\{\alpha,\;[\beta]\}$

   let lift x = x
   let inj  x = x
\end{lstlisting}

The function \lstinline{lift} puts a value into the ``bookkeeping injection'' domain for the first time, while
\lstinline{inj} plays the role of the injection itself. Their composition is analogous to what was
called ``$\uparrow_\forall$'' in the previous implementation:

\begin{lstlisting}
   val $\uparrow_\forall$ : $\alpha$ -> $\{\alpha,\;[\alpha]\}$
   let $\uparrow_\forall$ x = inj (lift x)
\end{lstlisting}

In order to deal with parametric types, we can again utilize generic programming. To handle the types with
one parameter, we introduce the following functor:

\begin{lstlisting}
   module FMap (T : sig type $\alpha$ t val fmap : ($\alpha$ -> $\beta$) -> $\alpha$ t -> $\beta$ t end) :
     sig
       val distrib : $\{\alpha,\;\beta\}$ T.t -> $\{\alpha$ T.t, $\beta$ T.t$\}$
     end =
     struct
       let distrib x = x
     end
\end{lstlisting}

Note, that we do not use the function ``\lstinline{T.fmap}'' in the implementation; however, first, we need an inhabitant of the
corresponding type to make sure we are indeed dealing with a functor, and next, we actually will use it in the
implementation of type-specific reification, see below.

In order to handle two-, three-, etc. parameter types we need higher-kinded polymorphism, which is
not supported in a direct form in OCaml. So, unfortunately, we need to introduce separate
functors for the types with two-, three- etc. parameters; existing works on higher-kinded
polymorphism in OCaml~\cite{HKinded} require the similar scaffolding to be erected as a bootstrap step.

Given the functor(s) of the described shape, we can implement logic representations for
all type's constructors. For example, for standard type \lstinline{$\alpha$ option} with two constructors
\lstinline{None} and \lstinline{Some} the implementation looks like as follows:

\begin{lstlisting}
   module FOption = FMap
     (struct
        type $\alpha$ t = $\alpha$ option
        let fmap = fmap$_{\mbox{\texttt{option}}}$
      end
     )

   val some : $\{\alpha, \beta\}$ -> $\{\alpha\mbox{\texttt{ option}},\;\beta\mbox{\texttt{ option}}\}$
   val none : unit  -> $\{\alpha\mbox{\texttt{ option}},\;\beta\mbox{\texttt{ option}}\}$

   let some x  = inj (FOption.distrib (Some x))
   let none () = inj (FOption.distrib None)
\end{lstlisting}

In other words, we can in a very systematic manner define logic representations for all constructors
of types of interest. These representations can be used in the relational code, providing a well-bookkept
typing~--- for each logical type we would be able to reconstruct its original, tagless preimage.

With the new implementation, the types of basic goal constructors have to be adjusted:

\begin{lstlisting}
   val (===) : $\{\alpha,\;[\beta]\}$ -> $\{\alpha,\;[\beta]\}$ -> goal
   val (=/=) : $\{\alpha,\;[\beta]\}$ -> $\{\alpha,\;[\beta]\}$ -> goal
\end{lstlisting}

As always, we require both arguments of unification and disequality constraint to be of the same type; in addition
we require the injected part of the type to be logical.

During the reification stage the bindings for the top-level variables, reconstructed using the final
substitution, have to be properly tagged. This process is implemented in a datatype-generic manner as well:
first, we have reifiers for all primitive types:

\begin{lstlisting}
   val reify$_{\mbox{\texttt{int}}}$ : helper -> $\{$int,$[$int$]\}$ -> $[$int$]$
   val reify$_{\mbox{\texttt{string}}}$ : helper -> $\{$string,$[$string$]\}$ -> $[$string$]$
   ...
\end{lstlisting}

and, then, we add the reifier to the output signature in all \lstinline{FMap}-like functors:

\begin{lstlisting}
   val reify: (helper -> $\{\alpha,\;\beta\}$ -> $\beta$) -> helper -> $\{\alpha$ T.t, $[\beta$ T.t$]$ as $\gamma\}$ -> $\gamma$
\end{lstlisting}

Note, since now \lstinline{reify} is a type-specific and, hence, constructed at the user-level, we refrain from passing
it a state (which is inaccessible on the user level). Instead, we wrap all state-specific functionality in
an abstract \lstinline{helper} data type, which encapsulates all state-dependent functionality needed for \lstinline{reify}
to work properly.
