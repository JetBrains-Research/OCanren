\section{Logic Variables and Injection}
\label{sec:injection}

Unification, considered in Section~\ref{sec:unification}, works for values of any types. However, it
is too generic to be used directly. As long as we use it to unify closed terms, it's ok (but rather meaningless); 
however, for terms with free variables the situation becomes more complicated. Indeed, as we've seen in the
previous section, a variable $x^\tau$ must have a runtime representation of some value of type $\tau$. As different
types may not have any values in common, if we wish to unify arbitrary types, we would need to specify the 
ground type for each logical variable explicitly, which would overthrow our implementation into a non-polymorphic 
realm with no type inference.

In this part we consider two approaches to safely handling typed terms, while preserving polymorphism and
type inference. The first is rather easy to develop and implement; unfortunately, the implementation demonstrates 
poor performance for a number of important applications. In order to fix this defficiency, we develop more 
elaborated technique, which nevertheless reuses some components from the previous one. In Section~\ref{sec:evaluation}
we present the results of performance evaluation for both approaches and compare them to that of original 
implementation.

\subsection{Tagged Logical Values}

The first natural solution is to restrict the unification to logic values only. We introduce the following polymorphic
type $[\alpha]$\footnote{In concrete syntax it is $\alpha\;\;$\lstinline{logic}}, which corresponds to a type, injected into
logic domain:

\begin{lstlisting}
   type $[\alpha]$ = Var of int | Value of $\alpha$
\end{lstlisting}

Note, this type cannot be disclosed to a end-user, since the only possible way to create a logic variable should 
still be by using the ``\lstinline{fresh}'' construct.

Informally speaking, any value of type $[\alpha]$ is either a value of type $\alpha$, or a free
logic variable. Now, we may redefine the signature of abstraction, unification and disequality primitives in the
following manner

\begin{lstlisting}
   val call_fresh : ($[\alpha]$ -> goal) -> goal

   val (===)      : $[\alpha]$ -> $[\alpha]$ -> goal
   val (=/=)      : $[\alpha]$ -> $[\alpha]$ -> goal
\end{lstlisting}

Both unification and disequality constraint would still use the same polymorphic unification; their external, visible type,
however, is restricted to logical types only.

Apart from variables, other logical values can be obtained by injection; conversely, sometimes logical value can be projected to 
a regular one. We supply two basic functions\footnote{``In concrete syntax the are \lstinline{inj}'' and ``\lstinline{prj}''.}
for these purposes

\begin{lstlisting}[mathescape=true]
   val ($\uparrow_\forall$) : $\alpha$ -> $[\alpha]$ 
   val ($\downarrow_\forall$) : $[\alpha]$ -> $\alpha$

   let ($\uparrow_\forall$) x = Value x
   let ($\downarrow_\forall$) = function Value x -> x | _ -> failwith $\mbox{``not a value''}$
\end{lstlisting}

which can be used to perform a \emph{shallow} injection/projection. As expected, the injection is total, while the projection is partial. 
Using these functions and type-specific functors, which can be derived automatically using a number of existing frameworks for 
generic programming, one can easily provide injection and projection for user-defined datatypes.

We consider user-defined list type as an example:

\begin{lstlisting}[mathescape=true]
   type ($\alpha$, $\beta$) list$_f$ = Nil | Cons of $\alpha$ * $\beta$
   
   type $\alpha$ list = ($\alpha$, $\alpha$ list) list$_f$
   type $\alpha$ list$_l$ = $[$($[\alpha]$, $\alpha$ list$_l$) list$]$

   let rec $\uparrow_{\mbox{\texttt{list}}}$ l = $\uparrow$($fmap_{\mbox{\texttt{list}}_f}$ ($\uparrow_\forall$) $\uparrow_{\mbox{\texttt{list}}}$ l) 
   let rec $\downarrow_{\mbox{\texttt{list}}}$ l = $fmap_{\mbox{\texttt{list}}_f}$ ($\downarrow_\forall$) $\downarrow_{\mbox{\texttt{list}}}$ ($\downarrow_\forall$ l)
\end{lstlisting}

Here ``\lstinline{list$_f$}'' is a custom functor for lists; note, that it is made more polymorphic, than usual~--- we abstracted it from itself 
and made it non-recursive (pragmatically speaking, it is desirable to make a type fully abstract, thus logic variables can be placed 
in arbitrary positions).

Then we provided two specialized versions~--- ``\lstinline{list}'', which corresponds to regular, non-logic lists, and ``\lstinline{list}$_l$'',
 which corresponds to logical lists with logical elements. Using a single type-specific function \lstinline{$fmap_{\mbox{\texttt{list}}_f}$}, we could easily 
provide both injection of type 

\begin{lstlisting}
   $\alpha$ list -> $\alpha$ list$_l$
\end{lstlisting} 

and projection of type 

\begin{lstlisting}
   $\alpha$ list$_l$ -> $\alpha$ list
\end{lstlisting}

Generally speaking, we can always implement injection/projection for arbitrary regular type, represented as a fixpoint of a functor~\cite{ALaCarte}, 
using only $\uparrow_\forall$, $\downarrow_\forall$ and a small set of datatype-generic combinators.

We now have to address the problem of variable identification in polymorphic unification. As we do not know the types, we cannot discriminate logical 
variables by their tags only and, thus, cannot simply use pattern matching. In our implementation we perform variable test 
as follows:

\begin{itemize}
\item in environment, we additionally keep some unique boxed value~--- \emph{anchor}~--- created by \lstinline{run} at the moment of initial
state generation; the anchor is inherited unchanged in all derived environments during the search session;
\item in each variable we additionally save the anchor, inherited from the environment, in which the variable was created;
\item inside a unification, in order to check, if we deal with a variable, we test the conjuction of the following properties:

  \begin{enumerate}
    \item the scrutenee is boxed; 
    \item the scrutenee's tag and layout correspond to those for variables;
    \item the scrutenee's anchor and current environment anchor addresses are equal.
  \end{enumerate}
\end{itemize}

Taking into account, that the state type is abstract at the user level, we guarantee, that only those variables, which were
created during current run session would pass the test, since the pointer to anchor is unique among all pointers to a boxed values 
and could not be disclosed elsewhere but in the variable creation primitive.

The only thing to describe now is the implementation of the refinement stage. The refinement is represented by the following 
function:

\begin{lstlisting}
   val refine : subst -> $[\alpha]$ -> $[\alpha]$ 
\end{lstlisting}

This function takes a substitution and a logical value and recursively substitutes all logical variables in that value w.r.t. 
the substitution until no occurrences of bound variables left. Since in our implementation the type of substitution is
not polymorphic, \lstinline{refine} is also implemented in an unsafe manner. However, it is easy to see, that \lstinline{refine} 
does not produce ill-typed terms. Indeed, all original types of variables are preserved in a substitution; unification does not 
change unified terms, so all terms, bound in a substitution, are well-typed. Hence, \lstinline{refine} always substitutes
some subterm in a well-typed term with another term of the same type, which preserves well-typedness.

In addition to performing substitutions, \lstinline{refine} also reifies disequality constrains. Reification 
attaches to each free variable in a refined term a list of \emph{refined} terms, describing the disequality constraints for that
free variable. Note, disequality can be established only for equally typed terms, which justifies type-safety of reification. 
Note also, additional care has to be taken to avoid infinite looping, since refinement and reification are
mutually recursive, and refinement of a variable can be potentially invoked from itself due to a chain of disequality 
constraints.

After refinement, the content of a logical value can be inspected via the following function:

\begin{lstlisting}
   val destruct : $[\alpha]$ -> [`Var of int * $[\alpha]$ list | `Value of $\alpha$]
\end{lstlisting}

Constructor \lstinline{`Var} corresponds to a free variable with unique integer identifier and a list of terms, 
representing all disequality constraints for this variable. These terms are refined as well.

\subsection{Untagged Logical Values and Type Bookkeeping}

% Возможно везде надо две k но я не уверен. Слова с одним k я в инете не нашел

The solution, presented in the previous subsection, suffers from the following defficiency: in order to perform unification,
we inject terms into logic domain, making them as twice as large. As a result, this implementation loses the original one in 
terms of performance in many important applications, which compromises the very idea of using OCaml as a host language.

Here we develop an advanced version, which eliminates this penalty. As a first step, let's try to eliminate the tagging with
a drastic measure:

\begin{lstlisting}
   type $[\alpha]$ = $\alpha$
\end{lstlisting}

What consequences would this have? Of course, we would not be able to create logical variables in a conventional way. However, 
we still could have a separate type for variables

\begin{lstlisting}
   type var = Var of int * anchor
\end{lstlisting}

and use \emph{the same} variable test procedure. As the type $[\alpha]$ is abstract, this modification does not change the interface. 
As we reuse the variable test, polymorphic unification can continue work \emph{almost} correctly. The problem is that
now it can introduce the occurrences of free logic variables in non-logical, untagged, data structures. These free logic variables 
do not get in the way of unification itself (since it can handle them properly, thanks to the variable test), but they can not
be disclosed to the outer world as is.

Our idea is to use this generally unsound representation for all internal actions, and perform tagging only during the refinement
stage. However, this scenario raises the following question: what whould the type of \lstinline{refine} be? It can not be simply

\begin{lstlisting}
   val refine : subst -> $[\alpha]$ -> $[\alpha]$
\end{lstlisting}

anymore since $[\alpha]$ now equals $\alpha$. We \emph{want}, however, it be something like

\begin{lstlisting}
   val refine : subst -> $[\alpha]$ -> $(\mbox{``tagged'' }[\alpha])$
\end{lstlisting}

If $\alpha$ is not a parametric type, we can simply test, if the value is a variable, and if yes, tag it with constructor \lstinline{Var};
otherwise we tag it with \lstinline{Value}, and we're done. This trick, however, would not work for parametric types. Consider, for example, 
the refinement of a value of type \lstinline{$[[$int$]$ list$]$}. The (hypotetical) approach being described would return a value of
type \lstinline{$(\mbox{``tagged'' }[[$int$]$ list$])$}, i.e. tagged only on the top level; we need to repeat the procedure
recursively. In other words, we need the following (meta) type for the refinement primitive:

\begin{lstlisting}
   val refine : subst -> $[\alpha]$ -> $\mbox{(``tagged''} [\beta])$
\end{lstlisting}

where $\beta$ is the result of tagging $\alpha$.

These considerations can be boiled down to the following concrete implementation. 

First, we roll back to the initial definition of $[\alpha]$~--- it will play role of our ``tagged'' type.
We introduce a new, two-parametric type\footnote{In concrete syntax it is $(\alpha,\;\beta)\;$\lstinline{injected}.}

\begin{lstlisting}
   type $\{\alpha,\;\beta\}$ = $\alpha$
\end{lstlisting}

Of course, this type is being kept abstract at the end-user level. Informally speaking, type $\{\alpha,\;\beta\}$ designates the
injection of untagged type $\alpha$ into tagged type $\beta$; the value itself is kept in the untagged form, but
tagged type can be used during the refinement stage as a constraint, which would allow to refine untagged
representation only in a feasible tagged one.

We introduce the following primivites for the type $\{\alpha,\;\beta\}$:

\begin{lstlisting}
   val lift : $\alpha$ -> $\{\alpha,\;\alpha\}$
   val inj  : $\{\alpha,\;\beta\}$ -> $\{\alpha,\;[\beta]\}$

   let lift x = x
   let inj  x = x
\end{lstlisting}

The function \lstinline{lift} puts a value into the ``bookeeping injection'' domain for the first time, while
\lstinline{inj} plays the role of bookeeping injection itself. Their composition is analogous to what was 
called ``$\uparrow_\forall$'' in the previous implementation:

\begin{lstlisting}
   val $\uparrow_\forall$ : $\alpha$ -> $\{\alpha,\;[\alpha]\}$
   let $\uparrow_\forall$ x = lift x |> inj
\end{lstlisting}

In order to deal with parametric types, we can again utilize generic programming. To handle the types with
one parameter, we introduce the following functor:

\begin{lstlisting}
   module FMap 
     (T : sig 
            type $\alpha$ t 
            val fmap : ($\alpha$ -> $\beta$) -> $\alpha$ t -> $\beta$ t 
          end
     )  : sig
            val distrib : $\{\alpha,\;\beta\}$ t -> $\{\alpha$ t, $\beta$ t$\}$
          end =
     struct
       let distrib x = x
     end
\end{lstlisting}

Note, that we do not use function ``\lstinline{T.fmap}'' in the implementation; we, however, need an inhabitant of
corresponding type to make sure we indeed deal with a functor.

In order to handle two-, three-, etc. parametric types we need higher-kinded polymorphism, which is
not supported in a direct form in OCaml. So, unfortunately, we need to introduce a separate
functors for the types with two-, three- etc. parameters; there some works on higher-kinded
polymorphism in OCaml~\cite{HKinded}, which still require the similar construction to be
erected as a bootstrap step.

Given the functor(s) of described shape, we can use it in the following manner:

\begin{lstlisting}
   module FOption = FMap 
     (struct 
        type $\alpha$ t = $\alpha$ option 
        let fmap = $fmap_{\mbox{\texttt{option}}}$ 
      end
     )

   val some : $\{\alpha, \beta\}$ -> $\{\alpha\mbox{\texttt{ option}},\;\beta\mbox{\texttt{ option}}\}$
   val none : unit  -> $\{\alpha\mbox{\texttt{ option}},\;\beta\mbox{\texttt{ option}}\}$
   
   let some x  = FOption.distrib (Some x) |> inj
   let none () = FOption.distrib None     |> inj
\end{lstlisting}

In other words, we can in a very systematic manner define \emph{logic representatives} for all constructors
of types of interest. These representatives can be used in the relational code, providing a well-bookept
typing~--- for each logical type we would be able to reconstruct its original, untagged preimage. However,
the representation of bookept types is untagged; their values are tagged only during the refinement
stage, which we can now properly type:

\begin{lstlisting}
   val refine : $\{\alpha,\;\beta\}$ -> $\beta$
\end{lstlisting}

We can implement refinement in a generic manner, adding extra functionality to the corresponding functor; we
omit the details here.

With the new implementation, the types for ``basic'' goal constructors have to be adjusted:

\begin{lstlisting}
   val (===) : $\{\alpha,\;[\beta]\}$ -> $\{\alpha,\;[\beta]\}$ -> goal
   val (=/=) : $\{\alpha,\;[\beta]\}$ -> $\{\alpha,\;[\beta]\}$ -> goal
\end{lstlisting}

As always, we require both arguments of unification and disequalify constraint to be of the same type; in addition
we require the ``injected'' part of the type to be logic.

Note, we implemented bookeeping in a completely type-safe manner, not making use of any unsafe features; in addition
we preserved polymorphism and type inference. Consider an example:

\begin{lstlisting}
   fresh (x) (x === some ($\uparrow_\forall$ 5))
\end{lstlisting}

Here the type of the term at the right hand side of unification goal is 

\begin{lstlisting}
   $\{\mbox{\texttt{int option}},\;[[\mbox{\texttt{int}}]\mbox{ \texttt{option}}]\}$
\end{lstlisting}

Hence the inferred type for the variable $x$ is the same.

