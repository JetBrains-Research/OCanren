\section{Introduction}
\label{intro}

Relational programming~\cite{TRS} is an attractive technique, based on the idea 
of constructing programs as relations.  As a result, relational programs can be
``queried'' in various ``directions'', making it possible, for example, to simulate
reversed execution. Apart from being interesting from purely theoretical standpoint, 
this approach may have a practical value: some problems look much simpler, 
if they are considered as queries to some relational specification. There is a 
number of appealing examples, confirming this observation: a type checker 
for simply typed lambda calculus (and, at the same time, type inferencer and solver 
for the inhabitation problem), an interpreter (capable of generating ``quines''~--- 
programs, producing themselves as a result)~\cite{Untagged}, list sorting (capable of 
producing all permutations), etc. 

Many logic programming languages, such as Prolog, Mercury\footnote{\url{https://mercurylang.org}}, 
or Curry\footnote{\url{http://www-ps.informatik.uni-kiel.de/currywiki}} to some extent
can be considered as relational. We have chosen miniKanren\footnote{\url{http://minikanren.org}} 
as model language, because it was specifically designed as relational DSL, embedded in Scheme/Racket. 
Being rather a minimalistic language, which can be implemented with just a few data structures and
combinators, miniKanren found its way in dozens of host languages, including Haskell, 
Standard ML, and OCaml.

TODO: motivation. Why OCaml?

There is, however, a predictable glitch in implementing miniKanren for a strongly typed language. 
Designed in a metaprogramming-friendly and dynamically typed realm of Scheme/Racket, original 
miniKanren implementation pays very little attention to what has a significant importance in (specifically) 
ML or Haskell. In particular, one of the capstone constructs of miniKanren~--- unification~--- has to work for 
different data structures, which may have types, different beyond parametricity.

There are a few ways to overcome this problem. The first one is simply to follow the untyped paradigm and
provide unification for some concrete type, rich enough to represent any reasonable data structures.
Some Haskell miniKanren libraries\footnote{\url{https://github.com/JaimieMurdock/HK}, \url{https://github.com/rntz/ukanren}}
as well as previous OCaml implementation\footnote{\url{https://github.com/lightyang/minikanren-ocaml}} take this approach. 
As a result, the original implementation can be retold with all its elegance; the relational specifications, however,
become weakly typed. Another approach is to utilize \emph{ad hoc} polymorphism and provide type-specific
unification for each ``interesting'' type; Molog\footnote{\url{https://github.com/acfoltzer/Molog}} and 
MiniKanrenT\footnote{\url{https://github.com/jvranish/MiniKanrenT}}, both for Haskell, can be mentioned as examples.
While preserving strong typing, this approach requires a lot of ``boilerplate'' code to be written, so some
automation, for example, using Template Haskell\footnote{\url{https://wiki.haskell.org/Template_Haskell}},
is desirable. There is, actually, another potential approach, but we do not know, if anybody tried
it: to implement unification for generic representation of types as sum-of-products and fixpoints of 
functors~\cite{InstantGenerics, ALaCarte}. Thus, unification would work for any types, for which representation
is provided. We assume that implementing representation would require less boilerplate code to be written.

As follows from this exposition, typed embedding of miniKanren in OCaml can be done with
a combination of datatype-generic programming~\cite{DGP} and \emph{ad hoc} polymorphism. There are a 
number of generic frameworks for OCaml (for example,~\cite{Deriving}). On the other hand, the support
for \emph{ad hoc} polymorphism in OCaml is weak; there is nothing comparable in power with Haskell 
type classes, and despite sometimes object-oriented layer of the language can be used to mimic
desirable behavior, the result as a rule is far from satisfactory. Existing proposals (for example, 
module implicits~\cite{Implicits}) require patching the compiler, which we tend to avoid.

We present an implementation of a set of relational combinators in OCaml, which, technically speaking, 
corresponds to $\mu$Kanren~\cite{MicroKanren} with disequality constraints~\cite{CKanren}; syntax extension 
for the ``\lstinline{fresh}'' construct is added as well. The contribution of our work is as follows:

\begin{enumerate}
\item Our implementation is based on \emph{polymorphic unification}, which, like polymorphic comparison,
can be used for almost arbitrary types. The implementation of polymorphic unification uses unsafe features and
relies on intrinsic knowledge of runtime representation of values; we show, however, that this does not
compromise type safety. Practically, we applied purely \emph{ad hoc} approach since the features, 
which would provide less \emph{ad hoc} solution, are not yet integrated into the mainstream language.

\item We describe a uniform and scalable pattern for using types for relational programming, which
helps in converting typed data to- and from relational domain. With this pattern, only one
generic feature (``\lstinline{map/morphism/Functor}'') is needed, and thus virtually any generic 
framework for OCaml can be used. Despite being rather a pragmatic observation, this pattern, as we
believe, would lead to  more regular and easy to maintain relational specifications.

\item We provide a simplified way to integrate relational and functional code. Our approach utilizes
well-known pattern~\cite{Unparsing, DoWeNeed} for variadic function implementation and makes it
possible to hide the refinement of the answers phase from an end-user.
\end{enumerate}

The rest of the paper is organized as follows: in the next section \ref{sec:goals} we establish some basic
notes about miniKanren. In the section \ref{sec:unification} we discuss polymorphic
unification, and show, that standard unification with triangular substitution respects
typing. Then we present our approach to handle user-defined types by injecting them 
into logic domain \ref{sec:injection}. The section \ref{refinement} describes top-level primitives and addresses the problem of
relational and functional code integration. Then, we present a complete example \ref{example} of relational
specification, written with the aid of our library. The final section concludes.

We expect from reader some familiarity with basic concepts behind original miniKanren 
implementation as well as principles of relational programming.

