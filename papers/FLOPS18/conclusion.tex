\section{Conclusion}
\label{conclusion}

We presented an improvement of a search strategy for relational programming, which is aimed at
improving refutationally completeness. We've proven, that in the case of a finite number of 
answers our modification is a strict improvement over the original strategy. Our evaluation 
shows, that w.r.t. the improved search many practically important refutationally incomplete 
queries became refutationally complete; in addition in a number of cases the performance was greatly 
improved since our modification, as a side effect, causes the search to choose more 
``optimistic'' branches. 

We can identify the following directions for future work. 

First, we believe, that our result on refutational improvement for a finite number of answers 
can be extended to the general case as well (note, in our current development we did not make 
any use of the \emph{completeness} property of miniKanren search). For this, we would also need 
another, more general, semantics. 

Another direction is extending the language with disequality constraints. Our evaluation has 
shown, that disequality constraints do not compromise our improvement in all user benchmarks, 
but we do not have a proof, that they are indeed harmless.

Next, we are working on a certified proof of the main theorem in Coq.

Finally, our practical evaluation is performed only for a prototype. 
We consider the embedding of our improvement in a full-fledged implementation to be
an important task.