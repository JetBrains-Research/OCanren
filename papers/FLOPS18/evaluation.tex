\section{Implementation and Evaluation}
\label{evaluation}

We implemented the improved version of the search, described in the previous section, 
as a prototype over existing miniKanren implementation for OCaml, called OCanren~\cite{OCanren}. 
For the simplicity, we switched to a deep embedding (most miniKanren implementations, 
including OCanren, use shallow embedding), and changed the syntax to make relational definitions
visible for the interpreter. Programs in OCanren can be easily converted to respect 
the new syntax.

With this implementation, we became capable to evaluate a number of useful, but refutationally 
incomplete miniKanren relations, including those considered earlier. In all cases the 
improved search managed to discover the divergence and fix the incompleteness, caused by
the non-commutativity of conjuction; moreover, the performance of some queries was improved 
significantly. In the following subsections we consider some interesting cases.

\subsection{Relations on lists}

As we have seen in Section~\ref{incompleteness}, for some simple relations, like \lstinline|append$^o$|,
a recursive call has to be placed last in the sequence of conjuncts in order for the 
relation to be refutationally complete. Specifically for \lstinline|append$^o$|, with the improved search 
the divergence is discovered and refutational incompleteness is fixed regardless the position of 
the recursive call.

As a more interesting example we consider the \lstinline|revers$^o$| relation. As we've
seen, in order for different queries to work it requires different orders of conjuncts 
to be used in the implementation. Again, the improved search lifts this requirement.
Even more interesting, that, for example, in the query 

\begin{lstlisting}
   fresh ($q$) (revers$^o$ (Cons (A, Nil)) $q$)
\end{lstlisting}

\noindent the divergence is discovered for the recursive call of \lstinline|appendo$^o$|, but,
as none of conjunct orders within the definition of \lstinline|append$^o$| help, the
improved search goes even more back and switches the conjuncts within the definition of
\lstinline|revers$^o$|. This example demonstrates the importance of operating on 
dynamic invocation order.

Another example we've already looked at is relational sorting/permutations. 
With the improved search, both queries \lstinline|fresh ($q$) (perm$^o$ l $q$)| and 
\lstinline|fresh ($q$) (perm$^o$ $q$ l)| terminate for any list \lstinline|l| of a reasonable 
length (7, 8, etc.), and now work in a reasonable time. Moreover, now
relational permutations can be used as a (we admit, somewhat exotic) way to calculate 
the number of permutations with repetitions.

\subsection{Binary arithmetics}

The implementation of a relational arithmetics is an important problem, since it is required
in a number of elaborated relational specifications. For performance reasons, it is 
preferrable to use binary numbers, not comfy Peano encoding, which makes the problem 
more complicated. As it is shown in~\cite{WillThesis}, the naive implementation of 
binary arithmetics is inapproptiate due to multiple problems. 

Fixing these problems take a lot of efforts: it requires some additional conditions, 
excess on the first glance, to be introduced in the specification to ensure the non-overlapping 
of cases and correctness of number representation. And still, even with all these 
improvements, arithmetic relations diverge for some routine queries with one order of 
consituents, and for others with another. To overcome this problem, 
arithmetics in the standard miniKanren implementation~\cite{TRS} uses digital logic and 
some advanced techniques of bounding the sizes of the terms~\cite{KiselyovArithmetic}. 
As a result, the implementation is quite complicated and takes time to understand.

At the same time, all these problems are exactly the consequences of the non-commutativity 
of conjunction. So, as expected, the improved search makes it possible to stick with
the naive version (for addition, multiplication, comparisons, division with a reminder)
without all these complicated optimisations. 

As the brightest example, for the division with reminder, instead of a very complicated recursive definition 
from~\cite{TRS} (20 lines of code, not including auxiliary functions), one can just right down the 
following definition

\begin{lstlisting}
   div$^o$ $\binds$ $\lambda\;x\;y\;q\;r$ . 
     (fresh (yq)        
        (mult$^o$ $y$ $q$ $yq$) /\
        (plus$^o$ $yq$ $r$ $x$) /\
        (lt$^o$ $r$ $y$)
     )
\end{lstlisting}

\noindent and for all queries

\begin{lstlisting}
   fresh ($q$ $r$) (div$^o$ $\overline{23}$ $\overline{5}$ $q$ $r$)
   fresh ($y$ $q$ $r$) (div$^o$ $\overline{19}$ $y$ $q$ $r$)
   fresh ($x$ $r$) (div$^o$ $x$ $\overline{17}$ $\overline{4}$ $r$)
\end{lstlisting}

\noindent the search terminates and shows the preformance, comparable with the advanced version 
(here $\overline{n}$ denotes a binary representation of the number $n$).

\subsection{Binary trees}

For a natural representation of binary trees using two constructors \lstinline|Leaf| and 
\lstinline|Node($l$, $r$)| it is easy to implement the relation to count the number of 
leaves in a tree (using arithmetic relation \lstinline|plus$^o$| for addition):

\begin{lstlisting}
   siz$^o$ $\binds$ $\lambda\;t\;s$ .
     (($t$ === $\;\;$Leaf) /\ ($s$ === $\;\;$$\overline{1}$)) \/
     (fresh ($l$ $r$ $sl$ $sr$)
        ($t$ === $\;$Node ($l$, $r$)) /\
        (siz$^o$ $l$ $sl$) /\
        (siz$^o$ $r$ $sr$) /\
        (plus$^o$ $sl$ $sr$ $s$)
     )
\end{lstlisting}

By running this relation backwards

\begin{lstlisting}
   fresh ($q$) (siz$^o$ $q$ $\overline{n}$)
\end{lstlisting}

\noindent it becomes possible to generate all binary trees with given number of leaves $n$. 
The improved search provides the termination of this query, and the number of discovered 
answers corresponds to the number of such trees, so this relational program may be seen 
as (an exotic) way of calculating the Catalan numbers.

\subsection{Interpreters}

Program synthesis with relational interpreters is one of the most useful applications of miniKanren. 
The construction of a relational interpreter for a small Scheme-like language is considered 
in details in~\cite{Untagged}. However, this simple interpreter also has some problems, caused by 
the non-commutativity. For example, the following obviously contradictory query

\begin{lstlisting}
   fresh ($e1$ $e2$ $r1$ $r2$) (eval$^o$ (list $e_1$ 3 $e_2$) Nil ($r_1$ 4 $r_2$))
\end{lstlisting}

\noindent leads to a divergence under the coventional search (and no simple reordering can fix it). 
Under the improved search, however, the conradiction is found and  the query terminates with
no answers. Relational interpreters, used in practice for more complex problems~\cite{unified}, 
include a lot of optimisations and take significant effort to implement. We hope, that some 
performance problems with them are caused by the non-commutativity and can be fixed automatically 
with the improved search.
