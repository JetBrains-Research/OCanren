\documentclass{llncs}

\usepackage{makeidx}
\usepackage{amssymb}
\usepackage{mathtools}
\usepackage{multirow}
\usepackage{listings}
\usepackage{indentfirst}
\usepackage{verbatim}
\usepackage{amsmath, amssymb}
\usepackage{graphicx}
\usepackage{xcolor}
\usepackage{url}
\usepackage{stmaryrd}
\usepackage{xspace}
\usepackage{comment}
\usepackage{wrapfig}
\usepackage[caption=false]{subfig}
%\usepackage{subcaption}
\usepackage{placeins}
\usepackage{tabularx}
\usepackage{ragged2e}

\def\transarrow{\xrightarrow}
\newcommand{\setarrow}[1]{\def\transarrow{#1}}

\newcommand{\trule}[2]{\frac{#1}{#2}}
\newcommand{\crule}[3]{\frac{#1}{#2},\;{#3}}
\newcommand{\withenv}[2]{{#1}\vdash{#2}}
\newcommand{\trans}[3]{{#1}\transarrow{#2}{#3}}
\newcommand{\ctrans}[4]{{#1}\transarrow{#2}{#3},\;{#4}}
\newcommand{\llang}[1]{\mbox{\lstinline[mathescape]|#1|}}
\newcommand{\pair}[2]{\inbr{{#1}\mid{#2}}}
\newcommand{\inbr}[1]{\left<{#1}\right>}
\newcommand{\highlight}[1]{\color{red}{#1}}
\newcommand{\ruleno}[1]{\eqno[\scriptsize\textsc{#1}]}
\newcommand{\rulename}[1]{\textsc{#1}}
\newcommand{\inmath}[1]{\mbox{$#1$}}
\newcommand{\lfp}[1]{fix_{#1}}
\newcommand{\gfp}[1]{Fix_{#1}}
\newcommand{\vsep}{\vspace{-2mm}}
\newcommand{\supp}[1]{\scriptsize{#1}}
\newcommand{\G}{\mathfrak G}
\newcommand{\sembr}[1]{\llbracket{#1}\rrbracket}
\newcommand{\cd}[1]{\texttt{#1}}
\newcommand{\miniKanren}{miniKanren\xspace}
\newcommand{\ocanren}{OCanren\xspace}
\newcommand{\free}[1]{\boxed{#1}}
\newcommand{\binds}{\;\mapsto\;}
\newcommand{\dbi}[1]{\mbox{\bf{#1}}}
\newcommand{\sv}[1]{\mbox{\textbf{#1}}}
\newcommand{\bnd}[2]{{#1}\mkern-9mu\binds\mkern-9mu{#2}}

\newcommand{\meta}[1]{{\cal{#1}}}

\lstdefinelanguage{ocanren}{
keywords={fresh, let, in, match, with, when, class, type,
object, method, of, rec, repeat, until, while, not, do, done, as, val, inherit,
new, module, sig, deriving, datatype, struct, if, then, else, open, private, virtual, include, success, failure,
true, false},
sensitive=true,
commentstyle=\small\itshape\ttfamily,
keywordstyle=\ttfamily\underbar,
identifierstyle=\ttfamily,
basewidth={0.5em,0.5em},
columns=fixed,
fontadjust=true,
literate={fun}{{$\lambda$}}1 {->}{{$\to$}}3 {===}{{$\equiv$}}1 {=/=}{{$\not\equiv$}}1 {|>}{{$\triangleright$}}3 {\\/}{{$\vee$}}2 {/\\}{{$\wedge$}}2 {^}{{$\uparrow$}}1,
morecomment=[s]{(*}{*)}
}

\lstset{
mathescape=true,
%basicstyle=\small,
identifierstyle=\ttfamily,
keywordstyle=\bfseries,
commentstyle=\scriptsize\rmfamily,
basewidth={0.5em,0.5em},
fontadjust=true,
language=ocanren
}

\usepackage{letltxmacro}
\newcommand*{\SavedLstInline}{}
\LetLtxMacro\SavedLstInline\lstinline
\DeclareRobustCommand*{\lstinline}{%
  \ifmmode
    \let\SavedBGroup\bgroup
    \def\bgroup{%
      \let\bgroup\SavedBGroup
      \hbox\bgroup
    }%
  \fi
  \SavedLstInline
}
\addtolength{\parskip}{-2pt}

\begin{document}

\mainmatter

\title{Improving Refutational Completeness\\
of Relational Search via Divergence Test}

\author{
  Dmitri Rozplokhas\inst{1} \and Dmitri Boulytchev\inst{2}
}

\institute{
\email{rozplokhas@gmail.com}\\
St.Petersburg Academic University
\and
\email{dboulytchev@math.spbu.ru}\\
St.Petersburg State University\\
JetBrains Research
}

\maketitle

\begin{abstract}
Abstract goes here
\end{abstract}

\section{Introduction}
\label{sec:intro}

Algebraic data types (ADT) is an important tool in functional programming which delivers a way to represent flexible and easy to manipulate data structures.
To inspect the contents of ADT values a generic construct~--- \emph{pattern matching}~--- is used. Pattern matching can be considered as a generalization of
conventional conditional control-flow construct ``\lstinline|if .. then .. else|'' and in principle can be decomposed into a nested hierarchy of those; from
this standpoint the problem of pattern matching implementation can be considered trivial. However, some decompositions are obviously better than others. We
repeat here an example from~\cite{maranget2008} to demonstrate this difference (see Fig.~\ref{fig:match-example}). Here we match a triple of boolean
values $x$, $y$, and $z$ against four pattern (Fig.~\ref{fig:matching-example1}; we use \textsc{OCaml}~\cite{ocaml} as reference language). The na\"{i}ve
implementation of this example is shown on Fig.~\ref{fig:matching-example2}; however if we decide to match $y$ first the result becomes much
better (Fig.~\ref{fig:matching-example3}).

\begin{figure}[ht]
\begin{subfigure}[t]{0.2\linewidth}
\centering
\begin{lstlisting}
match x, y, z with
| _, F, T -> 1
| F, T, _ -> 2
| _, _, F -> 3
| _, _, T -> 4
\end{lstlisting}
\vskip18.5mm
\caption{Pattern matching}
\label{fig:matching-example1}
\end{subfigure}
\hspace{0.5cm}
\begin{subfigure}[t]{0.26\linewidth}
\centering
\begin{lstlisting}
if x then
  if y then
    if z then 4 else 3
  else
    if z then 1 else 3
else
  if y then 2
  else
    if z then 1 else 3
\end{lstlisting}
\caption{A correct but non-optimal\\\phantom{(b)~}implementation}
\label{fig:matching-example2}
\end{subfigure}
\hspace{0.5cm}
\begin{subfigure}[t]{0.33\linewidth}
\centering
\begin{lstlisting}
if y then
  if x then
    if z then 4 else 3
  else 2
else
  if z then 1 else 3
\end{lstlisting}
\vskip13.5mm
\caption{Optimal implementation}
\label{fig:matching-example3}
\end{subfigure}
\caption{Pattern matching implementation example} 
\label{fig:match-example}
\end{figure}

\begin{comment}
\begin{figure}[ht]
\begin{minipage}[b]{0.3\linewidth}
\centering
\label{fig:figure1}
\end{minipage}
\hspace{0.5cm}
\begin{minipage}[b]{0.3\linewidth}
\centering
\begin{lstlisting}
switch x with 
| true -> 
    switch y with 
    | true -> 
       switch z with 
       | true -> 4
       | _ -> 3
    | _ -> 
      switch z with 
      | true -> 1
      | _ -> 3 
| _ -> 
   switch y with 
   | true -> 2 
   | _ -> if z then 1 else 3
\end{lstlisting}
\end{minipage}
\hspace{0.5cm}
\begin{minipage}[b]{0.3\linewidth}
\centering
\end{minipage}
\end{figure}
\end{comment}


%clasification 1
Although semantics of pattern matching can be given as a sequence of srutinee's sub expression comparisons (Figure~\ref{fig:matchpatts}) effective compilers don't follow
this approach. One can either optimise runtime cost by minimizing amount of checks performed or static cost by minimizing the size of generated code. \emph{Decision trees}~\cite{?}
are good for the first criteria, because they check every subexpression not more than once. \emph{Backtracking automata} are rather compact but in some cases can perform
repeated checks.

%clasification 2
\emph{For strict languages} checking sub-expressions of scrutinee in any order is allowed. \emph{For lazy languages} pattern matching should evaluate only those sub-expressions which are
necessary for performing pattern matching. If not careful pattern matching can change the termination behavior of the program. In general lazy languages setup more constraints on pattern matching and because of that allow lesser set of heuristics. Decision trees and backtracking automata can be used for compilation both  strict and lazy languages.

%clasification 3
The matching compilers for strict languages can work in \emph{direct} or \emph{indirect} styles. The first ones return efficient code immediately. In the second style to
construct final answer some post processing is required. It can vary from easy simplifications to complicated supercompilation techniques~\cite{sestoft1996}. The main
drawback of indirect style is that the size of intermediate data structures can be exponentially large.

% about lazy languages
A few approaches for checking sub-expressions in lazy languages has been proposed. In ~\cite{augustsson1985} simple left-to-right order of subexpression checking was proposed and was proved that it doesn't affect termination. The backtracking automaton being built has a form of a DAG to reduce code size. A few refinements has been added by~\cite{wadler1987} as a part of textbook~\cite{peytonjones1987} about implementing lazy functional languages. The implementation from this book is being used in the current version of GHC~\cite{ghc}. \cite{laville1991} models values in lazy languages
using \emph{partial terms}, although it doesn't scale to types with infinite constructor sets (like integers). The approach doesn't test all subexpressions from left to right as~\cite{augustsson1985} but aims to not perform unnecessary check by constructing \emph{lazy automaton}. 
%In~\cite{suarez1993} the similar approach is extended by special treatment of overlapping patterns.

% about decision trees
Pattern matching for lazy languages has been compiled also to decision trees~\cite{maranget1992} and later (\cite{maranget1994}) into
\emph{decision DAGs} which allow in some cases to make code smaller.

Minimizing the size of decision tree is known to be NP-hard~\cite{baudinet1985tree}, and as a rule various heuristics are applied during compilation, for example, the number of nodes,
the length of the longest path, the average length of all paths. The paper~\cite{Scott2000WhenDM} performs experimental evaluation of nine heuristics on the base of for strict language Standard ML of New Jersey.

%about automata
The inefficiency of backtracking automaton has been
improved in~\cite{maranget2001}. The approach utilizes matrix representation for pattern matching. It splits the current matrix according to constructors in the
first column and reduces the task to compiling matrices with less rows. The technique is indirect, in the end a few optimizations are performed by introducing
special \emph{exit} nodes to the compiled representation. No preprocessing is required for this scheme: or-pattern receives a special treatment during compilation process.
The approach from this paper is used in the current implementation of the \textsc{OCaml} compiler.

Previous approach uses first column to split the matrix. In~\cite{maranget2008} the \emph{necessity} heuristic has been introduced which recommends which column should be
used to perform the split. Good decision trees which are constructed in this work can perform better on corner cases than~\cite{maranget2001}, but for practical cases the
difference is insignificant.


While existing approaches deliver appropriate solutions for certain forms of pattern matching construct, they have to be extended in an \emph{ad hoc} manner each time
the syntax and semantics of pattern matching construct changes. For example, besides a simple conventional form of pattern matching there is a number of extensions
(guards~\cite{?}, disjunctive patterns~\cite{?}, non-linear patterns~\cite{mcbride1969symbol}, active patterns~\cite{activepatterns}, pattern matching for polymorphic variants~\cite{Garrigue98} and generalized
algebraic datatypes~\cite{?}) which require a separate customized algorithms to be developed.

\begin{comment}
\begin{minipage}[b]{0.5\textwidth}
There are a few different approaches for compiling pattern mathcing. For example, \textsc{GHC}~\cite{?} uses that presented in an influential paper~\cite{Jones1987},
implementation of pattern matching in \textsc{OCaml} is currently based on~\cite{maranget2001} although \cite{maranget2008} reports a slight improvements
of generated code efficiency. 
\end{minipage}
\end{comment}

We present an approach to pattern matching implementation based on application of relational programming~\cite{TRS,WillThesis} and, in particular, relational interpreters~\cite{unified}
and relational conversion~\cite{lozov2017}. Our approach is based on relational representation of the top-level source language semantics of pattern matching on the one hand, and
the semantics of intermediate-level implementation language on the other. We formulate the condition for a correct and complete implementation of pattern matching and use it to
construct a top-level goal which represents a search procedure for all correct and complete implementations. We also present a number of techniques which makes it possible to come up with an
\emph{optimal} solution as well as optimizations to improve the performance of the search. Our implementation\footnote{\url{https://github.com/kakadu/pat-match}} makes use of
\textsc{OCanren}\footnote{\url{https://github.com/JetBrains-Research/OCanren}}~--- a typed implementation of \textsc{miniKanren} for \textsc{OCaml}~\cite{OCanren},
and \textsc{noCanren}\footnote{\url{https://github.com/Lozov-Petr/noCanren}}~--- a convertor from the subset of plain \textsc{OCaml} into \textsc{OCanren}~\cite{lozov2017}.
An evaluation, performed for a set of benchmarks taken from other papers, initially has shown a good performance of our synthesizer. However, being aware of some pitfalls of
our approach, we came up with a set of counterexamples, on which it did not provide any results in observable time, so we do not consider the problem completely solved.
We also started a work on mechanized formalization\footnote{\url{https://github.com/dboulytchev/Coq-matching-workout}}, written in \textsc{Coq}~\cite{Coq}, to
make the justification of out approach more solid and easier to verify, but this formalization is not yet complete. 

 

\begin{figure*}[t]
\[
\begin{array}{cccll}
  &\mathcal{C} & = & \{C_i^{k_i}\} & \mbox{constructors with arities} \\
  &\mathcal{T}_X & = & X \cup \{C_i^{k_i} (t_1, \dots, t_{k_i}) \mid t_j\in\mathcal{T}_X\} & \mbox{terms over the set of variables $X$} \\
  &\mathcal{D} & = & \mathcal{T}_\emptyset & \mbox{ground terms}\\
  &\mathcal{X} & = & \{ x, y, z, \dots \} & \mbox{syntactic variables} \\
  &\mathcal{A} & = & \{ \alpha, \beta, \gamma, \dots \} & \mbox{semantic variables} \\
  &\mathcal{R} & = & \{ R_i^{k_i}\} &\mbox{relational symbols with arities} \\[2mm]
  &\mathcal{G} & = & \mathcal{T_X}\equiv\mathcal{T_X}   &  \mbox{unification} \\
  &            &   & \mathcal{G}\wedge\mathcal{G}     & \mbox{conjunction} \\
  &            &   & \mathcal{G}\vee\mathcal{G}       &\mbox{disjunction} \\
  &            &   & \mbox{\lstinline|fresh|}\;\mathcal{X}\;.\;\mathcal{G} & \mbox{fresh variable introduction} \\
  &            &   & R_i^{k_i} (t_1,\dots,t_{k_i}),\;t_j\in\mathcal{T_X} & \mbox{relational symbol invocation} \\[2mm]
  \phantom{XXXXXXXXXXXXXXX}&\mathcal{S} & = & \{R_i^{k_i} = \lambda\;x_1^i\dots x_{k_i}^i\,.\, g_i;\}\; g & \mbox{specification}
  %      ^
  %      |
  %  this ugly hack is due to non-working \centering
\end{array}
\]
\caption{The syntax of the source language}
\label{syntax}
\end{figure*}

\begin{figure}[t]
\centering
\[
\begin{array}{c}
  \mathcal{FV}\,(x)=\{x\}\\
  \mathcal{FV}\,(C_i^{k_i}\,(t_1,\dots,t_k))=\bigcup\mathcal{FV}\,(t_i)\\[2mm]
  \mathcal{FV}\,(t_1\equiv t_2)=\mathcal{FV}\,(t_1)\cup\mathcal{FV}\,(t_2)\\
  \mathcal{FV}\,(g_1\wedge g_2)=\mathcal{FV}\,(g_1)\cup\mathcal{FV}\,(g_2)\\
  \mathcal{FV}\,(g_1\vee g_2)=\mathcal{FV}\,(g_1)\cup\mathcal{FV}\,(g_2)\\
  \mathcal{FV}\,(\mbox{\lstinline|fresh|}\;x\;.\;g)=\mathcal{FV}\,(g)\setminus\{x\}\\
  \mathcal{FV}\,(R_i^{k_i}\,(t_1,\dots,t_k))=\bigcup\mathcal{FV}\,(t_i)
\end{array}
\]
\caption{Free variables in terms and goals}
\label{free}
\end{figure}

\section{The Language}
\label{language}
 
In this section we introduce the syntax of the language we use throughout the paper, describe the informal semantics and give some examples.

The syntax of the language is shown on Figure~\ref{syntax}. First, we fix a set of constructors $\mathcal{C}$ with known arities and consider
a set of terms $\mathcal{T}_X$ with constructors as functional symbols and variables from $X$. We parameterize this set with an alphabet of
variables, since in the semantic description we will need \emph{two} kinds of variables. The first kind, \emph{syntactic} variables, is denoted
by $\mathcal{X}$. We also consider an alphabet of \emph{relational symbols} $\mathcal{R}$ which are used to name relational definitions.
The central syntactic category in the language is \emph{goal}. In our case there are five types of goals: \emph{unification} of terms,
conjunction and disjunction of goals, fresh variable introduction and invocation of some relational definition. Thus, unification is used
as a constraint, and multiple constraints can be combined using conjunction, disjunction and recursion. For the sake of brevity we
abbreviate immediately nested ``\lstinline|fresh|'' constructs into the one, writing ``\lstinline|fresh $x$ $y$ $\dots$ . $g$|'' instead of
``\lstinline|fresh $x$ . fresh $y$ . $\dots$ $g$|''. The final syntactic category is \emph{specification} $\mathcal{S}$. It consists of a set
of relational definitions and a top-level goal. A top-level goal represents a search procedure which returns a stream of substitutions for
the free variables of the goal. The definition for set of free variables for both terms and goals is given on Figure~\ref{free}; as ``\lstinline|fresh|''
is the sole binding construct the definition is rather trivial. The language we defined is first-order, as goals can not be passed as parameters,
returned or constructed at runtime.

We now informally describe how relational search works. As we said, a goal represents a search procedure. This procedure takes a \emph{state} as input and returns a
stream of states; a state (among other information) contains a substitution which maps semantic variables into terms over semantic variables. Then the five types of
scenarios are possible (depending on the type of the goal):

\begin{itemize}
\item Unification ``\lstinline|$t_1$ === $t_2$|'' unifies terms $t_1$ and $t_2$ in the context of the substitution in the current state. If terms are unifiable,
  then their MGU is integrated into the substitution, and one-element stream is returned; otherwise the result is an empty stream.
\item Conjunction ``\lstinline|$g_1$ /\ $g_2$|'' applies $g_1$ to the current state and then applies $g_2$ to the each element of the result, concatenating
  the streams.
\item Disjunction ``\lstinline|$g_1$ \/ $g_2$|'' applies both its goals to the current state independently and then concatenates the results.
\item Fresh construct ``\lstinline|fresh $x$ . $g$|'' allocates a new semantic variable $\alpha$, substitutes all free occurrences of $x$ in $g$ with $\alpha$, and
  runs the goal.
\item Invocation ``$\lstinline|$R_i^{k_i}$ ($t_1$,...,$t_{k_i}$)|$'' finds a definition for the relational symbol $R_i^{k_i}=\lambda x_1\dots x_{k_i}\,.\,g_i$, substitutes
  all free occurrences of formal parameter $x_j$ in $g_i$ with term $t_j$ (for all $j$) and runs the goal in the current state.
\end{itemize}

We stipulate that the top-level goal is preceded by an implicit ``\lstinline|fresh|'' construct, which binds all its free variables, and that the final substitutions
for these variables constitute the result of the goal evaluation.

Conjunction and disjunction form a monadic~\cite{Monads} interface with conjunction playing role of ``bind'' and disjunction~--- of ``mplus''. In this description
we swept a lot of important details under the carpet~--- for example, in actual implementations the components of disjunction are not evaluated in isolation, but
both disjuncts are being evaluated incrementally with the control passing from one disjunct to another (\emph{interleaving}); instead streams the implementation
can be based on ``ferns''~\cite{BottomAvoiding} to defer divergent computations, etc. 

As an example consider the following specification:

\begin{lstlisting}
  append$^o$ = fun x y xy .
    ((x === Nil) /\ (xy === y)) \/
    (fresh h t ty .
       (x  === Cons (h, t))  /\
       (xy === Cons (h, ty)) /\
       (append$^o$ y t ty)
    );
  revers$^o$ = fun x y .
    ((x === Nil) /\ (y === Nil)) \/
    (fresh h t t' .
       (x === Cons (h, t)) /\
       (append$^o$ t' (Cons (h, Nil)) y) /\
       (revers$^o$ t t') 
    );
  revers$^o$ x x
\end{lstlisting}

Here we defined\footnote{We respect here a conventional tradition for \textsc{miniKanren} programming to superscript all relational names with ``$^o$''.}
two relational symbols~--- ``\lstinline|append$^o$|'' and ``\lstinline|revers$^o$|'',~--- and specified a top-level goal ``\lstinline|revers$^o$ x x|''.
The symbol ``\lstinline|append$^o$|'' defines a relational concatenation of lists~--- it takes three arguments and performs a case analysis on the first one. If the
first one is an empty list (``\lstinline|Nil|''), then the second and the third arguments are unified. Otherwise the first argument is deconstructed into a head ``\lstinline|h|''
and a tail ``\lstinline|t|'', and the tail is concatenated with the second argument using a recursive call to ``\lstinline|append$^o$|'' and additional variable ``\lstinline|ty|'', which
represents the concatenation of ``\lstinline|t|'' and ``\lstinline|y|''. Finally, we unify ``\lstinline|Cons (h, ty)|'' with ``\lstinline|xy|'' to form a final constraint. Similarly,
``\lstinline|revers$^o$|'' defines relational list reversing. The top-level goal represents a search procedure for all lists ``\lstinline|x|'', which are stable under reversing, i.e.
represent palindromes. Running it results in an infinite stream of substitutions:

\begin{lstlisting}
   $\alpha\;\mapsto\;$ Nil
   $\alpha\;\mapsto\;$ Cons ($\beta_0$, Nil)
   $\alpha\;\mapsto\;$ Cons ($\beta_0$, Cons ($\beta_0$, Nil))
   $\alpha\;\mapsto\;$ Cons ($\beta_0$, Cons ($\beta_1$, Cons ($\beta_0$, Nil)))
   $\dots$
\end{lstlisting}

where ``$\alpha$''~--- a \emph{semantic} variable, corresponding to ``\lstinline|x|'', ``$\beta_i$''~--- free semantics variables.

The syntax described above can be formalized in \textsc{Coq} in a natural way using inductive data types. We have made a few non-essential simplifications and modifications for the sake of convenience.
Specifically, we restrict the arities of constructors to be either zero or two:

\begin{lstlisting}[language=Coq,basicstyle=\footnotesize]
   Inductive term : Set :=
   | Var : var -> term
   | Cst : con_name -> term
   | Con : con_name -> term -> term -> term.
\end{lstlisting}

Here ``\lstinline[language=Coq]{var}'' and ``\lstinline[language=Coq]{con_name}''~--- the types representing variables and constructor names, whose definitions we omitted for the sake of brevity.
Similarly, we restrict relations to always have exactly one argument:

\begin{lstlisting}[language=Coq,basicstyle=\footnotesize]
   Definition rel : Set := term -> goal.
\end{lstlisting}

These restrictions do not make the language less expressive in any way since we can represent a sequence of terms as a list using constructors \lstinline|Nil$^0$| and \lstinline|Cons$^2$|.

We also introduce one additional type of goals~--- \emph{failure}~--- for deliberately unsuccessful computation (empty stream). As a result, the definition of goals looks as follows:

\begin{lstlisting}[language=Coq,basicstyle=\footnotesize]
   Inductive goal : Set :=
   | Fail   : goal
   | Unify  : term -> term -> goal
   | Disj   : goal -> goal -> goal
   | Conj   : goal -> goal -> goal
   | Fresh  : (var -> goal) -> goal
   | Invoke : rel_name -> term -> goal.
\end{lstlisting}

Note that in our formalization we use higher-order abstract syntax for variable binding~\cite{HOAS}. We preferred it to the first-order syntax because it gives us the ability
to use substitution and inductive principle provided by \textsc{Coq}. However, we still need to carefully ensure some expected properties on the structure of syntax trees.
For example, we should require that the definitions of relations do not contain unbound variables:

\begin{lstlisting}[language=Coq,basicstyle=\footnotesize]
   Definition closed_goal_in_context
     (c : list var) (g : goal) : Prop :=
     forall n, is_fv_of_goal n g -> In n c.
   Definition closed_rel (r : rel) : Prop :=
     forall (arg : term),
     closed_goal_in_context (fv_term arg) (r arg). 
   Definition def : Set := {r : rel | closed_rel r}.
\end{lstlisting}

In the snippet above ``\lstinline[language=Coq]{def}'' corresponds to a set of relational symbol definitions in a specification, ``\lstinline[language=Coq]{is_fv_of_goal}''
is inductively defined proposition for a free variable in a goal.

We set an arbitrary environment (a map from relational symbol to the definition of relation) to use further throughout the formalization. Failure goals allow us to define it as
a total function:

\begin{lstlisting}[language=Coq,basicstyle=\footnotesize]
   Definition env : Set := rel_name -> def.
   Variable Prog : env.
\end{lstlisting}



\begin{thebibliography}{99}
\bibitem{TRS}
Daniel P. Friedman, William E.Byrd, Oleg Kiselyov. The Reasoned Schemer. The MIT
Press, 2005.

\bibitem{MicroKanren}
Jason Hemann, Daniel P. Friedman. $\mu$Kanren: A Minimal Core for Relational Programming //
Proceedings of the 2013 Workshop on Scheme and Functional Programming (Scheme '13).

\bibitem{alphaKanren}
William E. Byrd, Daniel P. Friedman. alphaKanren: A Fresh Name in Nominal Logic Programming //
Proceedings of the 2007 Workshop on Scheme and Functional Programming (Scheme '07).

\bibitem{CKanren}
Claire E. Alvis, Jeremiah J. Willcock, Kyle M. Carter, William E. Byrd, Daniel P. Friedman.
cKanren: miniKanren with Constraints //
Proceedings of the 2011 Workshop on Scheme and Functional Programming (Scheme '11).

\bibitem{Untagged}
William E. Byrd, Eric Holk, Daniel P. Friedman.
miniKanren, Live and Untagged: Quine Generation via Relational Interpreters (Programming Pearl) //
Proceedings of the 2012 Workshop on Scheme and Functional Programming (Scheme '12).

\bibitem{Kumar}
Ramana Kumar. Mechanising Aspects of miniKanren in HOL. Bachelor Thesis, The Australian National University, 2010.

\bibitem{Unification}
Franz Baader, Wayne Snyder. Unification theory. In John Alan Robinsonand Andrei Voronkov, editors,
Handbook of Automated Reasoning. Elsevier and MIT Press, 2001.

\bibitem{UnificationRevisited}
J.-L. Lassez, M.J. Maher, K. Marriott. Unification Revisited // Foundations of Deductive Databases and Logic Programming, 
Morgan Kaufmann Publishers Inc., 1988.

\bibitem{Lambda}
Henk Barendregt. Lambda Calculi with Types, Handbook of Logic in Computer Science (Vol.~2), 1992.

\bibitem{WillThesis}
William E. Byrd. Relational Programming in miniKanren: Techniques, Applications, and Implementations. PhD Thesis,
Indiana University, Bloomington, IN, September 30, 2009.

\bibitem{OCanren}
Dmitry Kosarev, Dmitry Boulytchev. Typed Embedding of a Relational Language in OCaml // International Workshop on ML, 2016.

\bibitem{RelConversion}
Petr Lozov, Andrei Vyatkin, Dmitry Boulytchev. Typed Relational Conversion // International Symposium on Trends in Funtional
Programming, 2017.

\bibitem{Types}
Benjamin Pierce. Types and Programming Languages. MIT Press, 2002.

\bibitem{Felleisen}
Andrew Wright, Matthias Felleisen. A Syntactic Approach to Type Soundness // Information and Computation, Vol.~115, No.~1, 1994.

\bibitem{cardelli}
Luca Cardelli, Peter Wegner. On Understanding Types, Data Abstraction, and Polymorphism // ACM Computing Surveys, Vol.~17, No.~4, 1985.

\bibitem{unified}
William E. Byrd, Michael Ballantyne, Gregory Rosenblatt, Matthew Might. A Unified Approach to Solving Seven Programming Problems // 
Proceedings of the International Conference on Functional Programming, 2017.

\bibitem{WillOnHM}
William E. Byrd. Personal communications.

\end{thebibliography}

\end{document}

