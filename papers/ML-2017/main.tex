\documentclass[10pt, oneside, nocopyrightspace]{sigplanconf}
%\documentclass{llncs}

\usepackage{makeidx}
\usepackage{amssymb}
\usepackage{listings}
\usepackage{indentfirst}
\usepackage{verbatim}
\usepackage{amsmath, amssymb}
\usepackage{graphicx}
\usepackage{xcolor}
\usepackage{url}
\usepackage{stmaryrd}
\usepackage{xspace}
\usepackage{comment}

\sloppy

\def\transarrow{\xrightarrow}
\newcommand{\setarrow}[1]{\def\transarrow{#1}}

\newcommand{\trule}[2]{\frac{#1}{#2}}
\newcommand{\crule}[3]{\frac{#1}{#2},\;{#3}}
\newcommand{\withenv}[2]{{#1}\vdash{#2}}
\newcommand{\trans}[3]{{#1}\transarrow{#2}{#3}}
\newcommand{\ctrans}[4]{{#1}\transarrow{#2}{#3},\;{#4}}
\newcommand{\llang}[1]{\mbox{\lstinline[mathescape]|#1|}}
\newcommand{\pair}[2]{\inbr{{#1}\mid{#2}}}
\newcommand{\inbr}[1]{\left<{#1}\right>}
\newcommand{\highlight}[1]{\color{red}{#1}}
\newcommand{\ruleno}[1]{\eqno[\scriptsize\textsc{#1}]}
\newcommand{\inmath}[1]{\mbox{$#1$}}
\newcommand{\lfp}[1]{fix_{#1}}
\newcommand{\gfp}[1]{Fix_{#1}}
\newcommand{\vsep}{\vspace{-2mm}}
\newcommand{\supp}[1]{\scriptsize{#1}}
\newcommand{\G}{\mathfrak G}
\newcommand{\sembr}[1]{\llbracket{#1}\rrbracket}
\newcommand{\cd}[1]{\texttt{#1}}
\newcommand{\miniKanren}{\texttt{miniKanren}\xspace}
\newcommand{\OCanren}{\texttt{OCanren}\xspace}

\lstdefinelanguage{ocanren}{
keywords={fresh, let, in, match, with, when, class, type,
object, method, of, rec, repeat, until, while, not, do, done, as, val, inherit,
new, module, sig, deriving, datatype, struct, if, then, else, open, private, virtual, include, success, failure},
sensitive=true,
commentstyle=\small\itshape\ttfamily,
keywordstyle=\ttfamily\underbar,
identifierstyle=\ttfamily,
basewidth={0.5em,0.5em},
columns=fixed,
fontadjust=true,
literate={fun}{{$\lambda$}}1 {->}{{$\to$}}3 {===}{{$\equiv$}}1 {=/=}{{$\not\equiv$}}1 {|>}{{$\triangleright$}}3 {/\\}{{$\wedge$}}2 {\\/}{{$\vee$}}2 {^}{{$\uparrow$}}1,
morecomment=[s]{(*}{*)}
}

\lstset{
mathescape=true,
%basicstyle=\small,
identifierstyle=\ttfamily,
keywordstyle=\bfseries,
commentstyle=\scriptsize\rmfamily,
basewidth={0.5em,0.5em},
fontadjust=true,
language=ocanren
}

\usepackage{letltxmacro}
\newcommand*{\SavedLstInline}{}
\LetLtxMacro\SavedLstInline\lstinline
\DeclareRobustCommand*{\lstinline}{%
  \ifmmode
    \let\SavedBGroup\bgroup
    \def\bgroup{%
      \let\bgroup\SavedBGroup
      \hbox\bgroup
    }%
  \fi
  \SavedLstInline
}

\begin{document}

\title{Relational Conversion for OCaml}

\authorinfo{Petr Lozov \and Dmitri Boulytchev}
{St.Petersburg State University \\ 
  Saint-Petersburg, Russia }
{$\mathtt{lozov.peter@gmail.com}$ \and $\mathtt{dboulytchev@math.spbu.ru}$}

\maketitle

\begin{abstract}
\small
We address the problem of transforming typed functional programs into relational form. 
In this form a program can be run in various ``directions'' with various arguments left 
free, making it possible to acquire different behaviors from a single specification. 
We present an implementation of relational convertor for the subset of Objective Caml and
evaluate it on a number of benchmarks, obtaining some relational programs never written 
before.
\end{abstract}

\section{Introduction}
\label{intro}

Relational programming is an attractive technique, based on the idea of constructing programs as relations.
Many logic programming languages, such as Prolog, Mercury\footnote{\url{https://mercurylang.org}}, 
or Curry\footnote{\url{http://www-ps.informatik.uni-kiel.de/currywiki}} to some extent can be considered as relational. 
In this paper we, specifically, focus on \miniKanren~\cite{TRS,CKanren,MicroKanren}. 

\miniKanren\footnote{\url{http://minikanren.org}} was originally designed as a small relational DSL, embedded in Scheme/Racket. 
The advantage of this approach is the flexibility in combining functional and logic features; in addition \miniKanren possesses 
some quite appealing features (complete search, purity, declarativity, etc).

With relational approach, it becomes possible to give simple and elegant solutions for the problems, otherwise
considered as tricky, tough, tedious, or boring. For example, relational interpreters can be used to derive
\emph{quines}~--- programs, which reduce to themselves, as well as \emph{twines} or \emph{trines} (a pairs or triples of
programs, reducing to each other)~\cite{Untagged}; a straightforward relational description of
simply typed lambda calculus~\cite{Lambda} inference rules works both as type inferencer and inhabitation problem solver~\cite{WillThesis};
relational list sorting can be used to generate all permutations~\cite{ocanren}, etc. 

On the other hand, writing relational specification can sometimes be a tricky and error-prone task. Fortunately, many 
specifications can be written systematically by ``generalizing'' a certain functional program. From the very beginning 
the conversion from functional to relational form was considered as an element of relational programming thesaurus~\cite{TRS}. However,
the traditional approach~--- \emph{unnesting}~--- was formulated for untyped case, worked only for specifically written
programs and, to our knowledge, was never implemented.

We present a generalized form of relational conversion, which can be applied to typed terms in general form. We study the relational conversion 
for a certain subset of Objective Caml, retaining Hindley-Milner type system with let-polymorphism~\cite{Types}. As target
relational language we use \OCanren~\cite{ocanren}~--- a typed shallow embedding of \miniKanren in Objective Caml; as a matter of fact, 
we transform a functional language into its \emph{relational extension}. Our contribution includes the formal description of the semantics for 
both the source language and its relational extension and the proofs of static and dynamic correctness of the conversion. Due to space
considerations we do not present the formal part here; the report was accepted for presentation at the symposium on Trends in Functional 
Programming\footnote{https://www.cs.kent.ac.uk/events/tfp17}.

\section{Relational Extension and Conversion}

\OCanren can be seen as a relational extension for Objective Caml. The central notion in this extension is \emph{goal}, which can be an arbitrary 
expression of reserved goal type $\G$. There are only five syntactic forms of goals:

\begin{itemize}
  \item conjunction $g_1\wedge g_2$ and disjunction $g_1\vee g_2$ of goals $g_1$ and $g_2$;
  \item fresh variable introduction $\lstinline|fresh ($x$) $\;g$|$ for goal $g$;
  \item unification $t_1\;\equiv\;t_2$ and disequality constraint $t_1\;\not\equiv\;t_2$.
\end{itemize}

The two last forms constitute the basis for goal construction; here $t_1$ and $t_2$ are \emph{terms}. In \OCanren a term is
an arbitrary expression of polymorphic \emph{logic type} $\uparrow\!\alpha$. The simplest expression of logic type is a variable, 
bound in \lstinline|fresh|. Another example is a value, \emph{injected} into the logic domain with a built-in primitive 
``$\uparrow$'', such as $\uparrow\!3$ of type \lstinline|$\uparrow$int|. As an example of relational specification in \OCanren, 
consider the following code snippet:

\begin{lstlisting}[basicstyle=\small]
   let is_zero n b = (n === ^0 /\ b === ^True) \/ 
                      (n =/= ^0 /\ b === ^False)
\end{lstlisting}

Function \lstinline|is_zero| implements a binary relation between integers and booleans; when called with specific arguments, it
returns a goal, which can be executed, returning a \emph{stream of answers}. An element of the stream contains the description of 
certain constraints for logical variables, which have to be respected in order for the relation to hold. For example,

\begin{lstlisting}[basicstyle=\small]
   iz_zero q p$\;\leadsto\;$[(q=0,p=True);(q=_.0(=/=0),p=False)]
\end{lstlisting}

The returned stream of answers contains two elements: the first one assigns \lstinline{q} and \lstinline{p} values 
``\lstinline{0}'' and ``\lstinline{True}'', respectively; the second assigns ``\lstinline{False}'' to \lstinline{p} 
and the constraint to be everything, except zero, to \lstinline{q}; here ``\lstinline{_.0}'' corresponds to a free
variable, and \lstinline{(=/=0)} to its disequality constraint.

Our relational conversion is based on the following very simple idea on the type level: we transform a source
term of type $t$ into its relational counterpart of type $[t]$, where the type transformation function $[\bullet]$ 
is defined as follows:

$$
\begin{array}{rcl}
\left[a\right]              &=&a\to\G\\
\left[t_1\to t_2\right]     &=&\left[t_1\right]\to\left[t_2\right]\\
\left[\forall\alpha.t\right]&=&\forall\alpha.\left[t\right]
\end{array}
$$

\noindent where $a$~--- arbitrary ground (non-polymorphic, non-functional type). Despite its type-based description, 
the conversion itself does not make use of types. The static correctness property, which we've proven, claims, that
for properly typed source terms the results of conversion always properly typed in relational sense. The similar
result holds for dynamic correctness.

%There is an important limitation, however, which originates from the fundamental feature of \miniKanren~--- the use of
%syntactic unification~\cite{Unification}: we interpret the polymorphism in the source program as bounded~\cite{cardelli}. Namely,
%we assume, that a polymorphic type cannot be instantiated in su

\section{Evaluation}

We implemented the conversion\footnote{https://bitbucket.org/peter\_lozov/translator-to-minikanren} and applied it to a number of programs, 
providing their relational implementations. 

First, we implemented an interpreter for a simple Scheme-like language, converted it into relational interpreter and reproduced quines, twines 
and trines benchmarks~\cite{Untagged}. 

Next, we implemented the type inference for Hindley-Milner type system and tested it in various directions. For example, our relational
inferencer is capable of inferring types:

\begin{lstlisting}[basicstyle=\small]
   nat_type_inference (^Abst (^"x", ^Var ^"x")) q $\;\leadsto\;$ 
     [q=Just (TFun (TVar Z, TVar Z))]
\end{lstlisting}

\noindent as well as finding some inhabitants for a given type:

\begin{lstlisting}[basicstyle=\small]
   nat_type_inference q ^(Just ^TBool) $\;\leadsto\;$
     [q=Lit (LBool _.24); 
      q=Let (_.18, Lit (LInt _.32), Lit (LBool _.80)); 
      q=Let (_.18, Lit (LBool _.32), Lit (LBool _.80)); 
      q=Let (_.89, Lit (LBool _.32), Var _.89); 
      ...
     ]
\end{lstlisting}

The first answer corresponds to a boolean constant (\lstinline{true} of \lstinline{false}), the second and third~---
to expressions of the form \lstinline{let x = $A$ in $B$}, where $A$~--- some integer or boolean
constant, $B$~--- some boolean constant, and the fourth~--- to the expression \lstinline{let x = $B$ in x}, where
$B$~--- some boolean constant.

Our relational inferencer can also be used to deduce a complete term from an incomplete term and its desirable type:

\begin{lstlisting}[basicstyle=\small]
   nat_type_inference 
     (^Let (^"f", q, 
            ^App (^Var ^"f", 
                  ^Abst (^"x", 
                         ^App (^Var ^"f", 
                               ^Var ^"x"))))) 
     ^(Just ^TBool)) $\;\leadsto\;$
     [q=Abst (_.74, Var (_.74)); 
      q=Abst (_.44, Abst (_.90, Var (_.90))); 
      ...
     ]
\end{lstlisting}

Here we provide the type inferencer an incomplete term \lstinline{let f = $\Box$ in f (fun x -> f x)}, where $\Box$ is a hole, 
and expected type \lstinline{bool}. The answers provide us with the terms, which can be plugged into the hole: the first one 
is \lstinline{fun x -> x}, the second~--- \lstinline{fun x y -> y}. Note, that this example essentially uses the polymorphic
part of the type system, as the term \lstinline{fun f -> f (fun x -> f x)} can not be typed in STLC.

%------

\begin{comment}
\item Проверка населенности типа с помощью генерации термов, которые имеют данный тип.
\begin{lstlisting}[language=ocaml,mathescape=true,numberstyle=\small,stepnumber=1,numbersep=-5pt]
run (q) 5 (nat_type_inference ((===) q) (Just TBool))

==> {
q=Lit (LBool _.24); 
q=Let (_.18, Lit (LInt _.32), Lit (LBool _.80)); 
q=Let (_.18, Lit (LBool _.32), Lit (LBool _.80)); 
q=Let (_.89, Lit (LBool _.32), Var _.89); 
q=Let (_.18, Abst (_.26, Lit LInt (_.48)), Lit (LBool _.149)); 
}
\end{lstlisting}
\item Достраивание неполного терма. Имеется терм с дыркой \linebreak  В результате получаются различные термы (например, $lambda x. x$), которые можно подставить на место дырки, чтобы полученный терм типизировался заданным типом.
\begin{lstlisting}[language=ocaml,mathescape=true,numberstyle=\small,stepnumber=1,numbersep=-5pt]
run (q) 5 
  nat_type_inference 
    (Let ("f", q, App(Var "f", Abst("x", App(Var "f", Var "x"))))) 
    (Just TBool))

==> {
q=Abst (_.74, Var (_.74)); 
q=Abst (_.44, Abst (_.90, Var (_.90))); 
q=Let (_.44, Lit (LInt (_.58)), Abst (_.130, Var (_.130))); 
q=Abst (_.231 [=/= _.218], 
    Let (_.218, Lit (LInt (_.74)), Var (_.231 [=/= _.218]))); 
q=Let (_.44, Lit (LBool (_.58)), Abst (_.130, Var (_.130)));  
}
\end{lstlisting}
\end{itemize}
\end{comment}

Finally, we implemented an interpreter of \OCanren-like relational language and converted it into relational form. As a result, 
we can run relational programs relationally. For example, for this relational program with a hole ($\Box$) 

\begin{lstlisting}[basicstyle=\small]
   let rec add a b c =
      ((a === Z) /\ (b === c)) \/
      (fresh (a$_0$ c$_0$)
        (a === S a$_0$) /\
        $\Box$ /\
        (add a$_0$ b c$_0$)
      )
   in fresh (x y z) (add x y z)
  \end{lstlisting}

\noindent and a number of answers, describing the results of addition of natural numbers in Peano form, our relational interpreter for relational
language finds a feasible term to plug into the hole: \lstinline{c === S c$_0$}. Our relational language supports disequality
constraints as well, which makes it different from existing works\footnote{https://github.com/jasonhemann/micro-in-mini, https://github.com/jpt4/muko}.

\begin{thebibliography}{99}
\bibitem{TRS}
Daniel P. Friedman, William E.Byrd, Oleg Kiselyov. The Reasoned Schemer. The MIT
Press, 2005.

\bibitem{MicroKanren}
Jason Hemann, Daniel P. Friedman. $\mu$Kanren: A Minimal Core for Relational Programming //
Proceedings of the 2013 Workshop on Scheme and Functional Programming (Scheme '13).

\bibitem{CKanren}
Claire E. Alvis, Jeremiah J. Willcock, Kyle M. Carter, William E. Byrd, Daniel P. Friedman.
cKanren: miniKanren with Constraints //
Proceedings of the 2011 Workshop on Scheme and Functional Programming (Scheme '11).

\bibitem{Untagged}
William E. Byrd, Eric Holk, Daniel P. Friedman.
miniKanren, Live and Untagged: Quine Generation via Relational Interpreters (Programming Pearl) //
Proceedings of the 2012 Workshop on Scheme and Functional Programming (Scheme '12).

%\bibitem{Unification}
%Franz Baader, Wayne Snyder. Unification theory. In John Alan Robinsonand Andrei Voronkov, editors,
%Handbook of Automated Reasoning. Elsevier and MIT Press, 2001.

\bibitem{Lambda}
Henk Barendregt. Lambda Calculi with Types, Handbook of Logic in Computer Science (Vol.~2), 1992.

\bibitem{WillThesis}
William E. Byrd. Relational Programming in miniKanren: Techniques, Applications, and Implementations. PhD Thesis,
Indiana University, Bloomington, IN, September 30, 2009.

\bibitem{ocanren}
Dmitry Kosarev, Dmitry Boulytchev. Typed Embedding of a Relational Language in OCaml // International Workshop on ML, 2016.

\bibitem{Types}
Benjamin Pierce. Types and Programming Languages. MIT Press, 2002.

%\bibitem{cardelli}
%Luca Cardelli, Peter Wegner. On Understanding Types, Data Abstraction, and Polymorphism // ACM Computing Surveys, Vol.~17, No.~4, 1985.
\end{thebibliography}
\end{document}

