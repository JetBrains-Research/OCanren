\documentclass[submission,copyright,creativecommons]{eptcs}
\providecommand{\event}{TEASE-LP 2020} % Name of the event you are submitting to
\usepackage{breakurl}             % Not needed if you use pdflatex only.
\usepackage{underscore}           % Only needed if you use pdflatex.

\usepackage{amsmath,amssymb}
%\usepackage[english]{babel}
\usepackage{amssymb}
\usepackage{mathtools}
\usepackage{listings}
\usepackage{comment}
\usepackage{indentfirst}
\usepackage{hyperref}
\usepackage{amsthm}
\usepackage{xcolor}
\usepackage{stmaryrd}
\usepackage{eufrak}
\usepackage{placeins}
\newtheorem{theorem}{Theorem}
%\newtheorem{lemma}{Lemma}
%\newtheorem{corollary}{Corollary}
\newtheorem{hyp}{Hypothesis}
%\newtheorem{definition}{Definition}
\usepackage{caption}
\usepackage{subcaption}
%\usepackage[OT2]{fontenc}   %% Just for lulz

\lstdefinelanguage{minikanren}{
basicstyle=\normalsize,
keywords={fresh},
sensitive=true,
commentstyle=\itshape\ttfamily, % \footnotesize\itshape\ttfamily,
keywordstyle=\textbf,
identifierstyle=\ttfamily,
basewidth={0.5em,0.5em},
columns=fixed,
fontadjust=true,
literate={fun}{{$\lambda\;\;$}}1 {->}{{$\to$}}3 {===}{{$\,\equiv\,$}}1 {=/=}{{$\not\equiv$}}1 {|>}{{$\triangleright$}}3 {/\\}{{$\wedge$}}2 {\\/}{{$\vee$}}2,
morecomment=[s]{(*}{*)}
}


\lstdefinelanguage{ocaml}{
keywords={fresh, let, begin, end, in, match, type, and, fun, function, try, with, when, class, 
object, method, of, rec, repeat, until, while, not, do, done, as, val, inherit, 
new, module, sig, deriving, datatype, struct, if, then, else, open, private, virtual},
sensitive=true,
basicstyle=\small,
commentstyle=\small\itshape\ttfamily,
keywordstyle=\ttfamily\underbar,
identifierstyle=\ttfamily,
basewidth={0.5em,0.5em},
columns=fixed,
fontadjust=true,
literate={->}{{$\;\;\to\;\;$}}1,
morecomment=[s]{(*}{*)}
}

\lstset{
mathescape=true,
language=minikanren
}

\usepackage{letltxmacro}
\newcommand*{\SavedLstInline}{}
\LetLtxMacro\SavedLstInline\lstinline
\DeclareRobustCommand*{\lstinline}{%
  \ifmmode
    \let\SavedBGroup\bgroup
    \def\bgroup{%
      \let\bgroup\SavedBGroup
      \hbox\bgroup
    }%
  \fi
  \SavedLstInline
}

\def\transarrow{\xrightarrow}
\newcommand{\setarrow}[1]{\def\transarrow{#1}}

\def\padding{\phantom{X}}
\newcommand{\setpadding}[1]{\def\padding{#1}}

\def\subarrow{}
\newcommand{\setsubarrow}[1]{\def\subarrow{#1}}

\newcommand{\trule}[2]{\frac{#1}{#2}}
\newcommand{\crule}[3]{\frac{#1}{#2},\;{#3}}
\newcommand{\withenv}[2]{{#1}\vdash{#2}}
\newcommand{\trans}[3]{{#1}\transarrow{\padding{\textstyle #2}\padding}\subarrow{#3}}
\newcommand{\ctrans}[4]{{#1}\transarrow{\padding#2\padding}\subarrow{#3},\;{#4}}
\newcommand{\llang}[1]{\mbox{\lstinline[mathescape]|#1|}}
\newcommand{\pair}[2]{\inbr{{#1}\mid{#2}}}
\newcommand{\inbr}[1]{\left<{#1}\right>}
\newcommand{\highlight}[1]{\color{red}{#1}}
%\newcommand{\ruleno}[1]{\eqno[\scriptsize\textsc{#1}]}
\newcommand{\ruleno}[1]{\mbox{[\textsc{#1}]}}
\newcommand{\rulename}[1]{\textsc{#1}}
\newcommand{\inmath}[1]{\mbox{$#1$}}
\newcommand{\lfp}[1]{fix_{#1}}
\newcommand{\gfp}[1]{Fix_{#1}}
\newcommand{\vsep}{\vspace{-2mm}}
\newcommand{\supp}[1]{\scriptsize{#1}}
\newcommand{\sembr}[1]{\llbracket{#1}\rrbracket}
\newcommand{\cd}[1]{\texttt{#1}}
\newcommand{\free}[1]{\boxed{#1}}
\newcommand{\binds}{\;\mapsto\;}
\newcommand{\dbi}[1]{\mbox{\bf{#1}}}
\newcommand{\sv}[1]{\mbox{\textbf{#1}}}
\newcommand{\bnd}[2]{{#1}\mkern-9mu\binds\mkern-9mu{#2}}
\newcommand{\meta}[1]{{\mathcal{#1}}}
\newcommand{\dom}[1]{\mathtt{dom}\;{#1}}
%\newcommand{\primi}[2]{\mathbf{#1}\;{#2}}
\renewcommand{\dom}[1]{\mathcal{D}om\,({#1})}
\newcommand{\ran}[1]{\mathcal{VR}an\,({#1})}
\newcommand{\fv}[1]{\mathcal{FV}\,({#1})}
\newcommand{\tr}[1]{\mathcal{T}r_{#1}}
\newcommand{\diseq}{\not\equiv}
\newcommand{\reprfunset}{\mathcal{R}}
\newcommand{\reprfun}{\mathfrak{f}}
\newcommand{\cstore}{\Omega}
\newcommand{\cstoreinit}{\cstore_\epsilon^{init}}
\newcommand{\csadd}[3]{add(#1, #2 \diseq #3)}  %{#1 + [#2 \diseq #3]}
\newcommand{\csupdate}[2]{update(#1, #2)}  %{#1 \cdot #2}
\newcommand{\primi}[1]{\mathbf{#1}}
\newcommand{\sem}[1]{\llbracket #1 \rrbracket}
\newcommand{\ir}{\ensuremath{\mathcal{I\!R}}}

\let\emptyset\varnothing
\let\eps\varepsilon

\title{Relational Synthesis of Pattern Matching}
\author{Dmitry Kosarev
\institute{Saint Petersburg State University and \\ JetBrains Research, Russia}
\email{Dmitrii.Kosarev@protonmail.ch}
\and
Dmitry Boulytchev
\institute{Saint Petersburg State University and \\ JetBrains Research, Russia}
\email{dboulytchev@math.spbu.ru}
}
\def\titlerunning{Relational Synthesis of Pattern Matching}
\def\authorrunning{D. Kosarev \& D. Boulytchev}
\begin{document}
\maketitle

We present an approach to pattern matching code generation based on application of relational programming~\cite{TRS,WillThesis} and, in
particular, relational interpreters~\cite{unified}.

In a simplified case, we consider a finite set of constructors with arities $\mathcal{C}$, a set of values $\mathcal{V}$,
and a set of patterns $\mathcal{P}$

\[
 \begin{array}{rcll}
    \mathcal{C} & = & \{ C_1^{k_1}, \dots, C_n^{k_n} \}\\
    \mathcal{V} & = & \mathcal{C}\,\mathcal{V}^*\\  
    \mathcal{P} & = & \_ \mid \mathcal{C}\,\mathcal{P}^*
 \end{array}
\]

and address a problem of matching a value (\emph{scrutinee}) against an ordered list of patterns.

Our approach is based on two representations of pattern matching semantics. First, we use \emph{declarative semantics},
representing it as a relation

\[
match\, (s, ps, i)
\]

where $s\in\mathcal{V}$ is a scrutinee, $ps\in\mathcal{P}^*$~--- an ordered list of patterns, and $i\in\mathbb{N}$~--- a natural number.
This relation holds iff $s$ matches the $i$-th pattern of $ps$. For a fixed language of patterns $match$ can be implemented directly
in \textsc{miniKanren} once and for all.

On the other hand, we introduce a simple language $\mathcal{B}$ of test-and-branch constructs:

\[
\begin{array}{rcl}
  \mathcal M & = & \bullet \\
             &   & \mathcal M\,[\mathbb{N}] \\
  \mathcal B & = & \primi{return}\,\mathbb{N} \\
             &   & \primi{if}\;\mathcal{C}\;\mathcal{M}\;\primi{then}\;\mathcal{B}\;\primi{else}\;\mathcal{B}
\end{array}
\]

Here $\mathcal{M}$ stands for a \emph{matchable expression}, which is either a reference to a scrutinee ("$\bullet$") or
a denotation of some indexed subvalue of a matchable expression. The programs in $\mathcal{B}$ can discriminate on the
structure of matchable expressions, testing their top constructors and eventually returning natural numbers as results.
The language $\mathcal{B}$ is similar to the intermediate representations for pattern matching code, used in 
previous works on pattern matching implementation~\cite{maranget2001,maranget2008}.

We use a \emph{relational interpreter} for $\mathcal{B}$

\[
eval^o_{\mathcal B}\, (s, p, i)
\]

Here $s$ and $i$ have the same meaning as in declarative semantics description, $p\in\mathcal{B}$~--- a syntactic representation of
a program in $\mathcal{B}$. The relation $eval^o_{\mathcal B}$ encodes the operational semantics of $\mathcal{B}$; it holds, if
evaluating $p$ for $s$ returns $i$. Being relational interpreter, however, $eval^o_{\mathcal B}$ is capable of solving a
synthesis problem: by a scrutinee $s$ and a number $i$ calculate a program $p$ which makes the relation to hold.
Within this setting, we can formulate the pattern-matching synthesis problem as follows: \emph{for a given ordered list of patterns $ps$ find a program $p$, such that}

\[
\forall s\in\mathcal{V},\,\forall i\in\mathbb{N},\,eval^o_{\mathcal B}\, (s, p, i) \wedge\, match (s, ps, i)
\]

It is rather problematic to directly solve this synthesis problem with existing \textsc{miniKanren} implementations as
they provide a rather limited support for universal quantification~\cite{eigen,moiseenko}. However, in our concrete
case there is a simple way to alleviate this problem. Indeed, we may replace universal quantification over $i$ by
a finite conjunction, as the length of $ps$ is known at the synthesis time. As for the quantification over $s$, for
any concrete $ps$ we may precompute a \emph{complete set of examples} $\mathcal{E}(ps)\subseteq\mathcal{V}$ with the following
property:

\[
\forall i\in\mathbb{N},\,(\forall s\in\mathcal{E}(ps),\,match\, (s, ps, i) \Leftrightarrow \forall s\in\mathcal{V},\,match\, (s, ps, i))
\]

It easy to see, that for arbitrary $ps$ there exists a finite complete set of examples (indeed, any pattern describes the ``shape''
of a scrutinee up to some finite depth, beyond which all scrutinees become indistinguishable). Thus, for a given $ps$ we may
completely eliminate the quantification, reformulating the synthesis problem as

\[
\bigwedge_{i\in[1\dots|ps|]}\,\bigwedge_{s\in\mathcal{E}(ps)} (eval^o_{\mathcal B}\, (s, p, i) \wedge match\, (s, ps, i))
\]

We implemented the synthesis framework using \textsc{OCanren}~--- an embedding of \textsc{miniKanren} into \textsc{OCaml}~\cite{ocanren},~---
and evaluated it on the set of benchmarks, reported in the previous works on \emph{ad-hoc} algorithms for pattern matching
code generation~\cite{maranget2001,maranget2008}. In comparison with a simplified setting, presented above, our implementation
deals with a more elaborate pattern matching problem~--- in particular, we support \emph{guard expressions}, name bindings in
patterns and incorporate a deterministic top-down matching strategy, which is common in functional languages.

Initially, our synthesis did not demonstrate good results. However, we applied the following techniques to improve both the performance
and the quality of synthesized programs:

\begin{itemize}
\item we restricted the shape of scrutinees using type information;
\item we utilized tabling to memoize repeating search steps;
\item we implemented a pruning technique, which makes the search stop exploring a certain branch if the program, synthesized so far,
  contains too much nesting constructs (this factor can be precomputed by patterns analysis).
\end{itemize}

With these adjustments, our synthesis framework in a negligible time provides the same results as those reported in the previous works.
Our future steps include extending the pattern matching language to completely match that of \textsc{OCaml} (for
now we do not support GADTs), integrate the synthesis into the existing \textsc{OCaml} compiler and evaluate it on a
set of real-world programs. Another direction is extending the pattern matching language to incorporate features which
are known to be hard, tedious or error-prone to implement (for example, non-linear patterns).

\begin{comment}


Real world modern compilers are obliged to address a few problems which are NP-complete and hence can't have effective
algorithm to solve them. So, compilers use semi-optimal algorithms to find a decent solution. Optimal algorithms require
brute force search to get the best solution and  affect compilation speed negatively. In this work we apply relational
programming -- a convenient DSL for implementing search -- to compilation of pattern matching, one of a kind hard problems for compiler.

The task of compiling pattern matching for typed languages is well presented in literature~\cite{maranget2001,maranget2008}.


We test approach on simplified source language $PM$ where scrutinee is a value $\in\mathcal{V}$ of algebraic data type, only wildcards
and nested constructors are allowed as patterns $\mathcal{P}$ and right hand side of clause is its index. The source language is easy
extendable by pattern variables and optional pattern guards that test subterms of scrutinee using a function. The semantics of $PM$
is a function from concrete scrutinee $s$, concrete patterns $pats$ and concrete guards $gs$ to clause indexes, and is denoted
as $\sem{s,pats,gs}_{PM} = i$.

Compilation scheme translates sentences from $PM$ to $\ir$ language which has constructions for clause indexes and conditions which
test matchable values for specific constructor. Matchable values can be either a scrutinee, or a projection of matchable value that
returns one of its field indexed by natural numbers. $\ir$ language is easy extendable by tests for fixed number of pattern guards.
The semantics is straightforward and is denoted by $\sem{\cdot}_{\ir}$.

We deal with a task of compiling pattern matching as it is a synthesis problem. The goal of algorithm is to synthesize $ideal_\ir$
for concrete patterns $pats$ and guards that, firstly, will behave the same as original pattern matching for any possible
scrutinee $s$. Secondly, we want shortest solution because short code usually runs faster. Relational programming~\cite{OCanren}
will help with that because it has a tendency to generate short answers earlier, although this tendency is not strict.

$$
\forall s:\; \sem{s; ideal_\ir}_\ir = \sem{s;pats}_{PM}
$$

To eliminate universal quantifier we use the following observation: for \emph{finite} amount of patterns of \emph{finite} height
we can generate \emph{finite} amount of examples to test pattern matching semantics. In examples, very deep subterms can have
any value because they will not be tested during pattern matching. We can reformulate synthesis problem as follows:

$$
\mid  Examples\mid < \infty\quad \land\quad \left(\forall e \; (e\in Examples)\quad\land \quad\left( \sem{e; ideal_\ir}_\ir = \sem{e;pats}_{PM}\right)\right)
$$

For plain ADT the approach will generate required examples in finite time, but for GADTs it can diverge because inhabitancy problem
is semi-decidable~\cite{garrigue2017gadts}(chapter 5). Inhabitants generation as well as synthesis algorithm is
implemented\footnote{\url{https://github.com/Kakadu/pat-match/}} using relational programming.

Presented approach is good as general description of an idea but require a few tweaks to start working, for example, on presented
sample~\ref{fig:example1} . Firstly, synthesis procedure in a way as it is described doesn't take types into account, so it is
useful to give hints about which parts of scrutinee should be checked for which constructors. Second observation says that we
run $\sem{\cdot}_\ir$ in concrete direction, so it is possible to check periodically count of \texttt{IfTag} constructors in
result value and prune branches where it becomes too big. Thirdly, synthesis query generates a lot of similar queries, and
we use tabling to speedup search. All three observations are important, removing one of them leads to visible performance degradation.

The optimal (two \texttt{IfTag}'s) and the semi-optimal solution (three \texttt{IfTag}'s) for~\ref{fig:example1} are described
in~\cite{maranget2008}. Current implementation generates semi-optimal solution as 28th answer. Before that it generates optimal
solution (and it's equivalents) three times, other 24 answers are longer and less useful. All tasks (example generation, synthesis
and printing answers) take 3 seconds, which is unfortunate.

Shortly, we present following contributions
\begin{itemize}
\item Code synthesis for pattern matching works after implementing \emph{three optimizations} above.
\item GADTs, pattern binding and guards works for simple examples, the approach is easy extendable by them.
\end{itemize}

Future work is
\begin{itemize}
\item Discover other optimizations and enable current ones automatically using type information (at the moment we patched synthesis
  algorithm manually for concrete example).
\item When current implementation tests for \texttt{cons} tag it can't propagate constraint that tag equals to \texttt{nil} to
  the \texttt{else} branch, which partially explains why branch pruning is so useful.
\item Algorithm for inhabitant generation requires proper formulation and proof.
\item Apply current synthesis procedure for exhaustiveness checking which will give us \emph{single} procedure for compilation and exhaustiveness checking.
\item Test the approach on real world problems (embedding to OCaml compiler).
\end{itemize}


\begin{figure}[t]
  \[
  \begin{array}{rcll}
    \mathcal{C} & = & \{ C_1^{k_1}, \dots, C_n^{k_n} \}    &\mbox{(constructors)} \\
    \mathcal{V} & = & \mathcal{C}\,\mathcal{V}^*        &\mbox{(values)}       \\
    \mathcal{P} & = & \_ \mid \mathcal{C}\,\mathcal{P}^*&\mbox{(patterns)}     \\
    \mathcal{M} & = & \bullet \mid \mathcal{M} [\mathbb{N}]&

  \end{array}
  \]
\end{figure}

\begin{figure}
\centering
\begin{minipage}{.7\textwidth}
  \centering
\begin{align*}
\mathcal{C} =&\; \{ C_1^{k_1}, \dots, C_n^{k_n} \} \\
\mathcal{V} =&\;  \mathcal{C}\ \mathcal{V}^*\\
\mathcal{M} =&\;  \mathcal{S} \\
          \mid\; &\; \text{\texttt{Field}}\;  \mathcal{M}\times  \mathbb{N}\\
\mathcal{P} =&\;  \text{\texttt{Wildcard}} \\
          \mid\; &\; \text{\texttt{Var}}\  Name\\
          \mid\; &\; \text{\texttt{PConstructor}}\  \mathcal{C}\times  \mathcal{P}^*\\
\ir  =&\; \text{\texttt{Int}}\  \mathbb{N} \\
%           \mid\; &\;\mathcal{S} \\
           \mid\; &\; \text{\texttt{IfTag}}\; \mathcal{C}\times \mathcal{M}\times \ir\times \ir\\
           \mid\; &\; \text{\texttt{IfGuard}}\ \mathbb{N}\times (Name\times \mathcal{M})^*\times \ir\times \ir\\
Clause =&\;  \mathcal{P} \times \mathbb{N}? \times \ir
\end{align*}
  \captionof{figure}{Structure of $PM$ and \ir languages}
%  \label{fig:test1}
\end{minipage}%
\begin{minipage}{.3\textwidth}
  \centering
\begin{lstlisting}[language=ocaml]
match s with 
| ([], _)     -> 1
| (_, [])     -> 2
| (_::_,_::_) -> 3
\end{lstlisting}
  \captionof{figure}{Simple example of pattern matching problem from~\cite{maranget2008}}
\label{fig:example1}
\end{minipage}
\end{figure}

\end{comment}

%\nocite{*}
\bibliographystyle{eptcs}
\bibliography{matching}

\end{document}
