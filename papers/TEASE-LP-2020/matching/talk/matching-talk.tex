\def\pgfsysdriver{pgfsys-dvipdfm.def}
\documentclass[aspectratio=169
  , xcolor={svgnames}
  , hyperref={ colorlinks,citecolor=Blue
             , linkcolor=DarkRed,urlcolor=DarkBlue}
  , russian
  ]{beamer}
\usetheme{CambridgeUS}
\usefonttheme{professionalfonts}

% get rid of header navigation bar
\setbeamertemplate{headline}{}
% get rid of bottom navigation symbols
\setbeamertemplate{navigation symbols}{}

\usepackage{pgfpages}
%\setbeamertemplate{note page}[plain]
%\setbeameroption{show notes on second screen=right}

\usepackage{bibentry}
\usepackage{cite}
%\usepackage{natbib}
\def\newblock{\hskip .11em plus .33em minus .07em}
% get rid of footer
%\setbeamertemplate{footline}{}

\usepackage{xcolor}
\definecolor{YellowGreen} {HTML}{B5C28C}
\definecolor{ForestGreen} {HTML}{009B55}
\definecolor{MyBackground}{HTML}{F0EDAA}


\usepackage{xltxtra} % load xunicode
\usepackage{polyglossia}
\setmainlanguage{english}

\let\cyrillicfonttt\monofamily

%\newfontfamily\dejaVuSansMono{DejaVu Sans Mono}
% https://github.com/vjpr/monaco-bold/raw/master/MonacoB/MonacoB.otf
%\newfontfamily\monacoB{MonacoB}

\usepackage{listings}
\lstdefinelanguage{ocanren}{
keywords={run, conde, fresh, let, in, match, with, when, class, type,
object, method, of, rec, repeat, until, while, not, do, done, as, val, inherit,
new, module, sig, deriving, datatype, struct, if, then, else, open, private, virtual, include, success, failure,
true, false},
sensitive=true,
commentstyle=\small\itshape\ttfamily,
keywordstyle=\ttfamily\textbf,
identifierstyle=\ttfamily,
basewidth={0.5em,0.5em},
columns=fixed,
mathescape=true,
fontadjust=true,
literate={fun}{{$\lambda$}}1 {->}{{$\to$}}3 {===}{{$\equiv$}}1 {=/=}{{$\not\equiv$}}1 {|>}{{$\triangleright$}}3 {\\/}{{$\vee$}}2 {/\\}{{$\wedge$}}2 {^}{{$\uparrow$}}1,
morecomment=[s]{(*}{*)}
}

\lstset{
language=ocanren
}
  
%%%%%%%%%%%%%%%%%%%%%%%%%%%%%%%%%%%%%%%%%%%%
\setmainfont[
 Ligatures=TeX,
 Extension=.otf,
 BoldFont=cmunbx,
 ItalicFont=cmunti,
 BoldItalicFont=cmunbi,
]{cmunrm}
% С засечками (для заголовков)
\setsansfont[
 Ligatures=TeX,
 Extension=.otf,
 BoldFont=cmunsx,
 ItalicFont=cmunsi,
]{cmunss}


\usefonttheme{professionalfonts}
\usepackage{times}
\usepackage{tikz}
\usetikzlibrary{cd}
% \usepackage{tikz-cd}
\usepackage{amsmath}
%\DeclareMathOperator{->}{\rightarrow}
\newcommand\iso{\ensuremath{\cong}}
\usepackage{verbatim}
\usepackage{graphicx}
\usetikzlibrary{arrows,shapes}

\usepackage{fontawesome}
% \newfontfamily{\FA}{Font Awesome 5 Free} % some glyphs missing
\expandafter\def\csname faicon@facebook\endcsname{{\FA\symbol{"F09A}}}
\def\faQuestionSign{{\FA\symbol{"F059}}}
\def\faQuestion{{\FA\symbol{"F128}}}
\def\faExclamation{{\FA\symbol{"F12A}}}
\def\faUploadAlt{{\FA\symbol{"F093}}}
\def\faLemon{{\FA\symbol{"F094}}}
\def\faPhone{{\FA\symbol{"F095}}}
\def\faCheckEmpty{{\FA\symbol{"F096}}}
\def\faBookmarkEmpty{{\FA\symbol{"F097}}}

\newcommand{\faGood}{\textcolor{ForestGreen}{\faThumbsUp}}
\newcommand{\faBad}{\textcolor{red}{\faThumbsODown}}
\newcommand{\faWrong}{\textcolor{red}{\faTimes}}
\newcommand{\faMaybe}{\textcolor{blue}{\faQuestion}}
\newcommand{\faCheckGreen}{\textcolor{ForestGreen}{\faCheck}}

\usepackage{soul} % for \st that strikes through
\usepackage[normalem]{ulem} % \sout


\usepackage{tabulary}

% sudo aptget install ttf-mscorefonts-installer
\defaultfontfeatures{Ligatures={TeX}}
\setmainfont{Times New Roman}
\setsansfont{CMU Sans Serif}

\setmonofont[Scale=1.0,
    BoldFont=lmmonolt10-bold.otf,
    ItalicFont=lmmono10-italic.otf,
    BoldItalicFont=lmmonoproplt10-boldoblique.otf
]{lmmono9-regular.otf}

%\usepackage[cache=true]{minted}

\author[]{\underline{Dmitry Kosarev} \textsuperscript{1,2} \and Dmitry Boulytchev\inst{1,2}}
\institute[]{\textsuperscript{1} JetBrains Research \and \inst{2} Saint-Petersburg University}

\addtobeamertemplate{title page}{}{
  \begin{center}{\tiny Build date: \today}\end{center}
}

\usepackage{inconsolata}
\setmonofont[Scale=1.0,
    BoldFont=lmmonolt10-bold.otf,
    ItalicFont=lmmono10-italic.otf,
    BoldItalicFont=lmmonoproplt10-boldoblique.otf
]{lmmono9-regular.otf}

\newfontfamily{\monott}{Latin Modern Mono}

\usepackage{listings}
\lstdefinelanguage{ocanren}{
keywords={run, conde, fresh, let, in, match, with, when, class, type,
object, method, of, rec, repeat, until, while, not, do, done, as, val, inherit,
new, module, sig, deriving, datatype, struct, if, then, else, open, private, virtual, include, success, failure,
true, false},
sensitive=true,
commentstyle=\small\itshape\ttfamily
,keywordstyle=\fontfamily{monott}\selectfont \textbf
,identifierstyle=\fontfamily{monott}\selectfont%\ttfamily
,basewidth={0.5em,0.5em},
columns=fixed,
mathescape=true,
fontadjust=true,
literate={fun}{{$\lambda$}}1 {->}{{$\to$}}3 {===}{{$\equiv$}}1 {=/=}{{$\not\equiv$}}1 {|>}{{$\triangleright$}}3 {\\/}{{$\vee$}}2 {/\\}{{$\wedge$}}2 {^}{{$\uparrow$}}1,
morecomment=[s]{(*}{*)}
}

\setmonofont[Mapping=tex-text]{CMU Typewriter Text}
\lstdefinelanguage{ocaml}{
keywords={@type, function, fun, let, in, match, with, when, class, type,
nonrec, object, method, of, rec, repeat, until, while, not, do, done, as, val, inherit, and,
new, module, sig, deriving, datatype, struct, if, then, else, open, private, virtual, include, success, failure,
lazy, assert, true, false, end},
sensitive=true,
commentstyle=\small\itshape\ttfamily,
keywordstyle=\ttfamily\bfseries, %\underbar,
identifierstyle=\ttfamily,
basewidth={0.5em,0.5em},
columns=fixed,
fontadjust=true,
literate={->}{{$\to$}}3 {===}{{$\equiv$}}1 {=/=}{{$\not\equiv$}}1 {|>}{{$\triangleright$}}3 {\\/}{{$\vee$}}2 {/\\}{{$\wedge$}}2 {>=}{{$\ge$}}1 {<=}{{$\le$}} 1,
morecomment=[s]{(*}{*)}
}

\usepackage{fontawesome}
%\newcommand{\faGood}{\textcolor{ForestGreen}{\faThumbsUp}}
%\newcommand{\faBad}{\textcolor{red}{\faThumbsODown}}

\usepackage{subcaption}
\usepackage{etoolbox}


\usepackage{exercise}
\usepackage{tikz}
 \usepackage{tikz-qtree}
\usetikzlibrary{trees}
\usepackage[edges]{forest}
\forestset{.style={
%  for tree={l=1em, l sep=1em, s sep=1em}
  forked edges,
    for tree={    grow'=0,    draw,    align=c,    font=\sffamily,
        rounded corners  }
  }}

\newcommand{\lstquot}[1]{``\lstinline{#1}''}
\newcommand{\sembr}[1]{\llbracket{#1}\rrbracket}
\newcommand\false{$f\!alse$}
\newcommand\myif{i\!f}


\def\transarrow{\xrightarrow}
\newcommand{\setarrow}[1]{\def\transarrow{#1}}

\def\padding{\phantom{X}}
\newcommand{\setpadding}[1]{\def\padding{#1}}

\def\subarrow{}
\newcommand{\setsubarrow}[1]{\def\subarrow{#1}}

\newcommand{\trule}[2]{\dfrac{#1}{#2}}
\newcommand{\crule}[3]{\dfrac{#1}{#2},\;{#3}}
\newcommand{\withenv}[2]{{#1}\vdash{#2}}
\newcommand{\trans}[3]{{#1}\transarrow{\padding{\textstyle #2}\padding}\subarrow{#3}}
\newcommand{\ctrans}[4]{{#1}\transarrow{\padding#2\padding}\subarrow{#3},\;{#4}}
\newcommand{\llang}[1]{\mbox{\lstinline[mathescape]|#1|}}
\newcommand{\pair}[2]{\inbr{{#1}\mid{#2}}}
\newcommand{\inbr}[1]{\left<{#1}\right>}
\newcommand{\highlight}[1]{\color{red}{#1}}
%\newcommand{\ruleno}[1]{\eqno[\scriptsize\textsc{#1}]}
\newcommand{\ruleno}[1]{\mbox{[\textsc{#1}]}}
\newcommand{\rulename}[1]{\textsc{#1}}
\newcommand{\inmath}[1]{\mbox{$#1$}}
\newcommand{\lfp}[1]{fix_{#1}}
\newcommand{\gfp}[1]{Fix_{#1}}
\newcommand{\vsep}{\vspace{-2mm}}
\newcommand{\supp}[1]{\scriptsize{#1}}
\renewcommand{\sembr}[1]{\llbracket{#1}\rrbracket}
\newcommand{\cd}[1]{\texttt{#1}}
\newcommand{\free}[1]{\boxed{#1}}
\newcommand{\binds}{\;\mapsto\;}
\newcommand{\dbi}[1]{\mbox{\bf{#1}}}
\newcommand{\sv}[1]{\mbox{\textbf{#1}}}
\newcommand{\bnd}[2]{{#1}\mkern-9mu\binds\mkern-9mu{#2}}
\newcommand{\meta}[1]{{\mathcal{#1}}}
\newcommand{\dom}[1]{\mathtt{dom}\;{#1}}
%\newcommand{\primi}[2]{\mathbf{#1}\;{#2}}
\renewcommand{\dom}[1]{\mathcal{D}om\,({#1})}
\newcommand{\ran}[1]{\mathcal{VR}an\,({#1})}
\newcommand{\fv}[1]{\mathcal{FV}\,({#1})}
\newcommand{\tr}[1]{\mathcal{T}r_{#1}}
\newcommand{\diseq}{\not\equiv}
\newcommand{\reprfunset}{\mathcal{R}}
\newcommand{\reprfun}{\mathfrak{f}}
\newcommand{\cstore}{\Omega}
\newcommand{\cstoreinit}{\cstore_\epsilon^{init}}
\newcommand{\csadd}[3]{add(#1, #2 \diseq #3)}  %{#1 + [#2 \diseq #3]}
\newcommand{\csupdate}[2]{update(#1, #2)}  %{#1 \cdot #2}
\newcommand{\primi}[1]{\mathbf{#1}}
\newcommand{\sem}[1]{\llbracket #1 \rrbracket}
\newcommand{\ir}{\ensuremath{\mathcal{S}}}
\usepackage{tikz}
\newcommand*\circled[1]{\tikz[baseline=(char.base)]{
    \node[shape=circle,draw,inner sep=1pt] (char) {#1};}}

\let\emptyset\varnothing
\let\eps\varepsilon






%%%%%%%%%%%%%%%%%%%%%%%%%%%%%%%%%%%%%%%%

%%%%%%%%%%%%%%%%%%%%%%%%%%%%%%%%%%%%%%%%


\title{Relational Synthesis for Pattern Matching}

\date{May 2020}


\newcommand{\verbatimfont}[1]{\def\verbatim@font{#1}}
\usepackage{verbatimbox}


\begin{document}
\maketitle

% For every picture that defines or uses external nodes, you'll have to
% apply the 'remember picture' style. To avoid some typing, we'll apply
% the style to all pictures.
\tikzstyle{every picture}+=[remember picture] 

% By default all math in TikZ nodes are set in inline mode. Change this to
% displaystyle so that we don't get small fractions.
\everymath{\displaystyle}

% Uncomment these lines for an automatically generated outline.
\begin{frame}{Pattern matching}
Effective compilation of pattern matching is essential for functional programming
\vspace{1cm}

Two main approaches to compile:
\begin{itemize}
\item decision diagrams
\begin{itemize}
\item guaranteed amount of checks performed (speed)
\end{itemize}
\item backtracking automaton
\begin{itemize}
\item guaranteed code size
\end{itemize}
\end{itemize}
  \note{}
%  \note[enumerate]
%  {
%  \item Stress this first.
%  \item Then this.
%  }
  
\end{frame}


\begin{frame}[fragile]{An example of matching and it's two compiled represenations}
\begin{figure}[ht]
\begin{subfigure}[b]{0.2\linewidth}
\centering
\begin{lstlisting}
match x,y,z with
| _,F,T -> 1
| F,T,_ -> 2
| _,_,F -> 3
| _,_,T -> 4
\end{lstlisting}
\caption{Pattern matching}
\end{subfigure}
\hspace{0.5cm}
\begin{subfigure}[b]{0.32\linewidth}
\centering
\begin{lstlisting}
if x then
  if y then
    if z then 4 else 3
  else
    if z then 1 else 3
else
  if y then 2
  else
    if z then 1 else 3
\end{lstlisting}
\caption{Correct but semi-optimal compilation}
\label{fig:matching-example2}
\end{subfigure}
\hspace{0.5cm}
\begin{subfigure}[b]{0.35\linewidth}
\centering
\begin{lstlisting}
if y then
  if x then
    if z then 4 else 3
  else 2
else
  if z then 1 else 3
\end{lstlisting}
\caption{Optimal compilation}
\label{fig:matching-example3}
\end{subfigure}
\caption{Pattern matching compilation is non-trivial (example from~\cite{maranget2008}).}
\label{fig:match-example}
\end{figure}
 \note{This is an example from Luc Maranget's paper.
 
 Pattern matchin is on the left. We can start testing from $x$, and then $y$ and $z$, it will give us a program with 6 $if$ constructions. But we can start testing from $y$ and if it is $false$ we can not test $x$. It will give us a program with four tests.}
\end{frame}

\begin{frame}{Our goal}
Do \emph{not compile} with specific algorithm but \emph{synthesize} compiled representation on large enough but finite set of examples
\vspace{1cm}

Minimization criteria: less checks performed.
\vspace{1cm}

We are going to use relational programming for this, more precise OCanren~\cite{OCanrenWeb} from miniKanren~\cite{MiniKanrenWeb} family.
\vspace{1cm}


Our repo on Github: ~\cite{Repo}.
\end{frame}


\begin{frame}[fragile]{Pattern matching syntax}
\begin{figure}[ht]
\begin{subfigure}[b]{0.3\linewidth}
\[
 \begin{array}{rcll}
    \mathcal{C} & = & \{ C_1^{k_1}, \dots, C_n^{k_n} \}\\
    \mathcal{V} & = & \mathcal{C}\,\mathcal{V}^*\\  
    \mathcal{P} & = & \_ \mid \mathcal{C}\,\mathcal{P}^*
 \end{array}
\]
\end{subfigure}
\hspace{0.5cm}
\begin{subfigure}[b]{0.5\linewidth}
\[
\begin{array}{rcl}
  \mathcal M & = & \bullet \\
  &   & \mathcal M\,[\mathbb{N}] \\
  \ir & = & \primi{return}\,\mathbb{N} \\
  &   & \primi{switch}\;\mathcal{M}\;\primi{with}\; [\mathcal{C}\; \primi{\rightarrow}\; \ir]^*\;\primi{otherwise}\;\ir
\end{array}
\]
\end{subfigure}
\end{figure}
\end{frame}

\begin{frame}[fragile]{Example 1 (1/2)}
\begin{figure}
\begin{subfigure}[b]{0.3\linewidth}
\begin{forest}
  forked edges,
  for tree={    grow'=0,    draw,    align=c,    font=\sffamily,
      rounded corners  },
  highlight/.style={    thick,    font=\sffamily\bfseries  }
    [{pats}
      [{Cons}
      [{Cons}
            [{\_}]
            [{\_}]
      ]
      [{\_}]
      ]
      [{\_}]
    ]
\end{forest}
\end{subfigure}
\hspace{3.5cm}
\begin{subfigure}[b]{0.4\linewidth}
\begin{minted}{ocaml}
match (s : unit list) with 
| _ :: _ :: _ -> 1
| _           -> 2
\end{minted}
\vspace{1cm}
\begin{itemize}
\item Pattern matching (above)
\item with two branches
\end{itemize}
\end{subfigure}
\end{figure}
\end{frame}

\begin{frame}[fragile]{Example 1 (2/2)}
\begin{figure}
\begin{subfigure}[b]{0.3\linewidth}
\begin{forest}
  forked edges,
  for tree={    grow'=0,    draw,    align=c,    font=\sffamily,
      rounded corners  },
  highlight/.style={    thick,    font=\sffamily\bfseries  }
    [{all}
    [{Cons}
      [{unit}]
      [{Cons}
            [{unit}]
            [{Nil}]
      ]      
    ]
    [{Cons} [{unit}]  [{Nil}]]
    [{Nil}]
    ]
\end{forest}
\end{subfigure}
\hspace{3.5cm}
\begin{subfigure}[b]{0.4\linewidth}
\begin{minted}{ocaml}
match (s : unit list) with 
| _ :: _ :: _ -> 1
| _           -> 2
\end{minted}
\vspace{1cm}
\begin{itemize}
\item Pattern matching (above)
\item with example bound by height of patterns 
\end{itemize}
\end{subfigure}
\end{figure}
\end{frame}


\begin{frame}[fragile]{Example 2 }
\begin{figure}
\begin{subfigure}[b]{0.3\linewidth}
\begin{forest}
  forked edges,
  for tree={    grow'=0,    draw,    align=c,    font=\sffamily,
      rounded corners  },
  highlight/.style={    thick,    font=\sffamily\bfseries  }
    [{pats}
      [{Nil}]
      [{\_}]
    ]
\end{forest}
\end{subfigure}
\hspace{.5cm}
\begin{subfigure}[b]{0.6\linewidth}
\begin{minted}{ocaml}
match (s : unit list) with 
| []  -> 1
| _   -> 2
\end{minted}
\vspace{1cm}
\begin{itemize}
\item All inhabitants bound by height = 1 -- is a set of single inhabitant
\item It is not enough
\item We should test at least on some expression of height 2 (\mintinline{ocaml}{unit :: []}, at least)
\end{itemize}
\end{subfigure}
\end{figure}
\end{frame}

\begin{frame}{Current algorithm for examples generation}
\begin{figure}
\begin{subfigure}[b]{0.45\linewidth}
Algorithm:
\begin{itemize}
\item Evaluate height of patterns $h$
\item Synthesize all inhabitants, but
\item on depth $h+1$ use single predefined per type inhabitant 
\end{itemize}
\end{subfigure}
\hspace{.5cm}
\begin{subfigure}[b]{0.45\linewidth}
Pros and cons:
\begin{itemize}
\item[\texttt{+}] It is correct
\item[\texttt{-}] Exponential size of patterns
\end{itemize}
\vspace{1cm}
We are currently working on
\begin{itemize}
\item When we can generate less examples?
\item Overall synthesis speed
\begin{itemize}
\item it not only depend on a number of examples
\end{itemize}
\end{itemize}
\end{subfigure}
\end{figure}
\vspace{1cm}\pause
\begin{center}
{\Huge Thanks!}
\end{center}
\end{frame}

\begin{comment}
\begin{frame}[fragile]{Pattern matching semantics}
\begin{figure}[ht]
\begin{subfigure}[b]{0.3\linewidth}
\[
 \begin{array}{rcll}
    \mathcal{C} & = & \{ C_1^{k_1}, \dots, C_n^{k_n} \}\\
    \mathcal{V} & = & \mathcal{C}\,\mathcal{V}^*\\  
    \mathcal{P} & = & \_ \mid \mathcal{C}\,\mathcal{P}^*
 \end{array}
\]
\end{subfigure}
\hspace{0.5cm}
\begin{subfigure}[b]{0.4\linewidth}
 \renewcommand*{\arraystretch}{1.5}
 \[
 \begin{array}{cr}
   \inbr{v;\,\_} & \ruleno{Wildcard} \\
   \trule{\forall i\;\inbr{v_i;\,p_i}}{\inbr{C^k\,v_1\dots v_k;\,C^k\,p_1\dots p_k}},\,k\ge 0 & \ruleno{Constructor}
 \end{array}
 \]
\end{subfigure}
\end{figure}

\begin{figure}
   \renewcommand*{\arraystretch}{2}
%   \setarrow{\xrightarrow}
   \setsubarrow{_*}
   \[
   \begin{array}{cr}
     \trule{\inbr{v;\,p_1}}
           {\withenv{i}{\trans{\inbr{v;\,p_1,\dots,p_k}}{}{i}}} & \ruleno{MatchHead}\\
     \trule{\neg\inbr{v;\,p_1}\qquad\withenv{i+1}{\trans{\inbr{v;\,p_2,\dots,p_k}}{}{j}}}
           {\withenv{i}{\trans{\inbr{v;\,p_1,\dots,p_k}}{}{j}}} & \ruleno{MatchTail}\\
     \withenv{i}{\trans{\inbr{v;\,\eps}}{}{i}} & \ruleno{MatchOtherwise}\\
     \trule{\withenv{1}{\trans{\inbr{v;\,p_1,\dots,p_k}}{}{i}}}
           {\setsubarrow{}\trans{\inbr{v;\,p_1,\dots,p_k}}{}{i}} & \ruleno{Match}
   \end{array}
   \]
\end{figure}
\note{
This is pretty standart pattern mathcing semantics. Because of this talk is prerecorded you can easily pause and look into it more closely

Our syntax supports only constructors and wild cards. Pattern vriables can be easily added after right matching branch is decided.
}
\end{frame}

\begin{frame}[fragile]
\begin{figure}
\begin{subfigure}[b]{1\linewidth}
\[
\begin{array}{rcl}
  \mathcal M & = & \bullet \\
  &   & \mathcal M\,[\mathbb{N}] \\
  \ir & = & \primi{return}\,\mathbb{N} \\
  &   & \primi{switch}\;\mathcal{M}\;\primi{with}\; [\mathcal{C}\; \primi{\rightarrow}\; \ir]^*\;\primi{otherwise}\;\ir
\end{array}
\]
\end{subfigure}
\begin{subfigure}[b]{1\linewidth}
\renewcommand*{\arraystretch}{3}
\setarrow{\xrightarrow}
\setsubarrow{_{\mathcal S}}
  \[
  \begin{array}{cr}
    \withenv{v}{\trans{\primi{return}\;i}{}{i}} &  %\ruleno{Return}
    \\
    \trule{{\setsubarrow{_{\mathcal M}}\withenv{v}{\trans{m}{}{C^k\,v_1,\dots,v_k}}}\qquad \withenv{v}{\trans{s}{}{i}}}
          {\withenv{v}{\trans{\primi{switch}\;m\;\primi{with}\;[C^k\to s]s^*\;\primi{otherwise}\;s^\prime}{}{i}}} & 
          %\ruleno{SwitchMatched}
          \\
    \trule{{\setsubarrow{_{\mathcal M}}\withenv{v}{\trans{m}{}{D^n\,v_1,\dots,v_n}}}\qquad C^k\ne D^n\qquad \withenv{v}{\trans{\primi{switch}\;m\;\primi{with}\;s^*\;\primi{otherwise}\;s^\prime}{}{i}}}
          {\withenv{v}{\trans{\primi{switch}\;m\;\primi{with}\;[C^k\to s]s^*\;\primi{otherwise}\;s^\prime}{}{i}}} &
           %\ruleno{SwitchNotMatched}
          \\
    \trule{\withenv{v}{\trans{s}{}{i}}}{\withenv{v}{\trans{\primi{switch}\;m\;\primi{with}\;\eps\;\primi{otherwise}\;s}{}{i}}} & %\ruleno{SwitchOtherwise}
  \end{array}
  \]
\end{subfigure}
\end{figure}
\end{frame}
\end{comment}

\begin{frame}[t, allowframebreaks]
\frametitle{References}
\bibliographystyle{amsalpha}
\bibliography{references}
\vspace{1cm}
\end{frame}

\end{document}
