\documentclass[submission,copyright,creativecommons]{eptcs}
\providecommand{\event}{TEASE-LP 2020} % Name of the event you are submitting to
\usepackage{breakurl}             % Not needed if you use pdflatex only.
\usepackage{underscore}           % Only needed if you use pdflatex.

\usepackage{amsmath,amssymb}
%\usepackage[english]{babel}
\usepackage{amssymb}
\usepackage{mathtools}
\usepackage{listings}
\usepackage{comment}
\usepackage{indentfirst}
\usepackage{hyperref}
\usepackage{amsthm}
\usepackage{xcolor}
\usepackage{stmaryrd}
\usepackage{eufrak}
\usepackage{placeins}
\newtheorem{theorem}{Theorem}
%\newtheorem{lemma}{Lemma}
%\newtheorem{corollary}{Corollary}
\newtheorem{hyp}{Hypothesis}
%\newtheorem{definition}{Definition}

\lstdefinelanguage{minikanren}{
basicstyle=\normalsize,
keywords={fresh},
sensitive=true,
commentstyle=\itshape\ttfamily, % \footnotesize\itshape\ttfamily,
keywordstyle=\textbf,
identifierstyle=\ttfamily,
basewidth={0.5em,0.5em},
columns=fixed,
fontadjust=true,
literate={fun}{{$\lambda\;\;$}}1 {->}{{$\to$}}3 {===}{{$\,\equiv\,$}}1 {=/=}{{$\not\equiv$}}1 {|>}{{$\triangleright$}}3 {/\\}{{$\wedge$}}2 {\\/}{{$\vee$}}2,
morecomment=[s]{(*}{*)}
}

\lstset{
mathescape=true,
language=minikanren
}

\usepackage{letltxmacro}
\newcommand*{\SavedLstInline}{}
\LetLtxMacro\SavedLstInline\lstinline
\DeclareRobustCommand*{\lstinline}{%
  \ifmmode
    \let\SavedBGroup\bgroup
    \def\bgroup{%
      \let\bgroup\SavedBGroup
      \hbox\bgroup
    }%
  \fi
  \SavedLstInline
}

\def\transarrow{\xrightarrow}
\newcommand{\setarrow}[1]{\def\transarrow{#1}}

\def\padding{\phantom{X}}
\newcommand{\setpadding}[1]{\def\padding{#1}}

\def\subarrow{}
\newcommand{\setsubarrow}[1]{\def\subarrow{#1}}

\newcommand{\trule}[2]{\frac{#1}{#2}}
\newcommand{\crule}[3]{\frac{#1}{#2},\;{#3}}
\newcommand{\withenv}[2]{{#1}\vdash{#2}}
\newcommand{\trans}[3]{{#1}\transarrow{\padding{\textstyle #2}\padding}\subarrow{#3}}
\newcommand{\ctrans}[4]{{#1}\transarrow{\padding#2\padding}\subarrow{#3},\;{#4}}
\newcommand{\llang}[1]{\mbox{\lstinline[mathescape]|#1|}}
\newcommand{\pair}[2]{\inbr{{#1}\mid{#2}}}
\newcommand{\inbr}[1]{\left<{#1}\right>}
\newcommand{\highlight}[1]{\color{red}{#1}}
%\newcommand{\ruleno}[1]{\eqno[\scriptsize\textsc{#1}]}
\newcommand{\ruleno}[1]{\mbox{[\textsc{#1}]}}
\newcommand{\rulename}[1]{\textsc{#1}}
\newcommand{\inmath}[1]{\mbox{$#1$}}
\newcommand{\lfp}[1]{fix_{#1}}
\newcommand{\gfp}[1]{Fix_{#1}}
\newcommand{\vsep}{\vspace{-2mm}}
\newcommand{\supp}[1]{\scriptsize{#1}}
\newcommand{\sembr}[1]{\llbracket{#1}\rrbracket}
\newcommand{\cd}[1]{\texttt{#1}}
\newcommand{\free}[1]{\boxed{#1}}
\newcommand{\binds}{\;\mapsto\;}
\newcommand{\dbi}[1]{\mbox{\bf{#1}}}
\newcommand{\sv}[1]{\mbox{\textbf{#1}}}
\newcommand{\bnd}[2]{{#1}\mkern-9mu\binds\mkern-9mu{#2}}
\newcommand{\meta}[1]{{\mathcal{#1}}}
\newcommand{\dom}[1]{\mathtt{dom}\;{#1}}
\newcommand{\primi}[2]{\mathbf{#1}\;{#2}}
\renewcommand{\dom}[1]{\mathcal{D}om\,({#1})}
\newcommand{\ran}[1]{\mathcal{VR}an\,({#1})}
\newcommand{\fv}[1]{\mathcal{FV}\,({#1})}
\newcommand{\tr}[1]{\mathcal{T}r_{#1}}
\newcommand{\diseq}{\not\equiv}
\newcommand{\reprfunset}{\mathcal{R}}
\newcommand{\reprfun}{\mathfrak{f}}
\newcommand{\cstore}{\Omega}
\newcommand{\cstoreinit}{\cstore_\epsilon^{init}}
\newcommand{\csadd}[3]{add(#1, #2 \diseq #3)}  %{#1 + [#2 \diseq #3]}
\newcommand{\csupdate}[2]{update(#1, #2)}  %{#1 \cdot #2}

\let\emptyset\varnothing
\let\eps\varepsilon

\title{?Certified disequality in MiniKanren?}
\author{Dmitry Rozplokhas
\institute{Higher School of Economics and \\ JetBrains Research, Russia}
\email{rozplokhas@edu.hse.ru}
\and
%Andrey Vyatkin
%\institute{Saint Petersburg State University, Russia}
%\email{dewshick@gmail.com}
%\and
Dmitry Boulytchev
\institute{Saint Petersburg State University and \\ JetBrains Research, Russia}
\email{dboulytchev@math.spbu.ru}
}
\def\titlerunning{?Certified disequality in MiniKanren?}
\def\authorrunning{D. Rozplokhas \& D. Boulytchev}
\begin{document}
\maketitle

\begin{abstract}
?abstract in abstract?
\end{abstract}

This extension provides one additional type of base goal --- similarly to the unification of two terms we can state disequality of two terms: ``$term \diseq term$''. It has a natural interpretation: we want all solutions (the values of the free variables in the terms) for which the two given terms are not syntactically equal. In our framework, we formalize a denotational semantics of a goal as some subset of the set $\reprfunset$ of all \emph{representing functions}, i.e. the functions that map every free variable into a ground term. Although for convenient usage the domain of a representing function is the set of all free variables, we maintain the important \emph{completeness condition}, which states that only the values on free variables that occur in a goal determine whether a representing function belongs to the denotational semantics of this goal. In such terms, we have the following definition of denotational semantics for a disequality goal: \[ \sembr{t_1 \diseq t_2}  =  \{\reprfun \in \reprfunset \mid \overline{\reprfun}\,(t_1)=\overline{\reprfun}\,(t_2)\}, \] where $\overline{\reprfun}\,(t)$ applies $\reprfun$ to all free variables in $t$. This definition for a new type of goals fits nicely into the general inductive definition of denotational semantics of an arbitrary goal and preserve its properties such as completeness condition.

At the same time, there is variability in how the interpreter should be extended to be able to deal with disequalities. Different MiniKanren versions use different ways, which vary in simplicity, efficiency, and usability, and we want our operational semantics to capture many of these implementations at once. There is a common part: in standard MiniKanren the state of evaluation consists of a substitution (the information accumulated from unifications up to this point) and a number of fresh variables allocated, MiniKanren with disequality constraints augments this pair with a \emph{constraint store}, which is updated as new unifications and disequalities are encountered and in the end is used to presents information about disequalities in some form. However, the definition of a constraint store and its updates differ together with its denotational interpretation. We, therefore, parametrize our operational semantics with all this.

Specifically, we assume that we are given some definition of a constraint store objects, which we will denote by $\cstore_\sigma$ (we index every constraint store with the current substitution and expect it to be the correct representation of constraints only together with this substitution --- this flexibility is required for some of the implementations), and we also given three operations (including a nullary one) with it:

\begin{enumerate}
\item Initial constraint store: $\cstoreinit$ (where $\epsilon$ is the empty substitution). It is the store, that does not contain any constraints yet.
\item Addition of a constraint to a store: $\csadd{\cstore_\sigma}{t_1}{t_2}$, which may result in a new constraint store $\cstore'_\sigma$ or a failure $\bot$, if we want to signal that the new constraint store is incompatible with the current substitution.
\item Update of a substitution in a constraint store: $\csupdate{\cstore_\sigma}{\delta}$ when we want to compose the current substitution with an additional one, which may result in a new constraint store $\cstore'_{\sigma \delta}$ or a failure $\bot$, if we want to signal that the constraint store is incompatible with the new substitution.
\end{enumerate}

With these basic operations, we can easily extend the definition of operational semantics to incorporate disequalities: for a unification goal we just use the update operation and for a disequality goal we use the addition (and in both cases we finish the current branch if we get $\bot$). The definition of operational semantics for an arbitrary goal then turns a given state into a stream of resulting states, which represent the solutions.

The main objective for our two semantics of MiniKanren was to prove the soundness and completeness of the operational one with respect to the denotational one to ensure that the set of solutions that interleaving search in MiniKanren finds for a given program is actually equivalent to its mathematical model. To extend this result on the language with disequalities we first need a way to interpret the answers in the new form. For that, we also assume we are provided with a denotational interpretation of given constraint stores: $\sembr{\cstore_\sigma} \subset \reprfunset$, which we regard as the set of all solutions satisfying the constraints in a given store. Then every answer state $(\sigma, n_r, \cstore_\sigma)$ will represent the set $[\sigma] \cap \sembr{\cstore_\sigma}$, where $[\sigma]$ is the set of all representing functions extending the substitution $\sigma$. And we can formulate the soundness and completeness as follows.

\begin{theorem}[Operational semantics soundness and completeness]
If all free variables in a goal $g$ belong to the set $\{\alpha_1,\dots,\alpha_n\}$, then

\[
\bigcup_{(\sigma, n_r, \cstore_\sigma) \in Tr(g, (\epsilon, n, \cstoreinit))}  {[\sigma] \cap \sembr{\cstore_\sigma}}  \qquad =_n \qquad \sembr{g},
\]

where $Tr(g, st)$ is the stream of states that is the operational semantics of a goal $g$ with initial state $st$, and `$=_n$' means the sets are equal if we restrict the domains of representing functions in them to free variables $\{\alpha_1,\dots,\alpha_n\}$.
\end{theorem}

To be able to prove it we, of course, need certain requirements for the given operations on constraint stores. We came up with the following list of sufficient conditions on them for soundness and completeness.

\begin{enumerate}
\item $\sembr{\cstoreinit} = \reprfunset$
\item $\csadd{\cstore_\sigma}{t_1}{t_2} = \cstore'_\sigma \implies \sembr{\cstore'_\sigma} \cap [\sigma] = \sembr{\cstore_\sigma} \cap \sembr{t_1 \diseq t_2} \cap [\sigma]$
\item $\csadd{\cstore_\sigma}{t_1}{t_2} = \bot \implies \sembr{\cstore_\sigma} \cap \sembr{t_1 \diseq t_2} \cap [\sigma] = \emptyset$
\item $\csupdate{\cstore_\sigma}{\delta} = \cstore'_{\sigma \delta} \implies \sembr{\cstore'_{\sigma \delta}} \cap [\sigma \delta] = \sembr{\cstore_\sigma} \cap [\sigma \delta]$
\item $\csupdate{\cstore_\sigma}{\delta} = \bot \implies \sembr{\cstore_\sigma} \cap [\sigma \delta] = \emptyset$
\end{enumerate}

These conditions ensure that the given denotational interpretation and the given operations on constraint stores are adequate relative to each other. The conditions 2-5 describe exactly what we need to prove soundness and completeness for base goals (unification and disequality), at the same time, these conditions have relatively simple intuitive meaning in terms of these two operations and are expected to hold naturally in all reasonable implementations of constraint stores. We can prove that this is enough for soundness and completeness to hold for an arbitrary goal. However, contrary to our expectations, the proof for other types of goals, except base ones, is not working for this new formulation and has to be modified significantly in the case of \lstinline|fresh|. Specifically, we need one additional condition on constraint store in state $(\sigma, n, \cstore_\sigma)$: only the values on the first $n$ fresh variables determine whether a representing function belongs to the denotational semantics $[\sigma] \cap \sembr{\cstore_\sigma}$ of the state (note the similarity to the completeness condition). Luckily, we can infer this property for all states that can be constructed by our operational semantics from the necessary conditions in the list above.

Thus, for an arbitrary implementation, we need to give a formal definition of constraint store object and its denotational interpretation, provide three operations for it and prove five conditions on them, and by this we ensure that for arbitrary goal the interpretations of all solutions found by the search in this version of MiniKanren will cover exactly the mathematical interpretation of this goal. One thing we can not guarantee though is that the solutions will be presented to the user in a way that reflects their interpretation correctly. So we expect provided denotational interpretation for constraint store to be exactly what is put on the screen. {\color{red} Something that sounds more impressive is needed here probably.}

As well as our previous development this extension is certified in Coq\footnote{\color{red} TODO: create a new branch in the miniKanren-coq repository and put a link here}. We describe operational semantics and its soundness and completeness as modules parametrized by the definitions of constraint stores and our necessary conditions. We also provide a module with an example of a very obvious (and very impractical) suitable implementation of disequality constraints (with lists of pairs of terms as constraint stores) and extract a correct-by-construction reference interpreter for MiniKanren with disequality constraints from the operational semantics with this simple implementation passed as the parameter.

In addition to verification of correctness of different implementation of disequality constraints, we potentially can use our framework to formally state and prove some of its other important properties (for example, we currently allow meaningless answers with empty interpretations, which don't affect soundness or completeness, but we probably want ``good'' implementations not to have such things) or correctness of some optimizations inside specific implementation (for example, throwing out a constraint that is subsumed by some other constraint in the store).

\pagebreak





The optional arguments of {\tt $\backslash$documentclass$\{$eptcs$\}$} are
\begin{itemize}
\item at most one of
{\tt adraft},
{\tt submission} or
{\tt preliminary},
\item at most one of {\tt publicdomain} or {\tt copyright},
\item and optionally {\tt creativecommons},
  \begin{itemize}
  \item possibly augmented with
    \begin{itemize}
    \item {\tt noderivs}
    \item or {\tt sharealike},
    \end{itemize}
  \item and possibly augmented with {\tt noncommercial}.
  \end{itemize}
\end{itemize}
We use {\tt adraft} rather than {\tt draft} so as not to confuse hyperref.
The style-file option {\tt submission} is for papers that are
submitted to {\tt $\backslash$event}, where the value of the latter is
to be filled in in line 2 of the tex-file. Use {\tt preliminary} only
for papers that are accepted but not yet published. The final version
of your paper that is to be uploaded at the EPTCS website should have
none of these style-file options.

By means of the style-file option
\href{http://creativecommons.org/licenses/}{creativecommons}
authors equip their paper with a Creative Commons license that allows
everyone to copy, distribute, display, and perform their copyrighted
work and derivative works based upon it, but only if they give credit
the way you request. By invoking the additional style-file option {\tt
noderivs} you let others copy, distribute, display, and perform only
verbatim copies of your work, but not derivative works based upon
it. Alternatively, the {\tt sharealike} option allows others to
distribute derivative works only under a license identical to the
license that governs your work. Finally, you can invoke the option
{\tt noncommercial} that let others copy, distribute, display, and
perform your work and derivative works based upon it for
noncommercial purposes only.

Authors' (multiple) affiliations and emails use the commands
{\tt $\backslash$institute} and {\tt $\backslash$email}.
Both are optional.
Authors should moreover supply
{\tt $\backslash$titlerunning} and {\tt $\backslash$authorrunning},
and in case the copyrightholders are not the authors also
{\tt $\backslash$copyrightholders}.
As illustrated above, heuristic solutions may be called for to share
affiliations. Authors may apply their own creativity here \cite{multipleauthors}.

Exactly 46 lines fit on a page.
The rest is like any normal {\LaTeX} article.
We will spare you the details.
The rest is like any normal {\LaTeX} article.
We will spare you the details.\\
The rest is like any normal {\LaTeX} article.
We will spare you the details.\\
The rest is like any normal {\LaTeX} article.
We will spare you the details.\\
The rest is like any normal {\LaTeX} article.
We will spare you the details.\\
The rest is like any normal {\LaTeX} article.
We will spare you the details.\hfill6\\
The rest is like any normal {\LaTeX} article.
We will spare you the details.\\
The rest is like any normal {\LaTeX} article.
We will spare you the details.\\
The rest is like any normal {\LaTeX} article.
We will spare you the details.\\
The rest is like any normal {\LaTeX} article.
We will spare you the details.\\
The rest is like any normal {\LaTeX} article.
We will spare you the details.\hfill11\\
The rest is like any normal {\LaTeX} article.
We will spare you the details.\\
The rest is like any normal {\LaTeX} article.
We will spare you the details.

Here starts a new paragraph. The rest is like any normal {\LaTeX} article.
We will spare you the details.
The rest is like any normal {\LaTeX} article.
We will spare you the details.\\
The rest is like any normal {\LaTeX} article.
We will spare you the details.\hfill16\\
The rest is like any normal {\LaTeX} article.
We will spare you the details.\\
The rest is like any normal {\LaTeX} article.
We will spare you the details.\\
The rest is like any normal {\LaTeX} article.
We will spare you the details.\\
The rest is like any normal {\LaTeX} article.
We will spare you the details.\\
The rest is like any normal {\LaTeX} article.
We will spare you the details.\hfill21\\
The rest is like any normal {\LaTeX} article.
We will spare you the details.\\
The rest is like any normal {\LaTeX} article.
We will spare you the details.\\
The rest is like any normal {\LaTeX} article.
We will spare you the details.\\
The rest is like any normal {\LaTeX} article.
We will spare you the details.\\
The rest is like any normal {\LaTeX} article.
We will spare you the details.\hfill26\\
The rest is like any normal {\LaTeX} article.
We will spare you the details.\\
The rest is like any normal {\LaTeX} article.
We will spare you the details.\\
The rest is like any normal {\LaTeX} article.
We will spare you the details.\\
The rest is like any normal {\LaTeX} article.
We will spare you the details.\\
The rest is like any normal {\LaTeX} article.
We will spare you the details.\hfill31\\
The rest is like any normal {\LaTeX} article.
We will spare you the details.\\
The rest is like any normal {\LaTeX} article.
We will spare you the details.\\
The rest is like any normal {\LaTeX} article.
We will spare you the details.\\
The rest is like any normal {\LaTeX} article.
We will spare you the details.\\
The rest is like any normal {\LaTeX} article.
We will spare you the details.\hfill36\\
The rest is like any normal {\LaTeX} article.
We will spare you the details.\\
The rest is like any normal {\LaTeX} article.
We will spare you the details.\\
The rest is like any normal {\LaTeX} article.
We will spare you the details.\\
The rest is like any normal {\LaTeX} article.
We will spare you the details.\\
The rest is like any normal {\LaTeX} article.
We will spare you the details.\hfill41\\
The rest is like any normal {\LaTeX} article.
We will spare you the details.\\
The rest is like any normal {\LaTeX} article.
We will spare you the details.\\
The rest is like any normal {\LaTeX} article.
We will spare you the details.\\
The rest is like any normal {\LaTeX} article.
We will spare you the details.\\
The rest is like any normal {\LaTeX} article.
We will spare you the details.\hfill46\\
The rest is like any normal {\LaTeX} article.
We will spare you the details.
The rest is like any normal {\LaTeX} article.
We will spare you the details.
The rest is like any normal {\LaTeX} article.
We will spare you the details.
The rest is like any normal {\LaTeX} article.
We will spare you the details.
The rest is like any normal {\LaTeX} article.
We will spare you the details.
The rest is like any normal {\LaTeX} article.
We will spare you the details.
The rest is like any normal {\LaTeX} article.
We will spare you the details.
The rest is like any normal {\LaTeX} article.
We will spare you the details.
The rest is like any normal {\LaTeX} article.
We will spare you the details.
The rest is like any normal {\LaTeX} article.
We will spare you the details.
The rest is like any normal {\LaTeX} article.
We will spare you the details.
The rest is like any normal {\LaTeX} article.
We will spare you the details.
The rest is like any normal {\LaTeX} article.
We will spare you the details.
The rest is like any normal {\LaTeX} article.
We will spare you the details.
The rest is like any normal {\LaTeX} article.
We will spare you the details.
The rest is like any normal {\LaTeX} article.
We will spare you the details.
The rest is like any normal {\LaTeX} article.
We will spare you the details.

\section{Ancillary files}

Authors may upload ancillary files to be linked alongside their paper.
These can for instance contain raw data for tables and plots in the
article, or program code.  Ancillary files are included with an EPTCS
submission by placing them in a directory \texttt{anc} next to the
main latex file. See also \url{https://arxiv.org/help/ancillary_files}.
Please add a file README in the directory \texttt{anc}, explaining the
nature of the ancillary files, as in
\url{http://eptcs.org/paper.cgi?226.21}.

\section{Prefaces}

Volume editors may create prefaces using this very template,
with {\tt $\backslash$title$\{$Preface$\}$} and {\tt $\backslash$author$\{\}$}.

\section{Bibliography}

We request that you use
\href{http://eptcs.web.cse.unsw.edu.au/eptcs.bst}
{\tt $\backslash$bibliographystyle$\{$eptcs$\}$}
\cite{bibliographystylewebpage}, or one of its variants
\href{http://eptcs.web.cse.unsw.edu.au/eptcsalpha.bst}{eptcsalpha},
\href{http://eptcs.web.cse.unsw.edu.au/eptcsini.bst}{eptcsini} or
\href{http://eptcs.web.cse.unsw.edu.au/eptcsalphaini.bst}{eptcsalphaini}
\cite{bibliographystylewebpage}. Compared to the original {\LaTeX}
{\tt $\backslash$biblio\-graphystyle$\{$plain$\}$},
it ignores the field {\tt month}, and uses the extra
bibtex fields {\tt eid}, {\tt doi}, {\tt ee} and {\tt url}.
The first is for electronic identifiers (typically the number $n$
indicating the $n^{\rm th}$ paper in an issue) of papers in electronic
journals that do not use page numbers. The other three are to refer,
with life links, to electronic incarnations of the paper.

\paragraph{DOIs}

Almost all publishers use digital object identifiers (DOIs) as a
persistent way to locate electronic publications. Prefixing the DOI of
any paper with {\tt http://dx.doi.org/} yields a URI that resolves to the
current location (URL) of the response page\footnote{Nowadays, papers
  that are published electronically tend
  to have a \emph{response page} that lists the title, authors and
  abstract of the paper, and links to the actual manifestations of
  the paper (e.g.\ as {\tt dvi}- or {\tt pdf}-file). Sometimes
  publishers charge money to access the paper itself, but the response
  page is always freely available.}
of that paper. When the location of the response page changes (for
instance through a merge of publishers), the DOI of the paper remains
the same and (through an update by the publisher) the corresponding
URI will then resolve to the new location. For that reason a reference
ought to contain the DOI of a paper, with a life link to the corresponding
URI, rather than a direct reference or link to the current URL of
publisher's response page. This is the r\^ole of the bibtex field {\tt doi}.
{\bf EPTCS requires the inclusion of a DOI in each cited paper, when available.}

DOIs of papers can often be found through
\url{http://www.crossref.org/guestquery};\footnote{For papers that will appear
  in EPTCS and use \href{http://eptcs.web.cse.unsw.edu.au/eptcs.bst}
  {\tt $\backslash$bibliographystyle$\{$eptcs$\}$} there is no need to
  find DOIs on this website, as EPTCS will look them up for you
  automatically upon submission of a first version of your paper;
  these DOIs can then be incorporated in the final version, together
  with the remaining DOIs that need to found at DBLP or publisher's webpages.}
the second method {\it Search on article title}, only using the {\bf
surname} of the first-listed author, works best.  
Other places to find DOIs are DBLP and the response pages for cited
papers (maintained by their publishers).

\paragraph{The bibtex fields {\tt ee} and {\tt url}}

Often an official publication is only available against payment, but
as a courtesy to readers that do not wish to pay, the authors also
make the paper available free of charge at a repository such as
\url{arXiv.org}. In such a case it is recommended to also refer and
link to the URL of the response page of the paper in such a
repository.  This can be done using the bibtex fields {\tt ee} or {\tt
url}, which are treated as synonyms.  These fields should \textbf{not} be used
to duplicate information that is already provided through the DOI of
the paper.
You can find archival-quality URL's for most recently published papers
in DBLP---they are in the bibtex-field {\tt ee}---but please suppress
repetition of DOI information. In fact, it is often
useful to check your references against DBLP records anyway, or just find
them there in the first place.

\paragraph{Typesetting DOIs and URLs}

When using {\LaTeX} rather than {\tt pdflatex} to typeset your paper, by
default no linebreaking within long URLs is allowed. This leads often
to very ugly output, that moreover is different from the output
generated when using {\tt pdflatex}. This problem is repaired when
invoking \href{http://eptcs.web.cse.unsw.edu.au/breakurl.sty}
{\tt $\backslash$usepackage$\{$breakurl$\}$}: it allows linebreaking
within links and yield the same output as obtained by default with
{\tt pdflatex}. 
When invoking {\tt pdflatex}, the package {\tt breakurl} is ignored.

Please avoid using {\tt $\backslash$usepackage$\{$doi$\}$}, or
{\tt $\backslash$newcommand$\{\backslash$doi$\}$}.
If you really need to redefine the command {\tt doi}
use {\tt $\backslash$providecommand$\{\backslash$doi$\}$}.

The package {\tt $\backslash$usepackage$\{$underscore$\}$} is
recommended to deal with underscores in DOIs. This is not needed when
using {\tt $\backslash$usepackage$\{$breakurl$\}$} and not {\tt pdflatex}.

\paragraph{References to papers in the same EPTCS volume}

To refer to another paper in the same volume as your own contribution,
use a bibtex entry with
\begin{center}
  {\tt series    = $\{\backslash$thisvolume$\{5\}\}$},
\end{center}
where 5 is the submission number of the paper you want to cite.
You may need to contact the author, volume editors or EPTCS staff to
find that submission number; it becomes known (and unchangeable)
as soon as the cited paper is first uploaded at EPTCS\@.
Furthermore, omit the fields {\tt publisher} and {\tt volume}.
Then in your main paper you put something like:

\noindent
{\small \tt $\backslash$providecommand$\{\backslash$thisvolume$\}$[1]$\{$this
  volume of EPTCS, Open Publishing Association$\}$}

\noindent
This acts as a placeholder macro-expansion until EPTCS software adds
something like

\noindent
{\small \tt $\backslash$newcommand$\{\backslash$thisvolume$\}$[1]%
  $\{\{\backslash$eptcs$\}$ 157$\backslash$opa, pp 45--56, doi:\dots$\}$},

\noindent
where the relevant numbers are pulled out of the database at publication time.
Here the newcommand wins from the providecommand, and {\tt \small $\backslash$eptcs}
resp.\ {\tt \small $\backslash$opa} expand to

\noindent
{\small \tt $\backslash$sl Electronic Proceedings in Theoretical Computer Science} \hfill and\\
{\small \tt , Open Publishing Association} \hfill .

\noindent
Hence putting {\small \tt $\backslash$def$\backslash$opa$\{\}$} in
your paper suppresses the addition of a publisher upon expansion of the citation by EPTCS\@.
An optional argument like
\begin{center}
  {\tt series    = $\{\backslash$thisvolume$\{5\}[$EPTCS$]\}$},
\end{center}
overwrites the value of {\tt \small $\backslash$eptcs}.

\nocite{*}
\bibliographystyle{eptcs}
\bibliography{generic}
\end{document}
