\section{Related Works}

The study of formal semantics for logic programming languages, in the first place \textsc{Prolog}, is a well-established research domain. Early
works~\cite{JonesMycroftSemantics,DebrayMishraSemantics} addressed the computational aspects of both pure \textsc{Prolog} and its extension
with the cut construct. Recently, the application of certified/mechanized approaches came into focus as well. In particular,
in~\cite{CertifiedPrologEquivalences} the equivalence of a few differently defined operational semantics
for pure \textsc{Prolog} is proven, and in~\cite{CeritfiedDenotationalCut} a denotational semantics for \textsc{Prolog} with cut is presented; both
works provide \textsc{Coq}-mechanised proofs. It is interesting that the former one also advocates the use of higher-order
abstract syntax. We are not aware of any prior works on certified semantics for \textsc{Prolog} which contributed a correct-by-construction
interpreter. Our certified description of SLD resolution with cut can be considered as a certified semantics for \textsc{Prolog} modulo
occurs check in unification (which \textsc{Prolog} does not have by default).

The implementation of first-order unification in dependently typed languages constitutes a well-known challenge with a number of
known solutions. The major difficulty comes from the non-structural recursivity of conventional unification algorithms, which
requires to provide a witness for convergence. The standard approach is to define a generally-recursive function and a well-founded order
for its arguments. This route is taken in~\cite{MGUinLCF,MGUinMLTT,IdempMGUinCoq,TextbookMGUinCoq}, where the descriptions of
unification algoriths are given in \textsc{LCF}, \textsc{Alf}, \textsc{Coq} and \textsc{Coq} respectively. As a well-founded
order a lexicographically ordered tuples, containing the information about the number of different free variables and the sizes of
the arguments, is used. We implemented a similar approach, but we separated the test for the non-matching case into a dedicated
function. Thus, we make a recursive call only when the current substitution extension is guaranteed, which allows us to use the
number of different free variables as the order. Alternatively, in~\cite{StructuralMGU} a structurally recursive definition of
unification algorithm is given; this is achieved by indexing the arguments with the numbers of their free variables.

The use of higher-order abstract syntax (HOAS) for dealing with language constructs in \textsc{Coq} was addressed in earlier work~\cite{HOASinCoq},
where it was employed to describe lambda calculus. The inconsistence phenomenon of HOAS representation, mentioned in Section~\ref{language}, is called
there ``exotic terms'' and is handled using a dedicated inductive predicate ``\lstinline|Valid_v|''. The predicate has a non-trivial implementation based
on subtle observations on bindings behavior. Our case, however, is much simpler: there is not much variety in ``exotic terms'' (for example, we do not have
reductions in terms), and our predicate ``\lstinline|consistent_binding|'' can be considered as a limited version of ``\lstinline|Valid_v|'' for a
smaller language.

The study of formal semantics for \textsc{miniKanren} is also not a completely novel venture. In~\cite{RelConversion} a non-deterministic
small-step semantics is described, and in~\cite{DivTest} a big-step semantics for a finite number of answers is given;
neither uses proof mechanization, and in both works the interleaving is not addressed. 

The most important property of interleaving search~--- completeness~--- was postulated in the introductory paper~\cite{Search}, and is delivered by
all major implemetations. In~\cite{2016} a formal proof of completeness is presented; however, the completeness is understood as a
preservation of all answers during the interleaving of answer streams, i.e. in a more narrow sense than in our case since no relation
to denotational semantics is established; additionally no proof mechanization is used.

The work of~\cite{MechanisingMiniKanren} can be considered as our direct predecessor. It also introduces both denotational and
operational semantics and presents a \textsc{HOL}-certified proof for the soundness of the latter w.r.t. the former. The denotational
semantics resembles ours but considers only queries with a single free variable (we do not see this restriction as important).
On the other hand, the operational semantics is nondeterministic (similarly to~\cite{RelConversion}), which makes it
impossible to express interleaving and extract the interpreter in a direct way. In addition, a specific form of ``executable semantics''
is introduced, but its connection to the other two is not established. Finally, no completeness result is presented.
We consider our completeness proof as an essential improvement. 




