\section{Equivalence of Semantics}
\label{equivalence}

Now when we defined two different kinds of semantics for \textsc{miniKanren} we can relate them and show that the results given by these two semantics are the same for any specification.
This will actually say something important about the search in the language: since operational semantics describes precisely the behavior of the search and denotational semantics
ignores the search and describes what we \emph{should} get from a mathematical point of view, by proving their equivalence we establish \emph{completeness} of the search which
means that the search will get all answers satisfying the described specification and only those.

But first, we need to relate the answers produced by these two semantics as they have different forms: a trace of substitutions (along with numbers of allocated variables)
for the operational one and a set of representing functions for the denotational one. We can notice that the notion of representing function is close to substitution, with only two differences:

\begin{itemize}
\item representing function is total;
\item terms in the domain of representing function are ground.
\end{itemize}

Therefore we can easily extend (perhaps ambiguously) any substitution to a representing function by composing it with an arbitrary representing function and that will
preserve all variable dependencies in the substitution. So we can define a set of representing functions corresponding to substitution as follows:

\[
[\sigma] = \{\overline{\mathfrak f} \circ \sigma \mid \mathfrak{f}:\mathcal{A}\mapsto\mathcal{D}\}
\]

In \textsc{Coq} this notion boils down to the following definition:

\begin{lstlisting}[language=Coq,basicstyle=\footnotesize]
   Definition in_denotational_sem_subst
     (s : subst) (f : repr_fun) : Prop :=
       exists (f' : repr_fun),
         repr_fun_eq (subst_repr_fun_compose s f') f.
\end{lstlisting}

where ``\lstinline[language=Coq]|repr_fun_eq|'' stands for representing functions extensional equality, ``\lstinline[language=Coq]|subst_repr_fun_compose|''~---
for a composition of a substitution and a representing function.

And \emph{denotational analog} of an operational semantics (a set of representing functions corresponding to answers in the trace) for given extended state $s$ is
then defined as a union of sets for all substitution in the trace:

\[
\sembr{s}_{op} = \cup_{(\sigma, n) \in \tr{s}} [\sigma]
\]

In \textsc{Coq} we again use a proposition instead:

\begin{lstlisting}[language=Coq,basicstyle=\footnotesize]   
   Definition in_denotational_analog
      (t : trace) (f : repr_fun) : Prop :=
      exists s n, in_stream (Answer s n) t /\
             in_denotational_sem_subst s f.
   Notation "{| t , f |}" := (in_denotational_analog t f).
\end{lstlisting}

This allows us to state theorems relating the two semantics.

\begin{theorem}[Operational semantics soundness]
If indices of all free variables in a goal $g$ are limited by some number $n$, then

\[
\sembr{\inbr{g, \epsilon, n}}_{op} \subset \sembr{g}.
\]
\end{theorem}

It can be proven by nested induction, but first, we need to generalize the statement so that the inductive hypothesis would be strong enough for the inductive step.
To do so, we define denotational semantics not only for goals but for arbitrarily extended states. Note that this definition does not need to have any intuitive
interpretation, it is introduced only for proof to go smoothly. The definition of the denotational semantics for extended states is in Figure~\ref{denotational_semantics_of_states}.
The generalized version of the theorem uses it:

\begin{figure}[t]
  \[
  \begin{array}{ccl}
    \sembr{\Diamond}&=&\emptyset\\
    \sembr{\inbr{g, \sigma, n}}&=&\sembr{g}\cap[\sigma]\\
    \sembr{s_1 \oplus s_2}&=&\sembr{s_1}\cup\sembr{s_2}\\
    \sembr{s \otimes g}&=&\sembr{s}\cap\sembr{g}\\
  \end{array}
  \]
  \caption{Denotational semantics of states}
  \label{denotational_semantics_of_states}
\end{figure}

\begin{lemma}[Generalized soundness]
For any well-formed extended state $s$

\[
\sembr{s}_{op} \subset \sembr{s}.
\]
\end{lemma}

It can be proven by induction on the number of steps in which a given answer (more accurately, the substitution that contains it) occurs in the trace.
The induction step is proven by structural induction on the extended state $s$.

It would be tempting to formulate the completeness of operational semantics as the inverse inclusion, but it does not hold in such generality. The reason for
this is that denotational semantics encodes only dependencies between the free variables of a goal, which is reflected by the completeness condition, while
operational semantics may also contain dependencies between semantic variables allocated in ``\lstinline|fresh|''. Therefore we formulate the completeness
with representing functions restricted on the semantic variables allocated in the beginning (which includes all free variables of a goal). This does not
compromise our promise to prove the completeness of the search as \textsc{miniKanren} provides the result as substitutions only for queried variables,
which are allocated in the beginning.

\begin{theorem}[Operational semantics completeness]
If indices of all free variables in a goal $g$ are limited by some number $n$, then

\[
\{\mathfrak{f}|_{\{\alpha_1,\dots,\alpha_n\}} \mid \mathfrak{f} \in \sembr{g}\} \subset \{\mathfrak{f}|_{\{\alpha_1,\dots,\alpha_n\}} \mid \mathfrak{f} \in \sembr{\inbr{g, \epsilon, n}}_{op}\}.
\]
\end{theorem}


Similarly to the soundness, this can be proven by nested induction, but the generalization is required. This time it is enough to generalize it from goals
to states of the shape $\inbr{g, \sigma, n}$. We also need to introduce one more auxiliary semantics --- bounded denotational semantics:

\[
\sembr{\bullet}^l : \mathcal{G} \to 2^{\mathcal{A}\to\mathcal{D}}
\]

Instead of always unfolding the definition of a relation for invocation goal, it does so only given number of times. So for a given set of relational
definitions $\{R_i^{k_i} = \lambda\;x_1^i\dots x_{k_i}^i\,.\, g_i;\}$ the definition of bounded denotational semantics is exactly the same as in usual denotational semantics,
except that for the invocation case:

\[
\sembr{R_i^{k_i}\,(t_1,\dots,t_{k_i})}^{l+1} = \sembr{g_i[t_1/x_1^i, \dots, t_{k_i}/x_{k_i}^i]}^{l}
\]

It is convenient to define bounded semantics for level zero as an empty set:

\[
\sembr{g}^{0} = \emptyset
\]

Bounded denotational semantics is an approximation of a usual denotational semantics and it is clear that any answer in usual denotational semantics will also be in
bounded denotational semantics for some level:

\begin{lemma}
$\sembr{g} \subset \cup_l \sembr{g}^l$
\end{lemma}

Formally it can be proven using the definition of the least fixed point from Tarski-Knaster theorem: the set on the right-hand side is a closed set.

Now the generalized version of the completeness theorem is as follows:

\begin{lemma}[Generalized completeness]
For any set of relational definitions, for any level $l$, for any well-formed state $\inbr{g, \sigma, n}$,

\[
\{\mathfrak{f}|_{\{\alpha_1,\dots,\alpha_n\}} \mid \mathfrak{f} \in \sembr{g}^l \cap [\sigma]\} \subset \{\mathfrak{f}|_{\{\alpha_1,\dots,\alpha_n\}} \mid \mathfrak{f} \in \sembr{\inbr{g, \sigma, n}}_{op}\}.
\]
\end{lemma}

It is proven by induction on the level $l$. The induction step is proven by structural induction on the goal $g$.

The proofs of both theorems are certified in \textsc{Coq}; for completeness we can not just use the induction on proposition \lstinline|in_denotational_sem_goal|, as it would be natural to expect,
because the inductive principle it provides is not flexible enough. So we need to define bounded denotational semantics in our formalization too and perform
induction on the level explicitly:

\begin{lstlisting}[language=Coq,basicstyle=\footnotesize,morekeywords={where,at,level}]
  Reserved Notation "[| n | g , f |]" (at level 0).
  Inductive in_denotational_sem_lev_goal :
    nat -> goal -> repr_fun -> Prop :=
    ...
  | dslgInvoke : forall l r t f,
      [| l  | proj1_sig (Prog r) t , f |] ->
      [| S l | Invoke r t , f |]
  where "[| n | g , f |]" :=
    (in_denotational_sem_lev_goal n g f).
\end{lstlisting}

Recall that the environment ``\lstinline[language=Coq]|Prog|'' maps every relational symbol to the definition of relation,
which is a pair of a function from terms to goals and a proof that it is closed and consistent.
So ``\lstinline[language=Coq]|(proj1_sig (Prog r) t)|'' here simply takes the body of the corresponding relation.

The lemma relating bounded and unbounded denotational semantics in \textsc{Coq} looks as follows:

\begin{lstlisting}[language=Coq,basicstyle=\footnotesize]
  Lemma in_denotational_sem_some_lev:
    forall (g : goal) (f : repr_fun),
      [| g , f |] -> exists l, [| l | g , f |].
\end{lstlisting}

The statements of the theorems are as follows:

\begin{lstlisting}[language=Coq,basicstyle=\footnotesize]
  Theorem search_correctness:
    forall (g : goal) (k : nat) (f : repr_fun) (t : trace),
      closed_goal_in_context (first_nats k) g) ->
      op_sem (State (Leaf g empty_subst k)) t) ->
      {| t , f |} ->
      [| g , f |].
  Theorem search_completeness:
    forall (g : goal) (k : nat) (f : repr_fun) (t : trace),
      consistent_goal g ->
      closed_goal_in_context (first_nats k) g ->
      op_sem (State (Leaf g empty_subst k)) t ->
      [| g , f |] ->
      exists (f' : repr_fun),
        {| t , f' |} /\
        forall (x : var), In x (first_nats k) -> f x = f' x.
\end{lstlisting}

