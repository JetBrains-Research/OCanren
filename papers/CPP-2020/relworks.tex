\section{Related Works}

The study of formal semantics for logic programming languages, in the first place \textsc{Prolog}, is a well-established research domain. Early
works~\cite{JonesMycroftSemantics,DebrayMishraSemantics} addressed the computational aspects of both pure \textsc{Prolog} and its extension
with the cut construct. Recently the application of certified/mechanized approaches came into focus as well. In particular,
in~\cite{CertifiedPrologEquivalences} the equivalence of a few differently defined operational semantics
for pure \textsc{Prolog} is proven, and in~\cite{CeritfiedDenotationalCut} a denotational semantics for \textsc{Prolog} with cut is presented; both
works provide \textsc{Coq}-mechanised proofs. It is interesting that the former one also advocates the use of higher-order
abstract syntax. We are not aware of any prior works on certified semantics for \textsc{Prolog} which contributed a correct-by-construction
interpreter.

The study of formal semantics for \textsc{miniKanren} is also not a completely novel venture. In~\cite{RelConversion} a non-deterministic
small-step semantics is described, and in~\cite{DivTest} a big-step semantics for finite number of answers is given;
neither works use proof mechanization, and in both the intrleaving is not addressed. 

The work of~\citet{MechanisingMiniKanren} can be considered as our direct predecessor. It also introduces both denotational and
operational semantics and presents a HOL-certified proof for the soundness of the latter w.r.t. the former. The denotational
semantics resembles ours, but considers only queries with a single free variable (we do not see this restriction as important).
On the other hand, the operational semantics is nondeterministic (similarly to~\cite{RelConversion}), which makes it
impossible to express interleaving and extract an interpreter in a direct way. In addition, a specific form of ``executable semantics''
is introduced, but its connection to the other two is not established. Finally, no completeness result is presented.
We consider our completeness result as an essential improvement. 




