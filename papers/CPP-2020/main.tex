\documentclass[sigplan,10pt,anonymous,review]{acmart}
\settopmatter{printfolios=true,printccs=false,printacmref=false}

%%%%%%%%%%%%%%%%%%%%%%%%%%%%%%%%%%%%%%%%%%%%%%%%%%%%%%%%%%%%%%%%%%%%%%
%% Note: Authors migrating a paper from PACMPL format to traditional
%% SIGPLAN proceedings format must update the '\documentclass' and
%% topmatter commands above; see 'acmart-sigplanproc-template.tex'.
%%%%%%%%%%%%%%%%%%%%%%%%%%%%%%%%%%%%%%%%%%%%%%%%%%%%%%%%%%%%%%%%%%%%%%


%% Some recommended packages.
\usepackage{booktabs}   %% For formal tables:
                        %% http://ctan.org/pkg/booktabs
\usepackage{subcaption} %% For complex figures with subfigures/subcaptions
                        %% http://ctan.org/pkg/subcaption


\usepackage{amsmath,amssymb}
\usepackage[russian,english]{babel}
\usepackage{amssymb}
\usepackage{mathtools}
\usepackage{listings}
\usepackage{comment}
\usepackage{indentfirst}
\usepackage{hyperref}
\usepackage{amsthm}
\usepackage{stmaryrd}
\usepackage{eufrak}
\usepackage{lstcoq}

\newtheorem{theorem}{Theorem}
\newtheorem{lemma}{Lemma}
\newtheorem{corollary}{Corollary}
\newtheorem{hyp}{Hypethesis}
\newtheorem{definition}{Definition}

\lstdefinelanguage{minikanren}{
basicstyle=\small,
keywords={fresh},
sensitive=true,
commentstyle=\small\itshape\ttfamily,
keywordstyle=\textbf,
identifierstyle=\ttfamily,
basewidth={0.5em,0.5em},
columns=fixed,
fontadjust=true,
literate={fun}{{$\lambda\;\;$}}1 {->}{{$\to$}}3 {===}{{$\,\equiv\,$}}1 {=/=}{{$\not\equiv$}}1 {|>}{{$\triangleright$}}3 {/\\}{{$\wedge$}}2 {\\/}{{$\vee$}}2,
morecomment=[s]{(*}{*)}
}

\lstset{
mathescape=true,
language=minikanren
}

\usepackage{letltxmacro}
\newcommand*{\SavedLstInline}{}
\LetLtxMacro\SavedLstInline\lstinline
\DeclareRobustCommand*{\lstinline}{%
  \ifmmode
    \let\SavedBGroup\bgroup
    \def\bgroup{%
      \let\bgroup\SavedBGroup
      \hbox\bgroup
    }%
  \fi
  \SavedLstInline
}

\def\transarrow{\xrightarrow}
\newcommand{\setarrow}[1]{\def\transarrow{#1}}

\def\padding{\phantom{X}}
\newcommand{\setpadding}[1]{\def\padding{#1}}

\def\subarrow{}
\newcommand{\setsubarrow}[1]{\def\subarrow{#1}}

\newcommand{\trule}[2]{\frac{#1}{#2}}
\newcommand{\crule}[3]{\frac{#1}{#2},\;{#3}}
\newcommand{\withenv}[2]{{#1}\vdash{#2}}
\newcommand{\trans}[3]{{#1}\transarrow{\padding{\textstyle #2}\padding}\subarrow{#3}}
\newcommand{\ctrans}[4]{{#1}\transarrow{\padding#2\padding}\subarrow{#3},\;{#4}}
\newcommand{\llang}[1]{\mbox{\lstinline[mathescape]|#1|}}
\newcommand{\pair}[2]{\inbr{{#1}\mid{#2}}}
\newcommand{\inbr}[1]{\left<{#1}\right>}
\newcommand{\highlight}[1]{\color{red}{#1}}
%\newcommand{\ruleno}[1]{\eqno[\scriptsize\textsc{#1}]}
\newcommand{\ruleno}[1]{\mbox{\scriptsize{[\textsc{#1}]}}}
\newcommand{\rulename}[1]{\textsc{#1}}
\newcommand{\inmath}[1]{\mbox{$#1$}}
\newcommand{\lfp}[1]{fix_{#1}}
\newcommand{\gfp}[1]{Fix_{#1}}
\newcommand{\vsep}{\vspace{-2mm}}
\newcommand{\supp}[1]{\scriptsize{#1}}
\newcommand{\sembr}[1]{\llbracket{#1}\rrbracket}
\newcommand{\cd}[1]{\texttt{#1}}
\newcommand{\free}[1]{\boxed{#1}}
\newcommand{\binds}{\;\mapsto\;}
\newcommand{\dbi}[1]{\mbox{\bf{#1}}}
\newcommand{\sv}[1]{\mbox{\textbf{#1}}}
\newcommand{\bnd}[2]{{#1}\mkern-9mu\binds\mkern-9mu{#2}}
\newcommand{\meta}[1]{{\mathcal{#1}}}
\newcommand{\dom}[1]{\mathtt{dom}\;{#1}}
\newcommand{\primi}[2]{\mathbf{#1}\;{#2}}
\renewcommand{\dom}[1]{\mathcal{D}om\,({#1})}
\newcommand{\ran}[1]{\mathcal{VR}an\,({#1})}
\newcommand{\fv}[1]{\mathcal{FV}\,({#1})}
\newcommand{\tr}[1]{\mathcal{T}r_{#1}}

\newcommand{\searchRule}[6] {
  #1, #2 \vdash (#3, #4) \xRightarrow{#5} #6}
\newcommand{\extSearchRule}[8] {
  #1, #2, #3, #4 \vdash (#5, #6) \xRightarrow{#7}_{e} #8}
\newcommand{\q}{\hspace{0.5em}}
\newcommand{\bigcdot}{\boldsymbol{\cdot}}
\newcommand{\bigslant}[2]{{\raisebox{.2em}{$#1$}\left/\raisebox{-.2em}{$#2$}\right.}}

\let\emptyset\varnothing
\let\eps\varepsilon

\sloppy

%% Bibliography style
\bibliographystyle{ACM-Reference-Format}
%% Citation style
%% Note: author/year citations are required for papers published as an
%% issue of PACMPL.
\citestyle{acmauthoryear}   %% For author/year citations


\begin{document}

%% Title information
\title{Certified Semantics for Relational Programming} %% [Short Title] is optional;
                                           %% when present, will be used in
                                           %% header instead of Full Title.
%\titlenote{This work was partially suppored by the grant 18-01-00380 from The Russian Foundation for Basic Research} %% \titlenote is optional;
                                        %% can be repeated if necessary;
                                        %% contents suppressed with 'anonymous'
%\subtitle{Subtitle}                     %% \subtitle is optional
%\subtitlenote{with subtitle note}       %% \subtitlenote is optional;
                                        %% can be repeated if necessary;
                                        %% contents suppressed with 'anonymous'


%% Author information
%% Contents and number of authors suppressed with 'anonymous'.
%% Each author should be introduced by \author, followed by
%% \authornote (optional), \orcid (optional), \affiliation, and
%% \email.
%% An author may have multiple affiliations and/or emails; repeat the
%% appropriate command.
%% Many elements are not rendered, but should be provided for metadata
%% extraction tools.

\author{Dmitry Rozplokhas}
\affiliation{%
  \institution{Higher School of Economics}}
\affiliation{%
  \institution{JetBrains Research}
  \country{Russia}}
\email{darozplokhas@edu.hse.ru}

\author{Andrey Vyatkin}
\affiliation{%
  \institution{Saint Petersburg State University}
  \country{Russia}}
\email{dewshick@gmail.com}

\author{Dmitry Boulytchev}
\affiliation{%
  \institution{Saint Petersburg State University}}
\affiliation{%
  \institution{JetBrains Research}
  \country{Russia}}
\email{dboulytchev@math.spbu.ru}



%% Abstract
%% Note: \begin{abstract}...\end{abstract} environment must come
%% before \maketitle command
\begin{abstract}
  We present a formal study of semantics for relational programming language \textsc{miniKanren}. First,
  we formulate its denotational variant which corresponds to the minimal Herbrand model for definite logic
  programs. Second, we present operational semantics which models the destinctive feature of \textsc{miniKanren}
  implementation~---interleaving,~--- and prove its soundness and completeness w.r.t. the denotational semantics.
  Our development is supported by \textsc{Coq} specification, from which a reference interpreter can be
  extracted. We also derive from our main result a certified semantics (and a reference interpreter) for Prolog
  with cut and prove its soundness.   
\end{abstract}


%% 2012 ACM Computing Classification System (CSS) concepts
%% Generate at 'http://dl.acm.org/ccs/ccs.cfm'.
\begin{CCSXML}
<ccs2012>
<concept>
<concept_id>10003752.10003790.10003795</concept_id>
<concept_desc>Theory of computation~Constraint and logic programming</concept_desc>
<concept_significance>500</concept_significance>
</concept>
<concept>
<concept_id>10003752.10010124.10010131.10010133</concept_id>
<concept_desc>Theory of computation~Denotational semantics</concept_desc>
<concept_significance>500</concept_significance>
</concept>
<concept>
<concept_id>10003752.10010124.10010131.10010134</concept_id>
<concept_desc>Theory of computation~Operational semantics</concept_desc>
<concept_significance>500</concept_significance>
</concept>
</ccs2012>
\end{CCSXML}
\ccsdesc[500]{Theory of computation~Constraint and logic programming}
\ccsdesc[500]{Theory of computation~Denotational semantics}
\ccsdesc[500]{Theory of computation~Operational semantics}
%% End of generated code


%% Keywords
%% comma separated list
\keywords{Relational programming, denotational semantics, operational semantics, certified programming}  %% \keywords are mandatory in final camera-ready submission


%% \maketitle
%% Note: \maketitle command must come after title commands, author
%% commands, abstract environment, Computing Classification System
%% environment and commands, and keywords command.
\maketitle
\thispagestyle{empty}

\section{Introduction}


  \begin{comment}
We implemented the synthesis framework using \textsc{OCanren}~--- an embedding of \textsc{miniKanren} into \textsc{OCaml}~\cite{ocanren},~---
 and evaluated it on the set of benchmarks, reported in the previous works on \emph{ad-hoc} algorithms for pattern matching
 code generation~\cite{maranget2001,maranget2008}. In comparison with a simplified setting, presented above, our implementation
 deals with a more elaborate pattern matching problem~--- in particular, we support \emph{guard expressions}, name bindings in
 patterns and incorporate a deterministic top-down matching strategy, which is common in functional languages.
 
 Initially, our synthesis did not demonstrate good results. However, we applied the following techniques to improve both the performance
 and the quality of synthesized programs:
 
 \begin{itemize}
 \item we restricted the shape of scrutinees using type information;
 \item we utilized tabling to memoize repeating search steps;
 \item we implemented a pruning technique, which makes the search stop exploring a certain branch if the program, synthesized so far,
   contains too much nesting constructs (this factor can be precomputed by patterns analysis).
 \end{itemize}
 
 With these adjustments, our synthesis framework in a negligible time provides the same results as those reported in the previous works.
 Our future steps include extending the pattern matching language to completely match that of \textsc{OCaml} (for
 now we do not support GADTs), integrate the synthesis into the existing \textsc{OCaml} compiler and evaluate it on a
 set of real-world programs. Another direction is extending the pattern matching language to incorporate features which
 are known to be hard, tedious or error-prone to implement (for example, non-linear patterns).
 
\end{comment}


\begin{comment}

Algebraic data types are essential for typed functional programming and it's difficult to imagine effective compiler without effective compilation of pattern matching. 
There are a few different approaches for compiling pattern mathcing. GHC is using influential paper~\cite{Jones1987}, OCaml is currently based on~\cite{maranget2001} although a work~\cite{maranget2008} can slightly improve effectiveness of generated code. 

Also there are a number of possible extensions of pattern matching itself (guards, non-linear patterns, active patterns) and extensions of possible matchable values (polymorphic variants in OCaml, for example). Although having all these extensions can be helpful for programming in practice, they can complicate compilation schema or make it very difficult to generate effective code. Supporting a large number  of extensions can seriously complicate compiler's implementation too.

We present an approach to pattern matching code generation based on application of relational programming~\cite{TRS,WillThesis} and, in
 particular, relational interpreters~\cite{unified}. We expect that our approach can compile pattern mathcing to competitive code and will be easier to support during adding of new pattern matching extensions.
 
\end{comment}
 We use a \emph{relational interpreter} for $\ir$
 
 \[
 eval^o_{\ir}\, (s, p, i)
 \]
 
 Here $s$ and $i$ have the same meaning as in declarative semantics description, $p\in\ir$~--- a syntactic representation of
 a program in $\ir$. The relation $eval^o_{\ir}$ encodes the operational semantics of $\ir$; it holds, if
 evaluating $p$ for $s$ returns $i$. Being relational interpreter, however, $eval^o_{\ir}$ is capable of solving a
 synthesis problem: by a scrutinee $s$ and a number $i$ calculate a program $p$ which makes the relation to hold.
 Within this setting, we can formulate the pattern-matching synthesis problem as follows: \emph{for a given ordered list of patterns $ps$ find a program $p$, such that}
 
 \[
 \forall s\in\mathcal{V},\,\exists i,\,eval^o_{\ir}\, (s, p, i) \wedge\, match (s, ps, i)
 \]
 
 It is rather problematic to directly solve this synthesis problem with existing \textsc{miniKanren} implementations as
 they provide a rather limited support for universal quantification~\cite{eigen,moiseenko}. However, in our concrete
 case there is a simple way to alleviate this problem. Indeed, we may replace universal quantification over $i$ by
 a finite conjunction, as the length of $ps$ is known at the synthesis time. As for the quantification over $s$, for
 any concrete $ps$ we may precompute a \emph{complete set of examples} $\mathcal{E}(ps)\subseteq\mathcal{V}$ with the following
 property:
 
 \[
 \forall i\in\mathbb{N},\,(\forall s\in\mathcal{E}(ps),\,match\, (s, ps, i) \Leftrightarrow \forall s\in\mathcal{V},\,match\, (s, ps, i))
 \]
 
 It easy to see, that for arbitrary $ps$ there exists a finite complete set of examples (indeed, any pattern describes the ``shape''
 of a scrutinee up to some finite depth, beyond which all scrutinees become indistinguishable). Thus, for a given $ps$ we may
 completely eliminate the quantification, reformulating the synthesis problem as
 
 \[
 \bigwedge_{i\in[1\dots|ps|]}\,\bigwedge_{s\in\mathcal{E}(ps)} (eval^o_{\ir}\, (s, p, i) \wedge match\, (s, ps, i))
 \]
 
 We implemented the synthesis framework using \textsc{OCanren}~--- an embedding of \textsc{miniKanren} into \textsc{OCaml}~\cite{ocanren},~---
 and evaluated it on the set of benchmarks, reported in the previous works on \emph{ad-hoc} algorithms for pattern matching
 code generation~\cite{maranget2001,maranget2008}. In comparison with a simplified setting, presented above, our implementation
 deals with a more elaborate pattern matching problem~--- in particular, we support \emph{guard expressions}, name bindings in
 patterns and incorporate a deterministic top-down matching strategy, which is common in functional languages.
 
 Initially, our synthesis did not demonstrate good results. However, we applied the following techniques to improve both the performance
 and the quality of synthesized programs:
 
 \begin{itemize}
 \item we restricted the shape of scrutinees using type information;
 \item we utilized tabling to memoize repeating search steps;
 \item we implemented a pruning technique, which makes the search stop exploring a certain branch if the program, synthesized so far,
   contains too much nesting constructs (this factor can be precomputed by patterns analysis).
 \end{itemize}
 
 With these adjustments, our synthesis framework in a negligible time provides the same results as those reported in the previous works.
 Our future steps include extending the pattern matching language to completely match that of \textsc{OCaml} (for
 now we do not support GADTs), integrate the synthesis into the existing \textsc{OCaml} compiler and evaluate it on a
 set of real-world programs. Another direction is extending the pattern matching language to incorporate features which
 are known to be hard, tedious or error-prone to implement (for example, non-linear patterns).
 
 \begin{comment}
 
 
 Real world modern compilers are obliged to address a few problems which are NP-complete and hence can't have effective
 algorithm to solve them. So, compilers use semi-optimal algorithms to find a decent solution. Optimal algorithms require
 brute force search to get the best solution and  affect compilation speed negatively. In this work we apply relational
 programming -- a convenient DSL for implementing search -- to compilation of pattern matching, one of a kind hard problems for compiler.
 
 The task of compiling pattern matching for typed languages is well presented in literature~\cite{maranget2001,maranget2008}.
 
 
 We test approach on simplified source language $PM$ where scrutinee is a value $\in\mathcal{V}$ of algebraic data type, only wildcards
 and nested constructors are allowed as patterns $\mathcal{P}$ and right hand side of clause is its index. The source language is easy
 extendable by pattern variables and optional pattern guards that test subterms of scrutinee using a function. The semantics of $PM$
 is a function from concrete scrutinee $s$, concrete patterns $pats$ and concrete guards $gs$ to clause indexes, and is denoted
 as $\sem{s,pats,gs}_{PM} = i$.
 
 Compilation scheme translates sentences from $PM$ to $\ir$ language which has constructions for clause indexes and conditions which
 test matchable values for specific constructor. Matchable values can be either a scrutinee, or a projection of matchable value that
 returns one of its field indexed by natural numbers. $\ir$ language is easy extendable by tests for fixed number of pattern guards.
 The semantics is straightforward and is denoted by $\sem{\cdot}_{\ir}$.
 
 We deal with a task of compiling pattern matching as it is a synthesis problem. The goal of algorithm is to synthesize $ideal_\ir$
 for concrete patterns $pats$ and guards that, firstly, will behave the same as original pattern matching for any possible
 scrutinee $s$. Secondly, we want shortest solution because short code usually runs faster. Relational programming~\cite{OCanren}
 will help with that because it has a tendency to generate short answers earlier, although this tendency is not strict.
 
 $$
 \forall s:\; \sem{s; ideal_\ir}_\ir = \sem{s;pats}_{PM}
 $$
 
 To eliminate universal quantifier we use the following observation: for \emph{finite} amount of patterns of \emph{finite} height
 we can generate \emph{finite} amount of examples to test pattern matching semantics. In examples, very deep subterms can have
 any value because they will not be tested during pattern matching. We can reformulate synthesis problem as follows:
 
 $$
 \mid  Examples\mid < \infty\quad \land\quad \left(\forall e \; (e\in Examples)\quad\land \quad\left( \sem{e; ideal_\ir}_\ir = \sem{e;pats}_{PM}\right)\right)
 $$
 
 For plain ADT the approach will generate required examples in finite time, but for GADTs it can diverge because inhabitancy problem
 is semi-decidable~\cite{garrigue2017gadts}(chapter 5). Inhabitants generation as well as synthesis algorithm is
 implemented\footnote{\url{https://github.com/Kakadu/pat-match/}} using relational programming.
 
 Presented approach is good as general description of an idea but require a few tweaks to start working, for example, on presented
 sample~\ref{fig:example1} . Firstly, synthesis procedure in a way as it is described doesn't take types into account, so it is
 useful to give hints about which parts of scrutinee should be checked for which constructors. Second observation says that we
 run $\sem{\cdot}_\ir$ in concrete direction, so it is possible to check periodically count of \texttt{IfTag} constructors in
 result value and prune branches where it becomes too big. Thirdly, synthesis query generates a lot of similar queries, and
 we use tabling to speedup search. All three observations are important, removing one of them leads to visible performance degradation.
 
 The optimal (two \texttt{IfTag}'s) and the semi-optimal solution (three \texttt{IfTag}'s) for~\ref{fig:example1} are described
 in~\cite{maranget2008}. Current implementation generates semi-optimal solution as 28th answer. Before that it generates optimal
 solution (and it's equivalents) three times, other 24 answers are longer and less useful. All tasks (example generation, synthesis
 and printing answers) take 3 seconds, which is unfortunate.
 
 Shortly, we present following contributions
 \begin{itemize}
 \item Code synthesis for pattern matching works after implementing \emph{three optimizations} above.
 \item GADTs, pattern binding and guards works for simple examples, the approach is easy extendable by them.
 \end{itemize}
 
 Future work is
 \begin{itemize}
 \item Discover other optimizations and enable current ones automatically using type information (at the moment we patched synthesis
   algorithm manually for concrete example).
 \item When current implementation tests for \texttt{cons} tag it can't propagate constraint that tag equals to \texttt{nil} to
   the \texttt{else} branch, which partially explains why branch pruning is so useful.
 \item Algorithm for inhabitant generation requires proper formulation and proof.
 \item Apply current synthesis procedure for exhaustiveness checking which will give us \emph{single} procedure for compilation and exhaustiveness checking.
 \item Test the approach on real world problems (embedding to OCaml compiler).
 \end{itemize}
 
 
 \begin{figure}[t]
   \[
   \begin{array}{rcll}
     \mathcal{C} & = & \{ C_1^{k_1}, \dots, C_n^{k_n} \}    &\mbox{(constructors)} \\
     \mathcal{V} & = & \mathcal{C}\,\mathcal{V}^*        &\mbox{(values)}       \\
     \mathcal{P} & = & \_ \mid \mathcal{C}\,\mathcal{P}^*&\mbox{(patterns)}     \\
     \mathcal{M} & = & \bullet \mid \mathcal{M} [\mathbb{N}]&
 
   \end{array}
   \]
 \end{figure}
 
 \begin{figure}
 \centering
 \begin{minipage}{.7\textwidth}
   \centering
 \begin{align*}
 \mathcal{C} =&\; \{ C_1^{k_1}, \dots, C_n^{k_n} \} \\
 \mathcal{V} =&\;  \mathcal{C}\ \mathcal{V}^*\\
 \mathcal{M} =&\;  \mathcal{S} \\
           \mid\; &\; \text{\texttt{Field}}\;  \mathcal{M}\times  \mathbb{N}\\
 \mathcal{P} =&\;  \text{\texttt{Wildcard}} \\
           \mid\; &\; \text{\texttt{Var}}\  Name\\
           \mid\; &\; \text{\texttt{PConstructor}}\  \mathcal{C}\times  \mathcal{P}^*\\
 \ir  =&\; \text{\texttt{Int}}\  \mathbb{N} \\
 %           \mid\; &\;\mathcal{S} \\
            \mid\; &\; \text{\texttt{IfTag}}\; \mathcal{C}\times \mathcal{M}\times \ir\times \ir\\
            \mid\; &\; \text{\texttt{IfGuard}}\ \mathbb{N}\times (Name\times \mathcal{M})^*\times \ir\times \ir\\
 Clause =&\;  \mathcal{P} \times \mathbb{N}? \times \ir
 \end{align*}
   \captionof{figure}{Structure of $PM$ and \ir languages}
 %  \label{fig:test1}
 \end{minipage}%
 \begin{minipage}{.3\textwidth}
   \centering
 \begin{lstlisting}[language=ocaml]
 match s with 
 | ([], _)     -> 1
 | (_, [])     -> 2
 | (_::_,_::_) -> 3
 \end{lstlisting}
   \captionof{figure}{Simple example of pattern matching problem from~\cite{maranget2008}}
 \label{fig:example1}
 \end{minipage}
 \end{figure}
 
 \end{comment}
 

\begin{figure*}[t]
\[
\begin{array}{cccll}
  &\mathcal{C} & = & \{C_i^{k_i}\} & \mbox{constructors with arities} \\
  &\mathcal{T}_X & = & X \cup \{C_i^{k_i} (t_1, \dots, t_{k_i}) \mid t_j\in\mathcal{T}_X\} & \mbox{terms over the set of variables $X$} \\
  &\mathcal{D} & = & \mathcal{T}_\emptyset & \mbox{ground terms}\\
  &\mathcal{X} & = & \{ x, y, z, \dots \} & \mbox{syntactic variables} \\
  &\mathcal{A} & = & \{ \alpha, \beta, \gamma, \dots \} & \mbox{semantic variables} \\
  &\mathcal{R} & = & \{ R_i^{k_i}\} &\mbox{relational symbols with arities} \\[2mm]
  &\mathcal{G} & = & \mathcal{T_X}\equiv\mathcal{T_X}   &  \mbox{unification} \\
  &            &   & \mathcal{G}\wedge\mathcal{G}     & \mbox{conjunction} \\
  &            &   & \mathcal{G}\vee\mathcal{G}       &\mbox{disjunction} \\
  &            &   & \mbox{\lstinline|fresh|}\;\mathcal{X}\;.\;\mathcal{G} & \mbox{fresh variable introduction} \\
  &            &   & R_i^{k_i} (t_1,\dots,t_{k_i}),\;t_j\in\mathcal{T_X} & \mbox{relational symbol invocation} \\[2mm]
  \phantom{XXXXXXXXXXXXXXX}&\mathcal{S} & = & \{R_i^{k_i} = \lambda\;x_1^i\dots x_{k_i}^i\,.\, g_i;\}\; g & \mbox{specification}
  %      ^
  %      |
  %  this ugly hack is due to non-working \centering
\end{array}
\]
\caption{The syntax of the source language}
\label{syntax}
\end{figure*}

\begin{figure}[t]
\centering
\[
\begin{array}{c}
  \mathcal{FV}\,(x)=\{x\}\\
  \mathcal{FV}\,(C_i^{k_i}\,(t_1,\dots,t_k))=\bigcup\mathcal{FV}\,(t_i)\\[2mm]
  \mathcal{FV}\,(t_1\equiv t_2)=\mathcal{FV}\,(t_1)\cup\mathcal{FV}\,(t_2)\\
  \mathcal{FV}\,(g_1\wedge g_2)=\mathcal{FV}\,(g_1)\cup\mathcal{FV}\,(g_2)\\
  \mathcal{FV}\,(g_1\vee g_2)=\mathcal{FV}\,(g_1)\cup\mathcal{FV}\,(g_2)\\
  \mathcal{FV}\,(\mbox{\lstinline|fresh|}\;x\;.\;g)=\mathcal{FV}\,(g)\setminus\{x\}\\
  \mathcal{FV}\,(R_i^{k_i}\,(t_1,\dots,t_k))=\bigcup\mathcal{FV}\,(t_i)
\end{array}
\]
\caption{Free variables in terms and goals}
\label{free}
\end{figure}

\section{The Language}
\label{language}
 
In this section we introduce the syntax of the language we use throughout the paper, describe the informal semantics and give some examples.

The syntax of the language is shown on Figure~\ref{syntax}. First, we fix a set of constructors $\mathcal{C}$ with known arities and consider
a set of terms $\mathcal{T}_X$ with constructors as functional symbols and variables from $X$. We parameterize this set with an alphabet of
variables, since in the semantic description we will need \emph{two} kinds of variables. The first kind, \emph{syntactic} variables, is denoted
by $\mathcal{X}$. We also consider an alphabet of \emph{relational symbols} $\mathcal{R}$ which are used to name relational definitions.
The central syntactic category in the language is \emph{goal}. In our case there are five types of goals: \emph{unification} of terms,
conjunction and disjunction of goals, fresh variable introduction and invocation of some relational definition. Thus, unification is used
as a constraint, and multiple constraints can be combined using conjunction, disjunction and recursion. For the sake of brevity we
abbreviate immediately nested ``\lstinline|fresh|'' constructs into the one, writing ``\lstinline|fresh $x$ $y$ $\dots$ . $g$|'' instead of
``\lstinline|fresh $x$ . fresh $y$ . $\dots$ $g$|''. The final syntactic category is \emph{specification} $\mathcal{S}$. It consists of a set
of relational definitions and a top-level goal. A top-level goal represents a search procedure which returns a stream of substitutions for
the free variables of the goal. The definition for set of free variables for both terms and goals is given on Figure~\ref{free}; as ``\lstinline|fresh|''
is the sole binding construct the definition is rather trivial. The language we defined is first-order, as goals can not be passed as parameters,
returned or constructed at runtime.

We now informally describe how relational search works. As we said, a goal represents a search procedure. This procedure takes a \emph{state} as input and returns a
stream of states; a state (among other information) contains a substitution which maps semantic variables into terms over semantic variables. Then the five types of
scenarios are possible (depending on the type of the goal):

\begin{itemize}
\item Unification ``\lstinline|$t_1$ === $t_2$|'' unifies terms $t_1$ and $t_2$ in the context of the substitution in the current state. If terms are unifiable,
  then their MGU is integrated into the substitution, and one-element stream is returned; otherwise the result is an empty stream.
\item Conjunction ``\lstinline|$g_1$ /\ $g_2$|'' applies $g_1$ to the current state and then applies $g_2$ to the each element of the result, concatenating
  the streams.
\item Disjunction ``\lstinline|$g_1$ \/ $g_2$|'' applies both its goals to the current state independently and then concatenates the results.
\item Fresh construct ``\lstinline|fresh $x$ . $g$|'' allocates a new semantic variable $\alpha$, substitutes all free occurrences of $x$ in $g$ with $\alpha$, and
  runs the goal.
\item Invocation ``$\lstinline|$R_i^{k_i}$ ($t_1$,...,$t_{k_i}$)|$'' finds a definition for the relational symbol $R_i^{k_i}=\lambda x_1\dots x_{k_i}\,.\,g_i$, substitutes
  all free occurrences of formal parameter $x_j$ in $g_i$ with term $t_j$ (for all $j$) and runs the goal in the current state.
\end{itemize}

We stipulate that the top-level goal is preceded by an implicit ``\lstinline|fresh|'' construct, which binds all its free variables, and that the final substitutions
for these variables constitute the result of the goal evaluation.

Conjunction and disjunction form a monadic~\cite{Monads} interface with conjunction playing role of ``bind'' and disjunction~--- of ``mplus''. In this description
we swept a lot of important details under the carpet~--- for example, in actual implementations the components of disjunction are not evaluated in isolation, but
both disjuncts are being evaluated incrementally with the control passing from one disjunct to another (\emph{interleaving}); instead streams the implementation
can be based on ``ferns''~\cite{BottomAvoiding} to defer divergent computations, etc. 

As an example consider the following specification:

\begin{lstlisting}
  append$^o$ = fun x y xy .
    ((x === Nil) /\ (xy === y)) \/
    (fresh h t ty .
       (x  === Cons (h, t))  /\
       (xy === Cons (h, ty)) /\
       (append$^o$ y t ty)
    );
  revers$^o$ = fun x y .
    ((x === Nil) /\ (y === Nil)) \/
    (fresh h t t' .
       (x === Cons (h, t)) /\
       (append$^o$ t' (Cons (h, Nil)) y) /\
       (revers$^o$ t t') 
    );
  revers$^o$ x x
\end{lstlisting}

Here we defined\footnote{We respect here a conventional tradition for \textsc{miniKanren} programming to superscript all relational names with ``$^o$''.}
two relational symbols~--- ``\lstinline|append$^o$|'' and ``\lstinline|revers$^o$|'',~--- and specified a top-level goal ``\lstinline|revers$^o$ x x|''.
The symbol ``\lstinline|append$^o$|'' defines a relational concatenation of lists~--- it takes three arguments and performs a case analysis on the first one. If the
first one is an empty list (``\lstinline|Nil|''), then the second and the third arguments are unified. Otherwise the first argument is deconstructed into a head ``\lstinline|h|''
and a tail ``\lstinline|t|'', and the tail is concatenated with the second argument using a recursive call to ``\lstinline|append$^o$|'' and additional variable ``\lstinline|ty|'', which
represents the concatenation of ``\lstinline|t|'' and ``\lstinline|y|''. Finally, we unify ``\lstinline|Cons (h, ty)|'' with ``\lstinline|xy|'' to form a final constraint. Similarly,
``\lstinline|revers$^o$|'' defines relational list reversing. The top-level goal represents a search procedure for all lists ``\lstinline|x|'', which are stable under reversing, i.e.
represent palindromes. Running it results in an infinite stream of substitutions:

\begin{lstlisting}
   $\alpha\;\mapsto\;$ Nil
   $\alpha\;\mapsto\;$ Cons ($\beta_0$, Nil)
   $\alpha\;\mapsto\;$ Cons ($\beta_0$, Cons ($\beta_0$, Nil))
   $\alpha\;\mapsto\;$ Cons ($\beta_0$, Cons ($\beta_1$, Cons ($\beta_0$, Nil)))
   $\dots$
\end{lstlisting}

where ``$\alpha$''~--- a \emph{semantic} variable, corresponding to ``\lstinline|x|'', ``$\beta_i$''~--- free semantics variables.

The syntax described above can be formalized in \textsc{Coq} in a natural way using inductive data types. We have made a few non-essential simplifications and modifications for the sake of convenience.
Specifically, we restrict the arities of constructors to be either zero or two:

\begin{lstlisting}[language=Coq,basicstyle=\footnotesize]
   Inductive term : Set :=
   | Var : var -> term
   | Cst : con_name -> term
   | Con : con_name -> term -> term -> term.
\end{lstlisting}

Here ``\lstinline[language=Coq]{var}'' and ``\lstinline[language=Coq]{con_name}''~--- the types representing variables and constructor names, whose definitions we omitted for the sake of brevity.
Similarly, we restrict relations to always have exactly one argument:

\begin{lstlisting}[language=Coq,basicstyle=\footnotesize]
   Definition rel : Set := term -> goal.
\end{lstlisting}

These restrictions do not make the language less expressive in any way since we can represent a sequence of terms as a list using constructors \lstinline|Nil$^0$| and \lstinline|Cons$^2$|.

We also introduce one additional type of goals~--- \emph{failure}~--- for deliberately unsuccessful computation (empty stream). As a result, the definition of goals looks as follows:

\begin{lstlisting}[language=Coq,basicstyle=\footnotesize]
   Inductive goal : Set :=
   | Fail   : goal
   | Unify  : term -> term -> goal
   | Disj   : goal -> goal -> goal
   | Conj   : goal -> goal -> goal
   | Fresh  : (var -> goal) -> goal
   | Invoke : rel_name -> term -> goal.
\end{lstlisting}

Note that in our formalization we use higher-order abstract syntax for variable binding~\cite{HOAS}. We preferred it to the first-order syntax because it gives us the ability
to use substitution and inductive principle provided by \textsc{Coq}. However, we still need to carefully ensure some expected properties on the structure of syntax trees.
For example, we should require that the definitions of relations do not contain unbound variables:

\begin{lstlisting}[language=Coq,basicstyle=\footnotesize]
   Definition closed_goal_in_context
     (c : list var) (g : goal) : Prop :=
     forall n, is_fv_of_goal n g -> In n c.
   Definition closed_rel (r : rel) : Prop :=
     forall (arg : term),
     closed_goal_in_context (fv_term arg) (r arg). 
   Definition def : Set := {r : rel | closed_rel r}.
\end{lstlisting}

In the snippet above ``\lstinline[language=Coq]{def}'' corresponds to a set of relational symbol definitions in a specification, ``\lstinline[language=Coq]{is_fv_of_goal}''
is inductively defined proposition for a free variable in a goal.

We set an arbitrary environment (a map from relational symbol to the definition of relation) to use further throughout the formalization. Failure goals allow us to define it as
a total function:

\begin{lstlisting}[language=Coq,basicstyle=\footnotesize]
   Definition env : Set := rel_name -> def.
   Variable Prog : env.
\end{lstlisting}


\begin{figure}[t]
\[
\begin{array}{rcll}
  x\,[t/x] &=& t &\\
  y\,[t/x] &=& y & y\ne x\\
  C_i^{k_i}\,(t_1,\dots,t_{k_i})\,[t/x]&=&C_i^{k_i}\,(t_1\,[t/x],\dots,t_{k_i}\,[t/x])&\\[2mm]
  (t_1 \equiv t_2)\,[t/x]&=&t_1\,[t/x] \equiv t_2\,[t/x]&\\
  (g_1 \wedge g_2)\,[t/x]&=&g_1\,[t/x] \wedge g_2\,[t/x]&\\
  (g_1 \vee g_2)\,[t/x]&=&g_1\,[t/x] \vee g_2\,[t/x]&\\
  (\mbox{\lstinline|fresh|}\;x\,.\,g)\,[t/x]&=&\mbox{\lstinline|fresh|}\;x\,.\,g&\\
  (\mbox{\lstinline|fresh|}\;y\,.\,g)\,[t/x]&=&\mbox{\lstinline|fresh|}\;y\,.\,(g\,[t/x])&y\ne x\\
  (R_i^{k_i}\,(t_1,\dots,t_{k_i})\,[t/x]&=&R_i^{k_i}\,(t_1\,[t/x],\dots,t_{k_i}\,[t/x])&
\end{array}
\]
  \caption{Substitutions for terms and goals}
  \label{substitution}
\end{figure}

\section{Denotational Semantics}
\label{denotational}

In this section we present a denotational semantics for the language we defined above. We use a simple set-theoretic
approach which can be considered as an analogy to the least Herbrand model for definite logic programs~\cite{LHM}.
Strictly speaking, instead of developing it from scratch we could have just described the conversion of specifications
into definite logic form and took their least Herbrand model. However, in that case we would still need to define
the least Herbrand model semantics for definite logic programs in a certified way. In addition, while for
this concrete language the conversion to definite logic form is trivial, it may become less trivial for
its extensions (with, for examples, nominal constructs~\cite{AlphaKanren}) which we plan to do in future.

To motivate further development, we first consider the following example. Let us have the following goal:

\begin{lstlisting}
   x === Cons (y, z)
\end{lstlisting}

There are three free variables, and solving the goal delivers us the following single answer:

\begin{lstlisting}
   $\alpha\mapsto\;$ Cons ($\beta$, $\gamma$)
\end{lstlisting}

where semantic variables $\alpha$, $\beta$ and $\gamma$ correspond to the syntactic ones ``\lstinline|x|'', ``\lstinline|y|'', ``\lstinline|z|''. The
goal does not put any constraints on ``\lstinline|y|'' and ``\lstinline|z|'', so there are no bindings for ``$\beta$'' and ``$\gamma$'' in the answer.
This answer can be seen as the following ternary relation over the set of all ground terms:

\[
\{(\mbox{\lstinline|Cons ($\beta$, $\,\gamma$)|}, \beta, \gamma) \mid \beta\in\mathcal{D},\,\gamma\in\mathcal{D}\}\subset\mathcal{D}^3
\]

The order of ``dimensions'' is important, since each dimension corresponds to a certain free variable. Our main idea is to represent this relation as a set of total
functions 

\[
\mathfrak{f}:\mathcal{A}\mapsto\mathcal{D}
\]

from semantic variables to ground terms. We call these functions \emph{representing functions}. Thus, we may reformulate the same relation as

\[
\{(\mathfrak{f}\,(\alpha),\mathfrak{f}\,(\beta),\mathfrak{f}\,(\gamma))\mid\mathfrak{f}\in\sembr{\mbox{\lstinline|$\alpha$ === Cons ($\beta$, $\,\gamma$)|}}\}
\]

where we use conventional semantic brackets ``$\sembr{\bullet}$'' to denote the semantics. For the top-level goal we need to substitute its free syntactic
variables with distinct semantic ones, calculate the semantics, and build the explicit representation for the relation as shown above. The relation, obviously,
does not depend on concrete choice of semantic variables, but depends on the order in which the values of representing functions are tupled. This order can be
conventionalized, which gives us a completely deterministic semantics.

Now we implement this idea. First, for a representing function

\[
\mathfrak{f} : \mathcal{A}\to\mathcal{D}
\]

we introduce its homomorphic extension 

\[
  \overline{\mathfrak{f}}:\mathcal{T_A}\to\mathcal{D}
\]

which maps terms to terms:

\[
\begin{array}{rcl}

  \overline{\mathfrak f}\,(\alpha) & = & \mathfrak f\,(\alpha)\\
  \overline{\mathfrak f}\,(C_i^{k_i}\,(t_1,\dots.t_{k_i})) & = & C_i^{k_i}\,(\overline{\mathfrak f}\,(t_1),\dots \overline{\mathfrak f}\,(t_{k_i}))
\end{array}
\]

Let us have two terms $t_1, t_2\in\mathcal{T_A}$. If there is a unifier for $t_1$ and $t_2$ then, clearly, there is a substitution $\theta$ which
turns both $t_1$ and $t_2$ into the same \emph{ground} term (we do not require $\theta$ to be the most general). Thus, $\theta$ maps
(some) ground variables into ground terms, and its application to $t_{1(2)}$ is exactly $\overline{\theta}(t_{1(2)})$. This reasoning can be
performed in the opposite direction: a unification $t_1\equiv t_2$ defines the set of all representing functions $\mathfrak{f}$ for which
$\overline{\mathfrak{f}}(t_1)=\overline{\mathfrak{f}}(t_2)$. 

Then, the semantic function for goals is parameterized over environments which prescribe semantic functions to relational symbols:

\[
  \Gamma : \mathcal{R} \to (\mathcal{T_A}^*\to 2^{\mathcal{A}\to\mathcal{D}})
\]

An environment associates with relational symbol a function which takes a string of terms (the arguments of the relation) and returns a set of
representing functions. The signature for semantic brackets for goals is as follows:

\[
\sembr{\bullet}_{\Gamma} : \mathcal{G}\to 2^{\mathcal{A}\to\mathcal{D}}
\]

It maps a goal into the set of representing functions w.r.t. an environment $\Gamma$.

We formulate the following important \emph{completeness condition} for the semantics of a goal $g$:

\[
\forall\alpha\not\in FV\,(g)\; \forall d \in \mathcal{D}\; \forall\mathfrak{f} \in \sembr{g}\; \exists \mathfrak{f'} \in \sembr{g} \;:\; \mathfrak{f'}\,(\alpha)\; = d \wedge \forall \beta \neq \alpha:\; \mathfrak{f'}\,(\beta)\; = \mathfrak{f}\,(\beta)\; 
\]

In other words, representing functions for a goal $g$ restrict only the values of free variables of $g$ and do not introduce any ``hidden'' correlations.
This condition guarantees that our semantics is complete in the sense that it does not introduce artificial restrictions for the relation it defines. It
can be proven that the semantics of goals always satisfy this condition.

We remind conventional notions of pointwise modification of a function

\[
f\,[x\gets v]\,(z)=\left\{
\begin{array}{rcl}
  f\,(z) &,& z \ne x \\
  v      &,& z = x
\end{array}
\right.
\]

and substitution of a free variable with a term in terms and goals (see Figure~\ref{substitution}).

For a representing function $\mathfrak{f}:\mathcal{A}\to\mathcal{D}$ and a semantic variable $\alpha$ we define
the following \emph{generalization} operation:

\[
\mathfrak{f}\uparrow\alpha = \{ \mathfrak{f}\,[\alpha\gets d] \mid d\in\mathcal D\}
\]

Informally, this operation generalizes a representing function into a set of representing functions in such a way that the
values of these functions for a given variable cover the whole $\mathcal{D}$. We extend the generalization operation for sets of
representing functions $\mathfrak{F}\subseteq\mathcal{A}\to\mathcal{D}$:

\[
  \mathfrak{F}\uparrow\alpha = \bigcup_{\mathfrak{f}\in\mathfrak{F}}(\mathfrak{f}\uparrow\alpha)
\]

Now we are ready to specify the semantics for goals (see Figure~\ref{denotational_semantics_of_goals}). We've already given the motivation for
the semantics of unification: the condition $\overline{\mathfrak{f}}(t_1)=\overline{\mathfrak{f}}(t_2)$ gives us the set of all (otherwise
unrestricted) representing functions which ``equate'' terms $t_1$ and $t_2$. Set union and intersection provide a conventional interpretation
for disjunction and conjunction of goals, and the semantics of relational invocation reduces to the application of corresponding
function from the environment. The only interesting case is ``\lstinline|fresh $x$ . $g$|''. First, we take an arbitrary semantic variable $\alpha$,
not free in $g$, and substitute $x$ with $\alpha$. Then we calculate the semantics of $g\,[\alpha/x]$. The interesting part is the next step:
as $x$ can not be free in ``\lstinline|fresh $x$ . $g$|'', we need to generalize the result over $\alpha$ since in our model the semantics of a
goal specifies a relation over its free variables. We introduce some nondeterminism, by choosing arbitrary $\alpha$, but it can be proven by structural induction, that with different choices of free variable, semantics of a goal won't change. Consider the following example:

\begin{lstlisting}
   fresh y . ($\alpha$ ===  y) /\ (y === Zero)
\end{lstlisting}

As there is no invocations involved, we can safely omit the environment. Then:

\[
\begin{array}{lcr}
  \sembr{\mbox{\lstinline|fresh y . ($\alpha$ === y) $\,\wedge\,$ (y === Zero)|}}&=&\mbox{(by \textsc{Fresh$_D$})}\\[1mm]
  (\sembr{\mbox{\lstinline|($\alpha$ === $\beta$) $\,\wedge\,$ ($\beta$ === Zero)|}})\uparrow\beta&=&\mbox{(by \textsc{Conj$_D$})}\\[1mm]
  (\sembr{\mbox{\lstinline|$\alpha$ === $\beta$|}} \,\cap\, \sembr{\mbox{\lstinline|$\beta$ === Zero)|}})\uparrow\beta&=&\mbox{(by \textsc{Unify$_D$})}\\[1mm]
  (\{\mathfrak{f}\mid \overline{\mathfrak{f}}\,(\alpha)=\overline{\mathfrak{f}}\,(\beta)\} \,\cap\, \{\mathfrak{f}\mid \overline{\mathfrak{f}}\,(\beta)=\overline{\mathfrak{f}}\,(\mbox{\lstinline|Zero|})\})\uparrow\beta&=&\mbox{(by the definition of ``$\overline{\mathfrak{f}}$'')}\\[1mm]
  (\{\mathfrak{f}\mid \mathfrak{f}\,(\alpha)=\mathfrak{f}\,(\beta)\} \,\cap\, \{\mathfrak{f}\mid \mathfrak{f}\,(\beta)=\mbox{\lstinline|Zero|}\})\uparrow\beta&=&\mbox{(by the definition of ``$\cap$'')}\\[1mm]
  (\{\mathfrak{f}\mid \mathfrak{f}\,(\alpha)=\mathfrak{f}\,(\beta)=\mbox{\lstinline|Zero|}\})\uparrow\beta&=&\mbox{(by the definition of ``$\uparrow$'')}\\[1mm]
  \{\mathfrak{f}\mid \mathfrak{f}\,(\alpha)=\mbox{\lstinline|Zero|}, \mathfrak{f}\,(\beta)=d, d\in\mathcal{D}\}&=&\mbox{(by the totality of representing functions)}\\[1mm]
  \{\mathfrak{f}\mid \mathfrak{f}\,(\alpha)=\mbox{\lstinline|Zero|}\}&&
\end{array}
\]

In the end we've got the set of representing functions, each of which restricts only the value of free variable $\alpha$. 

\begin{figure}[t]
  \[
  \begin{array}{cclr}
    \sembr{t_1\equiv t_2}_\Gamma&=&\{\mathfrak f : \mathcal{A}\to\mathcal{D}\mid \overline{\mathfrak{f}}\,(t_1)=\overline{\mathfrak{f}}\,(t_2)\}& \ruleno{Unify$_D$}\\
    \sembr{g_1\wedge g_2}_\Gamma&=&\sembr{g_1}_\Gamma\cap\sembr{g_1}_\Gamma&\ruleno{Conj$_D$}\\
    \sembr{g_1\vee g_2}_\Gamma&=&\sembr{g_1}_\Gamma\cup\sembr{g_1}_\Gamma&\ruleno{Disj$_D$}\\
    \sembr{\mbox{\lstinline|fresh|}\,x\,.\,g}_\Gamma&=&(\sembr{g\,[\alpha/x]}_\Gamma)\uparrow\alpha,\;\alpha\not\in FV(g)& \ruleno{Fresh$_D$}\\
    \sembr{R\,(t_1,\dots,t_k)}_\Gamma&=&(\Gamma\,R)\,t_1\dots t_k & \ruleno{Invoke$_D$}
  \end{array}
  \]
  \caption{Denotational semantics of goals}
  \label{denotational_semantics_of_goals}
\end{figure}

The final component is the semantics of specifications. Given a specification

\[
\{R_i=\lambda\,x_1^i\dots x_{k_i}^i\,.\,g_i;\}_{i=1}^n\;g
\]

we have to construct a correct environment $\Gamma_0$ and then take the semantics of the top-level goal:

\[
\sembr{\{R_i=\lambda\,x_1^i\dots x_{k_i}^i\,.\,g_i;\}_{i=1}^n\;g}=\sembr{g}_{\Gamma_0}
\]

As the set of definitions can be mutually recursive we apply the fixed point approach. We consider the following
function

\[
\mathcal{F} : (\mathcal{T_A}^*\to 2^{\mathcal{A}\to\mathcal{D}})^n\to (\mathcal{T_A}^*\to 2^{\mathcal{A}\to\mathcal{D}})^n
\]

which represents a semantic for the set of definitions abstracted over themselves. The definition of this function is
rather standard:

\begin{gather*}
    \begin{array}{rcl}
      \mathcal{F}\,(p_1,\dots,p_n)& = &(t^1_1\dots t^1_{k_1}\mapsto\sembr{g^1\,[t^1_1/x^1_1,\dots,t^1_{k_1}/x^1_{k_1}]}_\Gamma,\\
                                  &  &\phantom{(}\dots\\
                                  &  &\phantom{(}t^n_1\dots t^n_{k_n}\mapsto\sembr{g^n\,[t^n_1/x^n_1,\dots,t^n_{k_n}/x^n_{k_n}]}_\Gamma)
    \end{array}\\
    \mbox{where}\;\Gamma\, R_i=p_i
\end{gather*}

Here $p_i$ is a semantic function for $i$-th definition; we build an environment $\Gamma$ which associates each relational symbol
$R_i$ with $p_i$ and construct a $n$-dimensional vector-function, where $i$-th component corresponds to a function which
calculates the semantics of $i$-th relational definition application to terms w.r.t. the environment $\Gamma$. Finally,
we take the least fixed point of $\mathcal{F}$ and define the top-level environment as follows:

\[
\Gamma_0\,R_i=(fix\;\mathcal{F})\,[i]
\]

where ``$[i]$'' denotes the $i$-th component of a vector-function.

The least fixed point exists by Knaster-Tarski~\cite{TarskiKnaster} theorem~--- the set $(\mathcal{T_A}^*\to 2^{\mathcal{A}\to\mathcal{D}})^n$
forms a complete lattice, and $\mathcal{F}$ is monotonic. 

To formalize denotational semantics in \textsc{Coq} we can define representing functions simply as \textsc{Coq} functions:

\begin{lstlisting}[language=Coq]
   Definition repr_fun : Set := var -> ground_term.
\end{lstlisting}

We define the semantics via inductive proposition ``\lstinline|in_denotational_sem_goal|'' such that

\[
\forall g,\mathfrak{f}\;:\;\mbox{\lstinline|in_denotational_sem_goal|}\;g\;\mathfrak{f}\Longleftrightarrow\mathfrak{f}\in\sembr{g}_\Gamma
\]

The definition is as follows:

\begin{lstlisting}[language=Coq]
   Inductive in_denotational_sem_goal : goal -> repr_fun -> Prop :=
   | dsgUnify  : forall f t1 t2, apply_repr_fun f t1 = apply_repr_fun f t2 ->
                            in_denotational_sem_goal (Unify t1 t2) f

   | dsgDisjL  : forall f g1 g2, in_denotational_sem_goal g1 f ->
                            in_denotational_sem_goal (Disj g1 g2) f

   | dsgDisjR  : forall f g1 g2, in_denotational_sem_goal g2 f ->
                            in_denotational_sem_goal (Disj g1 g2) f

   | dsgConj   : forall f g1 g2, in_denotational_sem_goal g1 f ->
                            in_denotational_sem_goal g2 f ->
                            in_denotational_sem_goal (Conj g1 g2) f

   | dsgFresh  : forall f fn a fg, (~ is_fv_of_goal a (Fresh fg)) ->
                              in_denotational_sem_goal (fg a) fn ->
                              (forall x, x <> a -> fn x = f x) ->
                              in_denotational_sem_goal (Fresh fg) f

   | dsgInvoke : forall r t f, in_denotational_sem_goal (proj1_sig (Prog r) t) f ->
                          in_denotational_sem_goal (Invoke r t) f.
\end{lstlisting}

Here we refer to a fixpoint ``\lstinline[language=Coq]|apply_repr_fun|'' which calculates the extension ``$\overline{\bullet}$'' for a representing
function, and inductive proposition ``\lstinline[language=Coq]|is_fv_of_goal|'' which encodes the set of free variables for a goal.

Recall that the environment ``\lstinline[language=Coq]|Prog|'' maps every relational symbol to the definition of relation,
which is a pair of a function from terms to goals and a proof that it has no unbound variables.
So in the last case ``\lstinline[language=Coq]|(proj1_sig (Prog r) t)|'' simply takes the body of the corresponding relation;
thus ``\lstinline[language=Coq]|Prog|'' in \textsc{Coq} specification plays role of a global environment $\Gamma$.

It is interesting that in \textsc{Coq} implementation we do not need to refer to Tarski-Knaster theorem explicitly since
the least fixpoint semantic is implicitly provided by inductive definitions.

\section{Operational Semantics}
\label{operational}

In this section we describe the operational semantics of \textsc{miniKanren}, which corresponds to the known
implementations with interleaving search. The semantics is given in the form of a labeled transition system (LTS)~\cite{LTS}. From now on we
assume the set of semantic variables to be linearly ordered ($\mathcal{A}=\{\alpha_1,\alpha_2,\dots\}$).

We introduce the notion of substitution

\[
  \sigma : \mathcal{A}\to\mathcal{T_A}
\]

as a (partial) mapping from semantic variables to terms over the set of semantic variables. We denote $\Sigma$ the
set of all substitutions, $\dom{\sigma}$~--- the domain for a substitution $\sigma$,
$\ran{\sigma}=\bigcup_{\alpha\in\mathcal{D}om\,(\sigma)}\fv{\sigma\,(\alpha)}$~--- its range (the set of all free variables in the image).

The \emph{non-terminal states} in the transition system have the following shape:

\[
S = \mathcal{G}\times\Sigma\times\mathbb{N}\mid S\oplus S \mid S \otimes \mathcal{G}
\]

As we will see later, an evaluation of a goal is separated into elementary steps, and these steps are performed interchangeably for different subgoals. 
Thus, a state has a tree-like structure with intermediate nodes corresponding to partially-evaluated conjunctions (``$\otimes$'') or
disjunctions (``$\oplus$''). A leaf in the form $\inbr{g, \sigma, n}$ determines a goal in a context, where $g$~--- a goal, $\sigma$~--- a substitution accumulated so far,
and $n$~--- a natural number, which corresponds to a number of semantic variables used to this point. For a conjunction node, its right child is always a goal since
it cannot be evaluated unless some result is provided by the left conjunct.

The full set of states also include one separate terminal state (denoted by $\diamond$), which symbolizes the end of the evaluation.

\[
\hat{S} = \diamond \mid S
\]

We will operate with the well-formed states only, which are defined as follows.

\begin{definition}
  Well-formedness condition for extended states:
  
  \begin{itemize}
  \item $\diamond$ is well-formed;
  \item $\inbr{g, \sigma, n}$ is well-formed iff $\fv{g}\cup\dom{\sigma}\cup\ran{\sigma}\subset\{\alpha_1,\dots,\alpha_n\}$;
  \item $s_1\oplus s_2$ is well-formed iff $s_1$ and $s_2$ well-formed;
  \item $s\otimes g$ is well-formed iff $s$ is well-formed and for all leaf triplets $\inbr{\_,\_,n}$ in $s$ it is true that $\fv{g}\subseteq\{\alpha_1,\dots,\alpha_n\}$.
  \end{itemize}
  
\end{definition}

Informally the well-formedness restricts the set of states to those in which all goals use only allocated variables.

Finally, we define the set of labels:

\[
L = \step \mid \Sigma\times \mathbb{N}
\]

The label ``$\step$'' is used to mark those steps which do not provide an answer; otherwise, a transition is labeled by a pair of a substitution and a number of allocated
variables. The substitution is one of the answers, and the number is threaded through the derivation to keep track of allocated variables.

\begin{figure*}
  \renewcommand{\arraystretch}{1.6}
  \[
  \begin{array}{cr}
    \inbr{t_1 \equiv t_2, \sigma, n} \xrightarrow{\step} \Diamond , \, \, \nexists\; mgu\,(t_1 \sigma, t_2 \sigma) &\ruleno{UnifyFail} \\
    \inbr{t_1 \equiv t_2, \sigma, n} \xrightarrow{(mgu\,(t_1 \sigma, t_2 \sigma) \circ \sigma),\, n)} \Diamond & \ruleno{UnifySuccess} \\
    \inbr{g_1 \lor g_2, \sigma, n} \xrightarrow{\step} \inbr{g_1, \sigma, n} \oplus \inbr{g_2, \sigma, n} & \ruleno{Disj} \\
    \inbr{g_1 \land g_2, \sigma, n} \xrightarrow{\step} \inbr{ g_1, \sigma, n} \otimes g_2 & \ruleno{Conj} \\
    \inbr{\mbox{\lstinline|fresh|}\, x\, .\, g, \sigma, n} \xrightarrow{\step} \inbr{g\,[\bigslant{\alpha_{n + 1}}{x}], \sigma, n + 1} & \ruleno{Fresh} \\
    \dfrac{R_i^{k_i}=\lambda\,x_1\dots x_{k_i}\,.\,g}{\inbr{R_i^{k_i}\,(t_1,\dots,t_{k_i}),\sigma,n} \xrightarrow{\step} \inbr{g\,[\bigslant{t_1}{x_1}\dots\bigslant{t_{k_i}}{x_{k_i}}], \sigma, n}} & \ruleno{Invoke}\\
    \dfrac{s_1 \xrightarrow{\step} \Diamond}{(s_1 \oplus s_2) \xrightarrow{\step} s_2} & \ruleno{DisjStop}\\
    \dfrac{s_1 \xrightarrow{r} \Diamond}{(s_1 \oplus s_2) \xrightarrow{r} s_2} & \ruleno{DisjStopAns}\\
    \dfrac{s \xrightarrow{\step} \Diamond}{(s \otimes g) \xrightarrow{\step} \Diamond} &\ruleno{ConjStop}\\
    \dfrac{s \xrightarrow{(\sigma, n)} \Diamond}{(s \otimes g) \xrightarrow{\step} \inbr{g, \sigma, n}}  & \ruleno{ConjStopAns}\\
    \dfrac{s_1 \xrightarrow{\step} s'_1}{(s_1 \oplus s_2) \xrightarrow{\step} (s_2 \oplus s'_1)} &\ruleno{DisjStep}\\
    \dfrac{s_1 \xrightarrow{r} s'_1}{(s_1 \oplus s_2) \xrightarrow{r} (s_2 \oplus s'_1)} &\ruleno{DisjStepAns}\\
    \dfrac{s \xrightarrow{\step} s'}{(s \otimes g) \xrightarrow{\step} (s' \otimes g)} &\ruleno{ConjStep}\\
    \dfrac{s \xrightarrow{(\sigma, n)} s'}{(s \otimes g) \xrightarrow{\step} (\inbr{g, \sigma, n} \oplus (s' \otimes g))} & \ruleno{ConjStepAns} 
  \end{array}
  \]
  \caption{Operational semantics of interleaving search}
  \label{lts}
\end{figure*}

The transition rules are shown in Figure~\ref{lts}. The first two rules specify the semantics of unification. If two terms are not unifiable under the current substitution
$\sigma$ then the evaluation stops with no answer; otherwise, it stops with the most general unifier applied to a current substitution as an answer.

The next two rules describe the steps performed when disjunction or conjunction is encountered on the top level of the current goal. For disjunction, it schedules both goals (using ``$\oplus$'') for
evaluating in the same context as the parent state, for conjunction~--- schedules the left goal and postpones the right one (using ``$\otimes$'').

The rule for ``\lstinline|fresh|'' substitutes bound syntactic variable with a newly allocated semantic one and proceeds with the goal; no answer provided at this step.

The rule for relation invocation finds a corresponding definition, substitutes its formal parameters with the actual ones, and proceeds with the body.

The rest of the rules specify the steps performed during the evaluation of two remaining types of the states~--- conjunction and disjunction. In all cases, the left state
is evaluated first. If its evaluation stops with a result then the right state (or goal) is scheduled for evaluation, and the label is propagated. If there is no result then
the conjunction evaluation stops with no result (\textsc{ConjStop}) as well while the disjunction evaluation proceeds with the right state (\textsc{DisjStop}).

The last four rules describe \emph{interleaving}, which occurs when the evaluation of the left state suspends with some residual state (with or without an answer). In the case of disjunction
the answer (if any) is propagated, and the constituents of the disjunction are swapped (\textsc{DisjStep}, \textsc{DisjStepAns}). In the case of conjunction, if the evaluation step in
the left conjunct did not provide any answer, the evaluation is continued in the same order since there is still no information to proceed with the evaluation of the right
conjunct (\textsc{ConjStep}); if there is some answer, then the disjunction of the right conjunct in the context of the answer and the remaining conjunction is
scheduled for evaluation (\textsc{ConjStepAns}).

The introduced transition system is completely deterministic: there is exactly one transition from any non-terminal state.
There was, however, some freedom in choosing the order of evaluation for conjunction and
disjunction states. For example, instead of evaluating the left substate first we could choose to evaluate the right one, etc. In each concrete case, we would
end up with a different (but still deterministic) system that would prescribe different semantics to a concrete goal. This choice reflects the inherent
non-deterministic nature of search in relational (and, more generally, logical) programming. However, as long as deterministic search procedures
are sound and complete, we can consider them ``equivalent''\footnote{There still can be differences in observable behavior of concrete goals under different
sound and complete search strategies. For example, a goal can be refutationally complete~\cite{WillThesis} under one strategy and non-complete under another.}.

It is easy to prove that transitions preserve well-formedness of states.

\begin{lemma}{(Well-formedness preservation)}
\label{lem:well_formedness_preservation}
For any transintion $s \xrightarrow{l} \hat{s}$, if $s$ is well-formed then $\hat{s}$ is also well-formed.
\end{lemma}

A derivation sequence for a certain state determines a \emph{trace}~--- a finite or infinite sequence of answers. The trace corresponds to the stream of answers
in the reference \textsc{miniKanren} implementations. We denote a set of answers in the trace for state $\hat{s}$ by $\tr{\hat{s}}$.

We can relate sets of answers for the partialy evaluated conjunction and disjunction with sets of answers for their constituents by the two following lemmas.

\begin{lemma}
\label{lem:sum_answers}
For any non-termial states $s_1$ and $s_2$, $\tr{s_1 \oplus s_2} = \tr{s_1} \cup \tr{s_2}$.
\end{lemma}

\begin{lemma}
\label{lem:prod_answers}
For any nontermial state $s$ and goal $g$,  $\tr{s \otimes g} = \bigcup_{(\sigma, n) \in \tr{s}} \tr{\inbr{g, \sigma, n}}$.
\end{lemma}

We also can easily desribe the criterion of termination for disjunctions.

\begin{lemma}
\label{lem:disj_termination}
For any goals $g_1$ and $g_2$, sunbstitution $\sigma$, and number $n$, the trace from the state $\inbr{g_1 \vee g_2, \sigma, n}$ is finite iff the traces from both $\inbr{g_1, \sigma, n}$ and $\inbr{g_2, \sigma, n}$ are finite.
\end{lemma}

These simple statements already allow us to prove two important properties of interleaving search as colloraries: `fairness' of disjuction~--- the fact that trace for disjunction contains all the answers from both streams for disjuncts~--- and `commutativity' of disjunctions~--- the fact that swapping two disjuncts (at the top level) does not change the termination of the goal evaluation. 

\section{Equivalence of Semantics}
\label{equivalence}

Now when we defined two different kinds of semantics for \textsc{miniKanren} we can relate them and show that the results given by these two semantics are the same for any specification.
This will actually say something important about the search in the language: since operational semantics describes precisely the behavior of the search and denotational semantics
ignores the search and describes what we \emph{should} get from mathematical point of view, by proving their equivalence we establish \emph{completeness} of the search which
means that the search will get all answers satisfying the described specification and only those.

But first, we need to relate the answers produced by these two semantics as they have different forms: a trace of substitutions (along with numbers of allocated variables)
for operational and a set of representing functions for denotational. We can notice that the notion of representing function is close to substitution, with only two differences:

\begin{itemize}
\item representing function is total;
\item terms in the domain of representing function are ground.
\end{itemize}

Therefore we can easily extend (perhaps ambiguously) any substitution to a representing function by composing it with an arbitrary representing function and that will
preserve all variable dependencies in the substitution. So we can define a set of representing functions corresponding to substitution as follows:

\[
[\sigma] = \{\overline{\mathfrak f} \circ \sigma \mid \mathfrak{f}:\mathcal{A}\mapsto\mathcal{D}\}
\]

In \textsc{Coq} this notion boils down to the following definition:

\begin{lstlisting}[language=Coq,basicstyle=\footnotesize]
   Definition in_denotational_sem_subst
     (s : subst) (f : repr_fun) : Prop :=
       exists (f' : repr_fun),
         repr_fun_eq (subst_repr_fun_compose s f') f.
\end{lstlisting}

where ``\lstinline[language=Coq]|repr_fun_eq|'' stands for representing functions extensional equality, ``\lstinline[language=Coq]|subst_repr_fun_compose|''~---
for a composition of a substitution and a representing function.

And \emph{denotational analog} of an operational semantics (a set of representing functions corresponding to answers in the trace) for given extended state $s$ is
then defined as a union of sets for all substitution in the trace:

\[
\sembr{s}_{op} = \cup_{(\sigma, n) \in \tr{s}} [\sigma]
\]

In \textsc{Coq} we again use a proposition instead:

\begin{lstlisting}[language=Coq,basicstyle=\footnotesize]   
   Definition in_denotational_analog
      (t : trace) (f : repr_fun) : Prop :=
      exists s n, in_stream (Answer s n) t /\
             in_denotational_sem_subst s f.
   Notation "{| t , f |}" := (in_denotational_analog t f).
\end{lstlisting}

This allows us to state theorems relating two semantics.

\begin{theorem}[Operational semantics soundness]
If indices of all free variables in a goal $g$ are limited by some number $n$, then

\[
\sembr{\inbr{g, \epsilon, n}}_{op} \subset \sembr{g}.
\]
\end{theorem}

It can be proven by nested induction, but first, we need to generalize the statement so that the inductive hypothesis would be strong enough for the inductive step.
To do so, we define denotational semantics not only for goals but for arbitrarily extended states. Note that this definition does not need to have any intuitive
interpretation, it is introduced only for proof to go smoothly. The definition of the denotational semantics for extended states is on Figure~\ref{denotational_semantics_of_states}.
The generalized version of the theorem uses it:

\begin{figure}[t]
  \[
  \begin{array}{ccl}
    \sembr{\Diamond}&=&\emptyset\\
    \sembr{\inbr{g, \sigma, n}}&=&\sembr{g}\cap[\sigma]\\
    \sembr{s_1 \oplus s_2}&=&\sembr{s_1}\cup\sembr{s_2}\\
    \sembr{s \otimes g}&=&\sembr{s}\cap\sembr{g}\\
  \end{array}
  \]
  \caption{Denotational semantics of states}
  \label{denotational_semantics_of_states}
\end{figure}

\begin{lemma}[Generalized soundness]
For any well-formed extended state $s$

\[
\sembr{s}_{op} \subset \sembr{s}.
\]
\end{lemma}

It can be proven by induction on the number of steps in which a given answer (more accurately, the substitution that contains it) occurs in the trace.
The induction step is proven by structural induction on the extended state $s$.

It would be tempting to formulate the completeness of operational semantics as the inverse inclusion, but it does not hold in such generality. The reason for
this is that denotational semantics encodes only dependencies between the free variables of a goal, which is reflected by the completeness condition, while
operational semantics may also contain dependencies between semantic variables allocated in ``\lstinline|fresh|''. Therefore we formulate the completeness
with representing functions restricted on the semantic variables allocated in the beginning (which includes all free variables of a goal). This does not
compromise our promise to prove the completeness of the search as \textsc{miniKanren} provides the result as substitutions only for queried variables,
which are allocated in the beginning.

\begin{theorem}[Operational semantics completeness]
If indices of all free variables in a goal $g$ are limited by some number $n$, then

\[
\{\mathfrak{f}|_{\{\alpha_1,\dots,\alpha_n\}} \mid \mathfrak{f} \in \sembr{g}\} \subset \{\mathfrak{f}|_{\{\alpha_1,\dots,\alpha_n\}} \mid \mathfrak{f} \in \sembr{\inbr{g, \epsilon, n}}_{op}\}.
\]
\end{theorem}


Similarly to the soundness, this can be proven by nested induction, but the generalization is required. This time it is enough to generalize it from goals
to states of the shape $\inbr{g, \sigma, n}$. We also need to introduce one more auxiliary semantics --- bounded denotational semantics:

\[
\sembr{\bullet}^l : \mathcal{G} \to 2^{\mathcal{A}\to\mathcal{D}}
\]

Instead of always unfolding the definition of a relation for invocation goal, it does so only given number of times. So for a given set of relational
definitions $\{R_i^{k_i} = \lambda\;x_1^i\dots x_{k_i}^i\,.\, g_i;\}$ the definition of bounded denotational semantics is exactly the same as in usual denotational semantics,
except that for the invocation case:

\[
\sembr{R_i^{k_i}\,(t_1,\dots,t_{k_i})}^{l+1} = \sembr{g_i[t_1/x_1^i, \dots, t_{k_i}/x_{k_i}^i]}^{l}
\]

It is convenient to define bounded semantics for level zero as an empty set:

\[
\sembr{g}^{0} = \emptyset
\]

Bounded denotational semantics is an approximation of a usual denotational semantics and it is clear that any answer in usual denotational semantics will also be in
bounded denotational semantics for some level:

\begin{lemma}
$\sembr{g} \subset \cup_l \sembr{g}^l$
\end{lemma}

Formally it can be proven using the definition of the least fixed point from Tarski-Knaster theorem: the set on the right-hand side is a closed set.

Now the generalized version of the completeness theorem is as follows:

\begin{lemma}[Generalized completeness]
For any set of relational definitions, for any level $l$, for any well-formed (w.r.t. that set of definitions) state $\inbr{g, \sigma, n}$,

\[
\{\mathfrak{f}|_{\{\alpha_1,\dots,\alpha_n\}} \mid \mathfrak{f} \in \sembr{g}^l \cap [\sigma]\} \subset \{\mathfrak{f}|_{\{\alpha_1,\dots,\alpha_n\}} \mid \mathfrak{f} \in \sembr{\inbr{g, \sigma, n}}_{op}\}.
\]
\end{lemma}

It is proven by induction on the level $l$. The induction step is proven by structural induction on the goal $g$.

The proofs of both theorems are certified in \textsc{Coq}; for completeness we can not just use the induction on proposition \lstinline|in_denotational_sem_goal|, as it would be natural to expect,
because the inductive principle it provides is not flexible enough. So we need to define bounded denotational semantics in our formalization too and perform
induction on the level explicitly:

\begin{lstlisting}[language=Coq,basicstyle=\footnotesize,morekeywords={where,at,level}]
  Reserved Notation "[| n | g , f |]" (at level 0).
  Inductive in_denotational_sem_lev_goal :
    nat -> goal -> repr_fun -> Prop :=
    ...
  | dslgInvoke : forall l r t f,
      [| l  | proj1_sig (Prog r) t , f |] ->
      [| S l | Invoke r t , f |]
  where "[| n | g , f |]" :=
    (in_denotational_sem_lev_goal n g f).
\end{lstlisting}

The lemma relating bounded and unbounded denotational semantics in \textsc{Coq} looks like follows:

\begin{lstlisting}[language=Coq,basicstyle=\footnotesize]
  Lemma in_denotational_sem_some_lev:
    forall (g : goal) (f : repr_fun),
      [| g , f |] -> exists l, [| l | g , f |].
\end{lstlisting}

The statements of the theorems are as follows:

\begin{lstlisting}[language=Coq,basicstyle=\footnotesize]
  Theorem search_correctness:
    forall (g : goal) (k : nat) (f : repr_fun) (t : trace),
      closed_goal_in_context (first_nats k) g) ->
      op_sem (State (Leaf g empty_subst k)) t) ->
      {| t , f |} ->
      [| g , f |].
  Theorem search_completeness:
    forall (g : goal) (k : nat) (f : repr_fun) (t : trace),
      consistent_goal g ->
      closed_goal_in_context (first_nats k) g ->
      op_sem (State (Leaf g empty_subst k)) t ->
      [| g , f |] ->
      exists (f' : repr_fun),
        {| t , f' |} /\
        forall (x : var), In x (first_nats k) -> f x = f' x.
\end{lstlisting}


\section{Conclusion}

We presented a strongly-typed implementation of \miniKanren for OCaml. Our implementation
passes all tests written for \miniKanren (including those for disequality constraints);
in addition we implemented many interesting relational programs known from
the literature. We claim that our implementation can be used both as a convenient
relational DSL for OCaml and an experimental framework for future research in the area of
relational programming.

%We also want to express our gratitude to William Byrd, who infected us with relational programming,
%and for the extra time he sacrificed as both our tutor and friend.


%Text of paper \ldots


%% Acknowledgments
%\begin{acks}                            %% acks environment is optional
                                        %% contents suppressed with 'anonymous'
  %% Commands \grantsponsor{<sponsorID>}{<name>}{<url>} and
  %% \grantnum[<url>]{<sponsorID>}{<number>} should be used to
  %% acknowledge financial support and will be used by metadata
  %% extraction tools.
 % This material is based upon work supported by the
  %\grantsponsor{GS100000001}{National Science
   % Foundation}{http://dx.doi.org/10.13039/100000001} under Grant
  %No.~\grantnum{GS100000001}{nnnnnnn} and Grant
  %No.~\grantnum{GS100000001}{mmmmmmm}.  Any opinions, findings, and
  %conclusions or recommendations expressed in this material are those
  %of the author and do not necessarily reflect the views of the
  %National Science Foundation.
%\end{acks}


%% Bibliography
\bibliography{main}


%% Appendix
%\appendix
%\section{Appendix}

%Text of appendix \ldots

\end{document}
