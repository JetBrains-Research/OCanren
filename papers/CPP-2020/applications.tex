\section{Application}

In this section we consider some applications of the framework and the results, described in the previous sections.

\subsection{Correctness of Transformations}

One important immediate corollary of these theorems is the correctness of certain program transformations. Since the results obtained by the search on a
specification are exactly the results from the mathematical model of this specification, after the transformations of relations that do not change their
mathematical meaning the search will obtain the same results. Note that this way we guarantee only the stability of results as the set of ground terms,
the other aspects of program behavior, such as termination, may be affected. This allows us to safely (to a certain extent) apply such natural
transformations as:

\begin{itemize}
\item changing the order of constituents in conjunction or disjunction;
\item swapping conjunction and disjunction using distributivity;
\item moving fresh variable introduction.
\end{itemize}

and even transform relational definitions to some kinds of normal form (like all fresh variables introduction on the top level with the
conjunctive normal form inside), which may be convenient, for example, for metacomputation.

\subsection{SLD Semantics}

Interlving to SLD --- changing order in disjunctions:

\[
  \begin{array}{cr}
    \dfrac{s_1 \xrightarrow{\circ} s'_1}{(s_1 \oplus s_2) \xrightarrow{\circ} (s'_1 \oplus s_2)} &\ruleno{DisjStep}\\[5mm]
    \dfrac{s_1 \xrightarrow{r} s'_1}{(s_1 \oplus s_2) \xrightarrow{r} (s'_1 \oplus s_2)} &\ruleno{DisjStepAns}\\[5mm]
  \end{array}
\]

\subsection{Cut}

Now, cuts.

New states:

\[
S = \mathcal{G}\times\Sigma\times\mathbb{N}\mid S\oplus S \mid  S \circledast S \mid S \otimes \mathcal{G}
\]


Changing \textsc{ConjStepAns} to

  \[
  \begin{array}{cr}
    \dfrac{s \xrightarrow{(\sigma, n)} s'}{(s \otimes g) \xrightarrow{\circ} (\inbr{g, \sigma, n} \circledast (s' \otimes g))} & \ruleno{ConjStepAns} 
  \end{array}
  \]
  
Adding simple rules for asterisk (the same as those for plus):

  \[
  \begin{array}{cr}
    \dfrac{s_1 \xrightarrow{\circ} \Diamond}{(s_1 \circledast s_2) \xrightarrow{\circ} s_2} & \ruleno{AstStop}\\[5mm]
    \dfrac{s_1 \xrightarrow{r} \Diamond}{(s_1 \circledast s_2) \xrightarrow{r} s_2} & \ruleno{AstStopAns}\\[5mm]
    \dfrac{s_1 \xrightarrow{\circ} s'_1}{(s_1 \circledast s_2) \xrightarrow{\circ} (s'_1 \circledast s_2)} &\ruleno{AstStep}\\[5mm]
    \dfrac{s_1 \xrightarrow{r} s'_1}{(s_1 \circledast s_2) \xrightarrow{r} (s'_1 \circledast s_2)} &\ruleno{AstStepAns}\\[5mm]
  \end{array}
\]
  
Cut signal creation:

  \[
  \begin{array}{cr}
    \inbr{!, \sigma, n} \xrightarrow{(\sigma, n)}_c \Diamond &\ruleno{UnifyFail} \\[2mm]
  \end{array}
\]

Cut signal for pluses and asterisks:

  \[
  \begin{array}{cr}
    \dfrac{s_1 \xrightarrow{\circ}_c \Diamond}{(s_1 \oplus s_2) \xrightarrow{\circ} \Diamond} & \ruleno{PlusStopC}\\[5mm]
    \dfrac{s_1 \xrightarrow{r}_c \Diamond}{(s_1 \oplus s_2) \xrightarrow{r} \Diamond} & \ruleno{PlusStopAnsC}\\[5mm]
    \dfrac{s_1 \xrightarrow{\circ}_c s'_1}{(s_1 \oplus s_2) \xrightarrow{\circ} s'_1} &\ruleno{PlusStepC}\\[5mm]
    \dfrac{s_1 \xrightarrow{r}_c s'_1}{(s_1 \oplus s_2) \xrightarrow{r} s'_1} &\ruleno{PlusStepAnsC}\\[5mm]
    \dfrac{s_1 \xrightarrow{\circ}_c \Diamond}{(s_1 \circledast s_2) \xrightarrow{\circ}_c \Diamond} & \ruleno{AstStopC}\\[5mm]
    \dfrac{s_1 \xrightarrow{r}_c \Diamond}{(s_1 \circledast s_2) \xrightarrow{r}_c \Diamond} & \ruleno{AstStopAnsC}\\[5mm]
    \dfrac{s_1 \xrightarrow{\circ}_c s'_1}{(s_1 \circledast s_2) \xrightarrow{\circ}_c s'_1} &\ruleno{AstStepC}\\[5mm]
    \dfrac{s_1 \xrightarrow{r}_c s'_1}{(s_1 \circledast s_2) \xrightarrow{r}_c s'_1} &\ruleno{AstStepAnsC}\\[5mm]
  \end{array}
  \]
  
And cut signal propogation in crosses (just adding $c$ to all arrows)
