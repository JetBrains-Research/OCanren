\documentclass[10pt, oneside, nocopyrightspace]{sigplanconf}

\usepackage{amssymb}
\usepackage{listings}
\usepackage{indentfirst}
\usepackage{verbatim}
\usepackage{amsmath, amsthm, amssymb}
\usepackage{graphicx}
\usepackage[hyphens]{url}
\usepackage[hidelinks]{hyperref}

\topmargin 2.0cm
\setlength{\textheight}{25.3cm}

\lstdefinelanguage{ocaml}{
keywords={fresh, let, begin, end, in, match, type, and, fun, function, try, with, when, class, 
object, method, of, rec, repeat, until, while, not, do, done, as, val, inherit, 
new, module, sig, deriving, datatype, struct, if, then, else, open, private, virtual},
sensitive=true,
basicstyle=\small,
commentstyle=\small\itshape\ttfamily,
keywordstyle=\ttfamily\underbar,
identifierstyle=\ttfamily,
basewidth={0.5em,0.5em},
columns=fixed,
fontadjust=true,
literate={->}{{$\;\;\to\;\;$}}1,
morecomment=[s]{(*}{*)}
}

\lstset{
basicstyle=\small,
identifierstyle=\ttfamily,
keywordstyle=\bfseries,
commentstyle=\scriptsize\rmfamily,
basewidth={0.5em,0.5em},
fontadjust=true,
%escapechar=~,
language=ocaml
}

\sloppy

\newcommand{\miniKanren}{\texttt{miniKanren}}

\begin{document}

\title{Typed Embedding of a Relational Language in OCaml}

\authorinfo{Dmitry Kosarev \and Dmitri Boulytchev}
{St.Petersburg State University \\ 
  Saint-Petersburg, Russia }
{$\mathtt{Dmitrii.Kosarev@protonmail.ch}$ \and $\mathtt{dboulytchev@math.spbu.ru}$}

\maketitle

\small
\begin{abstract}
\small
We present an implementation of the relational programming language miniKanren as a set 
of combinators and syntax extension for OCaml. The key feature of our approach is 
\emph{polymorphic unification}, which can be used to unify data structures of almost 
arbitrary types. In addition we provide a useful generic programming pattern to 
systematically develop relational specifications in a typed manner, and address 
the problem of relational and functional code integration.
\end{abstract}

\section{Introduction}
\label{intro}

Relational programming~\cite{TRS} is an attractive technique, based on the idea 
of constructing programs as relations.  As a result, relational programs can be
``queried'' in various ``directions'', making it possible, for example, to simulate
reversed execution. Apart from being interesting from purely theoretical standpoint, 
this approach may have a practical value: some problems look much simpler, 
if they are considered as queries to relational specification. There is a 
number of appealing examples, confirming this observation: a type checker 
for simply typed lambda calculus (and, at the same time, type inferencer and solver 
for the inhabitation problem), an interpreter (capable of generating ``quines''~--- 
programs, producing themselves as result)~\cite{Untagged}, list sorting (capable of 
producing all permutations), etc. 

Many logic programming languages, such as Prolog, Mercury\footnote{\url{https://mercurylang.org}}, 
or Curry\footnote{\url{http://www-ps.informatik.uni-kiel.de/currywiki}} to some extent
can be considered as relational. We have chosen miniKanren\footnote{\url{http://minikanren.org}} 
as model language, because it was specifically designed as relational DSL, embedded in Scheme/Racket. 
Being rather a minimalistic language, which can be implemented with just a few data structures and
combinators, miniKanren found its way in dozens of host languages, including Haskell, 
Standard ML, and OCaml.

There is, however, a predictable glitch in implementing miniKanren for a strongly typed language. 
Designed in a metaprogramming-friendly and dynamically typed realm of Scheme/Racket, original 
miniKanren implementation pays very little attention to what has a significant importance in (specifically) 
ML or Haskell. In particular, one of capstone constructs of miniKanren~--- unification~--- has to work for 
different data structures, which may have types, different beyond parametricity.

There are a few ways to overcome this problem. The first one is simply to follow the untyped paradigm and
provide unification for some concrete type, rich enough to represent any reasonable data structures.
Some Haskell miniKanren libraries\footnote{\url{https://github.com/JaimieMurdock/HK}, \url{https://github.com/rntz/ukanren}}
as well as existing OCaml implementation\footnote{\url{https://github.com/lightyang/minikanren-ocaml}} take this approach. 
As a result, the original implementation can be retold with all its elegance; relational specifications, however,
become weakly typed. Another approach is to utilize \emph{ad hoc} polymorphism and provide type-specific
unification for each ``interesting'' type; Molog\footnote{\url{https://github.com/acfoltzer/Molog}} and 
MiniKanrenT\footnote{\url{https://github.com/jvranish/MiniKanrenT}}, both for Haskell, can be mentioned as examples.
While preserving strong typing, this approach requires a lot of ``boilerplate'' code to be written, so some
automation, for example, using Template Haskell\footnote{\url{https://wiki.haskell.org/Template_Haskell}},
is desirable. There is, actually, another potential approach, but we do not know, if anybody tried
it: to implement unification for generic representation of types as sum-of-products and fixpoints of 
functors~\cite{InstantGenerics, ALaCarte}. Thus, unification would work for any types, for which representation
is provided. We assume that implementing representation would require less boilerplate code.

As follows from this exposition, typed embedding of miniKanren in OCaml can be done with
a combination of datatype-generic programming~\cite{DGP} and \emph{ad hoc} polymorphism. There are a 
number of generic frameworks for OCaml (for example,~\cite{Deriving}). On the other hand, the support
for \emph{ad hoc} polymorphism in OCaml is weak; there is nothing comparable in power with Haskell 
type classes, and despite sometimes object-oriented layer of the language can be used to mimic
desirable behavior, the result as a rule is far from satisfactory. Existing proposals (for example, 
module implicits~\cite{Implicits}) require patching the compiler, which we tend to avoid.

We present an implementation of a set of relational combinators in OCaml, which, 
technically speaking, corresponds to $\mu$Kanren~\cite{MicroKanren} with disequality 
constraints~\cite{CKanren}; syntax extension for ``\lstinline{fresh}'' construct is
added as well. The contribution of our work is as follows:

\begin{enumerate}
\item Our implementation is based on \emph{polymorphic unification}, which, like polymorphic comparison,
can be used for almost arbitrary types. The implementation of polymorphic unification uses unsafe features and
relies on intrinsic knowledge of runtime representation of values; we show, however, that this does not
compromise type safety. Practically, we applied purely \emph{ad hoc} approach since the features, 
which would provide less \emph{ad hoc} solution, are not yet integrated into the mainstream language.

\item We describe a uniform and scalable pattern for using types for relational programming, which
helps in converting typed data to- and from relational domain. With this pattern, only one
generic feature (``\lstinline{map/morphism/Functor}'') is needed, and thus virtually any generic 
framework for OCaml can be used. Despite being rather a pragmatic observation, this pattern, as we
believe, would lead to  more regular and easy to maintain relational specifications.

\item We provide a simplified way to integrate relational and functional code. Our approach utilizes
well-known pattern~\cite{Unparsing, DoWeNeed} for variadic function implementation and makes it
possible to hide refinement of answers phase from an end-user.
\end{enumerate}

The rest of the paper is organized as follows: in the next section we discuss polymorphic
unification, and show, that standard unification with triangular substitution respects
typing. Then we present our approach to handle user-defined types by injecting them 
into logic domain. Next section describes top-level primitives and addresses the problem of
relational and functional code integration. Then, we present a complete example of relational
specification, written with the aid of our library. The final section concludes.

We expect from reader some familiarity with basic concepts behind original miniKanren 
implementation as well as principles of relational programming.

\section{Polymorphic Unification}
\label{polyuni}

We consider it rather natural to employ polymorphic unification in the
language, already equipped with polymorphic comparison~--- a convenient, but
somewhat disputable\footnote{See, for example, \url{https://blogs.janestreet.com/the-perils-of-polymorphic-compare}} 
feature. Like polymorphic comparison, polymorphic unification performs traversal
of values, exploiting intrinsic knowledge of their runtime representation. 
The undeniable benefit of this solution is that in order to perform unification 
for user types no ``boilerplate'' code is needed. On the other hand, all pitfalls of
polymorphic comparison are inherited as well; in particular, unification can loop 
for cyclic data structures and does not work for functional values. Since we generally 
do not expect any reasonable outcome in these cases, the only remaining problem is that
the compiler is incapable of providing any assistance in identifying 
and avoiding them. Another drawback is that the implementation of polymorphic unification
relies on runtime representation of values and have to be fixed every time the representation changes. 
Finally, as it is written in unsafe manner using \lstinline{Obj} interface, it has to be
carefully developed and tested.

An important difference between polymorphic comparison and unification is that the former 
only inspects its operands, while the results of unification are recorded in a substitution
(mapping from logical variables to terms), which later is used to refine answers and reify 
constraints. So, generally speaking, we have to show, that no ill-typed terms are constructed 
as a result.

Polymorphic unification is introduced via the following function:

\begin{lstlisting}[mathescape=true]
   val unify : $\alpha$ logic -> $\;\;\alpha$ logic -> $\;\;$subst option -> 
     $\;\;$subst option
\end{lstlisting}

\noindent where ``\lstinline[mathescape=true]{$\alpha$ logic}'' stands for the type $\alpha$, 
injected into the logic domain, ``\lstinline{subst}''~--- for the type of substitution. Unification can 
fail (hence ``\lstinline{option}'' in the result type), is performed in the context of
existing substitution (hence ``\lstinline{subst}'' in the third argument) and
can be chained (hence ``\lstinline{option}'' in the third argument). Note, the 
type of substitution is not polymorphic, which means, that compiler completely looses the 
track of types for values, stored in a substitution. These types are recovered later during
refinement of answers.

To justify the correctness of unification, we consider a set of typed terms, each of which
has one of two forms

$$
x^\tau \mid C^\tau(t_1^{\tau_1},\dots,t_k^{\tau_k})
$$

\noindent where $x^\tau$ denotes a logical variable of type $\tau$, 
$C^\tau$~--- some constructor of type $\tau$, $t_i^{\tau_i}$~--- some terms of types $\tau_i$.
We reflect by $t_1^\tau[t_2^\rho]$ the fact of $t_2^\rho$ being a subterm of $t_1^\tau$, and
assume, that $\rho$ is unambiguously determined by $t_1$, $\tau$, and a position of $t_2$ 
``inside'' $t_1$.

Outside unification the compiler maintains typing, which means, that all 
terms, subterms, and variables agree in their types in all contexts. However, as 
our implementation resorts to unsafe features, we have to manually repeat this work for 
unification code.

We argue, that the following three invariants are maintained for any substitution $s$, involved 
in unification:

\begin{enumerate}
\item if \mbox{$t_1^{\_}[x^{\tau}]$} and \mbox{$t_2^{\_}[x^{\rho}]$}~--- two arbitrary terms (in particular, 
$t_1^{\_}$ and $t_2^{\_}$ may be the same), bound in $s$ and containing occurrences of variable $x$, 
then $\rho=\tau$ (different occurrences of the same variable in $s$ are attributed with the same type);

\item if \mbox{$(s\;\;x^\tau)$} is defined, then \mbox{$(s\;\;x^\tau) = t^\tau$} (a substitution always
binds a variable to a term of the same type);

\item each variable in $s$ preserves its type, assigned by the compiler (from the first two invariants 
it follows, that this type is unique; note also, that all variables are created and have their 
types assigned outside unification, in a type-safe world).
\end{enumerate}

The initial (empty) substitution trivially fulfills these invariants; hence, it is sufficient
to show, that they are preserved by unification.

The following snippet presents the implementation of unification with triangular 
substitution in only a little bit more abstract form, than actual code (for example, 
``occurs check'' is omitted):

\begin{lstlisting}[mathescape=true,numbers=left,numberstyle=\tiny,stepnumber=1,numbersep=-5pt]
   let rec walk $s$ = function
   | $x^\tau$ when $x\in dom(s)$ -> $\;\;$walk $s$ $(s\;\;x)^\tau$
   | $t^\tau$ -> $\;\;t^\tau$

   let rec unify $t_1^\tau$ $t_2^\tau$ = function
   | None -> None
   | Some $s$ as $sub$  ->
       match walk $s$ $t_1$, walk $s$ $t_2$ with
       | $x_1^\tau$, $x_2^\tau$ when $x_1$ = $x_2$ -> $\;\;sub$
       | $x_1^\tau$, $(t_2^\prime)^\tau$ -> $\;\;$Some ($s[x_1 \gets t_2^\prime]$)
       | $(t_1^\prime)^\tau$, $x_2^\tau$ -> $\;\;$Some ($s[x_2 \gets t_1^\prime]$)
       | $C^\tau(t_1^{\tau_1},\dots,t_k^{\tau_k})$, $C^\tau(p_1^{\tau_1},\dots,p_k^{\tau_k})$ -> 
           unify $t_k^{\tau_k}$ $p_k^{\tau_k}$(.. (unify $t_1^{\tau_1}$ $p_1^{\tau_1}$ $sub$)$..$)
       | $\_$, $\_$ -> $\;\;$None
\end{lstlisting}

Type annotations, included in the snippet above, can be justified by the following 
reasonings\footnote{We omit verbal description of unification algorithm; 
the details can be found in~\cite{MicroKanren}.}:

\begin{enumerate}
\item Line 2: the type of \mbox{$(s\;\;x^\tau)$} is $\tau$ due to invariant 2; hence, 
the type of \lstinline{walk} result coincides with the type of its second argument (technically,
an induction on the number of recursive invocations of \lstinline{walk} is needed).

\item Line 9: the substitution is left unchanged, hence all invariants are preserved.

\item Line 10 (and, symmetrically, line 11): first, note, that \mbox{$(s\;\;x_1)$} is undefined
(otherwise \lstinline{walk} would not return $x_1$). Then, $x_1$ and $t_2^\prime$ have the
same type, which justifies the preservation of invariant 2. Finally, either \mbox{$x_1=t_1$}
(and, then, $\tau$ is the type of $x_1$, assigned by the compiler), or $x_1$ is retrieved
from $s$ with type $\tau$~--- both cases justify invariants 1 and 3. The same applies to 
the pair $t_2^\prime$ and $t_2$.

\item The previous paragraph justifies the base case for inductive proof on the number of
recursive invocations of \lstinline{unify}.
\end{enumerate}

Function \lstinline{unify} is not directly accessible at the user level; it used
to implement both unification (``\lstinline{===}'') and disequality (``\lstinline{=/=}'') 
goals. The implementation generally follows~\cite{CKanren}.

\section{Logic Variables and Injection}
\label{logics}

Unification, considered in Section~\ref{polyuni}, works for values of type \lstinline[mathescape=true]{$\alpha$ logic}. 
Any value of this type can be seen as either value of type $\alpha$, or logical variable of type $\alpha$. The type 
itself is made abstract, but its values can be uncovered after refinement (see Section~\ref{refinement}).

Free variables solely can be created using ``\lstinline{fresh}'' construct of miniKanren. Note,  
since the unification is implemented in untyped manner, we can not use simple pattern matching to
distinguish logical variables from other logical values. Special attention was paid to implement
variable recognition in constant time.

Apart from variables, other logical values can be obtained by injection; conversely, sometimes
logical value can be projected to a regular one. We supply two functions\footnote{``\lstinline{inj}'' and ``\lstinline{prj}'' in concrete syntax.}
for these purposes

\begin{lstlisting}[mathescape=true]
   val ($\uparrow$) : $\alpha$ -> $\;\;\alpha$ logic
   val ($\downarrow$) : $\alpha$ logic -> $\;\;\alpha$
\end{lstlisting}

As expected, injection is total, while projection is partial. Using these functions and type-specific
``\lstinline{map}'', which can be derived automatically using a number of existing frameworks for
generic programming, one can easily provide injection and projection for user-defined datatypes. We
consider user-defined list type as an example:

\begin{lstlisting}[mathescape=true]
   type ($\alpha$, $\beta$) list = Nil | Cons of $\alpha$ * $\beta$
   
   type $\alpha$ glist = ($\alpha$, $\alpha$ glist) list
   type $\alpha$ llist = ($\alpha$ logic, $\alpha$ llist) list logic

   let rec inj_list l = $\uparrow$(map$_{\mbox{\texttt{list}}}$ ($\uparrow$) inj_list l) 
   let rec prj_list l = map$_{\mbox{\texttt{list}}}$ ($\downarrow$) prj_list ($\downarrow$ l)
\end{lstlisting}

Here ``\lstinline{list}'' is a custom type for lists; note, that it is made more
polymorphic, than usual~--- we abstracted it from itself and made it non-recursive 
(pragmatically speaking, it is desirable to make a type fully abstract, thus logic variables 
can be placed in arbitrary positions).

Then we provided two specialized versions~--- ``\lstinline{glist}'' (``ground'' list), which 
corresponds to regular, non-logic lists, and ``\lstinline{llist}'' (``logical'' list), which
corresponds to logical lists with logical elements. Using a single type-specific function
\lstinline[mathescape=true]{map$_{\mbox{\texttt{list}}}$}, we easily provided injection 
(of type {\lstinline[mathescape=true]{$\alpha$ glist -> $\;\;\alpha$ llist}}) and
projection (of type {\lstinline[mathescape=true]{$\alpha$ llist -> $\;\;\alpha$ glist}}).

In context of these definitions, now we can implement relational list concatenation, 
which is one of first-step examples of miniKanren programming:

\begin{lstlisting}[mathescape=true]
   let rec append$^o$ x y xy =
     conde [
       (x === $\uparrow$ Nil) &&& (xy === y);
       fresh (h t ty)
         (x  === $\uparrow$(Cons (h, t))
         (xy === $\uparrow$(Cons (h, ty))
         (append$^o$ t y ty)
     ]
\end{lstlisting}

Note, in the definition of \lstinline[mathescape=true]{append$^o$} we
used only default injection (``$\uparrow$''). Customized version most likely would 
appear in some top-level goal, for example:

\begin{lstlisting}[mathescape=true]
   (fun q -> append$^o$ (inj_list [1; 2; 3]) 
                    (inj_list [4; 5; 6]) 
                    q
   )
\end{lstlisting}

\section{Refinement and Top-Level Primitives}
\label{refinement}

The result of a relational program is a stream of substitutions, each of which represents
a certain answer. As a rule, a substitution binds many intermediate logical variables, 
created by ``\lstinline{fresh}'' in the course of execution. A meaningful answer has to be
\emph{refined}.

In our implementation refinement is represented by the following function:

\begin{lstlisting}[mathescape=true]
   val refine : subst -> $\alpha$ logic -> $\;\;\alpha$ logic
\end{lstlisting}

This function takes a substitution and a logical value and recursively substitutes
all logical variables in that value w.r.t. the substitution until no occurrences of 
bound variables are left. Since in our implementation the type of substitution is
not polymorphic, \lstinline{refine} is also implemented in an unsafe manner. However,
it is easy to see, that \lstinline{refine} does not produce ill-typed terms. Indeed,
all original types of variables are preserved in a substitution due to invariant
3 from Section~\ref{polyuni}. Unification does not change unified terms, so all terms, 
bound in a substitution, are well-typed. Hence, \lstinline{refine} always substitutes
some subterm in a well-typed term with another term of the same type, which preserves
well-typedness.

In addition to performing substitutions, \lstinline{refine} also \emph{reifies} 
disequality constrains. Reification attaches to each free variable in a refined
term a list of \emph{refined} terms, describing disequality constraint for that
free variable. Note, disequality can be established only for equally typed
terms, which justifies type-safety of reification. Note also, additional care has 
to be taken to avoid infinite looping, since refinement and reification are
mutually recursive, and refinement of a variable can be potentially invoked from 
itself due to a chain of disequality constraints.

After refinement, the content of a logical value can be inspected via the following 
function:

\begin{lstlisting}[mathescape=true]
   val destruct : $\alpha$ logic -> 
     [`Var of int * $\alpha$ logic list | `Value of $\alpha$]
\end{lstlisting}

Constructor \lstinline{`Var} corresponds to a free variable with unique
integer identifier and a list of terms, representing all disequality constraints
for this variable. These terms are refined as well.

We did not make \lstinline{refine} accessible for an end-user; instead we provided
a set of top-level combinators, which should be used to surround relational code
and perform refinement in a transparent manner. Note, from pragmatic
standpoint only variables, supplied as arguments for the top-level goal, have
to be refined (the original miniKanren implementation follows the same convention).

The toplevel primitive in our implementation is \lstinline{run}, which takes three
arguments. The exact type of \lstinline{run} is rather complex and non-instructive, 
so we better describe the typical form of its application:

\begin{lstlisting}[mathescape=true]
   run $\overline{n}$ (fun $l_1\dots l_n$ -> $\;\;G$) (fun $a_1\dots a_n$ -> $\;\;H$)
\end{lstlisting}

Here $\overline{n}$ stands for \emph{numeral}, which describes the number of
parameters for two other arguments of \lstinline{run}, \mbox{$l_1\dots l_n$}~---
free logical variables, $G$~--- a goal (which can make use of \mbox{$l_1\dots l_n$}), 
\mbox{$a_1\dots a_n$}~--- refined answers for \mbox{$l_1\dots l_n$}, respectively, and, 
finally, $H$~--- a \emph{handler} (which can make use of \mbox{$a_1\dots a_n$}). The types of 
\mbox{$l_1\dots l_n$} are inferred from $G$, and the types of \mbox{$a_1\dots a_n$} are
inferred from types of \mbox{$l_1\dots l_n$}: if $l_i$ has type \lstinline[mathescape=true]{$t$ logic}, then
$a_i$ has type \lstinline[mathescape=true]{$t$ logic stream}. In other words, user-defined handler
takes streams of refined answers for all variables, supplied to the top-level goal. All streams $a_i$ contains
coherent elements, so they all have the same length and $n$-th elements of all streams correspond 
to the $n$-th answer, produced by the goal $G$.

There are a few predefined numerals for one, two, etc. arguments (called, by tradition, 
\lstinline{q}, \lstinline{qr}, \lstinline{qrs} etc.), and a successor function, which 
can be applied to existing numeral to increment the number of expected arguments. The
technique, used to implement them, generally follows~\cite{Unparsing, DoWeNeed}.

\section{An Example}
\label{example}

Here we present an example of relational specification, written with the aid of our library. 
For this example we take list sorting; specifically, we present sorting for lists of
natural numbers in Peano form since our library already contains built-in
support for them. However our example can be easily extended for arbitrary (but
linearly ordered) types.

List sorting can be implemented in miniKanren in a variety of ways~--- 
virtually any existing algorithm can be rewritten relationally. We, however, 
try to be as much declarative as possible to demonstrate the
advantages of relational approach. From this standpoint, we can
claim, that sorted version of empty list is empty list, and sorted version
of non-empty list is its smallest element, concatenated with sorted
version of list, containing all its remaining elements.

The following snippet literally implements this definition:

\begin{lstlisting}[mathescape=true]
   let rec sort$^o$ x y = conde [
       (x === $\uparrow$Nil) &&& (y === $\uparrow$Nil);
       fresh (s xs xs')
         (y === $\uparrow$(Cons (s, xs')))
         (sort$^o$ xs xs')       
         (smallest$^o$ x s xs)
   ]
\end{lstlisting}

The meaning of the expression

\begin{lstlisting}[mathescape=true]
   smallest$^o$ x s xs
\end{lstlisting}

is

\begin{quotation}
\noindent ``\lstinline{s}'' is the smallest element of a (non-empty) list ``\lstinline{x}'', and 
``\lstinline{xs}'' is the list of all its remaining elements.
\end{quotation}

Now, \lstinline[mathescape=true]{smallest$^o$} can be implemented
using case analysis (note, that ``\lstinline{l}'' here is a non-empty 
list):

\begin{lstlisting}[mathescape=true]
   let rec smallest$^o$ l s l' = conde [       
       (l === $\uparrow$(Cons (s, $\uparrow$Nil))) &&& (l' === $\uparrow$Nil);
       fresh (h t s' t' max)
         (l' === $\uparrow$(Cons(max,t')))
         (l === $\uparrow$(Cons(h,t)))
         (minmax$^o$ h s' s max)
         (smallest$^o$ t s' t')
   ] 
\end{lstlisting}

Finally, we implement relational minimum-maximum calculation
primitive:

\begin{lstlisting}[mathescape=true]
   let minmax$^o$ a b min max = conde [
      (min === a) &&& (max === b) &&& (le$^o$ a b);
      (max === a) &&& (min === b) &&& (gt$^o$ a b)]
\end{lstlisting}

Here ``\lstinline[mathescape=true]{le$^o$}'' and ``\lstinline[mathescape=true]{gt$^o$}'' are
built-in comparison goals for natural numbers in Peano form.

Having relational \lstinline[mathescape=true]{sort$^o$}, we can implement 
sorting for regular integer lists:

\begin{lstlisting}[mathescape=true]
   let sort l =
     run q (sorto @@ inj_nat_list l)
           (fun qs -> prj_nat_list @@ Stream.hd qs)
\end{lstlisting}

Here \lstinline{Stream.hd} is a function, which takes a head from a 
lazy stream of answers. 

It is interesting, that since \lstinline[mathescape=true]{sort$^o$} is
relational, it can be used to calculate the list of all \emph{permutations}
for a given list. Indeed, each permutation, being sorted, results in the same list. 
So, the problem of finding all permutations can be relationally reformulated into 
the problem of finding all lists, which are converted by sorting into the given one:

\begin{lstlisting}[mathescape=true]
let perm l = map prj_nat_list @@
  run q (fun q -> fresh (r)
                    (sort$^o$ (inj_nat_list l) r) 
                    (sort$^o$ q r)
        )
        (Stream.take ~n:(fact @@ length l))
\end{lstlisting}

Note, for sorting original list we used exactly the same primitive. Note also, 
we requested exactly \lstinline{fact @@ length l} answers; requesting more
would result in infinite search for non-existing answers. This concludes our example.

\section{Conclusion}

We presented strongly typed implementation of miniKanren for OCaml. Our implementation
passes all tests, written for miniKanren (including those for disequality constraints);
in addition we implemented many interesting relational programs, known from
the literature. We claim, that our implementation can be used both as a convenient
relational DSL for OCaml and an experimental framework for future research in the area of
relational programming. 

The source code of our implementation is accessible from \url{https://github.com/dboulytchev/OCanren}.

We also want to express our gratitude to William Byrd, who infected us with relational programming, 
and for the extra time he sacrificed as both out tutor and friend.

\begin{thebibliography}{99}
\bibitem{TRS}
Daniel P. Friedman, William E.Byrd, Oleg Kiselyov. The Reasoned Schemer. The MIT
Press, 2005.

\bibitem{MicroKanren}
Jason Hemann, Daniel P. Friedman. $\mu$Kanren: A Minimal Core for Relational Programming //
Proceedings of the 2013 Workshop on Scheme and Functional Programming (Scheme '13).

\bibitem{CKanren}
Claire E. Alvis, Jeremiah J. Willcock, Kyle M. Carter, William E. Byrd, Daniel P. Friedman.
cKanren: miniKanren with Constraints // 
Proceedings of the 2011 Workshop on Scheme and Functional Programming (Scheme '11).

\bibitem{Untagged}
William E. Byrd, Eric Holk, Daniel P. Friedman.
miniKanren, Live and Untagged: Quine Generation via Relational Interpreters (Programming Pearl) //
Proceedings of the 2012 Workshop on Scheme and Functional Programming (Scheme '12).

\bibitem{Implicits}
Leo White, Fr\'ed\'eric Bour, Jeremy Yallop. 
Modular Implicits // Workshop on ML, 2014, arXiv:1512.01438.

\bibitem{Unparsing}
Olivier Danvy.
Functional Unparsing // Journal of Functional Programming, Vol.~8, Issue~6, November 1998.

\bibitem{DoWeNeed}
Daniel Fridlender, Mia Indrika.
Do we need dependent types? // Journal of Functional Programming, Vol.~10, Issue~4, July 2000.

\bibitem{DGP}
Jeremy Gibbons. Datatype-generic Programming //
Proceedings of the 2006 International Conference on Datatype-generic Programming.

\bibitem{Deriving}
Jeremy Yallop. 
Practical Generic Programming in OCaml // Proceedings of 2007 Workshop on ML.

\bibitem{InstantGenerics}
Manuel M. T. Chakravarty, Gabriel C. Ditu, Roman Leshchinskiy. 
Instant Generics: Fast and Easy. \url{http://www.cse.unsw.edu.au/~chak/papers/CDL09.html}, 2009.

\bibitem{ALaCarte}
Wouter Swierstra. Data Types \'a la Carte  // Journal of Functional Programming, Vol.~18, Issue~4, 2008.
\end{thebibliography}

\end{document}

