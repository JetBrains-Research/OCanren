\section{Implementation and Evaluation}
\label{evaluation}

We implemented the improved version of the search, described in the previous section, 
as a prototype over existing miniKanren implementation for OCaml, called OCanren~\cite{OCanren}. 
We added a history support and the divergence test, and changed the syntax to make relational definitions
visible for the interpreter. Programs in OCanren can be easily converted to respect the new syntax.
In our implementation the divergence propagation is implemented via exception mechanism, which is
known to be efficient in OCaml.

We evaluated our implementation on a number of benchmarks~--- virtually all available non-refulationally
complete queries, reported in the literature, whose incompleteness is caused by conjunction non-commutativity.
With no exceptions, the improved search was capable to fix refutational incompeteness. Thus, we do not know any
realistic cases, which are not improved by our approach. On the other hand, it is rather easy to construct
an artificial counterexample. For this we can define a relation

\begin{lstlisting}
   dummy $\;\binds\;\lambda\;x\;$ . dummy (S $\;x$)
\end{lstlisting}

\noindent and consider a query \lstinline|(dummy O) /\ fail|, where \lstinline|fail|~--- some never
suceeding goal (like \lstinline|A === B|). Since the argument of the recursive call to \lstinline|dummy| is
always performed with a more specific argument, the divergence test will never succeed, and the whole
query will diverge despite the absence of answers. 

In the following subsections we, first, describe the benchmarks in details and, then, present the
results of a quantitative evaluation.

\subsection{Relations on lists}

As we have seen in Section~\ref{incompleteness}, for some simple relations, like \lstinline|append$^o$|,
a recursive call has to be placed last in the sequence of conjuncts in order for the 
relation to be refutationally complete. Specifically for \lstinline|append$^o$|, with the improved search 
the divergence is discovered and refutational incompleteness is fixed regardless the position of 
the recursive call.

As a more interesting example, consider the \lstinline|revers$^o$| relation (see Section~\ref{incompleteness}
for the definition). As we've seen, in order for different queries to work it requires different orders of conjuncts 
to be used in the implementation. Again, the improved search lifts this requirement.
Even more interesting, in the query 

\begin{lstlisting}
   fresh ($q$) (revers$^o$ (Cons (A, Nil)) $q$)
\end{lstlisting}

\noindent the divergence is discovered for the recursive call of \lstinline|append$^o$|, but,
as none of conjunct orders within the definition of \lstinline|append$^o$| help, the
improved search goes even more back and switches the conjuncts within the definition of
\lstinline|revers$^o$|. This example demonstrates the importance of operating on 
a dynamic invocation order.

Another example we've already looked at is relational sorting/permutations. 
With the improved search, both queries \lstinline|fresh ($q$) (perm$^o$ l $q$)| and 
\lstinline|fresh ($q$) (perm$^o$ $q$ l)| terminate for any list \lstinline|l| of a reasonable 
length (7, 8, etc.), and now work in a reasonable time. Moreover, now
relational permutations can be used as a (we admit, somewhat exotic) way to calculate 
the number of permutations with repetitions.

\subsection{Binary arithmetics}

The implementation of a relational arithmetics is an important problem since it is utilized
in a number of elaborated relational specifications. For the performance reasons, it is 
preferrable to use binary numbers, not comfy Peano encoding, which makes the problem 
more complicated. As it is shown in~\cite{WillThesis}, the naive implementation of 
binary arithmetics turns out to be inappropriate due to multiple problems. 

Fixing these problems takes a lot of efforts: it requires some additional conditions, 
excess on the first glance, to be introduced in the specification to ensure the non-overlapping 
of the cases and the correctness of number representation. And still, even with all these 
improvements, arithmetic relations diverge for some routine queries with one order of 
consituents, and for others with another. To overcome this problem, 
arithmetics in the standard miniKanren implementation~\cite{TRS} uses digital logic and 
some advanced techniques of bounding the sizes of the terms~\cite{KiselyovArithmetic}. 
As a result, the implementation is quite complicated and takes time to understand.

At the same time, all these problems are exactly the consequences of the non-commutativity 
of conjunction. So, as expected, the improved search makes it possible to stick with
the naive version (for addition, multiplication, comparisons, division with a reminder)
without all these complicated optimizations. 

As the brightest example, for the division with reminder, instead of a very complicated recursive definition 
from~\cite{TRS} (20 lines of code, not including auxiliary functions), one can just write down the 
following definition

\begin{lstlisting}
   div$^o$ $\binds$ $\lambda\;x\;y\;q\;r$ . 
     (fresh (yq)        
        (mult$^o$ $y$ $q$ $yq$) /\
        (plus$^o$ $yq$ $r$ $x$) /\
        (lt$^o$ $r$ $y$)
     )
\end{lstlisting}

\noindent and for all queries

\begin{lstlisting}
   fresh ($q$ $r$) (div$^o$ $\overline{23}$ $\overline{5}$ $q$ $r$)
   fresh ($y$ $q$ $r$) (div$^o$ $\overline{19}$ $y$ $q$ $r$)
   fresh ($x$ $r$) (div$^o$ $x$ $\overline{17}$ $\overline{4}$ $r$)
\end{lstlisting}

\noindent the search terminates and shows the performance, comparable with the advanced version 
(here $\overline{n}$ denotes a binary representation of the number $n$).

\subsection{Binary trees}

For a natural representation of binary trees using two constructors \lstinline|Leaf| and 
\lstinline|Node ($l$, $r$)| it is easy to implement the relation to count the number of 
leaves in a tree (using arithmetic relation \lstinline|plus$^o$| for addition):

\begin{lstlisting}
   leaves$^o$ $\binds$ $\lambda\;t\;s$ .
     (($t$ === $\;\;$Leaf) /\ ($s$ === $\;\;$$\overline{1}$)) \/
     (fresh ($l$ $r$ $sl$ $sr$)
        ($t$ === $\;$Node ($l$, $r$)) /\
        (pos$^o$ $sl$) /\
        (pos$^o$ $sr$) /\
        (leaves$^o$ $l$ $sl$) /\
        (leaves$^o$ $r$ $sr$) /\
        (plus$^o$ $sl$ $sr$ $s$)
     )
\end{lstlisting}

By running this relation backwards

\begin{lstlisting}
   fresh ($q$) (siz$^o$ $q$ $\overline{n}$)
\end{lstlisting}

\noindent it becomes possible to generate all binary trees with given number of leaves $n$. 
The improved search provides the termination of this query; the number of discovered 
answers corresponds to the number of such trees, so this relational program may be seen 
as (an exotic) way of calculating the Catalan numbers.

\subsection{Interpreters}

Program synthesis with relational interpreters is one of the most useful applications of miniKanren. 
The construction of a relational interpreter for a small Scheme-like language is considered 
in details in~\cite{Untagged}. However, this simple interpreter also reveals some problems, caused by 
the non-commutativity of conjunction. For example, the following obviously contradictory query

\begin{lstlisting}
   fresh ($e1$ $e2$ $r1$ $r2$) (eval$^o$ (list $e_1$ 3 $e_2$) Nil ($r_1$ 4 $r_2$))
\end{lstlisting}

\noindent leads to a divergence under the conventional search; no simple reordering can fix it. 
Under the improved search, however, the contradiction is found and the query terminates with
no answers. Relational interpreters, used in practice for more complex problems~\cite{unified}, 
include a lot of optimisations and take significant effort to implement. We hope, that some 
performance problems with them are caused by the non-commutativity and can be fixed automatically 
with the improved search.

\subsection{A Quantitative Evaluation}

\begin{figure}[t]
  \small
  \begin{tabular}{ c | c | c c c  }
    relation     & input size & optimistic & improved search & pessimistic \\ 
    \hline
    append$^o$   & 100        &  0.091        &  0.111          &  0.305 \\
                 & 200        &  0.645        &  0.671          &  2.439 \\
                 & 300        &  2.106        &  2.122          &  8.173 \\
    \hline
    revers$^o$   & 30         &  0.032        &  0.059          &  0.301 \\
                 & 60         &  0.213        &  0.420          &  4.127 \\
                 & 90         &  0.717        &  1.402          &  20.204 \\
    \hline
    sort$^o$     & 2          &  0.003        &  0.005          &  0.003 \\
                 & 3          &  0.003        &  0.009          &  0.135 \\
                 & 4          &  0.004        &  0.020          &  --- \\
                 & 30         &  0.883        &  34.370         &  --- \\
    \hline
    perm$^o$     & 2          &  0.003        &  0.006          &  0.004 \\
                 & 3          &  0.005        &  0.027          &  --- \\
                 & 6          &  1.091        &  5.969          &  --- \\
    \hline
    mult$^o$     & 4          &  0.007        &  0.010          &  0.073 \\
                 & 5          &  0.017        &  0.021          &  20.650 \\
                 & 8          &  0.233        &  1.139          &  --- \\
    \hline
    div$^o$      & 3          &  0.010        &  0.486          &  0.044 \\
                 & 4          &  0.027        &  0.486          &  --- \\
                 & 5          &  0.112        &  4.928          &  --- \\
    \hline
    leaves$^o$   & 4          &  0.018        &  0.020          &  0.030 \\
                 & 5          &  0.064        &  0.082          &  68.841 \\
                 & 8          &  4.904        &  5.447          &  --- \\
    \hline
  \end{tabular}
  \caption{The results of a quantitative evaluation}
  \label{evaluation_results}
\end{figure}

While the improved search indeed fixes all considered realistic cases in a qualitative sense, it still introduces some runtime
overhead when no divergence is detected. In order to assess the overall performance gain we performed a quantitative
evaluation.

We ran our benchmarks for various input data sizes, and measured the time, needed for them to complete under the improved search. At the same time
we ran these benchmarks under the standard search in two cases: with ``optimistic'' and ``pessimistic'' orders of arguments in ``important'' conjunctions
(these conjunctions were determined experimentally). The running time in seconds is shown on Fig.~\ref{evaluation_results}.

For all benchmarks, under the improved search the convergence did not depend neither on the direction of queries, nor on the order of conjuncts; the same was true for
the running time. The behavior in the pessimistic case allows us to discriminate between two interesting situations:

\begin{itemize}
\item either in the pessimistic case the standard search was much slower, than the improved one, but still measurable, and the improved search worked with approximately the
same performance, as the standard search for the optimistic case;
  
\item or in the pessimistic case the standard search quickly became non-measurable (indicated ``---''), and the improved seach worked much slower, than the standard one for the
optimistic case, but still measurable.
\end{itemize}

We may conclude, that for the wide variety of realistic cases our improvement indeed delivers a fully-automatic and lightweight way to provide refutational completeness. In some cases,
however, on order to achieve the best performance a relational specification has to be optimized manually.

