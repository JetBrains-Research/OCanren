\section{Implementation and Evaluation}
\label{evaluation}

We implemented the improved version of the search, described in the previous section, 
as a prototype over existing miniKanren implementation for OCaml, called OCanren~\cite{OCanren}. 
We added a history support and the divergence test and changed the syntax to make relational definitions
visible for the interpreter. Programs in OCanren can be easily converted to respect the new syntax.
In our implementation the divergence propagation is implemented via exception mechanism that is
known to be efficient in OCaml. With our prototype implementation some answers for a certain query, being
evaluated under the improved search, can be repeated multiple times in comparison with those delivered by the
original search. While these two situations are indistinguishable w.r.t. our set-theoretic variant of the semantics,
we consider filtering out the repeating answers as a problem for future work.

We evaluated our implementation on a number of benchmarks~--- virtually all available non-refutationally
complete queries reported in the literature, whose incompleteness is caused by conjunction non-commutativity.
With no exceptions, the improved search was able to fix refutational incompleteness. Thus, we do not know any
realistic cases, which are not improved by our approach. On the other hand, it is rather easy to construct
an artificial counterexample. For this, we can define a relation

\begin{lstlisting}
   dummy $\;\binds\;\lambda\;x\;$ . dummy (S $\;x$)
\end{lstlisting}

\noindent and consider a query \lstinline|(dummy O) /\ fail|, where \lstinline|fail|~--- some never
succeeding goal (like \lstinline|A === B|). Since the argument of the recursive call to \lstinline|dummy| is
always performed with a more specific argument, the divergence test will never succeed, and the whole
query will diverge despite the absence of answers. 

In the following subsections we, first, describe the benchmarks in details and, then, present the
results of a quantitative evaluation.

\subsection{Relations on lists}

As we have seen in Section~\ref{incompleteness}, for some simple relations, like \lstinline|append$^o$|,
a recursive call has to be placed last in the sequence of conjuncts in order for the 
relation to be refutationally complete. Specifically for \lstinline|append$^o$|, with the improved search 
the divergence is discovered and refutational incompleteness is fixed regardless the position of 
the recursive call.

As a more interesting example, consider the \lstinline|revers$^o$| relation (see Section~\ref{incompleteness}
for the definition). As we've seen, in order for different queries to work it requires different orders of conjuncts 
to be used in the implementation. Again, the improved search lifts this requirement.
Even more interesting, in the query 

\begin{lstlisting}
   fresh ($q$) (revers$^o$ (Cons (A, Nil)) $q$)
\end{lstlisting}

\noindent the divergence is discovered for the recursive call of \lstinline|append$^o$|, but,
as none of conjunct orders within the definition of \lstinline|append$^o$| help, the
improved search goes even further back and switches the conjuncts within the definition of
\lstinline|revers$^o$|. This example demonstrates the importance of operating on 
a dynamic invocation order.

Another example we've already looked at is relational sorting/permutations. 
With the improved search, both queries \lstinline|fresh ($q$) (perm$^o$ l $q$)| and 
\lstinline|fresh ($q$) (perm$^o$ $q$ l)| terminate for any list \lstinline|l| and now work
in a reasonable time. Moreover, now relational permutations can be used as a (we admit,
somewhat exotic) way to calculate the number of permutations with repetitions.

\subsection{Binary arithmetics}

The implementation of a refutationally complete relational arithmetics is an important problem since it is utilized
in a number of elaborated relational specifications. For the performance reasons, it is preferrable to use binary numbers,
not comfy Peano encoding, which makes the problem more complicated. As it is shown in~\cite{WillThesis}, the naive
implementation of binary arithmetics turns out to be inappropriate due to multiple problems. 

Fixing these problems takes a lot of efforts: it requires some additional conditions, 
excess on the first glance, to be introduced in the specification to ensure the non-overlapping 
of the cases and the correctness of number representation. And still, even with all these 
improvements, arithmetic relations diverge for some routine queries with one order of 
consituents, and for others with another. To overcome this problem, arithmetics in the standard miniKanren
implementation~\cite{TRS} uses digital logic and some advanced techniques of bounding the sizes of
the terms~\cite{KiselyovArithmetic}. As a result, the implementation, proven to be refutationaly complete,
is quite complicated and takes time to understand. 

At the same time, some of these problems are exactly the consequences of the non-commutativity 
of conjunction. Thus, the improved search makes it possible to stick with the naive version (for addition,
multiplication, comparisons, division with a reminder) without complicated optimizations.  

As the most impressive example, for the division with a reminder, instead of a very complicated recursive definition 
from~\cite{TRS} (20 lines of code, not including auxiliary functions), one can just write down the 
following definition

\begin{lstlisting}
   div$^o$ $\binds$ $\lambda\;x\;y\;q\;r$ . 
     (fresh (yq)        
        (mult$^o$ $y$ $q$ $yq$) /\
        (plus$^o$ $yq$ $r$ $x$) /\
        (lt$^o$ $r$ $y$)
     )
\end{lstlisting}

\noindent and for all queries

\begin{lstlisting}
   fresh ($q$ $r$) (div$^o$ $\overline{23}$ $\overline{5}$ $q$ $r$)
   fresh ($y$ $q$ $r$) (div$^o$ $\overline{19}$ $y$ $q$ $r$)
   fresh ($x$ $r$) (div$^o$ $x$ $\overline{17}$ $\overline{4}$ $r$)
\end{lstlisting}

\noindent the search terminates and shows the performance, comparable with the advanced version 
(here $\overline{n}$ denotes a binary representation of the number $n$). However, in this case we do not
have a proof of refutational completeness.

\subsection{Binary trees}

For a natural representation of binary trees using two constructors \lstinline|Leaf| and 
\lstinline|Node ($l$, $r$)|, it is easy to implement the relation to count the number of 
leaves in a tree (using arithmetic relations \lstinline|plus$^o$| for addition and \lstinline|pos$^o$|
for positivity):

\begin{lstlisting}
   leaves$^o$ $\binds$ $\lambda\;t\;s$ .
     (($t$ === $\;\;$Leaf) /\ ($s$ === $\;\;$$\overline{1}$)) \/
     (fresh ($l$ $r$ $sl$ $sr$)
        ($t$ === $\;$Node ($l$, $r$)) /\
        (pos$^o$ $sl$) /\
        (pos$^o$ $sr$) /\
        (leaves$^o$ $l$ $sl$) /\
        (leaves$^o$ $r$ $sr$) /\
        (plus$^o$ $sl$ $sr$ $s$)
     )
\end{lstlisting}

By running this relation backwards

\begin{lstlisting}
   fresh ($q$) (leaves$^o$ $q$ $\overline{n}$)
\end{lstlisting}

\noindent it becomes possible to generate all binary trees with given number of leaves $n$. 
The improved search provides the termination of this query; the number of discovered 
answers corresponds to the number of such trees, so this relational program may be seen 
as (an exotic) way of calculating the Catalan numbers.

\subsection{Interpreters}

Program synthesis with relational interpreters is one of the most useful applications of miniKanren. 
The construction of a relational interpreter for a small Scheme-like language is considered 
in details in~\cite{Untagged}. However, this simple interpreter also reveals some problems, caused by 
the non-commutativity of conjunction. For example, consider the following query:

\begin{lstlisting}
   fresh ($e1$ $e2$ $r1$ $r2$) (eval$^o$ (list $e_1$ 3 $e_2$) Nil ($r_1$ 4 $r_2$))
\end{lstlisting}

Here the first argument of $eval^o$ is a program (a list of something ($e1$), $3$, and something ($e2$)),
which is evaluated in an empty environment ($Nil$) into a list of  something ($r1$), $4$, and something ($r2$).
Clearly, there are no $e1$, $e2$, $r1$, $r2$ to fulfill this contract, yet the evaluation leads to a divergence
under the conventional search; no simple reordering can fix it. Under the improved search, however, the
contradiction is found and the query terminates with no answers. Relational interpreters, used in practice for
more complex problems~\cite{unified}, include a lot of optimizations and take significant effort to implement.
We hope that some performance problems with them are caused by the non-commutativity and can be fixed automatically 
with the improved search.

\subsection{A Quantitative Evaluation}

%Current time: 0.6544
%Current time: 0.7859
%Current time: 0.9412
%Current time: 0.9872
%Current time: 0.7899
\def\permoxGximprovedxopt{0.8317}%permo_6_improved_opt
%Current time: 0.0307
%Current time: 0.0071
%Current time: 0.0098
%Current time: 0.0273
%Current time: 0.0311
\def\divoxJximprovedxopt{0.0212}%divo_4_improved_opt
%Current time: 0.0728
%Current time: 0.0802
%Current time: 0.0801
%Current time: 0.0839
%Current time: 0.0803
\def\appendoxIOOximprovedxpes{0.0795}%appendo_100_improved_pes
%Current time: 0.3809
%Current time: 0.3776
%Current time: 0.3930
%Current time: 0.4127
%Current time: 0.9122
\def\appendoxSOOximprovedxpes{0.4953}%appendo_200_improved_pes
%Current time: 0.0014
%Current time: 0.0016
%Current time: 0.0017
%Current time: 0.0000
%Current time: 0.0000
\def\sortoxJxoptimistic{0.0009}%sorto_4_optimistic
%Current time: 2.6521
%Current time: 3.9958
%Current time: 3.6779
%Current time: 3.8286
%Current time: 4.8376
\def\leavesoxXximprovedxpes{3.7984}%leaveso_8_improved_pes
%Current time: 4.3121
%Current time: 4.4847
%Current time: 4.6699
%Current time: 4.5118
%Current time: 4.5675
\def\permoxGximprovedxpes{4.5092}%permo_6_improved_pes
%Current time: 0.0002
%Current time: 0.0003
%Current time: 0.0002
%Current time: 0.0002
%Current time: 0.0002
\def\sortoxSxoptimistic{0.0002}%sorto_2_optimistic
%Current time: 0.0066
%Current time: 0.0063
%Current time: 0.0027
%Current time: 0.0059
%Current time: 0.0027
\def\multoxJxoptimistic{0.0049}%multo_4_optimistic
%Current time: 0.0008
%Current time: 0.0008
%Current time: 0.0009
%Current time: 0.0009
%Current time: 0.0009
\def\permoxSximprovedxopt{0.0009}%permo_2_improved_opt
%Current time: 0.1611
%Current time: 0.1722
%Current time: 0.1876
%Current time: 0.1799
%Current time: 0.1757
\def\reversoxROxpessimistic{0.1753}%reverso_30_pessimistic
%Current time: 0.0191
%Current time: 0.0075
%Current time: 0.0078
%Current time: 0.0162
%Current time: 0.0000
\def\leavesoxJximprovedxopt{0.0101}%leaveso_4_improved_opt
%Current time: 0.0062
%Current time: 0.0170
%Current time: 0.0231
%Current time: 0.0109
%Current time: 0.0055
\def\leavesoxJxpessimistic{0.0126}%leaveso_4_pessimistic
\def\permoxGxpessimistic{---}%permo_6_pessimistic
%Current time: 0.0076
%Current time: 0.0151
%Current time: 0.0239
%Current time: 0.0111
%Current time: 0.0176
\def\multoxBximprovedxopt{0.0151}%multo_5_improved_opt
\def\divoxJxpessimistic{---}%divo_4_pessimistic
%Current time: 0.5765
%Current time: 0.6081
%Current time: 0.5636
%Current time: 0.6036
%Current time: 0.5630
\def\reversoxPOxoptimistic{0.5830}%reverso_90_optimistic
%Current time: 0.0118
%Current time: 0.0041
%Current time: 0.0096
%Current time: 0.0057
%Current time: 0.0053
\def\divoxRxoptimistic{0.0073}%divo_3_optimistic
%Current time: 0.8341
%Current time: 1.2197
%Current time: 1.0539
%Current time: 1.0678
%Current time: 1.1343
\def\reversoxPOximprovedxpes{1.0620}%reverso_90_improved_pes
%Current time: 1.3231
%Current time: 1.3555
%Current time: 1.3942
%Current time: 1.5263
%Current time: 2.8026
\def\appendoxROOxoptimistic{1.6803}%appendo_300_optimistic
%Current time: 0.0563
%Current time: 0.0628
%Current time: 0.0563
%Current time: 0.0589
%Current time: 0.0497
\def\divoxRximprovedxpes{0.0568}%divo_3_improved_pes
%Current time: 3.3120
%Current time: 3.9231
%Current time: 4.2989
%Current time: 5.4268
%Current time: 4.6743
\def\reversoxGOxpessimistic{4.3270}%reverso_60_pessimistic
%Current time: 0.0001
%Current time: 0.0043
%Current time: 0.0039
%Current time: 0.0028
%Current time: 0.0000
\def\sortoxJximprovedxopt{0.0022}%sorto_4_improved_opt
%Current time: 15.9195
%Current time: 23.7064
%Current time: 18.2558
%Current time: 21.4184
%Current time: 19.6844
\def\multoxBxpessimistic{19.7969}%multo_5_pessimistic
%Current time: 0.0518
%Current time: 0.0358
%Current time: 0.0477
%Current time: 0.0489
%Current time: 0.0353
\def\reversoxROximprovedxopt{0.0439}%reverso_30_improved_opt
%Current time: 2.4812
%Current time: 3.0297
%Current time: 3.9702
%Current time: 4.5327
%Current time: 4.0538
\def\leavesoxXxoptimistic{3.6135}%leaveso_8_optimistic
%Current time: 0.1174
%Current time: 0.0758
%Current time: 0.0906
%Current time: 0.1223
%Current time: 0.1029
\def\divoxBximprovedxopt{0.1018}%divo_5_improved_opt
%Current time: 1.4922
%Current time: 1.6043
%Current time: 3.7089
%Current time: 1.3995
%Current time: 1.9031
\def\appendoxROOximprovedxpes{2.0216}%appendo_300_improved_pes
%Current time: 0.2180
%Current time: 0.1491
%Current time: 0.1877
%Current time: 0.1830
%Current time: 0.2056
\def\multoxXxoptimistic{0.1887}%multo_8_optimistic
%Current time: 0.0523
%Current time: 0.0620
%Current time: 0.0307
%Current time: 0.0567
%Current time: 0.0525
\def\leavesoxBximprovedxopt{0.0508}%leaveso_5_improved_opt
%Current time: 0.7866
%Current time: 0.9000
%Current time: 0.9902
%Current time: 0.8934
%Current time: 1.0409
\def\multoxXximprovedxpes{0.9222}%multo_8_improved_pes
%Current time: 0.3681
%Current time: 0.3075
%Current time: 0.2816
%Current time: 0.3120
%Current time: 0.3459
\def\reversoxGOximprovedxpes{0.3230}%reverso_60_improved_pes
\def\sortoxJxpessimistic{---}%sorto_4_pessimistic
%Current time: 0.0005
%Current time: 0.0043
%Current time: 0.0046
%Current time: 0.0042
%Current time: 0.0094
\def\sortoxRximprovedxpes{0.0046}%sorto_3_improved_pes
%Current time: 0.0014
%Current time: 0.0013
%Current time: 0.0013
%Current time: 0.0013
%Current time: 0.0013
\def\sortoxRxoptimistic{0.0013}%sorto_3_optimistic
\def\divoxBxpessimistic{---}%divo_5_pessimistic
%Current time: 0.0008
%Current time: 0.0007
%Current time: 0.0007
%Current time: 0.0007
%Current time: 0.0007
\def\sortoxSxpessimistic{0.0007}%sorto_2_pessimistic
%Current time: 0.0158
%Current time: 0.0225
%Current time: 0.0255
%Current time: 0.0089
%Current time: 0.0080
\def\permoxRximprovedxpes{0.0161}%permo_3_improved_pes
%Current time: 0.0792
%Current time: 0.0494
%Current time: 0.0526
%Current time: 0.0559
%Current time: 0.0394
\def\leavesoxBximprovedxpes{0.0553}%leaveso_5_improved_pes
%Current time: 35.7800
%Current time: 38.4145
%Current time: 36.6689
%Current time: 35.7528
%Current time: 34.8268
\def\sortoxROximprovedxpes{36.2886}%sorto_30_improved_pes
%Current time: 0.0061
%Current time: 0.0128
%Current time: 0.0087
%Current time: 0.0077
%Current time: 0.0005
\def\divoxRximprovedxopt{0.0072}%divo_3_improved_opt
\def\multoxXxpessimistic{---}%multo_8_pessimistic
%Current time: 1.1029
%Current time: 1.2271
%Current time: 1.4156
%Current time: 1.4442
%Current time: 1.1418
\def\sortoxROximprovedxopt{1.2663}%sorto_30_improved_opt
%Current time: 0.0227
%Current time: 0.0144
%Current time: 0.0165
%Current time: 0.0273
%Current time: 0.0135
\def\multoxBximprovedxpes{0.0189}%multo_5_improved_pes
%Current time: 3.0188
%Current time: 2.9048
%Current time: 4.0871
%Current time: 4.0273
%Current time: 3.2781
\def\appendoxSOOxpessimistic{3.4632}%appendo_200_pessimistic
\def\leavesoxXxpessimistic{---}%leaveso_8_pessimistic
%Current time: 0.2238
%Current time: 0.2320
%Current time: 0.2492
%Current time: 0.2307
%Current time: 0.2626
\def\multoxXximprovedxopt{0.2397}%multo_8_improved_opt
%Current time: 0.9312
%Current time: 1.0559
%Current time: 1.2230
%Current time: 1.2454
%Current time: 0.9794
\def\reversoxPOximprovedxopt{1.0870}%reverso_90_improved_opt
%Current time: 0.0689
%Current time: 0.0603
%Current time: 0.0721
%Current time: 0.0478
%Current time: 0.0710
\def\appendoxIOOxoptimistic{0.0640}%appendo_100_optimistic
%Current time: 0.0000
%Current time: 0.0029
%Current time: 0.0000
%Current time: 0.0015
%Current time: 0.0000
\def\permoxRxoptimistic{0.0009}%permo_3_optimistic
%Current time: 1.3583
%Current time: 1.5374
%Current time: 3.5605
%Current time: 1.3437
%Current time: 1.7455
\def\appendoxROOximprovedxopt{1.9091}%appendo_300_improved_opt
%Current time: 0.0099
%Current time: 0.0141
%Current time: 0.0156
%Current time: 0.0092
%Current time: 0.0053
\def\sortoxJximprovedxpes{0.0108}%sorto_4_improved_pes
%Current time: 0.0190
%Current time: 0.0064
%Current time: 0.0111
%Current time: 0.0181
%Current time: 0.0078
\def\leavesoxJxoptimistic{0.0125}%leaveso_4_optimistic
%Current time: 0.0050
%Current time: 0.0045
%Current time: 0.0045
%Current time: 0.0012
%Current time: 0.0047
\def\permoxSximprovedxpes{0.0040}%permo_2_improved_pes
%Current time: 0.0104
%Current time: 0.0193
%Current time: 0.0185
%Current time: 0.0141
%Current time: 0.0134
\def\leavesoxJximprovedxpes{0.0151}%leaveso_4_improved_pes
%Current time: 0.0008
%Current time: 0.0007
%Current time: 0.0007
%Current time: 0.0007
%Current time: 0.0000
\def\permoxSxoptimistic{0.0006}%permo_2_optimistic
%Current time: 0.0491
%Current time: 0.0480
%Current time: 0.0534
%Current time: 0.0366
%Current time: 0.0501
\def\multoxJxpessimistic{0.0475}%multo_4_pessimistic
%Current time: 0.0140
%Current time: 0.0268
%Current time: 0.0232
%Current time: 0.0295
%Current time: 0.0346
\def\divoxRxpessimistic{0.0256}%divo_3_pessimistic
%Current time: 0.0013
%Current time: 0.0000
%Current time: 0.0028
%Current time: 0.0042
%Current time: 0.0003
\def\permoxRximprovedxopt{0.0017}%permo_3_improved_opt
%Current time: 0.4259
%Current time: 0.4053
%Current time: 0.3882
%Current time: 0.4499
%Current time: 1.0350
\def\appendoxSOOximprovedxopt{0.5409}%appendo_200_improved_opt
%Current time: 20.7997
%Current time: 21.3724
%Current time: 20.4967
%Current time: 20.7788
%Current time: 20.6596
\def\reversoxPOxpessimistic{20.8215}%reverso_90_pessimistic
%Current time: 2.7126
%Current time: 3.9869
%Current time: 3.7247
%Current time: 3.7334
%Current time: 3.9466
\def\leavesoxXximprovedxopt{3.6208}%leaveso_8_improved_opt
%Current time: 0.3461
%Current time: 0.3460
%Current time: 0.3648
%Current time: 0.3779
%Current time: 0.3671
\def\appendoxSOOxoptimistic{0.3604}%appendo_200_optimistic
%Current time: 0.1662
%Current time: 0.1243
%Current time: 0.1828
%Current time: 0.1442
%Current time: 0.1994
\def\reversoxGOxoptimistic{0.1634}%reverso_60_optimistic
%Current time: 0.0335
%Current time: 0.0270
%Current time: 0.0233
%Current time: 0.0233
%Current time: 0.0182
\def\reversoxROxoptimistic{0.0251}%reverso_30_optimistic
%Current time: 0.0490
%Current time: 0.0446
%Current time: 0.0425
%Current time: 0.0603
%Current time: 0.0438
\def\sortoxRxpessimistic{0.0481}%sorto_3_pessimistic
%Current time: 0.5099
%Current time: 0.5000
%Current time: 0.5496
%Current time: 0.6052
%Current time: 0.5578
\def\permoxGxoptimistic{0.5445}%permo_6_optimistic
%Current time: 0.3439
%Current time: 0.3695
%Current time: 0.3146
%Current time: 0.3589
%Current time: 0.3848
\def\divoxJximprovedxpes{0.3543}%divo_4_improved_pes
%Current time: 0.0004
%Current time: 0.0003
%Current time: 0.0004
%Current time: 0.0004
%Current time: 0.0003
\def\sortoxSximprovedxopt{0.0003}%sorto_2_improved_opt
%Current time: 0.0691
%Current time: 0.0432
%Current time: 0.1089
%Current time: 0.0659
%Current time: 0.0659
\def\appendoxIOOximprovedxopt{0.0706}%appendo_100_improved_opt
%Current time: 55.6365
%Current time: 64.4739
%Current time: 59.6563
%Current time: 61.5575
%Current time: 59.3520
\def\leavesoxBxpessimistic{60.1352}%leaveso_5_pessimistic
%Current time: 0.0183
%Current time: 0.0177
%Current time: 0.0113
%Current time: 0.0158
%Current time: 0.0234
\def\divoxJxoptimistic{0.0173}%divo_4_optimistic
%Current time: 0.0010
%Current time: 0.0000
%Current time: 0.0000
%Current time: 0.0000
%Current time: 0.0012
\def\sortoxRximprovedxopt{0.0004}%sorto_3_improved_opt
%Current time: 0.0552
%Current time: 0.0356
%Current time: 0.0519
%Current time: 0.0453
%Current time: 0.0386
\def\reversoxROximprovedxpes{0.0453}%reverso_30_improved_pes
\def\sortoxROxpessimistic{---}%sorto_30_pessimistic
\def\permoxRxpessimistic{---}%permo_3_pessimistic
%Current time: 0.0239
%Current time: 0.0452
%Current time: 0.0388
%Current time: 0.0520
%Current time: 0.0595
\def\leavesoxBxoptimistic{0.0439}%leaveso_5_optimistic
%Current time: 0.0944
%Current time: 0.0917
%Current time: 0.1089
%Current time: 0.0814
%Current time: 0.0944
\def\divoxBxoptimistic{0.0942}%divo_5_optimistic
%Current time: 0.0197
%Current time: 0.0137
%Current time: 0.0112
%Current time: 0.0135
%Current time: 0.0143
\def\multoxBxoptimistic{0.0145}%multo_5_optimistic
%Current time: 0.0014
%Current time: 0.0003
%Current time: 0.0046
%Current time: 0.0041
%Current time: 0.0037
\def\sortoxSximprovedxpes{0.0028}%sorto_2_improved_pes
%Current time: 0.0146
%Current time: 0.0036
%Current time: 0.0053
%Current time: 0.0034
%Current time: 0.0117
\def\multoxJximprovedxpes{0.0077}%multo_4_improved_pes
%Current time: 0.0031
%Current time: 0.0020
%Current time: 0.0064
%Current time: 0.0088
%Current time: 0.0081
\def\multoxJximprovedxopt{0.0057}%multo_4_improved_opt
%Current time: 13.4180
%Current time: 15.0910
%Current time: 15.0819
%Current time: 14.8516
%Current time: 14.8944
\def\appendoxROOxpessimistic{14.6674}%appendo_300_pessimistic
%Current time: 0.0025
%Current time: 0.0000
%Current time: 0.0026
%Current time: 0.0034
%Current time: 0.0027
\def\permoxSxpessimistic{0.0022}%permo_2_pessimistic
%Current time: 0.2615
%Current time: 0.3034
%Current time: 0.3003
%Current time: 0.3353
%Current time: 0.3340
\def\reversoxGOximprovedxopt{0.3069}%reverso_60_improved_opt
%Current time: 4.1371
%Current time: 4.5179
%Current time: 4.8269
%Current time: 4.6465
%Current time: 4.4178
\def\divoxBximprovedxpes{4.5093}%divo_5_improved_pes
%Current time: 0.3286
%Current time: 0.3257
%Current time: 0.3491
%Current time: 0.3157
%Current time: 0.3504
\def\appendoxIOOxpessimistic{0.3339}%appendo_100_pessimistic
%Current time: 0.6082
%Current time: 0.6923
%Current time: 0.9072
%Current time: 0.7026
%Current time: 0.8264
\def\sortoxROxoptimistic{0.7473}%sorto_30_optimistic


\begin{figure}[t]
  \small
  \begin{tabular}{ c | c | c c c c  }
    relation     & size & optimistic & optimistic$^+$ & pessimistic & pessimistic$^+$  \\ 
    \hline
    append$^o$   & 100        & \appendoxIOOxoptimistic & \appendoxIOOximprovedxopt & \appendoxIOOxpessimistic & \appendoxIOOximprovedxpes \\
                 & 200        & \appendoxSOOxoptimistic & \appendoxSOOximprovedxopt & \appendoxSOOxpessimistic & \appendoxSOOximprovedxpes \\
                 & 300        & \appendoxROOxoptimistic & \appendoxROOximprovedxopt & \appendoxROOxpessimistic & \appendoxROOximprovedxpes \\
    \hline
    revers$^o$   & 30         & \reversoxROxoptimistic & \reversoxROximprovedxopt & \reversoxROxpessimistic & \reversoxROximprovedxpes \\
                 & 60         & \reversoxGOxoptimistic & \reversoxGOximprovedxopt & \reversoxGOxpessimistic & \reversoxGOximprovedxpes \\
                 & 90         & \reversoxPOxoptimistic & \reversoxPOximprovedxopt & \reversoxPOxpessimistic & \reversoxPOximprovedxpes \\
    \hline
    sort$^o$     & 2          & \sortoxSxoptimistic  & \sortoxSximprovedxopt  & \sortoxSxpessimistic  & \sortoxSximprovedxpes  \\
                 & 3          & \sortoxRxoptimistic  & \sortoxRximprovedxopt  & \sortoxRxpessimistic  & \sortoxRximprovedxpes  \\
                 & 4          & \sortoxJxoptimistic  & \sortoxJximprovedxopt  & \sortoxJxpessimistic  & \sortoxJximprovedxpes  \\
                 & 30         & \sortoxROxoptimistic & \sortoxROximprovedxopt & \sortoxROxpessimistic & \sortoxROximprovedxpes \\
    \hline
    perm$^o$     & 2          & \permoxSxoptimistic & \permoxSximprovedxopt & \permoxSxpessimistic & \permoxSximprovedxpes \\
                 & 3          & \permoxRxoptimistic & \permoxRximprovedxopt & \permoxRxpessimistic & \permoxRximprovedxpes \\
                 & 6          & \permoxGxoptimistic & \permoxGximprovedxopt & \permoxGxpessimistic & \permoxGximprovedxpes \\
    \hline
    mult$^o$     & 4          & \multoxJxoptimistic & \multoxJximprovedxopt & \multoxJxpessimistic & \multoxJximprovedxpes \\ 
                 & 5          & \multoxBxoptimistic & \multoxBximprovedxopt & \multoxBxpessimistic & \multoxBximprovedxpes \\
                 & 8          & \multoxXxoptimistic & \multoxXximprovedxopt & \multoxXxpessimistic & \multoxXximprovedxpes \\
    \hline
    div$^o$      & 3          & \divoxRxoptimistic & \divoxRximprovedxopt & \divoxRxpessimistic & \divoxRximprovedxpes \\ 
                 & 4          & \divoxJxoptimistic & \divoxJximprovedxopt & \divoxJxpessimistic & \divoxJximprovedxpes \\  
                 & 5          & \divoxBxoptimistic & \divoxBximprovedxopt & \divoxBxpessimistic & \divoxBximprovedxpes \\ 
    \hline
    leaves$^o$   & 4          & \leavesoxJxoptimistic & \leavesoxJximprovedxopt & \leavesoxJxpessimistic & \leavesoxJximprovedxpes \\
                 & 5          & \leavesoxBxoptimistic & \leavesoxBximprovedxopt & \leavesoxBxpessimistic & \leavesoxBximprovedxpes \\
                 & 8          & \leavesoxXxoptimistic & \leavesoxXximprovedxopt & \leavesoxXxpessimistic & \leavesoxXximprovedxpes \\
    \hline
  \end{tabular}
  \caption{The results of a quantitative evaluation}
  \label{evaluation_results}
\end{figure}


While the improved search indeed fixes all considered realistic cases in a qualitative sense, it still introduces some runtime
overhead when no divergence is detected. In order to assess the overall performance gain, we performed a quantitative
evaluation.

As we've pointed out earlier, the performance of the same specification essentially depends on the ``direction'' of
the query being evaluated. Moreover, the improved search can be faster than the original for one direction and
slower for another, which make the quiantitative evaluation problematic. In order to cope with this difficulty we
considered two important cases for each of the benchmarks ~--- ``optimistic''and ``pessimistic''.
Both cases were discovered by a careful analysis of each specification-query pair. As a result of the
analysis, the ``important'' conjunctions were discovered. The order of conjuncts in these conjunctions was adjusted to provide
the fastest convergence in the optimistic case and the slowest in the pessimistic one. Thus, to some extent these
cases provide the efficiency boundaries for a benchmark: when the ``direction'' of a query plays along with the
order of conjuncts in the specification, the whole specification-query performs as in optimistic case; otherwise
the pessimistic scenario takes place. Our conjecture was that the improved search would speed up the pessimistic
cases and slow down the optimistic, thus we've run the improved search in all cases and compared the running time
against that for the original search. The running time in seconds is shown on Fig.~\ref{evaluation_results} (the results
for the improved search are marked by ``$^+$'' superscript); by ``--'' we marked the cases, when the search did not
complete in two minutes\footnote{The evaluation repository is available at \url{https://github.com/rozplokhas/OCanren-improved-search}}. The workstation configuration was Intel Core i7 CPU M 620, 2.67GHz x 4, 8GB RAM running Ubuntu 16.04. We used a
native-code OCaml compiler with optimistic settings for the garbage collector to prevent it from interfering and blurring the
results.

For all benchmarks, under the improved search the convergence depended neither on the direction of queries, nor on the
order of conjuncts. With the exception of very small input sizes, the improved search provided a speedup of up to
few orders of magnitude in pessimistic cases. At the same time, the slowdown of optimistic cases under
the improved search did not exceed a factor of 2. The behavior in the pessimistic cases also allows us to discriminate
between two interesting situations:

\begin{itemize}
\item either in the pessimistic case the standard search was much slower than the improved one, but still measurable,
and the improved search worked with approximately the same performance, as the standard search for the optimistic case;
  
\item or in the pessimistic case the standard search quickly became non-measurable (indicated by ``---''), and the improved
search for the pessimistic case worked much slower, than the standard one for the optimistic case, but still measurable.
\end{itemize}

We may conclude that for the wide variety of realistic cases our improvement indeed delivers a fully-automatic and
lightweight way to provide refutational completeness. In some cases, however, in order to achieve the best performance,
a relational specification has to be optimized manually.
