\section{Related Works}
\label{sec:related_works}

The non-commutativity of conjunction evaluation in miniKanren is a well-known problem. In~\cite{WillThesis} some language
extensions are discussed, which, presumably, can be used to provide the commutativity. They include both simple enumeration
of conjunct orders and more advanced techniques, based on a combination of tabling, parallel goal evaluation, and continuations.
However, by now none of these proposals were implemented or evaluated. The tabling technique, described in the same work, can
indeed be used to provide the convergence of some queries, but it deals with the problems, orthogonal to the non-commutativity,
and, thus, does not heal the queries, which we do (but heals some other cases, like divergence of path-finding queries for
graphs with cycles, which we do not). 

For a number of problems, some \emph{ad-hoc} refutationally complete solutions were already presented before. For example,
in~\cite{TRS} a number of relations for binary arithmetics, implemented using the idea of bounding the sizes of terms, are
presented. In a follow-up paper~\cite{KiselyovArithmetic} this technique is explained in details, and the proof of refutational
completeness is given. Unfortunately, the specifications, written using this technique, are verbose and
hard to understand, and the implementation requires insight. Our improvement, on the other hand, makes it possible to stick
with the simplest definitions, and although we do not provide a proof of refutational completeness for each case, for the
majority of realistic queries they converge and demonstrate the same performance, as those, implemented with advanced methods.

In a broader context of logical programming, the problem of search convergence/termination was addressed multiple
times. However, it is rather hard to establish a direct correspondence between our proposal and the reported results,
since they were developed for essentially different language. For example, in~\cite{OLDresolution} a tabling-based
improvement of a resolution-based search~--- OLDT resolution~--- is described, and a complete search strategy is developed.
However, in miniKanren the original search is already complete, and we address rather a different issue. We can speculate,
that OLDT resolution roughly corresponds to the miniKanren with tabling and, thus, possesses the same properties, and relates to
our proposal in a similar manner.

An approach to a static transformation of logic programs is described in~\cite{BoundedNondeterminism}. Given a definitional program and
a query, the transformation results in a new program and a new query, which is guaranteed to converge in a finite number of steps
under arbitrary search strategy. The approach is proven to be sound and complete for queries, which deliver a finite number of solutions,
and utilizes the fact, that all these solutions can be discovered via a complete traversal of an SLD-tree up to a finite depth. The
authors introduce a sufficient condition for the finiteness of the number of solutions by utilizing the notion of level mapping and
present a method to calculate the limit for the SLD-tree depth. The important distinction with our proposal is that we perform a
dynamic analysis; as we argued in Section~\ref{incompleteness}, there are cases, when no concrete static form of a specification can
guarantee convergence for any query of interest. In addition, our approach detects the divergence rather than the
convergence of program evaluation.
