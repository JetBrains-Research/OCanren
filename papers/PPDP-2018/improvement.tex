\section{Search Improvement}
\label{improvement}

As we've seen in the previous section, the non-commutativity of conjunction in the presence of resursion
is one of the reasons for refutational incompleteness. Switching arguments of a certain conjunction
can sometimes improve the results; there is, however, no certain static order, beneficial in all cases.
Thus, we can make the following observations:

\begin{itemize}
\item the conjunction to change has to be properly identified;
\item the order of conjunct evaluation has to be a subject of a \emph{dynamic} choice.
\end{itemize}

Our improvement of the search is based on the idea of switching the order of conjuncts only when
the divergence of the first one is detected. More specifically: 

\begin{itemize}
\item during the search, we keep the track of all conjunctions being performed;
\item when we detect the divergence, we roll back to the nearest conjunction, for which 
we did not try all orders of constituents yet, switch its constituents, and rerun 
the search from that conjunction.
\end{itemize}

The important detail is the divergence test. Of course, due to the fundamental results in computability
theory, there is no hope to find a \emph{precise} computable test, which constitutes the necessary and 
sufficient condition of divergence. However, in our case a sufficient condition is sufficient. Indeed,  
a sufficient condition identifies a case, when the search, being continued, will lead to an incompleteness 
(since a divergence in our semantics always means incompleteness). Thus, it is no harm to try some other way. 

Another important question is the discipline of conjuncts reordering. Indeed, simply switching any two operands
of, for example, \mbox{$(g_1\wedge g_2)\wedge g_3$}, would not allow us to try \mbox{$(g_1\wedge g_3)\wedge g_2$}.
Thus, we have to flatten each ``cluster'' of nested conjunctions into a list of conjuncts\mbox{$\bigwedge g_i$}, 
where none of the goals $g_i$ is a conjunction. Then, it may seem at the first glance that the number of orderings to try 
is exponential on the number of conjuncts; we are going to show that, fortunately, this is not the case, and
a quadratic number of orders is suffucient.

In the rest of the section we address all these issues in details: first, we formally present the divergence
criterion and prove the necessity property; then, we describe an efficient reordering discipline. Finally, we present a
modified version of the semantics with incorporated divergence test and reordering. This semantics can be
considered as a modified version of the search, and we prove that this modification is a proper improvement.

\subsection{The Divergence Test}

Our divergence test is based on the following notion:

\begin{definition}
\normalfont 
We say that a vector of terms $\overline{a^{\phantom{x}}_i}$ is more general, than a vector of terms $\overline{b^{\phantom{x}}_i}$ (notation 
$\overline{a^{\phantom{x}}_i}\succeq\overline{b^{\phantom{x}}_i}$), if there is a substitution $\tau$, such that for $\forall i\;b_i = a_i \tau$.
\end{definition}

The idea of the divergence test is rather simple: it identifies a recursive call with more general arguments 
than (some) enclosing one. To state it formally and prove it using the semantics from section~\ref{language}, we need several definitions and lemmas.

\begin{definition}
\normalfont
A semantic variable $v$ is \emph{observable} w.r.t. the intrepretation $\iota$ and substitution $\sigma$, if there exists 
a syntactic variable $x$, such that \mbox{$v \in FV(\iota(x) \sigma)$}.
\end{definition}

\begin{definition}
\normalfont
A semantic statement 

$$
\otrans{\Gamma,\iota}{(\sigma,\,\delta)}{g}{S}
$$ 

\noindent is \emph{well-formed}, if \mbox{$dom(\sigma) \subseteq \delta$}, and any semantic variable, observable w.r.t. $\iota$ and $\sigma$, belongs to $\delta$.  
\end{definition}

Note, the root semantic statement \mbox{$\otrans{\Gamma,\bot}{(\epsilon,\,\emptyset)}{g}{S}$} is always well-formed.

\begin{lemma}
\label{one}
\normalfont
 For a well-formed semantic statement, every statement in its derivation tree is also well-formed.
\end{lemma}

The proof is by induction on the derivation tree. Note, we need to generalize the statement of the lemma, adding the well-formedness
condition w.r.t. $\iota$, $\sigma_r$, $\delta_r$ for any $(\sigma_r, \delta_r) \in S$.

The next lemma ensures that any substitution in the RHS of a semantic statement is a correct refinement of that in the LHS:

\begin{lemma}
\label{two}
\normalfont
For a well-formed semantic statement 

$$
\otrans{\Gamma,\iota}{(\sigma,\,\delta)}{g}{S}
$$ 

\noindent and any result \mbox{$(\sigma_r,\,\delta_r) \in S$}, there exists a substitution $\Delta$, such that:
  \begin{enumerate}
    \item \mbox{$\sigma_r = \sigma\circ\Delta$};
    \item any semantic variable \mbox{$v\in dom(\Delta)\cup ran(\Delta)$} either is observable w.r.t. $\iota$ and $\sigma$,
 or does not belong to $\delta$ (where \mbox{$ran(\Delta)=\bigcup_{v\in dom(\Delta)}FV(\Delta(v))$}).
  \end{enumerate}   
\end{lemma}

The proof is by induction on the derivation tree; we as well need to generalize the statement of the lemma, adding the condition that the 
set of all allocated semantic variables $\delta$ can only grow during the evaluation.

The final lemma formalizes the intuitive considerations that the evaluation for a more general state cannot fail, if the evaluation 
for a more specific state doesn't fail:

\begin{lemma}
\label{three}
\normalfont
Let 

$$
\otrans{\Gamma,\iota}{(\sigma,\,\delta)}{g}{S}
$$ 

and 

$$\otrans{\Gamma,\iota^\prime}{(\sigma^\prime,\,\delta^\prime)}{g}{S^\prime}
$$

\noindent be two well-formed semantic statements, and let $\tau$ be a substitution, such that 
for any syntactic variable $x$ \mbox{$\iota^\prime(x) \sigma^\prime = \iota(x) \sigma \tau$}. Then the 
derivation tree for the first statement has greater or equal height, then the derivation 
tree for the second statement.
\end{lemma}

The proof is by induction on the derivation tree for the second statement. We need to generalize the statement of the lemma, adding the requirement that 
for any substitution $s^\prime_r$ in the RHS of the second statement, there has to be a substitution $s_r$ in the RHS of the first statement,
such that there exists a substitution $\tau_r$, such that for any syntactic variable $x$ \mbox{$\iota^\prime(x) \sigma^\prime_r = \iota(x) \sigma_r \tau_r$}. 
In the cases of $\textsc{Fresh}$ and $\textsc{Invoke}$ rules, some semantic variables can become non-observable, and we need to define a substitution $\tau_r$ 
separately for these ``forgotten'' variables and those, which remain observable, using Lemma~\ref{two}.

Now we are ready to claim and prove the divergence criterion.

\setcounter{theorem}{0}
\begin{theorem}[Divergence criterion]
\label{criterion}
\normalfont
For any well-formed semantic statement 

$$
\otrans{\Gamma,\iota}{(\sigma,\,\delta)}{r^k\,t_1\dots t_k}{S}
$$ 

if its proper derivation subtree has a semantic statement 

$$
\otrans{\Gamma,\iota^\prime}{(\sigma^\prime,\,\delta^\prime)}{r^k\,t^\prime_1\dots t^\prime_k}{S^\prime}
$$

then \mbox{$\overline{t^\prime_i \iota^\prime \sigma^\prime} \not \succeq \overline{t^{\phantom{\prime}}_i \iota \sigma}$}. 
\end{theorem}
\begin{proof}
Assume that \mbox{$\overline{t^\prime_i \iota^\prime \sigma^\prime}\succeq \overline{t^{\phantom{\prime}}_i \iota \sigma}$}. 

By Lemma~\ref{one}, the semantic statement

$$
\otrans{\Gamma,\iota^\prime}{(\sigma^\prime,\,\delta^\prime)}{r^k\,t^\prime_1\dots t^\prime_k}{S^\prime}
$$

\noindent is well-formed.

By Lemma~\ref{three}, the derivation tree for

$$
\otrans{\Gamma,\iota^\prime}{(\sigma^\prime,\,\delta^\prime)}{r^k\,t^\prime_1\dots t^\prime_k}{S^\prime}
$$

\noindent has greater or equal height, than that for

$$
\otrans{\Gamma,\iota}{(\sigma,\,\delta)}{r^k\,t_1\dots t_k}{S}
$$ 

\noindent which contradicts the theorem condition.

\end{proof}

The theorem justifies that, indeed, our test constitutes a sufficient condition for a divergence: if the execution
reaches a relation call with more general arguments, than those of some enclosing one, then it has no derivation
in our semantics, and, thus, it is not terminating.

\setarrow{\xRightarrow}
\setsubarrow{_e}
\begin{figure*}
\begin{minipage}[t]{\textwidth}
\small
\[
\cotrans{\Gamma,\,\iota,\,h}{(\sigma,\,\delta)}{t_1\equiv t_2}{\emptyset}{mgu\,(t_1\iota\sigma,\,t_2\iota\sigma) = \bot}\ruleno{UnifyFail$^+$}
\]
\[
\cotrans{\Gamma,\,\iota,\,h}{(\sigma,\,\delta)}{t_1\equiv t_2}{(\sigma\circ\Delta,\,\delta)}{mgu\,(t_1\iota\sigma,\,t_2\iota\sigma) = \Delta\ne\bot}\ruleno{UnifySuccess$^+$}
\]
\[
\trule{\otrans{\Gamma,\,\iota,\,h}{(\sigma,\,\delta)}{g_1}{S_1};\quad
       \otrans{\Gamma,\,\iota,\,h}{(\sigma,\,\delta)}{g_2}{S_2}
      }
      {\otrans{\Gamma,\,\iota,\,h}{(\sigma,\,\delta)}{g_1\vee g_2}{S_1\cup S_2}}\ruleno{Disj$^+$}
\]
\[
\crule{\otrans{\Gamma,\,\iota[x\gets\alpha],\,h}{(\sigma,\,\delta\cup\{\alpha\})}{g}{S^\dagger}}
      {\otrans{\Gamma,\,\iota,\,h}{(\sigma,\,\delta)}{\lstinline|fresh($x$) $\;g$|}{S^\dagger}}
      {\alpha\in\meta{W}\setminus\delta}\ruleno{Fresh$^+$}
\]
\end{minipage}      
\caption{Improved search: inherited rules}
\label{improved-semantics-normal}
\end{figure*}

\begin{figure*}
\begin{minipage}[t]{\textwidth}
\small
\[
   \cotrans{\Gamma,\,\iota,\,h}{(\sigma,\,\delta)}{r^k t_1 \dots t_k}{\dagger}{v_i = t_i \iota \sigma, \; (v_1, \dots, v_k) \succeq h\,r^k}
   \ruleno{InvokeDiv$^+$}
\]

\[
\crule{\otrans{\Gamma,\,\epsilon[x_i\gets v_i],\,h[r^k\gets(v_1, \dots, v_k)]}{(\epsilon,\,\delta)}{g}{\{(\sigma_i,\,\delta_i)\}}}
      {\otrans{\Gamma,\,\iota,\,h}{(\sigma,\,\delta)}{r^k t_1 \dots t_k}{\{(\sigma\circ\sigma_i, \delta_i)\}}}
      {v_i=t_i\iota\sigma,\;\Gamma\,r^k=\lambda x_1 \dots x_k. g,\; (v_1, \dots, v_k) \nsucceq h\,r^k}
      \ruleno{Invoke$^+$}
\]
\end{minipage}      
\caption{Improved search: invocation and divergence detection}
\label{improved-semantics-invoke}
\end{figure*}

\begin{figure*}
\begin{minipage}[t]{\textwidth}
\small
\[
\trule{\otrans{\Gamma,\,\iota,\,h}{(\sigma,\,\delta)}{g_1}{\dagger}}
      {\otrans{\Gamma,\,\iota,\,h}{(\sigma,\,\delta)}{g_1\vee g_2}{\dagger}}\ruleno{DivDisjLeft$^+$}
\]
\[
\trule{\otrans{\Gamma,\,\iota,\,h}{(\sigma,\,\delta)}{g_2}{\dagger}}
      {\otrans{\Gamma,\,\iota,\,h}{(\sigma,\,\delta)}{g_1\vee g_2}{\dagger}}\ruleno{DivDisjRight$^+$}
\]
\[
\crule{\otrans{\Gamma,\,\epsilon[x_i\gets v_i],\,h[r^k\gets(v_1, \dots, v_k)]}{(\epsilon,\,\delta)}{g}{\dagger}}
      {\otrans{\Gamma,\,\iota,\,h}{(\sigma,\,\delta)}{r^k t_1 \dots t_k}{\dagger}}
      {v_i=t_i\iota\sigma,\;\Gamma\,r^k=\lambda x_1 \dots x_k. g,\; (v_1, \dots, v_k) \nsucceq h\,r^k}
      \ruleno{DivInvoke$^+$}
\]      
\end{minipage}      
\caption{Improved search: divergence propagation}
\label{improved-semantics-divergence-prop}
\end{figure*}

\subsection{Conjuncts Reordering}
\label{sec:reordering}

In this section we consider the discipline of conjuncts reordering. Recall, we flatten all nested conjunctions in 
clusters $\wedge g_i$, where none of $g_i$ is a conjunction. To evaluate a cluster, we have to evaluate
its conjuncts one after another, threading the results, starting from the initial substitution. Each time we
evaluate a conjunct, we can have three possible outcomes:

\begin{itemize}
\item The evaluation converges with some result. In this case we can proceed with the next conjunct.
\item The evaluation diverges undetected. In this case nothing can be done.
\item A divergence is detected by the test. This is the case when the reordering takes place.
\end{itemize}

In a general case, for each cluster there can be some converging prefix $\omega$ we've managed to evaluate so far (initially empty),
and the rest of the conjuncts $g_i$. Since $\omega$ converges, we have some set of substitutions $S_\omega$, which corresponds to the
result of $\omega$ evaluation.

Suppose none of $g_i$ converges on $S_\omega$ (i.e. there is atleast one substitution in $S_\omega$, on which
each $g_i$ diverges). We claim that reordering conjuncts inside $\omega$ would not help. Indeed, with any other order
of conjuncts, $\omega$ either diverges, or converges with the same result (up to the renaming of semantic variables). Thus,
making any permutations inside $\omega$ is superfluous.

Next, suppose we have two different goals $g_1$ and $g_2$, which both converge on $S_\omega$ (i.e. both converge on each
substitution in $S_\omega$). Do we need to try both cases ($g_1$ and $g_2$) to extend the converging prefix?
It is rather easy to see that if, say, $g_2$ converges on $S_\omega$, then it will as well converge on the result of evaluation
of $g_1$ on $S_\omega$. Indeed, for arbitrary \mbox{$\sigma\in S_\omega$} we have

\[
\otrans{\dots}{(\sigma,\dots)}{g_1}{S^\prime_\omega}
\]

where each \mbox{$\sigma^\prime\in S^\prime_\omega$} is ``more specific'', than $\sigma$, by Lemma~\ref{two}. By Lemma~\ref{three},
since $g_2$ converges on $\sigma$, it converges on each \mbox{$\sigma^\prime\in S^\prime_\omega$} as well.

In other words, to extend a converging prefix we can choose arbitrary conjunct, which converges immediately
after this prefix, and this choice will never has to be undone.

Now we can specify the reordering discipline. Since we never re-evaluate a converging prefix, we do not
represent it. Thus, each cluster we consider from now on is a suffix of some initial cluster after
evaluation of some converging prefix (and, perhaps, after some reorderings performed so far).

Let us have a cluster \mbox{$\bigwedge_{i=1}^k g_i$}. We evaluate it on some substitution $\sigma$ in the context of some integer
value $p$ (initially $p=1$), which describes, which conjunct we have to try next. We operate as follows:

\begin{enumerate}
\item\label{reorder:top} We try to evaluate $g_p$ on $\sigma$. If the evaluation succeeds with a result $S^\prime$, we 
remove $g_p$ from the cluster and evaluate the rest for each substitution in $S^\prime$ and $p=1$.
  
\item If a divergence is detected, and $p\le k$, then we switch $g_i$ with $g_p$, increment $p$,
and repeat from step~\ref{reorder:top}.
  
\item Otherwise, we give up and rollback to the enclosing cluster (if any).
\end{enumerate}

Thus, we apply a greedy approach: each time we have a converging prefix of conjuncts (possibly empty), and some tail.
We try to put each conjunct from the tail immediately after the prefix. If we find a converging conjunct, we attach
it to the prefix and continue; if no, then the list of conjuncts diverges. Thus, we can find a converging order
(if any) in a quadratic time. Note, for different substitutions in the result of a converging prefix evaluation
the order of remaining conjuncts can be different.

\begin{figure*}
\begin{minipage}[t]{\textwidth}
\small
\[
\trule{\setsubarrow{_r}\otrans{\Gamma,\,\iota,\,h,\,1}{(\sigma,\,\delta)}{\bigwedge\limits_{i=1}^n g_i}{S^\dagger}}
      {\otrans{\Gamma,\,\iota,\,h}{(\sigma,\,\delta)}{\bigwedge\limits_{i=1}^n g_i}{S^\dagger}}
      \ruleno{ClusterStart$^+$}
\]
\vskip3mm
\[
\crule{\otrans{\Gamma,\,\iota,\,h}{(\sigma,\,\delta)}{g_p}{\{(\sigma_j,\,\delta_j)\}};\quad
       \forall j\;:\;\otrans{\Gamma,\,\iota,\,h}{(\sigma_j,\,\delta_j)}{\bigwedge\limits_{i\ne p}g_i}{S_j}
      }
      {\setsubarrow{_r}\otrans{\Gamma,\,\iota,\,h,\,p}{(\sigma,\,\delta)}{\bigwedge\limits_{i=1}^n g_i}{\bigcup S_j}}
      {1 \le p \le n}
\ruleno{ClusterStep$^+$}
\]
\vskip3mm
\[
\crule{\otrans{\Gamma,\,\iota,\,h}{(\sigma,\,\delta)}{g_p}{\{(\sigma_j,\,\delta_j)\}};\quad
       \exists j\;:\;\otrans{\Gamma,\,\iota,\,h}{(\sigma_j,\,\delta_j)}{\bigwedge\limits_{i\ne p}g_i}{\dagger}
      }
      {\setsubarrow{_r}\otrans{\Gamma,\,\iota,\,h,\,p}{(\sigma,\,\delta)}{\bigwedge\limits_{i=1}^n g_i}{\dagger}}
      {1 \le p \le n}
\ruleno{ClusterDiv$^+$}
\]
\vskip3mm
\[
\crule{\otrans{\Gamma,\,\iota,\,h}{(\sigma,\,\delta)}{g_p}{\dagger};\quad
       {\setsubarrow{_r}\otrans{\Gamma,\,\iota,\,h,\,p+1}{(\sigma,\,\delta)}{\bigwedge\limits_{i=1}^n g_i}{S^\dagger}}
      }
      {\setsubarrow{_r}\otrans{\Gamma,\,\iota,\,h,\,p}{(\sigma,\,\delta)}{\bigwedge\limits_{i=1}^n g_i}{S^\dagger}}
      {1 \le p \le n}
\ruleno{ClusterNext$^+$}
\]
\vskip3mm
\[
{\setsubarrow{_r}\cotrans{\Gamma,\,\iota,\,h,\,p}{(\sigma,\,\delta)}{\bigwedge\limits_{i=1}^n g_i}{\dagger}{p>n}}
\ruleno{ClusterStop$^+$}
\]
\end{minipage}      
\caption{Improved search: conjuncts reordering}
\label{improved-semantics-reordering}
\end{figure*}

\subsection{Improved Search Semantics}

Here we combine all observations, presented in the preceding subsections~--- the divergence test, conjunct clustering
and reordering,~--- and express the improved search in terms of a big-step operational semantics, which is an extension
of the initial one, presented in Section~\ref{language}.

We denote ``$\xRightarrow{}_e$'' the semantic relation for the improved search, and we add another component to the
environment~--- a history $h$,~--- which maps a relational sybol to a list of fully interpreted terms as its arguments.
As we are (sometimes) capable of detecting the divergence, besides a regular set of answers $S$ as a result of evaluation
we can have a divergence signal, which we denote $\dagger$; we will use the notation $S^\dagger$, when the difference
between these two possible outcomes is not important.

For the convenience of presentation we split the set of semantic rules into a few groups. The first one is the inherited
rules (see Fig.~\ref{improved-semantics-normal})~--- those, which did not change (except for the extension in the
environment and evaluation result). Note, the rule \rulename{Disj$^+$} does not handle the divergence detection
in either of disjuncts.

The next group describes the invocation and divergence detection (see Fig.~\ref{improved-semantics-invoke}). On
relation invocation we first consult with the history. If the history indicates that the invocation is performed in the
context of the same relation evaluation with more specific arguments, then we raise the divergence signal; otherwise
we perform normally. Note, the rule \rulename{Invoke$^+$} does not handle the divergence in the \emph{body} of
invoked relation.

The next group describes the divergence signal propagation (see Fig.~\ref{improved-semantics-divergence-prop}). Here
the divergence signal, raised in one of disjuncts or in the body of relational definition, is propagated to the upper
levels of derivation tree.

The final group handles the conjunct reordering (see Fig.~\ref{improved-semantics-reordering}). As we need a reordering
parameter $p$ (see Section~\ref{sec:reordering}), we introduce another relation ``$\xRightarrow{}_r$'' with environment,
enriched by $p$.

The rule \rulename{ClusterStart$^+$} describes the case, when we make an attempt to evaluate a cluster. It can happen, when
we either first encounter an original cluster, or try to evaluate a suffix of some initial cluster past some converging
prefix. As the reordering starts now, we recurse to the reordering relation with the parameter \mbox{$p=1$} (which means,
that the first conjunct will be tried to evaluate next).

Two next rules describe the case, when the $p$-th conjunct, being tried to evaluate, succeds with some result. In the rule
\rulename{ClusterStep$^+$} we handle the case, when all other conjuncts can be evaluated in the context of that result: we
combine the outcomes, which completes the evaluation of the whole cluster. In the rule \rulename{ClusterDiv$^+$} we consider
the opposite case: now there is some conjunct $g_j$, which raises a divergence signal, being evaluated in the context of
the results, delivered by the evaluation of $g_p$. As we argued in Section~\ref{sec:reordering}, nothing can be done, and we
propagate the divergence signal.

The rule \rulename{ClusterNext$^+$} describes the case, when the $p$-th conjunct raises the divergence signal, and there are
some other conjuncts to try. We increment $p$ and proceed.

Finally, in the rule \rulename{ClusterStop$^+$} we handle the situation, when all available conjuncts in a cluster were tried to
evaluate first, and raised the divergence signal. We propagate the signal in this case.

The following theorem is rather easy to prove:

\begin{theorem} For arbitrary $\Gamma$ and $g$ if

  \[{\setsubarrow{}\otrans{\Gamma,\,\bot}{(\epsilon,\,\emptyset)}{g}{S}}\]

  then
  
  \[{\setsubarrow{_e}\otrans{\Gamma,\,\bot,\,\bot}{(\epsilon,\,\emptyset)}{g}{S}}\]

\end{theorem}

Indeed, due to Theorem~\ref{criterion}, from the condition we can conclude that the divergence signal is
never raised during the evaluation, according to ``$\xRightarrow{}_e$''; but in this case the evaluation
steps coinside with those, according to ``$\xRightarrow{}$''. Thus, the improved search is non-degrading.





