\section{The Syntax and Semantics of Relational Language}
\label{language}

\begin{figure*}
\small
\begin{minipage}[t]{\textwidth}
\[
\begin{array}{rcll}
\meta{C}    & = & \{C^k,\dots\}                                        & \;\;\mbox{\emph{(constructors)}}\\
\meta{T}(X) & = & x\in X \mid C^k\,(t_1,\dots,t_k),\,t_i\in \meta{T}(X) & \;\;\mbox{\emph{(terms)}}\\
\meta{V}    & = & \{x, y, z, \dots\}                                   & \;\;\mbox{\emph{(syntactic variables)}}\\
\meta{T_V}  & = & \meta{T}(\meta{V})                                   & \;\;\mbox{\emph{(syntactic terms)}}\\
\meta{R}    & = & \{r^k,\dots\}                                        & \;\;\mbox{\emph{(relational symbols)}}\\
\meta{G}    & = & t_1\equiv t_2,\,t_i\in\meta{T_V}                      & \;\;\mbox{\emph{(unification)}}\\ 
            &   & g_1\wedge g_2                                        & \;\;\mbox{\emph{(conjunction)}}\\
            &   & g_1\vee g_2                                          & \;\;\mbox{\emph{(disjunction)}}\\
            &   &\lstinline|fresh|\,(x)\;g                             & \;\;\mbox{\emph{(fresh variable introduction)}}\\
            &   &r^k\;t_1\dots t_k,\,t_i\in\meta{T_V}                   & \;\;\mbox{\emph{(relational reference)}}\\
\meta{D}    & = & r^k\binds\lambda x_1\dots x_k\,.\,g,\,x_i\in\meta{V}  & \;\;\mbox{\emph{(relational definition)}}\\
\meta{S}    & = & d_1,\dots,\,d_k; g                                    & \;\;\mbox{\emph{(specification)}}
\end{array}
\]
\end{minipage}
\caption{The syntax of the source language}
\label{syntax}
\end{figure*}

\setarrow{\xRightarrow}
\newcommand{\otrans}[4]{\withenv{#1}{\trans{#2}{\mbox{$#3$}}{#4}}}
\newcommand{\cotrans}[5]{\withenv{#1}{\ctrans{#2}{\mbox{$#3$}}{#4}{#5}}}
 
\begin{figure*}
\begin{minipage}[t]{\textwidth}
\small
\[
\cotrans{\Gamma,\,\iota}{(\sigma,\,\delta)}{t_1\equiv t_2}{\emptyset}{mgu\,(t_1\iota\sigma,\,t_2\iota\sigma) = \bot}\ruleno{UnifyFail}
\]

\[
\cotrans{\Gamma,\,\iota}{(\sigma,\,\delta)}{t_1\equiv t_2}{\{(\sigma\circ\Delta,\,\delta)\}}{mgu\,(t_1\iota\sigma,\,t_2\iota\sigma) = \Delta\ne\bot}\ruleno{UnifySuccess}
\]

\[
\trule{\otrans{\Gamma,\,\iota}{(\sigma,\,\delta)}{g_1}{S_1},\;\;\;\;
       \otrans{\Gamma,\,\iota}{(\sigma,\,\delta)}{g_2}{S_2}
      }
      {\otrans{\Gamma,\,\iota}{(\sigma,\,\delta)}{g_1\vee g_2}{S_1\cup S_2}}\ruleno{Disj}
\]

\[
\trule{\otrans{\Gamma,\,\iota}{(\sigma,\,\delta)}{g_1}{\bigcup_i\{(\sigma_i,\,\delta_i)\}},\;\;\;\;
       \forall i\;:\;\otrans{\Gamma,\,\iota}{(\sigma_i,\,\delta_i)}{g_2}{S_i}
      }
      {\otrans{\Gamma,\,\iota}{(\sigma,\,\delta)}{g_1\wedge g_2}{\bigcup_i S_i}}\ruleno{Conj}
\]

\[
\crule{\otrans{\Gamma,\,\iota[x\gets\alpha]}{(\sigma,\,\delta\cup\{\alpha\})}{g}{S}}
      {\otrans{\Gamma,\,\iota}{(\sigma,\,\delta)}{\lstinline|fresh($x$) $\;g$|}{S}}
      {\alpha\in\meta{W}\setminus\delta}\ruleno{Fresh}
\]

\[
\crule{\otrans{\Gamma,\,[x_i\gets v_i]}{(\epsilon,\,\delta)}{g}{\bigcup_j\{(\sigma_j,\,\delta_j)\}}}
      {\otrans{\Gamma,\,\iota}{(\sigma,\,\delta)}{r^k\,t_1\dots t_k}{\bigcup_j\{(\sigma\circ\sigma_j,\,\delta_j)\}}}
      {\Gamma(r^k)=\lambda x_1\dots x_k.g,\,v_i=t_i\iota\sigma}\ruleno{Invoke}
\]
\end{minipage}      
\caption{Big-step operational semantics}
\label{semantics}
\end{figure*}

In this section we describe the syntax and semantics of the language, which is used in the rest of the paper. To some extent this description serves as
a short introduction to miniKanren. The main distinction between ``the real'' miniKanren and our version is that we give a proper semantics only to converging programs, 
which deliver a finite set of answers, while in the reality of relational programming the result is represented as an infinite stream, from which any number of answers can be requested, and the request of a non-existing answer can
lead to a divergence. Our semantics, thus, corresponds to scenario, when \emph{all} answers are requested from the stream. On the other hand, we do not distinguish programs, calculating the infinite number 
of answers, from those diverging with no results at all.   
However, we consider the finite version of the semantics as an important case, which is justified by the evaluation, presented in Section~\ref{evaluation}.

The syntax of our relational language is shown on Fig.~\ref{syntax}. First, we introduce the alphabet of constructors $\meta{C}$, each of which is equipped with a 
non-negative arity. Then we in a conventional fashion inductively define the set of all terms $\meta{T}(X)$, parameterized by the set of variables $X$. We need this parameterization 
since later we will be dealing with two sorts of variables~--- \emph{syntactic} and \emph{semantic}, and, therefore, two sorts of terms. Next, we choose the set of syntactic variables
$\meta{V}$ and the set of \emph{relational symbols} $\meta{R}$ with arities, which will be used as names for relational definitions. We also introduce a shortcut
$\meta{T_V}$ for the set of all terms over syntactic variables since it will be used in all other syntactic definitions.

The core syntax category in the language is a \emph{goal}. There are five types of goals: unification of two terms, conjunction and disjunction of two
goals, fresh variable introduction and a call of some relational definition. We stipulate that the calls of relational definitions respect their arities; we 
will also use a shortcut form \lstinline|fresh ($x$ $y$ $z$ ...) ...| instead of \lstinline|fresh($x$) (fresh ($y$) (fresh ($z$) ...)| where needed.

Next, a \emph{relational definition} $\meta{D}$ binds some relational symbol to a parameterized goal; the number of parameters corresponds to the arity of the symbol, and
we assume that all parameter variables are pairwise distinct. Finally, the top-level syntax category is a \emph{specification} $\meta{S}$~--- a goal in the context 
of some relational definitions. 

As an example of a relational program we consider a canonical specification~--- list concatenation relation \lstinline|append$^o$|:

\begin{lstlisting}  
   append$^o$ $\binds$ $\lambda\;x\;y\;xy$ . 
     (($x$ === $\;$Nil) /\ ($xy$ === $\;y$)) \/
     (fresh ($h$ $t$ $ty$)
        ($x$  === $\;$Cons ($h$, $t$)) /\
        (append$^o$ $t$ $y$ $ty$) /\
        ($xy$ === $\;$Cons ($h$, $ty$)))
\end{lstlisting}

We respect here the convention to add the ``$^o$'' suffix to all names of relational entities; for the simplicity we omit the arities of constructors \lstinline|A|, \lstinline|Cons|, \lstinline|Nil| and relational symbol \lstinline|append$^o$|.

Note, we define here a language with first-order relations; in particular, we do not allow partial application. As we see later, our approach critically depends
on recursive call identification, which is a trivial task in the first-order case. Some existing frameworks for relational programming~\cite{OCanren,RelConversion}
do not impose such a limitation; extending our approach for a higher-order case is a subject of future research.


We describe the semantics of our language using a conventional big-step style inference system. First, we choose an infinite
set of \emph{semantic variables} $\meta{W}$. As we will see shortly, the \lstinline|fresh($x$)...| construct allocates a fresh variable, not being
used before, and associates it with the syntactic variable $x$. Thus, in the semantics we will need an infinite supply of fresh variables.

Next, we introduce the \emph{interpretation} of syntactic variables $\iota$ as a (partial) mapping

$$
\iota : \meta{V} \to \meta{T}(\meta{W})
$$

The role of the interpretation is twofold: first, it binds syntactic variables, used in the \lstinline|fresh| construct, to their semantic counterparts, and second, 
it binds relational parameters to their values~--- terms over semantic variables. For a syntactic term $t$ and an interpretation $\iota$ we denote
$t\iota$ the result of substitution of all syntactic variables in $t$ by their interpretations according to $\iota$; we assume $t\iota$ to be defined 
only when $\iota$ is defined for all variables in $t$. Thus, $t\iota$, if defined, is always an element of $\meta{T}(\meta{W})$.

Then, we borrow some conventional machinery from unification theory~\cite{Unification}. Namely, we define a substitution $\sigma$ to
be a partial mapping between semantic variables and semantic terms:

$$
\sigma : \meta{W} \to \meta{T}(\meta{W})
$$

For any substitution $\sigma$ we assume that the set of all free variables of all terms in the range of $\sigma$ has an empty intersection with the
domain of $\sigma$, and we denote by $\sigma\circ\theta$ the composition of substitutions, defined in a usual way. For arbitrary $t\in\meta{T}(\meta{W})$ and
a substitution $\sigma$ we denote the result of application of $\sigma$ to $t$ as $t\sigma$.

The basic inference relation for our semantics has the form

$$
\otrans{\Gamma,\iota}{(\sigma,\,\delta)}{g}{S}
$$

\noindent where $\Gamma$ is an environment, which binds relational symbols to their definitions, $\iota$~--- an interpretation, $\sigma$~--- a substitution, 
$\delta$~--- a set of allocated semantic variables, $g$~--- a goal, and $S$~--- a set of pairs $(\sigma^\prime,\,\delta^\prime)$, where $\sigma^\prime$ and
$\delta^\prime$~--- a substitution and a set of allocated semantic variables respectively. Informally speaking, we interpret a goal $g$ in the context of
relational definitions $\Gamma$, current interpretation $\iota$, current substitution $\sigma$ and current set of allocated semantic variables $\sigma$. As a 
result, we obtain a (possibly empty) set of answers. Each answer consists of a new substitution, accumulated through the execution of $g$, and a new set of
allocated semantic variables (note, in the original miniKanren a goal can produce the same answer multiple number of times, but this property is not important
in our case).

The inference rules themselves are shown on Fig.~\ref{semantics}. The first two rules handle two possible outcomes of the unification. Note, we use here the most 
general unifier ($mgu$) of two semantic terms; we assume ``occurs check'' to be incorporated in the unification algorithm. Since the unification goal is built of 
syntactic terms, we have to interpret them first (by applying $\iota$), and take into account current substitution $\sigma$.

The rule for the disjunction first interprets the constituents of the disjunction in the same state and then combines the outcomes.
% using the multiset union operation ``$\biguplus$''. 

The rule for the conjunction threads the execution of its subgoals in a left-to-right successive manner: first the
left conjunct is evaluated, providing a set $\{(\sigma_i,\,\delta_i)\}$. Then the second conjunct is evaluated for each element of the set, and the
results are eventually combined. Note, the evaluation of both conjuncts is performed under the \emph{same} interpretation $\iota$ since both of them occur in the 
\emph{same} bounding context. The substitution and the set of allocated semantic variables, on the other hand, are inherited from left to right since the evaluation 
of the right conjunct has to be performed in the context of the results, provided by the left one. 

The rule for the \lstinline|fresh| construct allocates arbitrary semantic variable, not taken before, and evaluates the single subgoal in the updated interpretation, which
associates the syntactic variable, bound in this \lstinline|fresh|, with the chosen semantic one.

\FloatBarrier

Finally, the rule for relational definition invocation describes its evaluation in a few steps. First, the body of the definition is found, using the environment $\Gamma$. 
Then, the terms $t_i$, specified as the arguments of the invocation, are converted into their semantic forms $v_i$ using current interpretation $\iota$ and current 
substitution $\sigma$. Next, the body of the definition is evaluated in the context of \emph{empty} substitution $\epsilon$ and an interpretation, containing nothing
else, than the bindings for the formal parameters of the definition. This way of handling interpretation models the behavior of a call stack in conventional languages
with no nested functions. Finally, the result substitutions are composed with the original one\footnote{We could use the original
substitution instead of the empty one without the need to use composition; however we found the approach we took more proof-friendly since each relational definition is evaluated
in initially empty substitution.}. 

Given this big-step evaluation relation for goals, we can describe the evaluation for the top-level specification $s=d_1,\dots,d_k;g$. First, we construct the associated environment
$\Gamma_s$, which properly binds all relational symbols in $s$ to their bodies. Then, we evaluate the top-level goal

$$
\otrans{\Gamma_s,\,\bot}{(\epsilon,\,\emptyset)}{g}{S_s}
$$

\noindent obtaining the set of results $S_s$; here we use empty (everywhere undefined) interpretation $\bot$ and empty substitution $\epsilon$ as a starting point. 
Finally, we choose all substitutions from $S_s$. 

Our semantics is almost deterministic~--- the only source of ambiguity is the rule for the \lstinline|fresh| construct, where we choose a new semantic variable
arbitrarily. If we fix the order, in which semantic variables are allocated, the semantics becomes completely deterministic. It is also easy to see that if each
relational symbol is unambiguously defined in the specification and called with a proper number of parameters, and all goals in all relational definitions and the 
top-level goal are closed (i.e. each variable occurrence is bound either in some \lstinline|fresh| construct or in a parameter list of enclosing definition), 
then during the evaluation all syntactic variables are properly interpreted~--- in other words, the execution cannot break down halfway through and either diverges or 
finishes with some results.

\begin{figure*}[t]
\arraycolsep=5pt
\def\arraystretch{2.2}
\subfloat[\label{appendo_eval_a}]{
$
\begin{array}{c|c}
  \multicolumn{2}{c}{\otrans{\bot}{(\epsilon,\,\emptyset)}{\lstinline|fresh ($q$)|\;...}{...}}\\
  \hline
  \multicolumn{2}{c}{\otrans{[\bnd{q}{\sv{0}}]}{(\epsilon,\,\{\sv{0}\})}{\lstinline|append$^o$ (Cons (A, Nil)) Nil $\;q$|}{...}}\\
  \hline
  \multicolumn{2}{c}{\otrans{[\bnd{x}{\lstinline|Cons(A, Nil)|},\,\bnd{y}{\lstinline|Nil|},\,\bnd{xy}{\sv{0}}]}{(\dots)}{\lstinline|(...) $\vee$ (...)|}{...}}\\
  \hline
  \otrans{\dots}{(\dots)}{\lstinline|($x\;$ === $\;$Nil) $\wedge$ (...)|}{\emptyset} & \multirow{2}*{Fig.~\ref{appendo_eval_b}}\\
  \cline{1-1}
  \otrans{\dots}{(\dots)}{\lstinline|$x\;$ === $\;$Nil|}{\emptyset} & 
\end{array}
$
}\\
\subfloat[\label{appendo_eval_b}]{
$
\begin{array}{c|c}
\multicolumn{2}{c}{\otrans{[\bnd{x}{\lstinline|Cons(A, Nil)|},\,\bnd{y}{\lstinline|Nil|},\,\bnd{xy}{\sv{0}}]}{(\epsilon,\,\{\sv{0}\})}{\lstinline|fresh($h\;t\;ty$) ...|}{...}}\\
\hline
\multicolumn{2}{c}{\otrans{[\dots,\,\bnd{h}{\sv{1}},\,\bnd{t}{\sv{2}},\,\bnd{ty}{\sv{3}}]}{(\epsilon,\,\{\sv{0}..\sv{3}\})}{\lstinline|($x\;$ === $\;$Cons ($h$, $\;t$)) $\wedge$ (...)|}{...}}\\
\hline
\otrans{\dots}{(\epsilon,\,\{\sv{0}..\sv{3}\})}{\lstinline|$x\;$ === $\;$Cons ($h$, $\;t$)|}{\{([\bnd{\sv{1}}{\lstinline|A|},\,\bnd{\sv{2}}{\lstinline|Nil|}],\,\{\sv{0}..\sv{3}\})\}} & \mbox{Fig.~\ref{appendo_eval_c}}
\end{array}
$
}\\
\subfloat[\label{appendo_eval_c}]{
$
\begin{array}{c|c|c}
\multicolumn{3}{c}{\otrans{\dots}{\{([\bnd{\sv{1}}{\lstinline|A|},\,\bnd{\sv{2}}{\lstinline|Nil|}],\,\{\sv{0}..\sv{3}\})\}}{\lstinline|(append$^o$ $\;t\;$ $y\;$ $ty$) $\wedge$ (...)|}{...}}
\\
\hline
\multicolumn{2}{c|}{\otrans{\dots}{(\dots)}{\lstinline|append$^o$ $\;t\;$ $y\;$ $ty$|}{...}} & 
\multirow{4}*{Fig.~\ref{appendo_eval_d}} \\
\cline{1-2}
\multicolumn{2}{c|}{\otrans{[\bnd{x}{\lstinline|Nil|},\,\bnd{y}{\lstinline|Nil|},\,\bnd{xy}{\sv{3}}]}{(\epsilon,\,\{\sv{0}..\sv{3}\})}{\lstinline|(...) $\vee$ (...)|}{...}} & \\
\cline{1-2}
\multicolumn{2}{c|}{\otrans{\dots}{(\dots)}{\lstinline|($x\;$ === $\;$Nil) $\wedge$ ($xy\;$ === $\;y$)|}{...}} &  \\
\cline{1-2}
\otrans{\dots}{(\dots)}{\lstinline|$x\;$ === $\;$Nil|}{(\dots)} & 
\otrans{\dots}{(\dots)}{\lstinline|$xy\;$ === $\;y$|}{\{([\bnd{\sv{3}}{\lstinline|Nil|}],\,\{\sv{0}..\sv{3}\})\}} &
\end{array}
$}\\
\subfloat[\label{appendo_eval_d}]{
$
\begin{array}{c}
\otrans{\dots}{([\bnd{\sv{1}}{\lstinline|A|},\,\bnd{\sv{2}}{\lstinline|Nil|},\,\bnd{\sv{3}}{\lstinline|Nil|}],\,\{\sv{0}..\sv{3}\})}{\lstinline|$xy\;$ === $\;$Cons ($h$, $\;ty$)|}{\{([\dots,\,\bnd{\sv{0}}{\lstinline|Cons (A, Nil)|}],\,\{\sv{0}..\sv{3}\})\}}
\end{array}
$}
\caption{An example of relational evaluation}
\label{appendo_eval}
\end{figure*}

We illustrate the evaluation, determined by this semantics, by the following query:

\begin{lstlisting}  
   fresh ($q$) (append$^o$ (Cons (A, Nil)) Nil $q$)
\end{lstlisting}

\noindent where \lstinline|append$^o$| is a list concatenation relation, presented earlier in this section. Since we require the 
top-level goal to be closed, from now on we conventionalize the use of a top-level \lstinline|fresh| construct as a binder for the variables, whose values we are 
interested in (in this particular example $q$).

The evaluation is illustrated on Fig.~\ref{appendo_eval}. Note, the derivation is shown in a top-down manner, opposite to the
direction, prescribed by the inference rules. We use numbers in bold font to denote semantic variables. For the sake of brevity and in order to
make the illustration observable we do not show the binding environment for relational definitions and as a rule denote by ellipses all inherited components
of derivation tree (the components in the left side of ``$\Rightarrow$'' are inherited top-down, in the right side~--- bottom-up).

We start from the top-level goal and first apply the rule $\rulename{Fresh}$ (see Fig.~\ref{appendo_eval_a}). Since 
we did not use any semantic variables yet, we allocate the first one ($\sv{0}$), update the interpretation and the set of used semantic variables and continue. The next construct
is the call for \lstinline|append$^o$|, so we unfold its definition, replace the interpretation of syntactic variables by the bindings for the formal parameters, and 
evaluate the body w.r.t. the empty substitution (which has no difference from the current one yet). The body of \lstinline|append$^o$| definition is a disjunction, so we
take its left constituent, which is a conjunction, so we in turn take its left constituent, which is a unification \mbox{\lstinline|$x\;$ === $\;$Nil|}. This unification clearly fails, as
current interpretation binds $x$ to \lstinline|Cons (A, Nil)|. This completes the whole branch for the first disjunct of \lstinline|append$^o$| with empty result.

The evaluation of the second disjunct is shown on Fig.~\ref{appendo_eval_b}. Its top-level construct is \lstinline|fresh ($h\;t\;ty$)|, so we allocate three successive 
semantic variables $\sv{1}$, $\sv{2}$ and $\sv{3}$ and save the bindings in the interpretation. The next construct is a conjunction of three goals (assuming
``$\wedge$'' is right-associative in the concrete syntax), and we proceed with the first one, which is a unification \mbox{\lstinline|$x\;$ === $\;$Cons($h$, $\;t$)|}. 
Since $x$, $h$ and $t$ are free in current substitution and $x$ is bound to \lstinline|Cons(A, Nil)| by current interpretation, the unification succeeds with the 
substitution \mbox{$[\bnd{\sv{1}}{\lstinline{A}},\,\bnd{\sv{2}}{\lstinline|Nil|}]$}. The evaluation of remaining conjuncts is shown on Fig.~\ref{appendo_eval_c}.

The first one is a recursive call to \lstinline|append$^o$|. We evaluate the actual parameters~--- $t$, $y$ and $ty$~--- in current interpretation and substitution, 
obtaining \lstinline|Nil|, \lstinline|Nil| and $\sv{3}$ respectively, replace the interpretation to bind formal parameters to these values, and recurse to the body with
empty current substitution. Again, we have the disjunction, and the first disjunct is a conjunction \mbox{\lstinline|($x\;$ === $\;$Nil) $\wedge$ ($xy\;$ === $\;y$)|}.
Now \mbox{\lstinline|$x\;$ === $\;$Nil|} succeeds, as $x$ is already bound to \lstinline|Nil| by the interpretation, and \mbox{\lstinline|$xy\;$ === $\;y$|} succeeds
as well, providing a new substitution \mbox{$[\bnd{\sv{3}}{\lstinline|Nil|}]$}. We omit the detailed evaluation of the second top-level disjunct of \lstinline|append$^o$| since
it contains a unification \lstinline|$x\;$ === $\;$Cons (_, _)| which, clearly, does not contribute anything.

Finally, we return from the recursive call to \lstinline|append$^o$| and take the composition of substitutions~--- the one before the call, and another after~--- which
gives us \mbox{$[\bnd{\sv{1}}{\lstinline|A|},\,\bnd{\sv{2}}{\lstinline|Nil|},\,\bnd{\sv{3}}{\lstinline|Nil|}]$} (see Fig.~\ref{appendo_eval_d}). We only need now to 
interpret the last conjunct of the second disjunct of \lstinline|append$^o$|~--- \mbox{\lstinline|$xy\;$ === $\;$Cons ($h$, $\;ty$)|}~--- which gives us the
final substitution \mbox{$[\bnd{\sv{1}}{\lstinline|A|},\,\bnd{\sv{2}}{\lstinline|Nil|},\,\bnd{\sv{3}}{\lstinline|Nil|},\,\bnd{\sv{0}}{\lstinline|Cons (A, Nil)|}]$}. Now, we
have to remember that the topmost bound variable of the top-level goal is $q$, and corresponding semantic variable is $\sv{0}$. Thus, the answer is 
\mbox{$q\;=\;\lstinline|Cons (A, Nil)|$}, which is rather expected.
