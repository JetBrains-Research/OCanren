\appendix

\section{The Proof of Theorem~\ref{thm:main}}
\label{sec:app}

First, we make a number of observations which will simplify following reasoning.

We need to relate two semantics~--- the angelic one presented in Section~\ref{sec:angelic-semantics} and
the semantics by well quasi-ordering presented in Section~\ref{sec:fair-semantics}~--- and prove that
arbitrary goal simultaneously converges or diverges in them. Both semantics are represented
as labeled transition systems with the same alphabet of labels and tree-shaped states in which interior
nodes correspond to disjunctions and leaves contain ordered conjunctions of relation applications (plus a
substitution, the first fresh semantic variable, etc.) Moreover, both semantics share the same
set of rules dealing with disjunction; both semantics are sound and complete w.r.t. the denotational
semantics of \mk. The completeness means that in a state no leaf can be excluded from consideration~---
each leaf of a state sooner of later (meaning: in a finite number of steps) will be considered and some
its relation application will be unfolded. This allows us to abstract from concrete shapes of states and
instead consider them as sets of leaves. The following lemma justifies this transition.

\begin{lemma}[Convergence of Leaves]
  \label{lem:convergence-of-leaves}
  Let $s$ be a state with leaves $\{\omega_i\}$. Then $s$ converges (in angelic semantics or the semantics by
  well quasi-ordering) iff each $\omega_i$ converges.
\end{lemma}
\begin{proof}
  The lemma follows from a more general fact: let

  \[
  s\to s_1\to s_2 \to \dots
  \]

  be a (finite or infinite) derivation sequence in either semantics of interest (we may consider
  only angelic semantics since the semantics by well quasi-ordering constitutes its special case).
  Then:

  \begin{itemize}
    \item each state $s_i$ can be represented as
      
      \[
      s_i\,[w_1,\,\dots,\,w_k]
      \]
      
      where $\{w_j\}$ is a collection of disjoint subtrees of $s_i$, containing all its leaves;
      
    \item each step can be represented either as
      
      \[
      s_i\,[\dots w_j\dots]\to s_i\,[\dots w^\prime_j\dots]
      \]

      where there is a derivation step

      \[
      w_j\to w^\prime_j
      \]

      or as

      \[
      s_i\,[\dots w_j\dots]\to s^\prime_i\,[\dots w_{j-1},\,w_{j+1},\,\dots]
      \]

      where $s^\prime_i$ is a reordering of all $\{w_l\mid l\ne j\}$;
     
    \item for each $w_j$ there is a derivation sequence

      \[
      \omega_k\to\dots\to w_j
      \]

      for some $k$.
  \end{itemize}

  Indeed, in the base case all leaves of $s$ are precisly $\{\omega_i\}$, which satisfy all requirements. Inductively, let
  us aleady have an intermediate state

  \[
  s_i\,[w_1,\,\dots,\,w_k]
  \]

  with all $\{w_j\}$ satisfying the claim. The application of rules for disjunction will traverse the state from the root to
  a certain leaf (if the state does not contain disjunctions then the whole case degenerates to $s_i=w_1$ and the claim
  holds trivially) and, hence, will arrive at a certain $w_j$. Then the next state can be represented as

  \[
  s^\prime_i\,[\dots w^\prime_j\dots]
  \]
  

  if $w_j\to w^\prime_j$ and $w^\prime\ne\times$ or as
  
  \[
  s_i\,[\dots w_{j-1},\,w_{j+1},\,\dots]
  \]

  if $w_j\to\times$ and $s^\prime_i[\dots,\,w_{j-1},\,w_{j+1},\,\dots]$ is a reordering of remaining $\{w_l\mid l\ne j\}$.
  
  In other words, any derivation for a state can be decomposed into the derivations for its leaves and vice-versa.  
\end{proof}

The next observation concerns the ``commutativity'' of sequential unifications. Let $p_{1,2},\,q_{1,2}$ be some terms. Then

\[
\begin{array}{c}
  mgu\,(p_1,\,p_2)\cdot mgu\,(q_1,\,q_2)=\\
  \phantom{XXXXXXX}mgu\,(q_1,\,q_2)\cdot mgu\,(p_1,\,p_2)
\end{array}
\]

Indeed, both

\[
mgu\,(p_1,\,p_2)\cdot mgu\,(q_1,\,q_2)
\]

and

\[
mgu\,(q_1,\,q_2)\cdot mgu\,(p_1,\,p_2)
\]

are most general unifiers for pairs $(p_1,\,p_2)$ and $(q_1,\,q_2)$ simultaneously. If there were different, then terms $C\,(p_1,\,q_1)$ and
$C\,(p_2,\,q_2)$, where $C$ is some binary constructor, would have \emph{different} most general unifiers, which is impossible\footnote{Strictly
  speaking, the validity of these claims depends on the representation of substitution; for example, in concrete representation
  the substitutions $[x\mapsto y]$ and $[y\mapsto x]$ may not be equal, but equivalent.}. Taking this into account we arrive at the
following property. Let

\[
\theta\xrightarrow{u_1;\,u_2;\dots u_k}{\theta^\prime}
\]

be a sequence of unifications which transforms a substitution $\theta$ into a substitution $\theta^\prime$. Then

\[
\theta\xrightarrow{u_{\pi(1)};\,u_{\pi(2)};\dots u_{\pi(k)}}{\theta^\prime}
\]

for any permutation $\pi$.

Another subtlety concerns the order of fresh semantic variables allocation. Let us have a state $\inbr{\theta,\,i,\,ab}$, where
$\theta$ is a substitution, $a$ and $b$ are relation invocations and $i$ is the next free semantic variable. In angelic semantics we
may perform the unfoldings of $a$ and $b$ in different orders. Since the bodies for both $a$ and $b$ may contain \lstinline|fresh| constructs,
these constructs will be evaluated in different orders delivering different concrete semantic variables and different substitutions. However
these substitutions will be $\alpha$-equivalent; this observation reflects the intuitively trivial fact that the order of semantic variable
allocation is not essential as long as no variable is allocated more than once in the evalutaion within the same branch. Thus, we discard
the \lstinline|fresh| construct from considerations and will treat all substitutions up to $\alpha$-equivalence. This, in particular,
will allow us to omit the next fresh variable component from states.

In the light of all previously made observations we may reconsider the way the evaluations are performed in both
angelic semantics and semantics by well quasi-ordering. Namely, the semantics build a tree of states in the form
$\inbr{\theta,\,\rho}$, where $\theta$ is a substitution and $\rho$ is a conjunction (a list) of relation application.
Each time a one-step unfolding is performed for a state a number of descendants for the corresponding node is
added to the tree; we will denote this step as

\[
\inbr{\theta,\,\rho}\to\{\inbr{\theta_i,\,\rho_i}\}
\]

A finite derivation tree corresponds to a converging evaluation; otherwise, by the K\H{o}nig's lemma, it will contain an infinite path. 
According to the semantics of one-step unfolding, any finite branch of a derivation tree can either end in a \emph{contradiction} state ``$\times$''
or in \emph{answer} state $\inbr{\theta,\,\epsilon}$, where $\epsilon$ designates an empty conjunction, ``true''. We say that a derivation
converges to ``$\times$'' if it converges to a finite tree with all leaves being ``$\times$''.

For a state $s$ we denote $h\,(s)$ the height of some finite derivation tree in angelic semanics; althought this value is defined ambiguously, this
definition turnes out to be sufficient.

\begin{lemma}
  \label{lem:one-step-liveness}
  Let

  \[
  \inbr{\theta,\, \phi (r,\, h) \psi}
  \]

  be a state in the semantics by well quasi-ordering, where $h$ is a history and $r$ is a relation application which
  is unfolded in the first step in the derivation according to  ``$\rightarrow_{\mathcal{P}_\trianglelefteq}$''. Then each infinite branch
  of the derivation tree (if any) will contain an unfolding of one of the relation applications in $\phi$ or $\psi$.
\end{lemma}
\begin{proof}
By contradiction.

Take an infinite branch and assume that none of the relation applications from $\phi$ and $\psi$ unfolded.
Then we have an infinite branch of states

     \[   
     \inbr{\theta,\, \phi (r,\, h) \psi} \rightarrow \inbr{\theta_1,\, \phi \pi_1 \psi} \rightarrow \ldots \rightarrow \inbr{\theta_d,\, \phi \pi_d \psi} \rightarrow \ldots
     \]      
        
By the property of well quasi-ordering there is a number

   \[
   K = \max_{\substack{d\in\mathbb{N}\\(r,h) \in \pi_d}}\,length\,(h)
   \]

since no history can be of infinite length. Thus, the maximal possible number of unfoldings can be estimated. Indeed,
the conjunct $r$ can be unfolded no more than $T = K - length\,(h)$ times (otherwise, the length of the history
of descendants will exceed $K$). Also, each unfolding adds no more than $S$ new conjuncts, where $S$ is maximum
number of conjuncts in all branches of all relation definition (the maximum exists, since there are a finite number of relations with
finite number of branches each). Thus, the total number of unfoldings $R$ can be estimated

    \[
    R \le \sum_{k=0}^{T}S^k = \frac{S^{T + 1} - 1}{S - 1}
    \]  
        
    which contradicts the infinity of the branch.
\end{proof}

\begin{lemma}
  \label{lem:histories-devastation}
  Let $\inbr{\theta,\, \phi}$ be a state in the semantics by well quasi-ordering. Then each infinite branch of the derivation tree (if any)
  will contain a state $\inbr{\theta^\prime,\, \psi}$ with empty histories for each conjunct in $\psi$.
\end{lemma}

\begin{proof}
  Set $t = n - k$, where $n$ is the total number of conjuncts in the state and $k$ is the index of conjunct $r$ which is unfolded first.
  If we don't have conjunct which would be unfolded first then set $t = 0$.

  Induction on $t$.
          
 \begin{itemize}
    \item Base case: $t = 0$.      
          In this case the next step will be \rulen{ConjClear}. After this step we will get a state with empty histories for each conjunct.
    \item Induction step: $t = m$. We also assume that the lemma holds for $t < m$.
          We will complete the unfolding in the conjunct $r$ in each infinite branch of the derivation tree by Lemma~\ref{lem:one-step-liveness}.
          If the next conjunct which will be unfolded is on the right of $r$ then $t$ decreases, which satisfies the induction hypothesis. If the next
          conjunct which will be unfolded is on the left of $r$ then the previous step was \rulen{ConjClear}, and the previous state satisfies the
          induction hypothesis with $t = 0$.
 \end{itemize}
\end{proof}

\begin{lemma}
  \label{lem:empty-stories-liveness}
  Let $\inbr{\theta,\, \phi r \psi}$ be a state in the semantics by well quasi-ordering where $r$ is a relation application
  with empty history. Then the conjunct $r$ will be unfolded in any infinite branch in the derivation tree.
\end{lemma}

\begin{proof}
 Set $t = m - k$ where $m$ is the index of conjunct $r^\prime$ which will be unfolded first and $k$ is index of conjunct $r$.

Induction on $t$.

\begin{itemize}
    \item Base case: $t = 0$. In this case $m = k$ and $r = r^\prime$. Moreover, the history for $r$ is empty. This means that the conjunct $r$ will
    be unfolded at current step.
    \item Induction step: $t = m$. We also assume that the lemma holds for $t < m$.
    We will complete the unfolding in the conjunct $r^\prime$ in each infinite branch of the derivation tree by Lemma~\ref{lem:one-step-liveness}.
    The next conjunct which will be unfolded is to right of $r$ (because we have a conjunct with empty history). And the next conjunct is either $r$ or
    on the left of $r$ (because $r$ has empty history). Thus, $m$ increases and $t$ decreases for the new state. As a result this state satisfies
    the induction hypothesis in both cases.
\end{itemize}
\end{proof}


\begin{lemma}[Liveness of the Semantics by Well Quasi-Ordering]
  \label{lem:liveness}
Let $\inbr{\theta,\, \phi r \psi}$ be a state in the semantics by well quasi-ordering. Then the conjunct $r$ will be unfolded in each infinite branch in the derivation tree.
\end{lemma}
\begin{proof}
If $r$ has empty history we just apply Lemma~\ref{lem:empty-stories-liveness}.

If $r$ has a nonempty history we can apply Lemma~\ref{lem:histories-devastation}. As a result, we get a new state $\inbr{\theta^\prime,\, \pi}$ where all conjuncts
has empty history in each infinite branch of the derivation.

\begin{itemize}
    \item If $r$ already was unfolded then this infinite branch of the derivation satisfy the requirements of the lemma.
    \item If $r$ was not unfolded yet then $\pi = \pi_1 r \pi_2$ and $r$ has an empty history. Due to this, we can apply 
    Lemma~\ref{lem:empty-stories-liveness} and $r$ will be unfolded in the each infinite branch of the derivation tree for the state $\inbr{\theta^\prime,\, \pi}$.
    And all these branches are continuations of branches for the state $\inbr{\theta,\, \phi r \psi}$.
\end{itemize}
\end{proof}

\begin{lemma}[Effect of One-step Unfolding]
\label{lem:one-step-unfolding}

Let $\inbr{\theta,\,r}$ be a state; let

\[
\inbr{\theta,\,r}\to\{\inbr{\theta_i,\,\rho_i}\}
\]

Then for each $i$ there is a sequence of unifications $u^i_1\dots u^i_{k_i}$ such that

\[
\theta\xrightarrow{u^i_1\dots u^i_{k_i}}{\theta_i}
\]
\end{lemma}
\begin{proof}
By the induction on derivation for one-step unfolding.
\end{proof}

Thus, we may introduce a denotation $[r\to\rho_i]$ for the effect of the sequence of unifications performed during a one-step unfolding of $r$ which
delivers a state $\rho_i$. 

\begin{lemma}[Weakening]
  \label{lem:weakening}

  Let $\inbr{\theta,\,\omega}$ converges in angelic semantics (or the semantics by well quasi-ordering). Then for arbitrary substitution $\sigma$ 

  \[
  \inbr{\theta\cdot\sigma,\,\omega}
  \]

  converges and

  \[
  h\,(\inbr{\theta\cdot\sigma,\,\omega})\le h\,(\inbr{\theta,\,\omega})
  \]
  
\end{lemma}
\begin{proof}
  By the induction on derivation.
\end{proof}

\begin{lemma}[Weakening in Contradiction]
  \label{lem:weakening-contradiciton}

  Let $\inbr{\theta,\,\omega}$ converges to $\times$ in angelic semantics (or the semantics by well quasi-ordering). Then for arbitrary substitution $\sigma$ and
  arbitrary list of applications $\psi$

  \[
  \inbr{\theta\cdot\sigma,\,\omega\psi}
  \]

  converges to $\times$ and

  \[
  h\,(\inbr{\theta\cdot\sigma,\,\omega\psi})\le h\,(\inbr{\theta,\,\omega})
  \]
  
\end{lemma}
\begin{proof}
  By the induction on derivation.
\end{proof}

These two lemmas show rather a trivial fact that putting additional constraints (by considering a more specific substitution and adding
more conjuncts) can not make a converging program diverge.

\begin{lemma}[Convergence Preservation]
  \label{lem:convergence-preservation}

  Let $\inbr{\theta,\,\phi a \phi^\prime}$ be a state; let

  \[
  \inbr{\theta,\,a}\to\{\inbr{\theta\cdot[a\to\alpha_i],\,\alpha_i}\}
  \]
  
  Then

  \[
  \inbr{\theta,\,\phi a \phi^\prime}
  \]

  converges iff all

  \[
  \{\inbr{\theta\cdot[a\to\alpha_i],\,\phi\alpha_i\phi^\prime}\}
  \]

  converge.
\end{lemma}
\begin{proof}

  \mbox{}
  
  \begin{itemize}
  \item[$\Leftarrow$] Let all $\{\inbr{\theta\cdot[a\to\alpha_i],\,\phi\alpha_i\phi^\prime}\}$ converge. Then unfolding of $a$ delivers a converging derivation for $\inbr{\theta,\,\phi a \phi^\prime}$.
  \item[$\Rightarrow$] Let $\inbr{\theta,\,\phi a \phi^\prime}$ converges. Two cases are possible.
    \begin{enumerate}
    \item There is no unfolding of $a$ in the converging derivation. Then
      
      \[
      \inbr{\theta,\,\phi a \phi^\prime}
      \]
      
      converges to $\times$ (each state contains $a$). Moreover
      
      \[
      \inbr{\theta,\,\phi \phi^\prime}\eqno{(*)}
      \]
      
      converges to $\times$ (we can repeat the converging derivation of $\inbr{\theta,\,\phi a \phi^\prime}$).      
      By weakening in contradiction all
      
      \[
      \inbr{\theta\cdot[a\to\alpha_i],\,\phi\alpha_i\phi^\prime}
      \]
      
      converge to $\times$ taking into account $(*)$.
      
    \item There is a sequence of derivation
      
      \[
      \inbr{\theta,\,\phi a \phi^\prime}\to\dots\to \inbr{\theta\cdot\sigma,\,\pi a \rho}\eqno{(**)}
      \]
      
      where $\sigma$ is some sequence of unifications and the next step is unfolding of $a$. Then its result is
      
      \[
      \{\inbr{\theta\cdot\sigma\cdot[a\to\alpha_i],\,\pi \alpha_i \rho}\}
      \]
      
      Taking the state
      
      \[
      \inbr{\theta\cdot[a\to\alpha_i],\,\phi\alpha_i \phi^\prime}
      \]
      
      and repeating the unfoldings of $(**)$ we can (at most) arrive at the state
      
      \[
      \inbr{\theta\cdot[a\to\alpha_i]\cdot\sigma,\,\pi \alpha_i \rho}
      \]
      
      which converges since
      
      \[
      \theta\cdot[a\to\alpha_i]\cdot\sigma=\theta\cdot\sigma\cdot[a\to\alpha_i]
      \]
    \end{enumerate}
  \end{itemize}
\end{proof}

%\begin{corollary}
%  \label{corollary:corollary}
%  Let $\inbr{\theta,\,\omega}$~--- some state. Let $\{\inbr{\theta^\prime,\,\omega^\prime}\}$ be a set of states, obtained from $\inbr{\theta,\,\omega}$
%  in a finite number of unfoldings. Then $\inbr{\theta,\,\omega}$ converges iff all $\{\inbr{\theta^\prime,\,\omega^\prime}\}$~--- converge.
%\end{corollary}

\begin{lemma}[Derivation Height Monotonicity]
  \label{lem:height-mono}
  Let $\inbr{\theta,\,\phi a \omega b \psi}$ be a state converging in angelic semantics. Let
  
  \[
  \inbr{\theta,\,b}\to\{\inbr{\theta\cdot[b\to\beta_i],\,\beta_i}\}
  \]

  Then for each $i$ $\inbr{\theta\cdot[b\to\beta_i],\,\phi a \omega \beta_i \psi}$ converges as well,
  and

  \[
  h\,(\inbr{\theta\cdot[b\to\beta_i],\,\phi a \omega \beta_i \psi})\le h\,(\inbr{\theta,\,\phi a \omega b \psi})
  \]  
\end{lemma}
\begin{proof}
  The convergence follows immediately from Lemma~\ref{lem:convergence-preservation}.
  
  For the height, consider the following state

  \[
   \inbr{\theta\cdot[b\to\beta_i],\,\phi a \omega b \psi}
  \]

  By weakening,

  \[
  h\,(\inbr{\theta\cdot[b\to\beta_i],\,\phi a \omega b \psi})\le h\,( \inbr{\theta,\,\phi a \omega b \psi})
  \]

  In a converging derivation for $\inbr{\theta\cdot[b\to\beta_i],\,\phi a \omega b \psi}$ there can be branches which do not unfold $b$; these
  branches will not be affected if we change $b$ to $\beta_i$. Let $\inbr{\theta\cdot[\dots],\,\pi b \delta}$ be some state in the derivation
  tree in which an unfolding of $b$ is performed, where $[\dots]$ is an effect of unifications accumulated along the path from the root state.
  Then one of its descendants corresponds to the state

  \[
  \inbr{\theta\cdot[\dots]\cdot[b\to\beta_i],\,\pi\beta_i\delta}
  \]
  
  If we change $b$ to $\beta_i$ in the root state we won't be able to perform the unfolding of $b$ anymore; however, the state in which this
  unfolding was performed will turn into

  \[
  \inbr{\theta\cdot[b\to\beta_i]\cdot[\dots],\,\pi\beta_i\delta}
  \]

  Thus, replacing $b$ with $\beta_i$ can only lead to shrinking the derivation tree.
\end{proof}

\begin{corollary}
  \label{cor:corollary}
  Let $\inbr{\theta,\,\phi a \psi}$ be a converging state in the angelic semantics. Let

  \[
  \inbr{\theta,\,\phi\psi}\to\dots\to\inbr{\theta^\prime,\,\phi^\prime\psi^\prime}
  \]

  be a path in the derivation tree. Then

  \[
  \inbr{\theta^\prime,\,\phi^\prime a \psi^\prime}
  \]

  converges in the angelic semantics and

  \[
  h\,(\inbr{\theta^\prime,\,\phi^\prime a \psi^\prime})\le h\,(\inbr{\theta,\,\phi a \psi})
  \]  
\end{corollary}

The corollary establishes an intuitively trivial fact that a computation can not take longer if some of its sub-computations were
already performed beforehead.

\begin{proof}[The Proof of Theorem~\ref{thm:main}]
  By Lemma~\ref{lem:convergence-of-leaves} it is enough to prove the theorem for arbitrary state $s=\inbr{\theta,\,\rho}$. Since $s$
  converges in the angelic semantics (by the condition of the theorem) we can take its finite derivation tree. The proof proceeds by induction
  on its height.

  Base case: $h\,(s)=1$. This means that a single unfolding was made in the angelic semantics to construct a finite tree. Only two cases
  are possible.

  \begin{itemize}
  \item $\rho=r$ (a single application) and $\inbr{\theta,\,r}\to\{\inbr{\theta^\prime,\,\epsilon}\}$ (its unfolding in the context of $\theta$
    eliminates all relation applications in the body of $r$ and immediately comes to an answer $\theta^\prime$). In the semantics by well quasi-ordering
    only one step is possible~--- the unfolding of $r$~--- which delivers the same result.
    
  \item $\rho=\phi r \psi$ and the unfolding of $r$ in the context of $\theta$ delivers a contradiction:

    \[
    \inbr{\theta,\,r}\to\times
    \]

    In this case the angelic semantics ``guessed'' an application to unfold which immediately comes to a contradiction and thus eliminates the
    need for further computations.

    Take the derivation tree for $s$ in the semantics by well quasi-ordering. If it finite, the proof completed. Otherwise there will be an infinite branch. By
    Lemma~\ref{lem:liveness}, one state on this branch unfolds $r$. This state has a form $\inbr{\theta\cdot[\dots],\,\pi r \delta}$ where
    $[\dots]$~--- an effect of some unifications performed along the path from $s$. By Lemma~\ref{lem:weakening-contradiciton}
    
    \[
    \inbr{\theta\cdot[\dots],\,r}\to\times
    \]

    which contradicts the divergence and concludes the proof.
    
  \end{itemize}

  Induction step: let $h\,(s)=k+1$. Take a converging derivation in angelic semantics; as the first step it performs an unfolding of
  some application, say, ``$a$'':

  \[
  \inbr{\theta,\,\phi a \psi} \to \{\inbr{\theta\cdot[a\to\alpha_i],\,\phi\alpha_i\psi}\}
  \]

  Two cases are possible.

  \begin{enumerate}
    
  \item Exactly the same unfolding is performed according to the semantics by well quasi-ordering. Then we arrive at
    exactly the same descendant states, for each of which there is exists a derivation tree in the angelic semantics
    with the height of no more than $k$. Application of inductive hypothesis completes the proof.
    
  \item An unfolding of some other application, say, $b$ is performed in the semantics by well quasi-ordering. If we
    rewrite the initial conjunction as $\rho=\phi a \omega b \psi$ then in angelic semantics we have

    \[
    \inbr{\theta,\,\phi a \omega b\psi}\to\{\inbr{\theta\cdot[a\to\alpha_i],\,\phi \alpha_i \omega b\psi}\}
    \]

    as the first step of derivation while in the semantics by well quasi-ordering

    \[
    \inbr{\theta,\,\phi a \omega b\psi}\to_{\mathcal{P}_\trianglelefteq}\{\inbr{\theta\cdot[b\to\beta_j],\,\phi a \omega \beta_j\psi}\}
    \]

    If the derivation in the semantics by well quasi-ordering converges, the proof completed. Otherwise, the derivation
    tree will have an infinite branch, and by Lemma~\ref{lem:liveness} there will be a state in which
    application $a$ is unfolded. This state will have the form

    \[
    \inbr{\theta\cdot[\dots],\,\pi a \delta}
    \]

    where 

    \[
    \inbr{\theta,\,\phi a \omega b\psi}\to_{\mathcal{P}_\trianglelefteq}\dots\to_{\mathcal{P}_\trianglelefteq}\inbr{\theta\cdot[\dots],\,\pi a \delta}
    \]

    in the semantics by well quasi-ordering. We can repeat exactly the same unfolding steps in the
    angelic semantics for each of the the states

    \[
    \inbr{\theta\cdot[a\to\alpha_i],\,\phi \alpha_i \omega b\psi}
    \]

    since none of them touches $a$, which gives us the derivations

    \[
    \begin{array}{c}
      \inbr{\theta\cdot[a\to\alpha_i],\,\phi \alpha_i \omega b \psi}\to\dots\to \\
      \phantom{XXXXXXXXXX}\inbr{\theta\cdot[a\to\alpha_i]\cdot[\dots],\,\pi\alpha_i\delta}
    \end{array}\eqno{(***)}
    \]

    On the other hand, we know that the semantics by well quasi-ordering performs the step

    \[
    \begin{array}{c}
      \inbr{\theta\cdot[\dots],\,\pi a \delta}\to_{\mathcal{P}_\trianglelefteq}\\
      \phantom{XXXXXX}\inbr{\theta\cdot[\dots]\cdot[a\to\alpha_i],\,\pi \alpha_i \delta}
    \end{array}
    \]

    There is a little subtlety in this reasoning: indeed, we perform the steps prescribed by the semantics
    by well quasi-ordering in a \emph{different} substitution ($\theta\cdot[a\to\alpha_i]$ vs. $\theta$). This
    can potentially lead to discovering some contradictions and discarding some of the branches. However, the same
    branches will be discarded when the semantics by well quasi-ordering will perform the unfolding of ``$a$''
    due to the commuativity of unifications.

    Applying Corollary~\ref{cor:corollary} to $(***)$, we get

    \[
    \begin{array}{c}
      h\,(\inbr{\theta\cdot[a\to\alpha_i]\cdot[\dots],\,\pi\alpha_i\delta})\le \\
      \phantom{XXXXXXX}h\,(\inbr{\theta\cdot[a\to\alpha_i],\,\phi \alpha_i \omega b \psi}) \le k
    \end{array}
    \]

    which allows us to apply the inductive hypothesis and establish the contradiction to the divergence in the
    semantics by well quasi-ordering.
  \end{enumerate}  
  
\end{proof}
