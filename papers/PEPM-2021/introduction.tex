\section{Introduction}
\label{sec:intro}

\mk~\cite{fair:TheReasonedSchemer,fair:micro} is known for its capability to express solutions for complex problems~\cite{fair:seven,fair:quines,fair:theorem-prover}
in the form of compact declarative specifications. This minimalistic language has various extensions~\cite{fair:CKanren,fair:WillThesis,fair:alphaKanren,fair:Guided}
designed to increase its expressiveness and declarativeness. However, \emph{conjunction} in \mk has somewhat imperative flavor. The evaluation of conjunction is asymmetrical
and the order of conjuncts affects not only the performance of the program but even the convergence. As a result, the order of conjuncts determines control flow in
a relational program. We call this directed behavior of conjunction \textit{unfair}.

The contribution of this paper is a more declarative approach to the evaluation of relational programs in \mk. This approach executes the conjuncts alternately, choosing
a more optimal execution order. The \emph{fair} conjunction that we propose is comparable in efficiency to the classic unfair one, but the order of the conjuncts weakly
affects both the performance and convergence. Our approach also demonstrates a more convergent behavior: we present some examples where classical conjunction diverges for
any order of conjuncts while the fair conjunction converges.

The paper is organized as follows. In Section~\ref{sec:exposition} we introduce \mk and informally describe its semantics. We also discuss the advantages and drawbacks of its
conventional left-biased  conjunction. Section~\ref{sec:angelic-semantics} contains a description of a specific version of \mk formal semantics~--- \emph{angelic} semantics~---
establishes some properties and formally introduces the fairness notion. Section~\ref{sec:fair-semantics} presents a practically important class of deterministic semantics
parameterized with unfolding predicate, and shows that a specific form of this predicate, based on the notion of \emph{well quasi-ordering}, turns the semantic into a fair one.
In Section~\ref{sec:eval} the results of practical evaluation are presented. Section~\ref{sec:related} observes related works. The final section concludes.
