\section{Fair conjunction by structural recursion}
\label{sec:structural}

% В этой секции мы рассмотрим обобщенную семантику \mk с равномерной конъюнкцией, которая определяет глубину раскркутки динамически. Также мы рассмотрим её конкретную реализацию, основонную на структурной рекурсии отношений.
In this section we consider a generalized semantics of \mk with a fair conjunction which determines the unfolding bound dynamically. We also consider its specific implementation
which makes use of structural recursion of relations.

% В общем случае мы хотим параметризовать семантику предикатом. Этот предикат принимает вызов в качестве аргумента. Он возвращает истину, если конъюнкт необходимо раскручивать дальше. И возвращает ложь, если необходимо перейти к следующему конъюнкту. Также мы оставим параметр N, определяющий количество раскруток. Он необходим для обработки случая, когда предикат ложен для всех вызовов.
In the general case we want to parameterize the semantics with an unfolding predicate $pred$. This predicate takes a substitution and a call as arguments. It returns \lstinline{true}
if the call needs to be unfolded further, and \lstinline{false}, if we need to move on to the next conjunct. We do not get rid of the unfolding bound completely since it is still needed to handle
the case when $pred$ is false for all calls in a leaf.

\begin{figure}[h!]
\[\begin{array}{cr}

      {\inbr{\sigma;\, i;\, \epsilon} \xrightarrow{\sigma} \emptyset}  
&     \ruleno{Answer} \\[2mm]
\dfrac{\bigvee_{j=1}^n pred(\sigma, c_j) = \bot}
      {\inbr{\sigma;\, i;\, c_1^0 \land \ldots \land c_n^0} \xrightarrow{\circ} \inbr{\sigma;\, i;\, c_1^N \land \ldots \land c_n^N}}
&     \ruleno{ConjZero} \\[4mm]
\dfrac{m_k > 0 \qquad \bigvee_{j=1}^n pred(\sigma, c_j) = \bot \qquad (\sigma, i) \vdash c_k \Rightarrow T \qquad set(T, m_k - 1) = \bar{T}}
      {\inbr{\sigma;\, i;\, c_1^0 \land \ldots \land c_{k-1}^0 \land c_k^{m_k} \land \ldots \land c_n^{m_n}} \xrightarrow{\circ} push(c_1^0 \land \ldots \land c_i^0 \land \Box \land \ldots \land c_n^{m_n}, \bar{T})}
&     \ruleno{ConjUnfold} \\[4mm]
\dfrac{\bigvee_{j=1}^{k-1} pred(\sigma, c_j) = \bot \quad pred(\sigma, c_k) = \top \quad (\sigma, i) \vdash c_k \Rightarrow T \quad set(T, \max(0, \, m_k - 1)) = \bar{T}}
      {\inbr{\sigma;\, i;\, c_1^{m_1} \land \ldots \land c_{k-1}^{m_{k-1}} \land c_k^{m_k} \land C_2} \xrightarrow{\circ} push(c_1^{m_1} \land \ldots \land c_{k-1}^{m_{k-1}} \land \Box \land C_2, \bar{T})}
&     \ruleno{ConjUnfoldPred} \\[4mm]
\dfrac{T_1 \xrightarrow{\alpha} \emptyset}
      {T_1 \lor T_2 \xrightarrow{\alpha} T_2}
&     \ruleno{Disj} \\[4mm]
\dfrac{T_1 \xrightarrow{\alpha} \bar{T_1} \qquad \bar{T_1} \not= \emptyset}
      {T_1 \lor T_2 \xrightarrow{\alpha} T_2 \lor\bar{T_1}}
&     \ruleno{DisjStep}
\end{array}\]
\caption{Semantics of fair conjunction by structural recursion}
\label{fair:structural-recursion-semantics}
\end{figure}

% Семантика, параметризованная предикатом раскрутки представлена на изображении 10. Так как мы обновляем только поведение конъюнкции, то правила [Answer], [Disj] и [DisjStep]  остаются без изменений. Но за обработку конъюнкций теперь отвечают три обновленных правила. Если предикат истинен хотя бы для одного вызова, то мы применяем правило [ConjUnfoldPred] и раскручиваем самый левый такой вызов и уменьшаем его счетчик. Если предикат ложен на всех вызовах, но есть хотя бы один вызов с ненулевым счетчиком, то мы применяем правило [ConjUnfold] и раскручиваем самый левый такой вызов и уменьшаем его счетчик. Если предикат ложен на всех вызовах и все счетчики равны нулю, то мы применяем правило [ConjZero] и обновляем все счетчики.
The semantics parameterized by the unfolding predicate is shown in Fig.~\ref{fair:structural-recursion-semantics}. Since we modify only the behavior of conjunction the rules \rulen{Answer},
\rulen{Disj} and \rulen{DisjStep} remain unchanged. Three updated rules are responsible for conjunction behavior. If the predicate $pred$ is $\top$ for at least one call, then we
apply the rule \rulen{ConjUnfoldPred}, which unfolds the leftmost such call and decrements its counter. If the predicate $pred$ is $\bot$ for all calls, but there is at least one
call with a nonzero counter, then we apply the rule \rulen{ConjUnfold}, which unfolds the leftmost such call and decrements its counter. If the predicate is $\bot$ for all calls
and all the counters are equal to zero, then we apply the rule \rulen{ConjZero}, which sets all the counters to unfolding bound.

% В качестве предиката нам необходим критерий, отличающий вызов, который выгодно раскрутить сейчас от вызова, который стоит отложить. Мы предлагаем критерий, который корректно работает на отношениях со структурной рекурсией. У таких отношений есть хотя бы один аргумент, который структурно убывает с каждым шагом рекурсии. Это свойство позволит нам контролировать грубину раскрутки. Предлагаемый критерий состоит в следующем
As for the predicate, we need a criterion that can tell a call that is profitable to unfold now apart from a call which is worth deferring. We propose a criterion that works correctly
for structurally recursive relations. Such relations have at least one argument which structurally decreases with each step of the recursion. This property allows us to control the depth
of unfolding. We propose to use the following predicate:

\[
pred(\sigma, F^k(t_1, \ldots, t_k)) = \left\{
\begin{array}{cl}
      & \mbox{if } F^k \mbox{ is structural recursion relation, } \\
\top, & i \mbox { is number of structural recursion argument, } \\
      & t_i \mbox { is not fresh variable in } \sigma \\
\bot, & \mbox{otherwise.}
\end{array}
\right.
\]

% Пока хотя бы один аргумент, по которому ведется структурная рекурсия не является свободной переменной, мы продолжаем разворачивать этот вызов. Если все такие аргументы свободны, то в текущей подстановке отношение разойдется, поэтому мы переходим к вычислению следующего вызова. Так как аргументы структурной рекурсии убывают, то за конечное число шагов вычисление либо завершится, либо все аргументы структурной рекурсии станут свободными переменными.

As long as at least one argument along which structural recursion is performed is not a free variable, we continue to unfold this call. If all such arguments are free, then the call
will diverge in the current substitution, so we proceed to evaluate the next call. Since structurally recursive arguments decrease, in a finite number of steps the evaluation will
either complete, or all arguments of structural recursion will become free variables.


\begin{figure}[h!]
\centering
\begin{tabular}{c}
\begin{lstlisting}
let rec append$^o$ x y xy =
  (x === [] /\ y === xy) \/
  fresh (e xs xys) (
    x === e : xs /\ 
    xs === e : xys /\ 
    append$^o$ xs y xys)
\end{lstlisting}
\end{tabular}
\caption{Relational concatenation}
\label{fair:lst-appendo}
\end{figure}

% Например, отношение appendo является структурно рекурсивным по первому и третьему аргументу. Действительно, вложенный вызов appendo в качестве первого аргумента принимает xs, который является подтермом x. Также в качестве третьего аргумента appendo принимает xys, который является подтермом xy. Если хотя бы один из них --- список фиксированной длины, то отношение сойдется. В противном случае, оба имеют вид: $x = t_1 : \ldots t_n : \alpha$ и $xy = \bar{t}_1 : \ldots \bar{t}_m : \alpha$. Следовательно через max(n, m) шагов оба аргумента станут свободными перееменными. 
For example, the relation \lstinline{append$^o$} (Fig.~\ref{fair:lst-appendo}) is structurally recursive on its first and third arguments. Indeed, the nested call \lstinline{append$^o$}
takes \lstinline{xs} as its first argument, which is a subterm of \lstinline{x}. Also, \lstinline{append$^o$} takes \lstinline{xys} as the third argument, which is a subterm of
\lstinline{xy}. If at least one of them is a fixed-length list, then the relation will converge. Otherwise, $\mbox{\lstinline{x}} \equiv t_1 : \ldots t_n : \alpha_1$
and $\mbox{\lstinline{xy}} \equiv \bar{t}_1 : \ldots \bar{t}_m : \alpha_2$. Therefore, in $max(n, m)$ steps, both arguments become free variables.

% На текущий момент мы работаем над доказательством независимости данной семантики от порядка конъюнктов в случае, когда все отношения являются отношениями со структурной рекурсией.
We are currently working on proving the independence of this semantics from the order of the conjuncts in the case when all relations are structurally recursive.
