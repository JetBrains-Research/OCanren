\section{The Semantics of Directed Conjunction}
% В этом разделе мы введем операционную семантику малого шага для определения поведения оригинального miniKanren с направленной конъюнкцией.
In this section we introduce the small step operational semantics to define the behavior of the original \mk with directed conjunction. 

% Отметим, что на данный момент существует сертифицированная семантика языка miniKanren, однако она делает большие различия между первым и вторым конъюнктом, что сильно усложняет задачу перестановки конъюнктов в процессе исполнения программы. Поэтому для данной работы мы разработали новую семантику, которая изначально не делает различий между конъюнктами.

Note, althought at the moment there exists a certified semantics~\cite{fair:semantics} of \mk, however, it makes a distinction between the first
and the second conjuncts, which greatly complicates the task of rearranging the conjuncts in the process of program evaluation. 
Therefore, for this research, we have developed a new semantics which does not distinguish between conjuncts from the start.

% Семантика, которую мы предлагаем, основана на развертке вызовов отношений. На каждом шаге в текущем состоянии программы выбирается некоторый вызов, который разворачивается. Этот процесс продолжается, пока в текущем представлении остаются вызовы. Ниже мы опишим эту семантику более формально.

The semantics that we propose is based on unfolding of relational calls. At each step we select a call from current state of the program and unfold it.
This process continues until no calls remain in the state. Below we define this semantics formally.

% Прежде всего мы определим синтаксис языка \mk на изображении 2. Конструкторы с арностью являются стандартным представлением данных. Синтаксические переменные необходимы для описания аргументов отношения и свежих переменных в операции fresh. Семантические переменные в исходной программе отсутствуют, однако они вводятся при исполнении операции fresh. Из конструкторов, синтаксический переменных и семантических переменных мы определили множество синтаксических термов и множество семантических переменных. Также мы определили множество имен отношений с арностью. Далее, мы определили цели. Цели являются унификацией термов, дизъюнкцией, конъюнкцией введением свежей переменной или вызовом отношения. Наконец мы определяем множество отношений. Каждое отношение состоит из имени с арностью, списка имен переменных и тела отношения.

First, we define the syntax of the language (Fig.~\ref{fair:syntax}). We define the set of syntax terms $\mathcal{T}_\mathcal{X}$ and the set of semantic terms
$\mathcal{T}_\mathcal{A}$ using constructors $\mathcal{C}$, syntactic variables $\mathcal{X}$ and semantic variables $\mathcal{A}$. Syntactic variables $\mathcal{X}$
are needed to intriduce relation arguments and fresh variables. Semantic variables $\mathcal{A}$ can not occur in the source program, but they are introduced during
the evaluation of  \lstinline{fresh} construct.

Also we define the set $\mathcal{F}$ of relation names with arity. Next, we describe the set of goals $\mathcal{G}$. A goals is either a unification of terms, or a disjunction, conjunction, introduction
of fresh variable, call of a relation. Finally, we define the set of relations $\mathcal{S}$. Each relation consists of the name with arity, the list of variable names,
and the goal as its body.

\begin{figure}[h]
\[
  \begin{array}{rcll}
     \mathcal{C} & = & \{ C_1^{k_1}, C_2^{k_2}, \ldots \} 
     & \mbox{constructors with arities} 
     \\
     \mathcal{X} & = & \{x_1, x_2, \ldots \} 
     & \mbox{syntax variables} 
     \\
     \mathcal{A} & = & \{\alpha_1, \alpha_2, \ldots \} 
     & \mbox{semantic variables} 
     \\
     \mathcal{T}_\mathcal{X} & = & \mathcal{X} \mid C_i^{k_i}(\mathcal{T}_\mathcal{X}^1, \ldots, \mathcal{T}_\mathcal{X}^{k_i})
     & \mbox{syntax terms} 
     \\
     \mathcal{T}_\mathcal{A} & = & \mathcal{A} \mid C_i^{k_i}(\mathcal{T}_\mathcal{A}^1, \ldots, \mathcal{T}_\mathcal{A}^{k_i})
     & \mbox{semantic terms} 
     \\
     \mathcal{F} & = & \{F_1^{k_1}, F_2^{k_2}, \ldots \} 
     & \mbox{names of relations with arities} 
     \\
     \mathcal{G} & =    & \mathcal{T}_\mathcal{X} \equiv \mathcal{T}_\mathcal{X} & \mbox{unification} \\
                 & \mid & \mathcal{G} \lor \mathcal{G} & \mbox{disjunction} \\
                 & \mid & \mathcal{G} \land \mathcal{G} & \mbox{conjunction} \\
                 & \mid & \mbox{\lstinline|fresh|} \; (\mathcal{X}) \; \mathcal{G} & \mbox{fresh variable introduction} \\
                 & \mid &  F_i^{k_i}(\mathcal{T}_\mathcal{X}^1, \ldots, \mathcal{T}_\mathcal{X}^{k_i}) & \mbox{relation call} \\
    \mathcal{S} & = &F_i^{k_i} = \lambda \mathcal{X}_1 \ldots \mathcal{X}_n. \; \mathcal{G} & \mbox{relations}
  \end{array}
\]
    \caption{The syntax of relational language}
    \label{fair:syntax}
\end{figure}

% Помимо синтаксиса нам понадбится промежуточное состояние реляционной программы. Это состояние является деревом дизъюнкций. Внутренние узел \circ соотвествует дизъюнкции двух состояний-потомков. Его листья содержат промежуточные подстановки, индекс семантических переменных i, и список вызовов. Подстановка --- это отображение из семантических переменных в семантические термы. Оно содержит информацию о текущих переменных и обновляется при исполнении унификаций. Индекс семантических переменных необходим для исполнения операции fresh, которая вводит семантическую переменную с новым индексом. Список вызовов содержит вызовы отношений c_i, которые необходимо довычислить в этой ветке. Список \epsilon соответствует пустому списку, а оператор (:) определяет добавление вызова в список.

In addition to syntax we need an intermediate state of relational program

\[
\begin{array}{rcl}
  \textgoth{T} & = & \textgoth{T} \circ \textgoth{T}\\
               &   & \inbr{\sigma;\, i;\, c_1\dots c_n},\, c_j = F_j^{k_j}\,(t_\mathcal{A}^1,\dots,t_\mathcal{A}^{k_j})
\end{array}
\]

The state has a shape of disjunction tree. An internal node ``$\circ$'' corresponds to the disjunction of two descendant states.
Its leaves contain intermediate substitutions $\sigma$, a counter of semantic variables $i$ and a list of calls. Substitution $\sigma$ is a
mapping from semantic variables to semantic terms. The counter of semantic variables is necessary to evaluate the \lstinline{fresh} operation,
which introduces a semantic variable with a new counter. The list of calls contains calls $c_i$ which must be evaluated in this branch.

% Мы расширим множество состояний пустым состоянием, которое соотвествует завершению вычисления. 

We expand the set of states with an empty state
\[
\bar{\textgoth{T}} = \emptyset \mid \textgoth{T},
\]
which corresponds to the completion of the evaluation.

% Также мы введем две вспомогательных функции для работы с состоянием программы. Первая функция union (изображение 3) производит объединение двух состояний.
Also we introduce two auxiliary functions for working with the program state. The first function $union$ (fig.~\ref{fair:union-semantics}) combines two extended states.

\begin{figure}[h!]
\[
union(T_1, T_2) =
\left\{
\begin{array}{rl}
T_2, & \mbox{if } T_1 = \emptyset \\
T_1, & \mbox{if } T_1 \not= \emptyset \mbox{ and } T_2 = \emptyset \\
(T_1 \circ T_2), & \mbox{otherwise}
\end{array}
\right.
\]
\caption{Auxiliary function $union$}
\label{fair:union-semantics}
\end{figure}

% Если одно из состояний пустое, функция union вернёт второе. Если оба состояния не пусты, то функция вернёт объединенное состояние.

If one of the states is empty, the function $union$ returns the second state. If both states are not empty, then the function returns the combined state.

\begin{figure}[h!]
\[
push(C, T) =
\left\{
\begin{array}{rl}
\inbr{\sigma, i, c_1 : \ldots : c_i : \bar{c_1} : \ldots : \bar{c_k} : cs}, & \mbox{if } T = \inbr{\sigma, i, \bar{c_1} : \ldots : \bar{c_k} : \epsilon} \mbox{ and } C = c_1 : \ldots : c_i : \Box : cs \\
(push(C, T_1) \circ push(C, T_2)), & \mbox{if } T = (T_1 \circ T_2)
\end{array}
\right.
\]
\caption{Auxiliary function $push$}
\label{fair:push-semantics}
\end{figure}

% Следующая вспомогательная функция push (изображение 4) необходима для конструирования состояние после развертки. Первый аргумент --- это список вызовов, который содержит дырку. На этом месте стоял вызов, который мы развернули. Второй аргумент --- состояние, которое является результатом развертки. Данная функция рекурсивно проходит по состоянию, а в каждом листе объединяет вызовы с дыкой и вызовы из листа. 

The following auxiliary push function (fig.~\ref{fair:push-semantics}) is needed to construct the state after the unfolding. 
The first argument of this function is a call list that contains a hole. 
The hole corresponds to the position of the call that we unfold. 
The second argument is the state that results from the unfolding. 
This function recursively traverse by state, and for any leafs it combines calls with hole and leaf calls.

% Теперь мы определим семантику для операции развертки. Эта семантика приобразовывает вызов отношения и подстановку в состояние, которое соответствует телу этого отношения. Так как развертка вызова --- конечный процесс, мы можем описать развертку как семантику большого шага.
Now we define the semantics for the unfolding operation. This semantics (fig.~\ref{fair:unfolding-semantics}) evaluates a call of a relation and a substitution into a state that corresponds to the body of this relation. Since the call unfolding is the finite process, we can describe the unfolding as the big step semantics.

\begin{figure}[h!]
\[\begin{array}{cr}

\dfrac{ F = \lambda \bar{x}. b \qquad \inbr{\sigma, i, \epsilon} \vdash b[\bar{x} \leftarrow \bar{t}] \leadsto T}
      {(\sigma, i) \vdash F(\bar{t}) \Rightarrow T}
&     \ruleno{Unfold} \\[3mm]
\dfrac{\not\exists \, mgu(t_1, t_2, \sigma)}
      {\inbr{\sigma, i, cs} \vdash (t_1 \equiv t_2) \leadsto \emptyset}
&     \ruleno{UnifyFail}  \\[3mm]
\dfrac{\bar\sigma = mgu(t_1, t_2, \sigma)}
      {\inbr{\sigma, i, cs} \vdash (t_1 \equiv t_2) \leadsto \inbr{\bar\sigma, i, cs}}
&     \ruleno{UnifySuccess}  \\[3mm]
      {\inbr{\sigma, i, cs} \vdash F(\bar{t}) \leadsto \inbr{ \sigma, i, F(\bar{t}) : cs}}
&     \ruleno{Call} \\[2mm]
\dfrac{\inbr{\sigma, i+1, cs} \vdash g[x \leftarrow \alpha_i] \leadsto T}
      {\inbr{\sigma, i, cs} \vdash (\mbox{\lstinline|fresh|} \, x. \, g) \leadsto T}
&     \ruleno{Fresh}  \\[3mm]
\dfrac{\inbr{\sigma, i, cs} \vdash g_1 \leadsto T_1 \qquad \inbr{\sigma, i, cs} \vdash g_2 \leadsto T_1}
      {\inbr{\sigma, i, cs} \vdash (g_1 \lor g_2) \leadsto union(T_1, T_2)}
&     \ruleno{DisjGoal}  \\[3mm]
\dfrac{g \not= g_1 \lor g_2 \qquad T_1 \vdash g \leadsto T_3 \qquad T_2 \vdash g \leadsto T_4}
      {(T_1 \circ T_2) \vdash g \leadsto union(T_3, T_4)}
&     \ruleno{DisjState}  \\[3mm]
\dfrac{\inbr{\sigma, i, cs} \vdash g_1 \leadsto \emptyset}
      {\inbr{\sigma, i, cs} \vdash (g_1 \land g_2) \leadsto \emptyset}
&     \ruleno{ConjFail}  \\[3mm]
\dfrac{\inbr{\sigma, i, cs} \vdash g_1 \leadsto T \qquad T \vdash g_2 \leadsto \bar{T}}
      {\inbr{\sigma, i, cs} \vdash (g_1 \land g_2) \leadsto \bar{T}}
&     \ruleno{ConjSuccess}  \\[3mm]
\end{array}\]

\caption{Big step semantics of Unfolding}
\label{fair:unfolding-semantics}
\end{figure}

% Правило [Unfold] является внешним. Поэтому оно единственное содержит символ =>. Оно запускает процесс развертки вызова F(t) в контексте подстановки и счетчика семантических переменных. Прежде всего, вызов заменяется на тело отношения. Производится подстановка аргументов, инициализируется состояние. Далее запускается преобразование тела отношения в соответствующее состояние.
The \rulen{Unfold} rule is external. 
Therefore, it only contains the symbol ($\Rightarrow$). 
It starts the unfolding process of call $F$ with list of arguments $\bar{t}$ in the context of substitution $\sigma$ and the counter of semantic variables $i$.
First of all, the call $F$ is replaced by the body $b$ of the relation. Also the arguments $\bar{x}$ are substituted by terms $\bar{t}$, the state $\inbr{\sigma, i, \epsilon}$ is initialized. Next, we evaluate the relation body into the correspond state using the rest of rules.

% Правила [UnifyFail] и [UnifySuccess] исполняют унификацию. Если существует ноиболее общий унификатор (MGU), то мы применяем правило [UnifySuccess], которое обновляет подстановку. Если MGU не существует, то мы применияем правило [UnifyFail], что приводи к пустому состоянию.
The rules \rulen{UnifyFail} and \rulen{UnifySuccess} perform unification. If the most common unifier (MGU) exists, then we apply the \rulen{UnifySuccess} rule, which updates the substitution $\sigma$. If the MGU does not exist, then we apply the \rulen{UnifyFail} rule, which leads to an empty state.

% Так как развертка должна развернуть вызов ровно один раз, все вложенные вызовы мы оставляем без изменений. Данное поведение описано в правиле [Call]. Вложенный вызов не вычисляется, а помещаяется в список вызовов состояния.
Since this semantics should unfold the call exactly once, we leave all the nested calls unchanged. This behavior is described in the \rulen{Call} rule. The nested call is not evaluated, but placed on the call list, which is contained into the state.

% Правило [Fresh] соответствует введению свежей переменной. В данном правиле мы заменяем синтаксическую переменную на семантическую переменную. Также мы увеличиваем счетчик семантических переменных. 
The \rulen{Fresh} rule corresponds to the introduction of a fresh variable. In this rule, we replace a syntax variable $x$ with a semantic variable $\alpha_i$. We also increment the counter of semantic variables.

% Правила [DisjGoal] и [DisjState] необходимы для исполнения дизъюнкции. Первое правило вычисляет оба дизъюнкта и объединяет их в новое состояние с помощью вспомогательной функции union. Второе правило обрабатывает дизъюнкцию, содержащуюся в состоянии. Как и в первом правиле мы производим два независимых вычисления, а затем объединяем результаты в новое состоения.
The rules \rulen{DisjGoal} and \rulen{DisjState} are required to evaluate a disjunction. The first rule evaluates both disjuncts and combines them into a new state using the auxiliary function $union$. The second rule handles a disjunction which is contain in the state. As in the previous rule, we perform two independent evaluations, and then combine the results in a new state.

% Оставшиеся два правила описывают вычисление конъюнкции. Если первый конъюнкт вычислился в пустое состояние, то мы применяем правило [ConjFail], которое возвращает пустое состояние как результат вычисления всей конъюнкции. В противном случае вычисляем первый конъюнкт в состояние T, а затем вычисляем второй конъюнкт в контесте состояния T. Таким образом, второй конъюнкт будет исполнен в контексте всех листьев состояния Т.
The last two rules \rulen{ConjFail} and \rulen{ConjSuccess} describe the evaluation of conjunction. If the first conjunct is calculated to an empty state, then we apply the \rulen{ConjFail} rule, which returns an empty state as a result of calculating the entire conjunction. Otherwise, we evaluate the first conjunct to state $T$, and then we evaluate the second conjunct in the context of state $T$. Thus, the second conjunct will be evaluated in the context of all leaves of the state $T$.

% Теперь у нас есть всё необходимое, чтобы определить семанику реляционного языка с направленной конъюнкцией. Данная семантика малого шага последовательно преобразовывает состояние и периодически производит ответы. Так как \mk недетеримированный язык, количество ответов, которое может получиться неограничено. Если в процессе исполнения программы будет обнаружен ответ, он будет помещён над символом перехода (->). В противном случае над символом перехода будет помещен \circ.

Now we have everything we need to define the semantics of a relational language with directed conjunction (fig.~\ref{fair:classic-semantics}). This small step semantics sequentially evaluates a state and periodically produces substitutions which are answers. Since \mk is a non-deterministic language, the number of answers that can be obtained is unlimited. If an answer is found during program evaluation, it will be placed above the transition symbol ($\xrightarrow{}$). Otherwise, ($\circ$) will be placed above the transition symbol.

\begin{figure}[h!]
\[\begin{array}{cr}

      {\inbr{\sigma, i, \epsilon} \xrightarrow{\sigma} \emptyset}  
&     \ruleno{Answer} \\[2mm]
\dfrac{(\sigma, i) \vdash c \Rightarrow T}
      {\inbr{\sigma, i, c : cs} \xrightarrow{\circ} push(\Box : cs, T)}
&     \ruleno{ConjUnfold} \\[2mm]
\dfrac{T_1 \xrightarrow{\alpha} \emptyset}
      {(T_1 \lor T_2) \xrightarrow{\alpha} T_2}
&     \ruleno{Disj} \\[4mm]
\dfrac{T_1 \xrightarrow{\alpha} \bar{T_1}}
      {(T_1 \lor T_2) \xrightarrow{\alpha} (T_2 \lor\bar{T_1})}
&     \ruleno{DisjStep} \\[4mm]
\end{array}\]
\caption{Semantics with directed conjunction}
\label{fair:classic-semantics}
\end{figure}

% Если текущее состояние является листом и не содержит вызовов, значит мы получили ответ. В этом случае мы применяем правило [Answer].
If the current state is a leaf and does not contain calls, then we got an answer. In this case, we apply the rule \rulen{Answer}.

% Если текущее состояние является листом, но содержит хотя бы один вызов, мы применяем правило [ConjUnfold]. Оно производит развертку самого левого вызова. Затем мы конструируем новое состояние из оставшищся вызовов и результата развертки с помощью функции push.
Also, if the current state is a leaf but contains at least one call, we apply the \rulen{ConjUnfold} rule. In this case we unfold the leftmost call $c$. Then we construct a new state from the remaining calls $cs$ and the unfolding result $T$ using the function $push$.

% Наконец, если текущее состояние является дизюнкцией, то мы производим вычисление в левом дизъюнкте T_1. В зависимости от результата мы приминяем или правило [Disj], или правило [DisjStep]. Первое правило соответствует пустому состоянию и возвращает второй дизъюнкт в качестве результата. Второе правило соответствует непустому состоянию и возвращает новое сотояние (T_2 \circ T_1). Перестановка дизъюнктов --- необходимое действие, которое называется интерливинг. Оно гарантирует полноту поиска.
Finally, if the current state is a disjunction, then we evaluate the left disjunct $T_1$. We apply either the \rulen{Disj} rule or the \rulen{DisjStep} rule depending on the result of evaluation the left disjunct. The first rule corresponds to an empty state and returns the second disjunct $T_2$ as a result. The second rule corresponds to a non-empty state and returns a new state ($T_2 \circ \bar{T_1}$). Rearrangement of disjuncts is a necessary action called interleaving~\cite{fair:interleaving}. It guarantees the completeness of the search.

% Для того чтобы сделать начальное состояние из вызова c, нам необходимо заменить в нем все синтаксические переменные на семантические. Тогда мы получим
In order to make the initial state from a call $c$, we need to replace all syntactic variables in it with semantic ones. Then we get
\[
\inbr{\{\}, n, c[x_0 \leftarrow \alpha_0, \ldots, x_{n-1} \leftarrow \alpha_{n-1}] : \epsilon}.
\]

% Данная семантика отличается от сертифицированной семантики и классических реализаций прежде всего размером шага. Операция развертки выполняет множество действий подряд, а в классическом случае после каждой элементарного действия выполняется интерливинг. Однако, общие черты данная семантика сохранила: дизъюнкты меняют порядок после каждого unfolding, а конъюнкты выполняются строго слева направо.
This semantics differs from certified semantics and classical implementations primarily in step size. The unfolding operation performs many actions in a row. But in the classic case, interleaving is performed after each elementary action. However, this semantics has retained common features: disjuncts change order after each step, and conjunctions are evaluated strictly from left to right. 


\begin{comment}
\[
\begin{array}{l}
\inbr{\{\}, 1, \mbox{\lstinline{revers}}^o \, [1] \; \alpha_0 : \epsilon} 
\xrightarrow{\circ} \\
\inbr{\{\alpha_1 = 1; \alpha_2 = []\}, 4, \mbox{\lstinline{revers}}^o \, \alpha_2 \; \alpha_3 : \mbox{\lstinline{append}}^o \, \alpha_3 \; [\alpha_1] \; \alpha_0 : \epsilon}
\xrightarrow{\circ} \\
\inbr{\{\alpha_1 = 1; \alpha_2 = []; \alpha_3 = []\}, 4, \mbox{\lstinline{append}}^o \, \alpha_3 \; [\alpha_1] \; \alpha_0 : \epsilon} 
\xrightarrow{\circ} \\
\inbr{\{\alpha_1 = 1; \alpha_2 = []; \alpha_3 = []\; \alpha_0 = [\alpha_1]\}, \epsilon}
\xrightarrow{\{\alpha_1 = 1; \alpha_2 = []; \alpha_3 = []\; \alpha_0 = [\alpha_1]\}} \emptyset
\end{array}
\]
\end{comment}

% Порядок конъюнктов сильно влияет на результат вычисления именно из-за строгого порядка исполнения конъюнктов. В следующих разделах мы предложем две семантики, которые более гибко обрабатывают конъюнкты.
The order of the conjuncts strongly affects the result of the evaluation precisely because of the strict fixed order of evaluation of the conjuncts. In the following sections, we offer two semantics that handle conjunctions more flexibly.
