\section{Conclusion and future work}

We presented algorithm for pattern matching compilation via program synthesis. The performance of proposed approach is very sensitive of the shape of scrutinee's type. 
But it can incrementally produce new compilation schemes,  step by step producing more performant answers while other approaches to compilation always return single compilation scheme.

The performance can be improved by searching new ways to cut search space and by speeding up implementation of relations and of structural constraint. Also it can be interesting to integrate structural constraint more closely to \textsc{OCanren} core. Optimal order of samples and ways to reduce required number samples for special cases of patterns and scrutinee's type were not studied too.

The language of compiled representation can be altered too. It is interesting to add to intermediate language so called \emph{exit nodes} described in~\cite{maranget2001}.
Straightforward implementation of them will require nominal unification but we are not aware of \textsc{miniKanren} implementation that has both disequality constraints and nominal unification~\cite{alphaKanren}.

At the moment we support only simple pattern matching without any extensions. It looks    technically easy to extend our approach for non-linear and or- patterns. It will increase search space and new optimizations may be required.



