\section{Introduction}
\label{sec:intro}

Verifying a solution for a problem is much easier than finding one~--- this common wisdom can be confirmed by anyone who used 
both to learn and to teach. This observation can be justified by its theoretical applications, thus being more than informal knowledge. For example, let us have a language $\mathcal{L}$. If there is a predicate $V_\mathcal{L}$ such~that
\[
\forall\omega\;:\;\omega\in\mathcal{L}\;\Longleftrightarrow\;\exists p_\omega\;:\;V_\mathcal{L}(\omega,p_\omega)
\]
(with $p_\omega$ being of size, polynomial on $\omega$) and we can recognize $V_\mathcal{L}$ in a polynomial time, then we call $\mathcal{L}$ to be in the class $NP$~\cite{Garey:1990:CIG:574848}. Here $p_\omega$ plays role of a justification (or proof) for the fact $\omega\in\mathcal{L}$. For example, if
$\mathcal{L}$ is a language of all hamiltonian graphs, then $V_\mathcal{L}$ is a predicate which takes a graph $\omega$ and some path $p_\omega$ and verifies that $p_\omega$ is indeed a hamiltonial path in $\omega$. The implementation of the predicate $V_\mathcal{L}$, however, tells us very little about the \emph{search procedure} which would calculate $p_\omega$ as a function of $\omega$. For the whole class of $NP$-complete problems no polynomial search procedures are known, and their existence at all is a long-standing problem in the complexity theory.

There is, however, a whole research area of \emph{relational interpreters}, in which a very close problem is addressed. Given a language $\mathcal{L}$, its \emph{interpreter} is a function \lstinline|eval$_\mathcal{L}$| which takes a program $p^\mathcal{L}$ in the language $\mathcal{L}$ and an input $i$ and calculates some output such that
\[
\mbox{\lstinline|eval$_\mathcal{L}$|}(p^\mathcal{L}, i)=\sembr{p^\mathcal{L}}_{\mathcal L}\,(i)
\]
where $\sembr{\bullet}_{\mathcal L}$ is the semantics of the language $\mathcal{L}$. In these terms, a verification predicate $V_\mathcal{L}$ can be
considered as an interpreter which takes a program $\omega$, its input $p_\omega$ and returns $true$ or \false. A \emph{relational} interpreter is an interpreter which is implemented not as a function \lstinline|eval$_\mathcal{L}$|, which calculates the output from a program and its input, but as a relation \lstinline|eval$^o_\mathcal{L}$|
which connects a program with its input and output. This alone would not have much sense, but if we allow the arguments of \lstinline|eval$^o_\mathcal{L}$|
to contain \emph{variables} we can consider relational interpreter as a generic search procedure which determines the values for these variables making the
relation hold. Thus, with relational interpreter it is possible not only to calculate the output from an input, but also to run a program in 
an opposite ``direction'', or to synthesize a program from an input-output pair, etc. In other words, relational verification predicate is capable
(in theory) to both \emph{verify} a solution and \emph{search} for it.

Implementing relational interpreters amounts to writing it in a relational language. In principle, any conventional language for logic programming
(Prolog~\cite{lozov:prolog}, Mercury~\cite{somogyi1996execution}, etc.) would make the job. However, the abundance of extra-logical features and the incompleteness of default search
strategy put a number of obstacles on the way. There is, however, a language specifically designed for pure relational programming, and, in a
narrow sense, for implementing relational interpreters~--- \textsc{miniKanren}~\cite{lozov:TheReasonedSchemer}. Relational interpreters, implemented
in \textsc{miniKanren}, demonstrate all their expected potential: they can synthesize programs by example, search for errors in partially defined programs~\cite{lozov:seven}, produce self-evaluated programs~\cite{lozov:quines}, etc. However, all these results are obtained for a family
of closely related Scheme-like languages and require a careful implementation and even some \emph{ad-hoc} optimizations in the relational
engine. 

From a theoretical standpoint a single relational interpreter for a Turing-complete language is sufficient: indeed, any other interpreter
can be turned into a relational one just by implementing it in a language, for which relational interpreter already exists. However, the overhead
of additional interpretation level can easily make this solution impractical. The standard way to tackle the problem is partial evaluation or specialization~\cite{jones1993partial}.
A \emph{specializer} \lstinline|spec$_\mathcal{M}$| for a language $\mathcal{M}$ for any program $p^\mathcal{M}$ in this language and its partial input $i$ returns some program which, being applied to the residual input $x$, works exactly as the original program on both $i$ and~$x$:
\[
\forall x\;:\;\sembr{\mbox{\lstinline|spec$_\mathcal{M}$|}\,(p^\mathcal{M}, i)}_\mathcal{M}\,(x)=\sembr{p^\mathcal{M}}_\mathcal{M}\,(i, x).
\]

If we apply a specializer to an interpreter and a source program, we obtain what is called \emph{the first Futamura projection}~\cite{futamura1971partial}:
\[
\forall i\;:\; \sembr{\mbox{\lstinline|spec$_\mathcal{M}$|}\,(\mbox{\lstinline|eval$^\mathcal{M}_\mathcal{L}$|}, p^\mathcal{L})}_\mathcal{M}\,(i)=\sembr{\mbox{\lstinline|eval$^\mathcal{M}_\mathcal{L}$|}}_\mathcal{M}\,(p^\mathcal{L}, i).
\]
Here we added an upper index $\mathcal{M}$ to \lstinline|eval$_\mathcal{L}$| to indicate that we consider it as a program in 
the language $\mathcal{M}$. In other words, the first Futamura projection specializes an interpreter for a concrete program, 
delivering the implementation of this program in the language of interpreter implementation. An important property of
a specializer is \emph{Jones-optimality}~\cite{jones1993partial}, which holds when it is capable to completely
eliminate the interpretation overhead in the first Futamura projection. In our case $\mathcal{M}=\mbox{\textsc{miniKanren}}$, 
from which we can conclude that in order to eliminate the interpretation overhead we need a Jones-optimal specializer for \textsc{miniKanren}. 
Although implementing a Jones-optimal specializer is not an easy task even for simple functional languages, there is a Jones-optimal specializer for a logical language~\cite{leuschel2004specialising}, but not for \textsc{miniKanren}. 

The contribution of this paper is as follows:

\begin{itemize}
\item We demonstrate the applicability of relational programming and, in particular, relational interpreters for the task of
turning verifiers into solvers.
\item To obtain a relational verifier from a functional specification we apply \emph{relational conversion}~\cite{lozov:miniKanren,lozov:conversion}~---
a technique which for a first-order functional program directly constructs its relational counterpart. Thus, we introduce a number
of new relational interpreters for concrete search problems.
\item We employ supercompilation in the form of conjunctive partial deduction (CPD)~\cite{de1999conjunctive} to
eliminate the redundancy of a generic search algorithm caused by partial knowledge of its input.
\item We give a number of examples and perform an evaluation of various solutions for the approach we address.
\end{itemize}

Both relational conversion and conjunctive partial deduction are done in an automatic manner. The only thing one needs to specify is the known arguments or the execution direction of a relation. 

As concrete implementation of \textsc{miniKanren} we use \textsc{OCanren}~\cite{lozov:ocanren}~--- its embedding in \textsc{OCaml}; we use
\textsc{OCaml} to write functional verifiers; our prototype implementation of conjunctive partial deduction is written in \textsc{Haskell}.

The paper is organized as follows. In Section~\ref{sec:example} we give a complete example of solving a concrete problem~--- searching for
a path in a graph,~--- with relational verifier. Section \ref{sec:conversion} recalls the cornerstones of relational programming in 
\textsc{miniKanren} and the relational conversion technique. In Section~\ref{sec:cpd} we describe how conjunctive partial deduction was 
adapted for relational programming. Section~\ref{sec:eva} presents the evaluation results for concrete solvers built using the technique
in question. The final section concludes.
