\section{Introduction}
\label{sec:intro}

Algebraic data types (ADT) are an important tool in functional programming which delivers a way to represent flexible and easy to manipulate data structures.
To inspect the contents of ADT values a generic construct~--- \emph{pattern matching}~--- is used. Pattern matching can be considered as a generalization of
conventional conditional control-flow construct ``\lstinline|if .. then .. else|'' and in principle can be decomposed into a nested hierarchy of those; from
this standpoint the problem of pattern matching implementation can be considered trivial. However, some decompositions are obviously better than others. We
repeat here an example from~\cite{maranget2008} to demonstrate this difference (see Fig.~\ref{fig:match-example}). Here we match a triple of boolean
values $x$, $y$, and $z$ against four patterns (Fig.~\ref{fig:matching-example1}; we use \textsc{OCaml}~\cite{ocaml} as reference language). The na\"{i}ve
implementation of this example is shown on Fig.~\ref{fig:matching-example2}; however if we decide to match $y$ first the result becomes much
better (Fig.~\ref{fig:matching-example3}).

\begin{figure}[ht]
\begin{subfigure}[t]{0.2\linewidth}
\centering
\begin{lstlisting}
match x, y, z with
| _, F, T -> 1
| F, T, _ -> 2
| _, _, F -> 3
| _, _, T -> 4
\end{lstlisting}
\vskip18.5mm
\caption{Pattern matching}
\label{fig:matching-example1}
\end{subfigure}
\hspace{0.5cm}
\begin{subfigure}[t]{0.26\linewidth}
\centering
\begin{lstlisting}
if x then
  if y then
    if z then 4 else 3
  else
    if z then 1 else 3
else
  if y then 2
  else
    if z then 1 else 3
\end{lstlisting}
\caption{A correct but non-optimal\\\phantom{(b)~}implementation}
\label{fig:matching-example2}
\end{subfigure}
\hspace{0.5cm}
\begin{subfigure}[t]{0.33\linewidth}
\centering
\begin{lstlisting}
if y then
  if x then
    if z then 4 else 3
  else 2
else
  if z then 1 else 3
\end{lstlisting}
\vskip13.5mm
\caption{Optimal implementation}
\label{fig:matching-example3}
\end{subfigure}
\caption{Pattern matching implementation example} 
\label{fig:match-example}
\end{figure}

\begin{comment}
\begin{figure}[ht]
\begin{minipage}[b]{0.3\linewidth}
\centering
\label{fig:figure1}
\end{minipage}
\hspace{0.5cm}
\begin{minipage}[b]{0.3\linewidth}
\centering
\begin{lstlisting}
switch x with 
| true -> 
    switch y with 
    | true -> 
       switch z with 
       | true -> 4
       | _ -> 3
    | _ -> 
      switch z with 
      | true -> 1
      | _ -> 3 
| _ -> 
   switch y with 
   | true -> 2 
   | _ -> if z then 1 else 3
\end{lstlisting}
\end{minipage}
\hspace{0.5cm}
\begin{minipage}[b]{0.3\linewidth}
\centering
\end{minipage}
\end{figure}
\end{comment}


%clasification 1
%Although semantics of pattern matching can be given as a sequence of srutinee's sub expression comparisons (Figure~\ref{fig:matchpatts}) effective compilers don't follow
%this approach. 

The quality of pattern matching implementation can be measured in various ways. One can either optimise the runtime cost by minimizing amount of checks performed, or static
cost by minimizing the size of generated code. \emph{Decision trees}~\cite{?} considered suitable for the first criterion as they check every subexpression no more than once.
However, minimizing the size of decision tree is known to be NP-hard~\cite{baudinet1985tree}, and as a rule various heuristics, using, for example,
the number of nodes, the length of the longest path, the average length of all paths are applied during compilation. In \cite{Scott2000WhenDM} the results of experimental
evaluation of nine heuristics for Standard ML of New Jersey are reported.

For minimizing the static cost \emph{backtracking automata} can be used since they admit a compact representation but in some cases can perform repeated checks.

%clasification 2
There is a certain difference in dealing with pattern matching in strict and non-strict languages. For strict languages checking sub-expressions of scrutinee in any order is allowed.
The pattern matching implementation for strict languages can operate in \emph{direct} or \emph{indirect} styles. In direct case an efficient implementation is constructed
explicitly. In indirect the construction of implementation requires some post-processing, which can vary from easy simplifications to complicated supercompilation
techniques~\cite{sestoft1996}. The main drawback of indirect approach is that the size of intermediate data structures can be exponentially large.

For non-strict languages pattern matching should evaluate only those sub-expressions which are necessary for performing pattern matching. If not done carefully pattern matching can
change the termination behavior of the program. In general non-strict languages put more constraints on pattern matching and thus admit a lesser set of heuristics. 
A few approaches for checking sub-expressions in lazy languages have been proposed. In ~\cite{augustsson1985} a simple left-to-right order of subexpression checking was proposed
and was proved that it doesn't affect termination. The backtracking automaton being built has a form of a DAG to reduce the code size. A few refinements have been added in~\cite{wadler1987}
as a part of textbook~\cite{peytonjones1987} on implementation of lazy functional languages. The approach from this book is used in the current version of GHC~\cite{ghc}.
\cite{laville1991} models values in lazy languages using \emph{partial terms}, although it doesn't scale to types with infinite constructor sets (like integers). The approach doesn't
test all subexpressions from left to right as~\cite{augustsson1985} but is aimed at avoiding to perform unnecessary checks by constructing \emph{lazy automaton}. Pattern matching for
lazy languages has been compiled also to decision trees~\cite{maranget1992} and later into \emph{decision DAGs} which allow in some cases to make the code
smaller~\cite{maranget1994}.

%about automata
The inefficiency of backtracking automaton has been
improved in~\cite{maranget2001}. The approach utilizes matrix representation for pattern matching. It splits the current matrix according to constructors in the
first column and reduces the task to compiling matrices with less rows. The technique is indirect, in the end a few optimizations are performed by introducing
special \emph{exit} nodes to the compiled representation. %No preprocessing is required for this scheme: or-pattern receives a special treatment during compilation process.
The approach from this paper is used in the current implementation of the \textsc{OCaml} compiler.

The previous approach uses the first column to split the matrix. In~\cite{maranget2008} the \emph{neccessity} heuristic has been introduced which recommends which column should be
used to perform the split. Good decision trees which are constructed in this work can perform better in corner cases than~\cite{maranget2001}, but for practical use the
difference is insignificant.

While existing approaches deliver appropriate solutions for certain forms of pattern matching construct, they have to be extended in an \emph{ad hoc} manner each time
the syntax and semantics of pattern matching construct changes. For example, besides a simple conventional form of pattern matching there is a number of extensions:
guards (first time appeared in KRC language~\cite{turner2013}), disjunctive patterns~\cite{?}, non-linear patterns~\cite{mcbride1969symbol}, active patterns~\cite{activepatterns}, pattern matching for polymorphic variants~\cite{Garrigue98} 
%and generalized algebraic datatypes~\cite{?}) 
which require a separate customized algorithms to be developed.

We present an approach to pattern matching implementation based on application of relational programming~\cite{TRS,WillThesis} and, in particular, relational interpreters~\cite{unified}
and relational conversion~\cite{lozov2017}. Our approach is based on relational representation of the source language pattern matching semantics on the one hand, and
the semantics of intermediate-level implementation language on the other. We formulate the condition for a correct and complete implementation of pattern matching and use it to
construct a top-level goal which represents a search procedure for all correct and complete implementations. We also present a number of techniques which makes it possible to come up with an
\emph{optimal} solution as well as optimizations to improve the performance of the search; similarly to many other prior works we use synthesized code size, which can be measured statically.
Our implementation\footnote{\url{https://github.com/kakadu/pat-match}} makes use of \textsc{OCanren}\footnote{\url{https://github.com/JetBrains-Research/OCanren}}~--- a typed
implementation of \textsc{miniKanren} for \textsc{OCaml}~\cite{OCanren}, and \textsc{noCanren}\footnote{\url{https://github.com/Lozov-Petr/noCanren}}~--- a convertor from the subset
of plain \textsc{OCaml} into \textsc{OCanren}~\cite{lozov2017}. An evaluation, performed for a set of benchmarks taken from other papers, initially has shown a good performance of our synthesizer.
However, being aware of some pitfalls of our approach, we came up with a set of counterexamples, on which it did not provide any results in observable time, so we do not consider the problem
completely solved. We also started a work on mechanized formalization\footnote{\url{https://github.com/dboulytchev/Coq-matching-workout}}, written in \textsc{Coq}~\cite{Coq}, to
make the justification of out approach more solid and easier to verify, but this formalization is not yet complete. 

 
