\section{The Pattern Matching Synthesis Problem}

We start from a simplified view on pattern matching which does not incorporate some practically important aspects of the construct such as
name bindings in patterns, guards or even semantic actions in branches. However, it in a purified form represents the essence of pattern
matching as ``inspect-and-branch'' procedure. Once we come up with the solution for the essential part of the problem we embellish it with
other features to a complete form.

First, we introduce a finite set of \emph{constructors} $\mathcal C$, equipped with arities, a set of values $\mathcal{V}$
and a set of patterns $\mathcal{P}$:
 
\[
 \begin{array}{rcll}
    \mathcal{C} & = & \{ C_1^{k_1}, \dots, C_n^{k_n} \}\\
    \mathcal{V} & = & \mathcal{C}\,\mathcal{V}^*\\  
    \mathcal{P} & = & \_ \mid \mathcal{C}\,\mathcal{P}^*
 \end{array}
\]

We define a matching of a value $v$ (\emph{scrutinee}) against an ordered non-empty sequence of patterns $p_1,\dots,p_k$ by mean of the following
relation

\[
\setarrow{\xrightarrow}
\trans{\inbr{v;\,p_1,\dots,p_k}}{}{i},\,1\le i\le k+1
\]

which gives us the index of the leftmost matched pattern or $k+1$ if no such pattern exists. We use an auxiliary relation $\inbr{;}\subseteq\mathcal{V}\times\mathcal{P}$
to specify the notion of a value matched by an individual pattern (see Fig.~\ref{fig:match1pat}). The rule \ruleno{Wildcard} says that
a wildcard pattern ``\_'' matches any value, and \ruleno{Constructor} specifies that a constructor pattern matches exactly those values which
have the same constructor at the top level and all subvalues matched by corresponding subpatterns. The definition of ``$\xrightarrow{}{}$'' is
shown on Fig.~\ref{fig:matchpatts}. An auxiliary relation ``$\xrightarrow{}{}_*$'' is introduced to specify the left-to-right matching strategy, and we
use current index as environment. An important rule $\ruleno{MatchOtherwise}$ specifies that if we exhausted all the patterns with no matching we stop with
the current index (which in this case is equal to the number of patterns plus one).

\begin{figure}
   \renewcommand*{\arraystretch}{2}
   \[
   \begin{array}{cr}
     \inbr{v;\,\_} & \ruleno{Wildcard} \\
     \trule{\forall i\;\inbr{v_i;\,p_i}}{\inbr{C^k\,v_1\dots v_k;\,C^k\,p_1\dots p_k}},\,k\ge 0 & \ruleno{Constructor}
   \end{array}
   \]
   \caption{Matching against a single pattern}
   \label{fig:match1pat}
\end{figure}

\begin{figure}
   \renewcommand*{\arraystretch}{3}
   \setarrow{\xrightarrow}
   \setsubarrow{_*}
   \[
   \begin{array}{cr}
     \trule{\inbr{v;\,p_1}}
           {\withenv{i}{\trans{\inbr{v;\,p_1,\dots,p_k}}{}{i}}} & \ruleno{MatchHead}\\
     \trule{\neg\inbr{v;\,p_1}\qquad\withenv{i+1}{\trans{\inbr{v;\,p_2,\dots,p_k}}{}{j}}}
           {\withenv{i}{\trans{\inbr{v;\,p_1,\dots,p_k}}{}{j}}} & \ruleno{MatchTail}\\
     \withenv{i}{\trans{\inbr{v;\,\epsilon}}{}{i}} & \ruleno{MatchOtherwise}\\
     \trule{\withenv{1}{\trans{\inbr{v;\,p_1,\dots,p_k}}{}{i}}}
           {\setsubarrow{}\trans{\inbr{v;\,p_1,\dots,p_k}}{}{i}} & \ruleno{Match}
   \end{array}
   \]
   \caption{Matching against an ordered sequence of patterns}
   \label{fig:matchpatts}
\end{figure}

The relation ``$\xrightarrow{}{}$'' gives us a \emph{declarative} semantics of pattern matching. Since we are interested in
synthesizing implementations, we need a \emph{programmatical} view on the same problem. Thus, we introduce a language $\mathcal S$
(the ``switch'' language) of test-and branch constructs:

\[
\begin{array}{rcl}
  \mathcal M & = & \bullet \\
  &   & \mathcal M\,[\mathbb{N}] \\
  \ir & = & \primi{return}\,\mathbb{N} \\
  &   & \primi{switch}\;\mathcal{M}\;\primi{with}\; [\mathcal{C}\; \primi{\rightarrow}\; \ir]^*\;\primi{otherwise}\;\ir
\end{array}
\]
 
Here $\mathcal{M}$ stands for a \emph{matching expression}, which is either a reference to a scrutinee ``$\bullet$'' or
a denotation of some indexed subvalue of a matching expression. Programs in the switch language can discriminate on the
structure of matching expressions, testing their top-level constructors and eventually returning natural numbers as results.
The switch language is similar to the intermediate representations for pattern matching code used in 
previous works on pattern matching implementation~\cite{maranget2001,maranget2008}.

The semantics of the switch language is given by mean of relations ``$\xrightarrow{}{}_{\mathcal M}$'' and ``$\xrightarrow{}{}_{\mathcal S}$''
(see Fig.~\ref{fig:matchexpr} and \ref{fig:test-and-branch}). The first one describes the semantics of matching expression, while
the second~--- the semantics of the switch language itself. In both cases the scrutinee is used as an environment.


\begin{figure}
  \renewcommand*{\arraystretch}{2}
  \setarrow{\xrightarrow}
  \setsubarrow{_{\mathcal M}}
  \[
  \begin{array}{cr}
    \withenv{v}{\trans{\bullet}{}{v}} & \ruleno{Scrutinee} \\
    \trule{\withenv{v}{\trans{m}{}{C^k\,v_1,\dots,v_k}}}{\withenv{v}{\trans{m[i]}{}{v_i}}} & \ruleno{SubMatch} 
  \end{array}
  \]
  \caption{Semantics of matching expression}
  \label{fig:matchexpr}
\end{figure}

\begin{figure}
  \renewcommand*{\arraystretch}{3}
  \setarrow{\xrightarrow}
  \setsubarrow{_{\mathcal S}}
  \[
  \begin{array}{cr}
    \withenv{v}{\trans{\primi{return}\;i}{}{i}} & \ruleno{Return}\\
    \trule{{\setsubarrow{_{\mathcal M}}\withenv{v}{\trans{m}{}{C^k\,v_1,\dots,v_k}}}\qquad \withenv{v}{\trans{s}{}{i}}}
          {\withenv{v}{\trans{\primi{switch}\;m\;\primi{with}\;[C^k\to s]s^*\;\primi{otherwise}\;s^\prime}{}{i}}} & \ruleno{SwitchMatched}\\
    \trule{{\setsubarrow{_{\mathcal M}}\withenv{v}{\trans{m}{}{D^n\,v_1,\dots,v_n}}}\qquad C^k\ne D^n\qquad \withenv{v}{\trans{\primi{switch}\;m\;\primi{with}\;s^*\;\primi{otherwise}\;s^\prime}{}{i}}}
          {\withenv{v}{\trans{\primi{switch}\;m\;\primi{with}\;[C^k\to s]s^*\;\primi{otherwise}\;s^\prime}{}{i}}} & \ruleno{SwitchNotMatched}\\
    \trule{\withenv{v}{\trans{s}{}{i}}}{\withenv{v}{\trans{\primi{switch}\;m\;\primi{with}\;\epsilon\;\primi{otherwise}\;s}{}{i}}} & \ruleno{SwitchOtherwise}
  \end{array}
  \]
  \caption{Semantics of switch programs}
  \label{fig:test-and-branch}
\end{figure}

The following observations can be easily proven by structural induction.

\begin{Observation}
  For arbitrary pattern the set of matching values is non-empty:

  \[
  \forall p\in\mathcal P : \{v\in\mathcal V\mid \inbr{v;\,p}\}\ne\emptyset
  \]
\end{Observation}

\begin{Observation}
  Relations ``$\xrightarrow{}{}$'' and ``$\xrightarrow{}{}_{\mathcal S}$'' are functional and deterministic respectively:

  \[
  \setarrow{\xrightarrow}
  \begin{array}{rcl}
    \forall p_1,\dots,p_k\in\mathcal P,\,\forall v\in \mathcal V,\,\forall \pi\in\mathcal S & : & |\{i\in\mathbb N\mid \trans{\inbr{v;\,p_1,\dots,p_k}}{}{i}\}|=1 \\
                                                                 &  & {\setsubarrow{_{\mathcal S}}|\{i\in\mathbb N\mid \withenv{v}{\trans{\pi}{}{i}}\}|\le 1}
  \end{array}
  \]
\end{Observation}

With these definitions, we can formulate the \emph{pattern matching synthesis problem} as follows: for a given ordered sequence of patterns $p_1,\dots,p_k$ find
a switch program $\pi$, such that

\[
\setarrow{\xrightarrow}
\forall v\in \mathcal V,\; \forall 1\le i\le n+1 : \trans{\inbr{v;\,p_1,\dots,p_n}}{}{i}\Longleftrightarrow{\setsubarrow{_{\mathcal S}}\withenv{v}{\trans{\pi}{}{i}}}\eqno{(\star)}
\]

In other words, program $\pi$ delivers a correct and complete implementation for pattern matching.
