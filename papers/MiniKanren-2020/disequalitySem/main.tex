\documentclass[acmlarge]{acmart}

%% Bibliography style
\bibliographystyle{ACM-Reference-Format}
%% Citation style
%% Note: author/year citations are required for papers published as an
%% issue of PACMPL.
\citestyle{acmauthoryear}   %% For author/year citations


%%%%%%%%%%%%%%%%%%%%%%%%%%%%%%%%%%%%%%%%%%%%%%%%%%%%%%%%%%%%%%%%%%%%%%
%% Note: Authors migrating a paper from PACMPL format to traditional
%% SIGPLAN proceedings format must update the '\documentclass' and
%% topmatter commands above; see 'acmart-sigplanproc-template.tex'.
%%%%%%%%%%%%%%%%%%%%%%%%%%%%%%%%%%%%%%%%%%%%%%%%%%%%%%%%%%%%%%%%%%%%%%


%% Some recommended packages.
\usepackage{booktabs}   %% For formal tables:
                        %% http://ctan.org/pkg/booktabs
\usepackage{subcaption} %% For complex figures with subfigures/subcaptions
                        %% http://ctan.org/pkg/subcaption


\usepackage{amsmath,amssymb}
\usepackage[russian,english]{babel}
\usepackage{amssymb}
\usepackage{mathtools}
\usepackage{listings}
\usepackage{comment}
\usepackage{indentfirst}
\usepackage{hyperref}
\usepackage{amsthm}
\usepackage{stmaryrd}
\usepackage{eufrak}
\usepackage{lstcoq}

\newtheorem{theorem}{Theorem}
\newtheorem{lemma}{Lemma}
\newtheorem{corollary}{Corollary}
\newtheorem{hyp}{Hypethesis}
\newtheorem{definition}{Definition}

\lstdefinelanguage{minikanren}{
keywords={fresh},
sensitive=true,
commentstyle=\small\itshape\ttfamily,
keywordstyle=\textbf,
identifierstyle=\ttfamily,
basewidth={0.5em,0.5em},
columns=fixed,
fontadjust=true,
literate={fun}{{$\lambda\;\;$}}1 {->}{{$\to$}}3 {===}{{$\,\equiv\,$}}1 {=/=}{{$\not\equiv$}}1 {|>}{{$\triangleright$}}3 {/\\}{{$\wedge$}}2 {\\/}{{$\vee$}}2,
morecomment=[s]{(*}{*)}
}

\lstset{
mathescape=true,
language=minikanren
}

\usepackage{letltxmacro}
\newcommand*{\SavedLstInline}{}
\LetLtxMacro\SavedLstInline\lstinline
\DeclareRobustCommand*{\lstinline}{%
  \ifmmode
    \let\SavedBGroup\bgroup
    \def\bgroup{%
      \let\bgroup\SavedBGroup
      \hbox\bgroup
    }%
  \fi
  \SavedLstInline
}

\def\transarrow{\xrightarrow}
\newcommand{\setarrow}[1]{\def\transarrow{#1}}

\def\padding{\phantom{X}}
\newcommand{\setpadding}[1]{\def\padding{#1}}

\def\subarrow{}
\newcommand{\setsubarrow}[1]{\def\subarrow{#1}}

\newcommand{\trule}[2]{\frac{#1}{#2}}
\newcommand{\crule}[3]{\frac{#1}{#2},\;{#3}}
\newcommand{\withenv}[2]{{#1}\vdash{#2}}
\newcommand{\trans}[3]{{#1}\transarrow{\padding{\textstyle #2}\padding}\subarrow{#3}}
\newcommand{\ctrans}[4]{{#1}\transarrow{\padding#2\padding}\subarrow{#3},\;{#4}}
\newcommand{\llang}[1]{\mbox{\lstinline[mathescape]|#1|}}
\newcommand{\pair}[2]{\inbr{{#1}\mid{#2}}}
\newcommand{\inbr}[1]{\left<{#1}\right>}
\newcommand{\highlight}[1]{\color{red}{#1}}
%\newcommand{\ruleno}[1]{\eqno[\scriptsize\textsc{#1}]}
\newcommand{\ruleno}[1]{\mbox{[\textsc{#1}]}}
\newcommand{\rulename}[1]{\textsc{#1}}
\newcommand{\inmath}[1]{\mbox{$#1$}}
\newcommand{\lfp}[1]{fix_{#1}}
\newcommand{\gfp}[1]{Fix_{#1}}
\newcommand{\vsep}{\vspace{-2mm}}
\newcommand{\supp}[1]{\scriptsize{#1}}
\newcommand{\sembr}[1]{\llbracket{#1}\rrbracket}
\newcommand{\cd}[1]{\texttt{#1}}
\newcommand{\free}[1]{\boxed{#1}}
\newcommand{\binds}{\;\mapsto\;}
\newcommand{\dbi}[1]{\mbox{\bf{#1}}}
\newcommand{\sv}[1]{\mbox{\textbf{#1}}}
\newcommand{\bnd}[2]{{#1}\mkern-9mu\binds\mkern-9mu{#2}}
\newcommand{\meta}[1]{{\mathcal{#1}}}
\newcommand{\dom}[1]{\mathtt{dom}\;{#1}}
\newcommand{\primi}[2]{\mathbf{#1}\;{#2}}
\renewcommand{\dom}[1]{\mathcal{D}om\,({#1})}
\newcommand{\ran}[1]{\mathcal{VR}an\,({#1})}
\newcommand{\fv}[1]{\mathcal{FV}\,({#1})}
\newcommand{\tr}[1]{\mathcal{T}r_{#1}}
\newcommand{\diseq}{\not\equiv}
\newcommand{\reprfunset}{\mathcal{R}}
\newcommand{\reprfun}{\mathfrak{f}}
\newcommand{\cstore}{\Omega}
\newcommand{\cstoreinit}{\cstore_\epsilon^{init}}
\newcommand{\csadd}[3]{\mathbf{add}\,(#1, #2 \diseq #3)}  %{#1 + [#2 \diseq #3]}
\newcommand{\csupdate}[2]{\mathbf{update}\,(#1, #2)}  %{#1 \cdot #2}
\newcommand{\cupdate}[2]{\mathbf{update_{constr}}\,(#1, #2)}  %{#1 \cdot #2}
\newcommand{\eqrestr}{=_n}

\newcommand{\searchRule}[6] {
  #1, #2 \vdash (#3, #4) \xRightarrow{#5} #6}
\newcommand{\extSearchRule}[8] {
  #1, #2, #3, #4 \vdash (#5, #6) \xRightarrow{#7}_{e} #8}
\newcommand{\q}{\hspace{0.5em}}
\newcommand{\bigcdot}{\boldsymbol{\cdot}}
\newcommand{\bigslant}[2]{{\raisebox{.2em}{$#1$}\left/\raisebox{-.2em}{$#2$}\right.}}

\let\emptyset\varnothing
\let\eps\varepsilon

\sloppy

\begin{document}

%% Title information
\title{Certified Semantics for disequality} %% [Short Title] is optional;
                                           %% when present, will be used in
                                           %% header instead of Full Title.
\titlenote{This work was partially suppored by the grant 18-01-00380 from The Russian Foundation for Basic Research} %% \titlenote is optional;
                                        %% can be repeated if necessary;
                                        %% contents suppressed with 'anonymous'
%\subtitle{Subtitle}                     %% \subtitle is optional
%\subtitlenote{with subtitle note}       %% \subtitlenote is optional;
                                        %% can be repeated if necessary;
                                        %% contents suppressed with 'anonymous'


%% Author information
%% Contents and number of authors suppressed with 'anonymous'.
%% Each author should be introduced by \author, followed by
%% \authornote (optional), \orcid (optional), \affiliation, and
%% \email.
%% An author may have multiple affiliations and/or emails; repeat the
%% appropriate command.
%% Many elements are not rendered, but should be provided for metadata
%% extraction tools.

\author{Dmitry Rozplokhas}
\affiliation{%
  \institution{Higher School of Economics}}
\affiliation{%
  \institution{JetBrains Research}
  \country{Russia}}
\email{darozplokhas@edu.hse.ru}

\author{Dmitry Boulytchev}
\affiliation{%
  \institution{Saint Petersburg State University}}
\affiliation{%
  \institution{JetBrains Research}
  \country{Russia}}
\email{dboulytchev@math.spbu.ru}



%% Abstract
%% Note: \begin{abstract}...\end{abstract} environment must come
%% before \maketitle command
\begin{abstract}
We present an extension of our prior work on certified semantics for core miniKanren, introducing disequality constraints in the language.
Semantics is parameterized by an exact definition of constraint stores, allowing us to cover different implementations, and we provide a list of sufficient conditions on this definition for search completeness.
We also give two examples of concrete implemantations of constraint stores that satisfy those sufficient conditions.
The description and proofs are partially certified in Coq and a correct-by-construction interpreter is extracted.
\end{abstract}


%% 2012 ACM Computing Classification System (CSS) concepts
%% Generate at 'http://dl.acm.org/ccs/ccs.cfm'.
\begin{CCSXML}
<ccs2012>
<concept>
<concept_id>10003752.10003790.10003795</concept_id>
<concept_desc>Theory of computation~Constraint and logic programming</concept_desc>
<concept_significance>500</concept_significance>
</concept>
<concept>
<concept_id>10003752.10010124.10010131.10010133</concept_id>
<concept_desc>Theory of computation~Denotational semantics</concept_desc>
<concept_significance>500</concept_significance>
</concept>
<concept>
<concept_id>10003752.10010124.10010131.10010134</concept_id>
<concept_desc>Theory of computation~Operational semantics</concept_desc>
<concept_significance>500</concept_significance>
</concept>
</ccs2012>
\end{CCSXML}
%% \ccsdesc[500]{Theory of computation~Constraint and logic programming}
%% \ccsdesc[500]{Theory of computation~Denotational semantics}
%% \ccsdesc[500]{Theory of computation~Operational semantics}
%% End of generated code


%% Keywords
%% comma separated list
%% \keywords{Relational programming, denotational semantics, operational semantics, certified programming}  %% \keywords are mandatory in final camera-ready submission


%% \maketitle
%% Note: \maketitle command must come after title commands, author
%% commands, abstract environment, Computing Classification System
%% environment and commands, and keywords command.
\maketitle
\thispagestyle{empty}

\section{Introduction}


  \begin{comment}
We implemented the synthesis framework using \textsc{OCanren}~--- an embedding of \textsc{miniKanren} into \textsc{OCaml}~\cite{ocanren},~---
 and evaluated it on the set of benchmarks, reported in the previous works on \emph{ad-hoc} algorithms for pattern matching
 code generation~\cite{maranget2001,maranget2008}. In comparison with a simplified setting, presented above, our implementation
 deals with a more elaborate pattern matching problem~--- in particular, we support \emph{guard expressions}, name bindings in
 patterns and incorporate a deterministic top-down matching strategy, which is common in functional languages.
 
 Initially, our synthesis did not demonstrate good results. However, we applied the following techniques to improve both the performance
 and the quality of synthesized programs:
 
 \begin{itemize}
 \item we restricted the shape of scrutinees using type information;
 \item we utilized tabling to memoize repeating search steps;
 \item we implemented a pruning technique, which makes the search stop exploring a certain branch if the program, synthesized so far,
   contains too much nesting constructs (this factor can be precomputed by patterns analysis).
 \end{itemize}
 
 With these adjustments, our synthesis framework in a negligible time provides the same results as those reported in the previous works.
 Our future steps include extending the pattern matching language to completely match that of \textsc{OCaml} (for
 now we do not support GADTs), integrate the synthesis into the existing \textsc{OCaml} compiler and evaluate it on a
 set of real-world programs. Another direction is extending the pattern matching language to incorporate features which
 are known to be hard, tedious or error-prone to implement (for example, non-linear patterns).
 
\end{comment}


\begin{comment}

Algebraic data types are essential for typed functional programming and it's difficult to imagine effective compiler without effective compilation of pattern matching. 
There are a few different approaches for compiling pattern mathcing. GHC is using influential paper~\cite{Jones1987}, OCaml is currently based on~\cite{maranget2001} although a work~\cite{maranget2008} can slightly improve effectiveness of generated code. 

Also there are a number of possible extensions of pattern matching itself (guards, non-linear patterns, active patterns) and extensions of possible matchable values (polymorphic variants in OCaml, for example). Although having all these extensions can be helpful for programming in practice, they can complicate compilation schema or make it very difficult to generate effective code. Supporting a large number  of extensions can seriously complicate compiler's implementation too.

We present an approach to pattern matching code generation based on application of relational programming~\cite{TRS,WillThesis} and, in
 particular, relational interpreters~\cite{unified}. We expect that our approach can compile pattern mathcing to competitive code and will be easier to support during adding of new pattern matching extensions.
 
\end{comment}
 We use a \emph{relational interpreter} for $\ir$
 
 \[
 eval^o_{\ir}\, (s, p, i)
 \]
 
 Here $s$ and $i$ have the same meaning as in declarative semantics description, $p\in\ir$~--- a syntactic representation of
 a program in $\ir$. The relation $eval^o_{\ir}$ encodes the operational semantics of $\ir$; it holds, if
 evaluating $p$ for $s$ returns $i$. Being relational interpreter, however, $eval^o_{\ir}$ is capable of solving a
 synthesis problem: by a scrutinee $s$ and a number $i$ calculate a program $p$ which makes the relation to hold.
 Within this setting, we can formulate the pattern-matching synthesis problem as follows: \emph{for a given ordered list of patterns $ps$ find a program $p$, such that}
 
 \[
 \forall s\in\mathcal{V},\,\exists i,\,eval^o_{\ir}\, (s, p, i) \wedge\, match (s, ps, i)
 \]
 
 It is rather problematic to directly solve this synthesis problem with existing \textsc{miniKanren} implementations as
 they provide a rather limited support for universal quantification~\cite{eigen,moiseenko}. However, in our concrete
 case there is a simple way to alleviate this problem. Indeed, we may replace universal quantification over $i$ by
 a finite conjunction, as the length of $ps$ is known at the synthesis time. As for the quantification over $s$, for
 any concrete $ps$ we may precompute a \emph{complete set of examples} $\mathcal{E}(ps)\subseteq\mathcal{V}$ with the following
 property:
 
 \[
 \forall i\in\mathbb{N},\,(\forall s\in\mathcal{E}(ps),\,match\, (s, ps, i) \Leftrightarrow \forall s\in\mathcal{V},\,match\, (s, ps, i))
 \]
 
 It easy to see, that for arbitrary $ps$ there exists a finite complete set of examples (indeed, any pattern describes the ``shape''
 of a scrutinee up to some finite depth, beyond which all scrutinees become indistinguishable). Thus, for a given $ps$ we may
 completely eliminate the quantification, reformulating the synthesis problem as
 
 \[
 \bigwedge_{i\in[1\dots|ps|]}\,\bigwedge_{s\in\mathcal{E}(ps)} (eval^o_{\ir}\, (s, p, i) \wedge match\, (s, ps, i))
 \]
 
 We implemented the synthesis framework using \textsc{OCanren}~--- an embedding of \textsc{miniKanren} into \textsc{OCaml}~\cite{ocanren},~---
 and evaluated it on the set of benchmarks, reported in the previous works on \emph{ad-hoc} algorithms for pattern matching
 code generation~\cite{maranget2001,maranget2008}. In comparison with a simplified setting, presented above, our implementation
 deals with a more elaborate pattern matching problem~--- in particular, we support \emph{guard expressions}, name bindings in
 patterns and incorporate a deterministic top-down matching strategy, which is common in functional languages.
 
 Initially, our synthesis did not demonstrate good results. However, we applied the following techniques to improve both the performance
 and the quality of synthesized programs:
 
 \begin{itemize}
 \item we restricted the shape of scrutinees using type information;
 \item we utilized tabling to memoize repeating search steps;
 \item we implemented a pruning technique, which makes the search stop exploring a certain branch if the program, synthesized so far,
   contains too much nesting constructs (this factor can be precomputed by patterns analysis).
 \end{itemize}
 
 With these adjustments, our synthesis framework in a negligible time provides the same results as those reported in the previous works.
 Our future steps include extending the pattern matching language to completely match that of \textsc{OCaml} (for
 now we do not support GADTs), integrate the synthesis into the existing \textsc{OCaml} compiler and evaluate it on a
 set of real-world programs. Another direction is extending the pattern matching language to incorporate features which
 are known to be hard, tedious or error-prone to implement (for example, non-linear patterns).
 
 \begin{comment}
 
 
 Real world modern compilers are obliged to address a few problems which are NP-complete and hence can't have effective
 algorithm to solve them. So, compilers use semi-optimal algorithms to find a decent solution. Optimal algorithms require
 brute force search to get the best solution and  affect compilation speed negatively. In this work we apply relational
 programming -- a convenient DSL for implementing search -- to compilation of pattern matching, one of a kind hard problems for compiler.
 
 The task of compiling pattern matching for typed languages is well presented in literature~\cite{maranget2001,maranget2008}.
 
 
 We test approach on simplified source language $PM$ where scrutinee is a value $\in\mathcal{V}$ of algebraic data type, only wildcards
 and nested constructors are allowed as patterns $\mathcal{P}$ and right hand side of clause is its index. The source language is easy
 extendable by pattern variables and optional pattern guards that test subterms of scrutinee using a function. The semantics of $PM$
 is a function from concrete scrutinee $s$, concrete patterns $pats$ and concrete guards $gs$ to clause indexes, and is denoted
 as $\sem{s,pats,gs}_{PM} = i$.
 
 Compilation scheme translates sentences from $PM$ to $\ir$ language which has constructions for clause indexes and conditions which
 test matchable values for specific constructor. Matchable values can be either a scrutinee, or a projection of matchable value that
 returns one of its field indexed by natural numbers. $\ir$ language is easy extendable by tests for fixed number of pattern guards.
 The semantics is straightforward and is denoted by $\sem{\cdot}_{\ir}$.
 
 We deal with a task of compiling pattern matching as it is a synthesis problem. The goal of algorithm is to synthesize $ideal_\ir$
 for concrete patterns $pats$ and guards that, firstly, will behave the same as original pattern matching for any possible
 scrutinee $s$. Secondly, we want shortest solution because short code usually runs faster. Relational programming~\cite{OCanren}
 will help with that because it has a tendency to generate short answers earlier, although this tendency is not strict.
 
 $$
 \forall s:\; \sem{s; ideal_\ir}_\ir = \sem{s;pats}_{PM}
 $$
 
 To eliminate universal quantifier we use the following observation: for \emph{finite} amount of patterns of \emph{finite} height
 we can generate \emph{finite} amount of examples to test pattern matching semantics. In examples, very deep subterms can have
 any value because they will not be tested during pattern matching. We can reformulate synthesis problem as follows:
 
 $$
 \mid  Examples\mid < \infty\quad \land\quad \left(\forall e \; (e\in Examples)\quad\land \quad\left( \sem{e; ideal_\ir}_\ir = \sem{e;pats}_{PM}\right)\right)
 $$
 
 For plain ADT the approach will generate required examples in finite time, but for GADTs it can diverge because inhabitancy problem
 is semi-decidable~\cite{garrigue2017gadts}(chapter 5). Inhabitants generation as well as synthesis algorithm is
 implemented\footnote{\url{https://github.com/Kakadu/pat-match/}} using relational programming.
 
 Presented approach is good as general description of an idea but require a few tweaks to start working, for example, on presented
 sample~\ref{fig:example1} . Firstly, synthesis procedure in a way as it is described doesn't take types into account, so it is
 useful to give hints about which parts of scrutinee should be checked for which constructors. Second observation says that we
 run $\sem{\cdot}_\ir$ in concrete direction, so it is possible to check periodically count of \texttt{IfTag} constructors in
 result value and prune branches where it becomes too big. Thirdly, synthesis query generates a lot of similar queries, and
 we use tabling to speedup search. All three observations are important, removing one of them leads to visible performance degradation.
 
 The optimal (two \texttt{IfTag}'s) and the semi-optimal solution (three \texttt{IfTag}'s) for~\ref{fig:example1} are described
 in~\cite{maranget2008}. Current implementation generates semi-optimal solution as 28th answer. Before that it generates optimal
 solution (and it's equivalents) three times, other 24 answers are longer and less useful. All tasks (example generation, synthesis
 and printing answers) take 3 seconds, which is unfortunate.
 
 Shortly, we present following contributions
 \begin{itemize}
 \item Code synthesis for pattern matching works after implementing \emph{three optimizations} above.
 \item GADTs, pattern binding and guards works for simple examples, the approach is easy extendable by them.
 \end{itemize}
 
 Future work is
 \begin{itemize}
 \item Discover other optimizations and enable current ones automatically using type information (at the moment we patched synthesis
   algorithm manually for concrete example).
 \item When current implementation tests for \texttt{cons} tag it can't propagate constraint that tag equals to \texttt{nil} to
   the \texttt{else} branch, which partially explains why branch pruning is so useful.
 \item Algorithm for inhabitant generation requires proper formulation and proof.
 \item Apply current synthesis procedure for exhaustiveness checking which will give us \emph{single} procedure for compilation and exhaustiveness checking.
 \item Test the approach on real world problems (embedding to OCaml compiler).
 \end{itemize}
 
 
 \begin{figure}[t]
   \[
   \begin{array}{rcll}
     \mathcal{C} & = & \{ C_1^{k_1}, \dots, C_n^{k_n} \}    &\mbox{(constructors)} \\
     \mathcal{V} & = & \mathcal{C}\,\mathcal{V}^*        &\mbox{(values)}       \\
     \mathcal{P} & = & \_ \mid \mathcal{C}\,\mathcal{P}^*&\mbox{(patterns)}     \\
     \mathcal{M} & = & \bullet \mid \mathcal{M} [\mathbb{N}]&
 
   \end{array}
   \]
 \end{figure}
 
 \begin{figure}
 \centering
 \begin{minipage}{.7\textwidth}
   \centering
 \begin{align*}
 \mathcal{C} =&\; \{ C_1^{k_1}, \dots, C_n^{k_n} \} \\
 \mathcal{V} =&\;  \mathcal{C}\ \mathcal{V}^*\\
 \mathcal{M} =&\;  \mathcal{S} \\
           \mid\; &\; \text{\texttt{Field}}\;  \mathcal{M}\times  \mathbb{N}\\
 \mathcal{P} =&\;  \text{\texttt{Wildcard}} \\
           \mid\; &\; \text{\texttt{Var}}\  Name\\
           \mid\; &\; \text{\texttt{PConstructor}}\  \mathcal{C}\times  \mathcal{P}^*\\
 \ir  =&\; \text{\texttt{Int}}\  \mathbb{N} \\
 %           \mid\; &\;\mathcal{S} \\
            \mid\; &\; \text{\texttt{IfTag}}\; \mathcal{C}\times \mathcal{M}\times \ir\times \ir\\
            \mid\; &\; \text{\texttt{IfGuard}}\ \mathbb{N}\times (Name\times \mathcal{M})^*\times \ir\times \ir\\
 Clause =&\;  \mathcal{P} \times \mathbb{N}? \times \ir
 \end{align*}
   \captionof{figure}{Structure of $PM$ and \ir languages}
 %  \label{fig:test1}
 \end{minipage}%
 \begin{minipage}{.3\textwidth}
   \centering
 \begin{lstlisting}[language=ocaml]
 match s with 
 | ([], _)     -> 1
 | (_, [])     -> 2
 | (_::_,_::_) -> 3
 \end{lstlisting}
   \captionof{figure}{Simple example of pattern matching problem from~\cite{maranget2008}}
 \label{fig:example1}
 \end{minipage}
 \end{figure}
 
 \end{comment}
 

\section{Review of syntax and semantics for the basic language}

In this section, we revisit our definition of syntax and two semantics for the basic language without disequality constraints and our main result~--- equivalence of the two semantics.

\subsection{Syntax of the basic language}
\label{subsec_syntax}

\begin{figure}[t]
\centering
\[
\begin{array}{ccll}
  \mathcal{C} & = & \{C_i^{k_i}\} & \mbox{constructors with arities} \\
  \mathcal{T}_X & = & X \cup \{C_i^{k_i} (t_1, \dots, t_{k_i}) \mid t_j\in\mathcal{T}_X\} & \mbox{terms over the set of variables $X$} \\
  \mathcal{D} & = & \mathcal{T}_\emptyset & \mbox{ground terms}\\
  \mathcal{X} & = & \{ x, y, z, \dots \} & \mbox{syntactic variables} \\
  \mathcal{A} & = & \{ \alpha, \beta, \gamma, \dots \} & \mbox{semantic variables} \\
  \mathcal{R} & = & \{ R_i^{k_i}\} &\mbox{relational symbols with arities} \\[2mm]
  \mathcal{G} & = & \mathcal{T_X}\equiv\mathcal{T_X}   &  \mbox{unification} \\
              &   & \mathcal{G}\wedge\mathcal{G}     & \mbox{conjunction} \\
              &   & \mathcal{G}\vee\mathcal{G}       &\mbox{disjunction} \\
              &   & \mbox{\lstinline|fresh|}\;\mathcal{X}\;.\;\mathcal{G} & \mbox{fresh variable introduction} \\
              &   & R_i^{k_i} (t_1,\dots,t_{k_i}),\;t_j\in\mathcal{T_X} & \mbox{relational symbol invocation} \\[2mm]
  \mathcal{S} & = & \{R_i^{k_i} = \lambda\;x_1^i\dots x_{k_i}^i\,.\, g_i;\}\; g & \mbox{specification}
\end{array}
\]
\caption{The syntax of the basic language}
\label{syntax}
\end{figure}

The syntax of the language is shown in Figure~\ref{syntax}. First, we fix a set of constructors $\mathcal{C}$ with known arities and consider
a set of terms $\mathcal{T}_X$ with constructors as functional symbols and variables from $X$. We parameterize this set with an alphabet of
variables since in the semantic description we will need \emph{two} kinds of variables. The first kind, \emph{syntactic} variables, is denoted
by $\mathcal{X}$. We also consider an alphabet of \emph{relational symbols} $\mathcal{R}$ which are used to name relational definitions.
The central syntactic category in the language is a \emph{goal}. In our case, there are five types of goals: \emph{unification} of terms,
conjunction and disjunction of goals, fresh variable introduction, and invocation of some relational definition. Thus, unification is used
as a constraint, and multiple constraints can be combined using conjunction, disjunction, and recursion. For the sake of brevity we
abbreviate immediately nested ``\lstinline|fresh|'' constructs into the one, writing ``\lstinline|fresh $x$ $y$ $\dots$ . $g$|'' instead of
``\lstinline|fresh $x$ . fresh $y$ . $\dots$ $g$|''. The final syntactic category is \emph{specification} $\mathcal{S}$. It consists of a set
of relational definitions and a top-level goal. A top-level goal represents a search procedure which returns a stream of substitutions for
the free variables of the goal. The language we defined is first-order, as goals can not be passed as parameters,
returned or constructed at runtime.

As an example consider the specification for the standard \lstinline|append$^o$| relation and a query to it that splits the list consisting of three constants \lstinline|A|, \lstinline|B| and \lstinline|C| in two parts.

\begin{lstlisting}
  append$^o$ = fun x y xy .
    ((x === Nil) /\ (xy === y)) \/
    (fresh h t ty .
       (x  === Cons (h, t))  /\
       (xy === Cons (h, ty)) /\
       (append$^o$ y t ty)
    );
  append$^o$ x y (Cons (A, Cons (B, Cons (C, Nil))))
\end{lstlisting}

\subsection{Denotational sematics}

For denotational semantics, we use a simple set-theoretic approach which can be considered as an analogy to the least Herbrand model for definite logic programs~\cite{LHM}.

Intuitively, the mathematical model for every goal should be a finitary relation between semantic variables that occur free in this goal. We represent this relation as a set of total
functions 

\[
\mathfrak{f}:\mathcal{A}\mapsto\mathcal{D}
\]

from semantic variables to ground terms. We call these functions \emph{representing functions}.

Then, the semantic function for goals is parameterized over environments which prescribe semantic functions to relational symbols:

\[
  \Gamma : \mathcal{R} \to (\mathcal{T_A}^*\to 2^{\mathcal{A}\to\mathcal{D}})
\]

An environment associates with relational symbol a function which takes a string of terms (the arguments of the relation) and returns a set of
representing functions. The signature for semantic brackets for goals is as follows:

\[
\sembr{\bullet}_{\Gamma} : \mathcal{G}\to 2^{\mathcal{A}\to\mathcal{D}}
\]

It maps a goal into the set of representing functions w.r.t. an environment $\Gamma$.

We formulate the following important \emph{completeness condition} for the semantics of a goal $g$: for any goal $g$ and two representing functions ${\mathfrak f}$ and ${\mathfrak f'}$, such that $\left.{\mathfrak f}\right|_{FV(g)} = \left.{\mathfrak f'}\right|_{FV(g)}$
\[ {\mathfrak f} \in \sembr{g} \Leftrightarrow {\mathfrak f'} \in \sembr{g} \]

In other words, representing functions for a goal $g$ restrict only the values of free variables of $g$ and do not introduce any ``hidden'' correlations.
This condition guarantees that our semantics is complete in the sense that it does not introduce artificial restrictions for the relation it defines.
We proved that the semantics of goals always satisfy this condition.

To define the semantic function we need a few operations for representing functions:

\begin{itemize}
\item A homomorphic extension of a representing function 

\[
  \overline{\mathfrak{f}}:\mathcal{T_A}\to\mathcal{D}
\]

which maps terms to terms:

\[
\begin{array}{rcl}

  \overline{\mathfrak f}\,(\alpha) & = & \mathfrak f\,(\alpha)\\
  \overline{\mathfrak f}\,(C_i^{k_i}\,(t_1,\dots.t_{k_i})) & = & C_i^{k_i}\,(\overline{\mathfrak f}\,(t_1),\dots \overline{\mathfrak f}\,(t_{k_i}))
\end{array}
\]

\item A pointwise modification of a function

\[
f\,[x\gets v]\,(z)=\left\{
\begin{array}{rcl}
  f\,(z) &,& z \ne x \\
  v      &,& z = x
\end{array}
\right.
\]

\item A \emph{generalization} operation:

\[
\mathfrak{f}\uparrow\alpha = \{ \mathfrak{f}\,[\alpha\gets d] \mid d\in\mathcal D\}
\]

Informally, this operation generalizes a representing function into a set of representing functions in such a way that the
values of these functions for a given variable cover the whole $\mathcal{D}$. We extend the generalization operation for sets of
representing functions $\mathfrak{F}\subseteq\mathcal{A}\to\mathcal{D}$:

\[
  \mathfrak{F}\uparrow\alpha = \bigcup_{\mathfrak{f}\in\mathfrak{F}}(\mathfrak{f}\uparrow\alpha)
\]

\end{itemize}

The semantics for goals is shown on Figure~\ref{denotational_semantics_of_goals}.

\begin{figure}[t]
  \[
  \begin{array}{cclr}
    \sembr{t_1\equiv t_2}_\Gamma&=&\{\mathfrak f : \mathcal{A}\to\mathcal{D}\mid \overline{\mathfrak{f}}\,(t_1)=\overline{\mathfrak{f}}\,(t_2)\}& \ruleno{Unify$_D$}\\
    \sembr{g_1\wedge g_2}_\Gamma&=&\sembr{g_1}_\Gamma\cap\sembr{g_1}_\Gamma&\ruleno{Conj$_D$}\\
    \sembr{g_1\vee g_2}_\Gamma&=&\sembr{g_1}_\Gamma\cup\sembr{g_1}_\Gamma&\ruleno{Disj$_D$}\\
    \sembr{\mbox{\lstinline|fresh|}\,x\,.\,g}_\Gamma&=&(\sembr{g\,[\alpha/x]}_\Gamma)\uparrow\alpha,\;\alpha\not\in FV(g)& \ruleno{Fresh$_D$}\\
    \sembr{R\,(t_1,\dots,t_k)}_\Gamma&=&(\Gamma\,R)\,t_1\dots t_k & \ruleno{Invoke$_D$}
  \end{array}
  \]
  \caption{Denotational semantics of goals}
  \label{denotational_semantics_of_goals}
\end{figure}

The final component is the semantics of specifications. Given a specification

\[
\{R_i=\lambda\,x_1^i\dots x_{k_i}^i\,.\,g_i;\}_{i=1}^n\;g
\]

we construct a correct environment $\Gamma_0$ and then take the semantics of the top-level goal:

\[
\sembr{\{R_i=\lambda\,x_1^i\dots x_{k_i}^i\,.\,g_i;\}_{i=1}^n\;g}=\sembr{g}_{\Gamma_0}
\]

As the set of definitions can be mutually recursive we apply the fixed point approach and define $\Gamma_0$ as the least
fixed point of a specific function $F$ that takes an environment $\Gamma$ and returns new environment in which semantics
of a body of each definition is evaluated with environment $\Gamma$.


\subsection{Operational sematics}

The operational semantics of \textsc{miniKanren}, which we described, corresponds to the known
implementations with interleaving search. The semantics is given in the form of a labeled transition system (LTS~\cite{LTS}).

The states in the transition system have the following shape:

\[
S = \mathcal{G}\times\Sigma\times\mathbb{N}\mid S\oplus S \mid S \otimes \mathcal{G}
\]

A state has a tree-like structure with intermediate nodes corresponding to partially-evaluated conjunctions (``$\otimes$'') or
disjunctions (``$\oplus$''). A leaf in the form $\inbr{g, \sigma, n}$ determines a goal in a context, where $g$~--- a goal, $\sigma$~--- a substitution accumulated so far,
and $n$~--- a natural number, which corresponds to a number of semantic variables used to this point. For a conjunction node, its right child is always a goal since
it cannot be evaluated unless some result is provided by the left conjunct.

We also need extended states

\[
\overline{S} = \diamond \mid S
\]

where $\diamond$ symbolizes the end of the evaluation.

The set of labels is defined as follows:

\[
L = \circ \mid \Sigma\times \mathbb{N}
\]

The label ``$\circ$'' is used to mark those steps which do not provide an answer; otherwise, a transition is labeled by a pair of a substitution and a number of allocated
variables. The substitution is one of the answers, and the number is threaded through the derivation to keep track of the allocated variables.

\begin{figure*}
  \renewcommand{\arraystretch}{1.6}
  \[
  \begin{array}{cr}
    \inbr{t_1 \equiv t_2, \sigma, n} \xrightarrow{\circ} \Diamond , \, \, \nexists\; mgu\,(t_1 \sigma, t_2 \sigma) &\ruleno{UnifyFail} \\
    \inbr{t_1 \equiv t_2, \sigma, n} \xrightarrow{(mgu\,(t_1 \sigma, t_2 \sigma) \circ \sigma),\, n)} \Diamond & \ruleno{UnifySuccess} \\
    \inbr{g_1 \lor g_2, \sigma, n} \xrightarrow{\circ} \inbr{g_1, \sigma, n} \oplus \inbr{g_2, \sigma, n} & \ruleno{Disj} \\
    \inbr{g_1 \land g_2, \sigma, n} \xrightarrow{\circ} \inbr{ g_1, \sigma, n} \otimes g_2 & \ruleno{Conj} \\
    \inbr{\mbox{\lstinline|fresh|}\, x\, .\, g, \sigma, n} \xrightarrow{\circ} \inbr{g\,[\bigslant{\alpha_{n + 1}}{x}], \sigma, n + 1} & \ruleno{Fresh} \\
    \dfrac{R_i^{k_i}=\lambda\,x_1\dots x_{k_i}\,.\,g}{\inbr{R_i^{k_i}\,(t_1,\dots,t_{k_i}),\sigma,n} \xrightarrow{\circ} \inbr{g\,[\bigslant{t_1}{x_1}\dots\bigslant{t_{k_i}}{x_{k_i}}], \sigma, n}} & \ruleno{Invoke}\\
    \dfrac{s_1 \xrightarrow{\circ} \Diamond}{(s_1 \oplus s_2) \xrightarrow{\circ} s_2} & \ruleno{DisjStop}\\
    \dfrac{s_1 \xrightarrow{r} \Diamond}{(s_1 \oplus s_2) \xrightarrow{r} s_2} & \ruleno{DisjStopAns}\\
    \dfrac{s \xrightarrow{\circ} \Diamond}{(s \otimes g) \xrightarrow{\circ} \Diamond} &\ruleno{ConjStop}\\
    \dfrac{s \xrightarrow{(\sigma, n)} \Diamond}{(s \otimes g) \xrightarrow{\circ} \inbr{g, \sigma, n}}  & \ruleno{ConjStopAns}\\
    \dfrac{s_1 \xrightarrow{\circ} s'_1}{(s_1 \oplus s_2) \xrightarrow{\circ} (s_2 \oplus s'_1)} &\ruleno{DisjStep}\\
    \dfrac{s_1 \xrightarrow{r} s'_1}{(s_1 \oplus s_2) \xrightarrow{r} (s_2 \oplus s'_1)} &\ruleno{DisjStepAns}\\
    \dfrac{s \xrightarrow{\circ} s'}{(s \otimes g) \xrightarrow{\circ} (s' \otimes g)} &\ruleno{ConjStep}\\
    \dfrac{s \xrightarrow{(\sigma, n)} s'}{(s \otimes g) \xrightarrow{\circ} (\inbr{g, \sigma, n} \oplus (s' \otimes g))} & \ruleno{ConjStepAns} 
  \end{array}
  \]
  \caption{Operational semantics of interleaving search}
  \label{lts}
\end{figure*}

The transition rules are shown in Figure~\ref{lts}. The introduced transition system is completely deterministic.

A derivation sequence for a certain state $s$ determines a \emph{trace} $\tr{s}$~--- a finite or infinite sequence of answers. The trace corresponds to the stream of answers
in the reference \textsc{miniKanren} implementations.

\subsection{Semantics Equivalence}

After we defined two different kinds of semantics for \textsc{miniKanren} we related them and showed that the results given by these two semantics are the same for any specification.
By proving this equivalence we established \emph{completeness} of the search which means that the search will get all answers satisfying the described specification and only those.

To do it we had to relate the answers produced by these two semantics as they have different forms: a trace of substitutions (along with numbers of allocated variables)
for operational and a set of representing functions for denotational. There is a natural way to extend any substitution to a representing function: composing it with an arbitrary representing function will preserve all variable dependencies in the substitution. So we define a set of representing functions corresponding to substitution as follows:

\[
\sembr{\sigma} = \{\overline{\mathfrak f} \circ \sigma \mid \mathfrak{f}:\mathcal{A}\mapsto\mathcal{D}\}
\]

And \emph{denotational analog} of an operational semantics (a set of representing functions corresponding to answers in the trace) for given extended state $s$ is
then defined as a union of sets for all substitution in the trace:

\[
\sembr{s}_{op} = \cup_{(\sigma, n) \in \tr{s}} \sembr{\sigma}
\]

This allows us to state the theorem relating two semantics.

\begin{theorem}[Operational semantics soundness and completeness]
For any specification $\{\dots\}\; g$, for which the indices of all free variables in $g$ are limited by some number $n$

\[
\sembr{\inbr{g, \epsilon, n}}_{op} \eqrestr \sembr{\{\dots\}\; g}.
\]
\end{theorem}

Where `$\eqrestr$' means that we compare representing functions of these sets only on the semantic variables from $\{\alpha_1, \dots, \alpha_n\}$:

\[
S_1 \eqrestr S_2 \xLeftrightarrow{def}  \{\mathfrak{f}|_{\{\alpha_1,\dots,\alpha_n\}} \mid \mathfrak{f} \in S_1 \} = \{\mathfrak{f}|_{\{\alpha_1,\dots,\alpha_n\}} \mid \mathfrak{f} \in S_2 \}.
\]

The stronger version that simply states that the sets acquired from two semantics are equal does not hold.
The reason for this is that denotational semantics encodes only dependencies between the free variables of a goal, which is reflected by the completeness condition, while
operational semantics may also contain dependencies between semantic variables allocated in ``\lstinline|fresh|''.
Therefore we have to restrict representing functions on the semantic variables allocated in the beginning (which includes all free variables of a goal). This does not
compromise our promise to prove the completeness of the search as \textsc{miniKanren} provides the result as substitutions only for queried variables,
which are allocated in the beginning.

The proof of this main theorem was certified in \textsc{Coq}.

\section{Extention with disequality constraints}

In this section, we present extensions of our two semantics on the language with disequality constraints and revised version of the soundness and completeness theorem.

Disequality constraint introduces one additional type of base goal~--- a disequality of two terms: $t_1 \diseq t_2$

The extension of denotational semantics is straightforward (as disequality constraint is complementary to unification):

\[ \sembr{t_1 \diseq t_2}  =  \{\reprfun \in \reprfunset \mid \overline{\reprfun}\,(t_1) \neq \overline{\reprfun}\,(t_2)\}, \]

This definition for a new type of goals fits nicely into the general inductive definition of denotational semantics of an arbitrary goal
and preserve its properties, such as completeness condition.

In the operational case, we deviate from describing one specific search implementation, since there are several distinct ways to embed disequality constraints
in the language and we would like to be able to give semantics (and subsequently prove correctness) for all of them. Therefore we base the extended operational
semantics on a number of abstract definitions concerning constraint stores for which different concrete implementations may be substituted.

So we assume that we are given a set of constraint store objects, which we denote by $\cstore_\sigma$ (indexing every constraint store with
some substitution $\sigma$ and assuming the store and the substitution are consistent with each other), and three following operations:

\begin{enumerate}
\item Initial constraint store $\cstoreinit$ (where $\epsilon$ is empty substitution), which does not contain any constraints yet.
\item Adding a disequality constraint to a store $\csadd{\cstore_\sigma}{t_1}{t_2}$, which may result in a new constraint store $\cstore^\prime_\sigma$ or a failure $\bot$,
  if the new constraint store is inconsistent with the substitution $\sigma$.
\item Updating a substitution in a constraint store $\csupdate{\cstore_\sigma}{\delta}$ to intergate a new substitution $\delta$ into the current one,
  which may result in a new constraint store $\cstore^\prime_{\sigma \delta}$ or a failure $\bot$, if the constraint store is inconsistent with the new substitution.
\end{enumerate}

The change in operational semantics for the language with disequality constraints is now straightforward: we add a constraint store to a basic (leaf) state $\inbr{g, \sigma, \cstore_\sigma, n}$, as well as in the label form $(\sigma, \cstore_\sigma, n)$, and this store is simply transmitted in all the rules, except the ones for unification. We change the rules for unification using $\mathbf{update}$ operation and add the rules for disequality constraint using $\mathbf{add}$. In both cases, the search in the current branch is pruned if these primitives return $\bot$.

 \[
  \begin{array}{cr}
    \inbr{t_1 \equiv t_2, \sigma, \cstore_\sigma, n} \xrightarrow{\circ} \Diamond , \, \, \nexists\; mgu\,(t_1, t_2, \sigma) &\ruleno{UnifyFailMGU} \\[2mm]
    \inbr{t_1 \equiv t_2, \sigma, \cstore_\sigma, n} \xrightarrow{\circ} \Diamond , \, \, mgu\,(t_1, t_2, \sigma) = \delta, \, \, \csupdate{\cstore_\sigma}{\delta} = \bot &\ruleno{UnifyFailUpdate} \\[2mm]
    \inbr{t_1 \equiv t_2, \sigma, \cstore_\sigma, n} \xrightarrow{(\sigma \delta, \, \cstore'_{\sigma\delta}, \, n)} \Diamond , \, \, mgu\,(t_1, t_2, \sigma) = \delta, \, \, \csupdate{\cstore_\sigma}{\delta} = \cstore'_{\sigma\delta} & \ruleno{UnifySuccess} \\[2mm]
    \inbr{t_1 \diseq t_2, \sigma, \cstore_\sigma, n} \xrightarrow{\circ} \Diamond , \, \, \csadd{\cstore_\sigma}{t_1}{t_2} = \bot &\ruleno{DiseqFail} \\[2mm]
    \inbr{t_1 \diseq t_2, \sigma, \cstore_\sigma, n} \xrightarrow{(\sigma \delta, \, \cstore'_{\sigma}, \, n)} \Diamond , \, \, \csadd{\cstore_\sigma}{t_1}{t_2} = \cstore'_\sigma &\ruleno{DiseqSucess} \\[2mm]
  \end{array}
\]

The initial state naturally has an initial constraint store $\inbr{g, \eps, \cstoreinit, n}$.

To state the soundness and completeness result now we need to revise our definition of the denotational analog of an answer $(\sigma, \cstore_\sigma, n)$
since we have to take into account the restrictions that a constraint store $\cstore_\sigma$ encodes.
To do it we need one more abstract definition~--- a denotational interpretation of a constraint store $\sembr{\cstore_\sigma}$ as a set of representing functions.
We prove the soundness and completeness w.r.t. this interpretation and expect it to adequately reflect how the restrictions of constraint stores in the answers will be presented.
The denotational analog of operational semantics for an arbitrary extended state is then redefined as follows.

 \[
\sembr{s}_{op} = \cup_{(\sigma, \cstore_\sigma, n) \in \tr{s}} \sembr{\sigma} \cap \sembr{\cstore_\sigma}
\]

The statement of the soundness and completeness theorem stays the same with regard to this updated definitions of semantics and denotational analog.

\begin{theorem}[Operational semantics soundness and completeness for extended language]
For any specification $\{\dots\}\; g$, for which the indices of all free variables in $g$ are limited by some number $n$

\[
\sembr{\inbr{g, \epsilon, \cstoreinit, n}}_{op} \eqrestr \sembr{\{\dots\}\; g}.
\]
\end{theorem}

To be able to prove it we, of course, need certain requirements for the given operations on constraint stores. We came up with the following list of sufficient
conditions on them for soundness and completeness.

\begin{enumerate}
\item $\sembr{\cstoreinit} = \{\mathfrak{f}:\mathcal{A}\mapsto\mathcal{D}\}$;
\item $\csadd{\cstore_\sigma}{t_1}{t_2} = \cstore^\prime_\sigma \implies \sembr{\cstore_\sigma} \cap \sembr{t_1 \diseq t_2} \cap \sembr{\sigma} = \sembr{\cstore^\prime_\sigma} \cap \sembr{\sigma}$;
\item $\csadd{\cstore_\sigma}{t_1}{t_2} = \bot \implies \sembr{\cstore_\sigma} \cap \sembr{t_1 \diseq t_2} \cap \sembr{\sigma} = \emptyset$;
\item $\csupdate{\cstore_\sigma}{\delta} = \cstore^\prime_{\sigma \delta} \implies \sembr{\cstore_\sigma} \cap \sembr{\sigma \delta} = \sembr{\cstore^\prime_{\sigma \delta}} \cap \sembr{\sigma \delta}$;
\item $\csupdate{\cstore_\sigma}{\delta} = \bot \implies \sembr{\cstore_\sigma} \cap \sembr{\sigma \delta} = \emptyset$.
\end{enumerate}

These conditions state that given denotational interpretation and the given operations on constraint stores are adequate to each other.
Condition 1 states that interpretation of the initial constraint store is the whole domain of representing function since it does not impose any restrictions.
Conditions 2 states that when we add a constraint to a store $\cstore_\sigma$ the interpretation of the result contains exactly those functions that simultaneously belong to
the interpretation of the store $\cstore_\sigma$ and satisfy the constraint if we consider only extensions of the substitution $\sigma$,
and condition 3 states that addition could fail only if no such functions exist.
Conditions 4 state that the result of updating a store with an additional substitution should have the same interpretation if we consider only extensions of the updated substitution,
and condition 5 states that update could fail only if no such functions exist.
The conditions 2-5 describe exactly what we need to prove the soundness and completeness for base goals (unification and disequality); at the same time,
since these conditions have relatively simple intuitive meaning in terms of these two operations they are expected to hold naturally
in all reasonable implementations of constraint stores.

We can prove that this is enough for soundness and completeness to hold for an arbitrary goal. However,
contrary to our expectations, the previous proof can not be just reused for all non-basic types of goals and has to be modified
significantly in the case of \lstinline|fresh|. Specifically, we need one additional condition on constraint store in state $(\sigma, n, \cstore_\sigma)$:
only the values on the first $n$ fresh variables determine whether a representing function belongs to the denotational semantics $\sembr{\sigma} \cap \sembr{\cstore_\sigma}$
of the state (note the similarity to the completeness condition). Luckily, we can infer this property for all states that can be constructed by our operational
semantics from the necessary conditions in the list above.

Thus, for an arbitrary implementation, we need to give a formal definition of constraint store object and its denotational interpretation, provide three
operations for it and prove five conditions on them, and by this we ensure that for arbitrary specification the interpretations of all solutions found by the
search in this version of MiniKanren will cover exactly the mathematical model of this specification.

As well as our previous development this extension is certified in Coq\footnote{\url{https://github.com/dboulytchev/miniKanren-coq/tree/disequality}}. We describe operational semantics and its soundness and completeness as modules parametrized by the definitions of constraint
stores and proofs of the necessary conditions for them.

\section{Concrete Implementatoins}

In this section, we define two concrete implementations of constraint stores, which can be incorporated in operational semantics: the trivial one (implementation A) and the one that is close to one real implementation in a certain version of \textsc{miniKanren} (implementation B).

\subsection{Implementation A}

This trivial implementation simply stores all pairs of terms that the search encounters in a multiset and never uses them.

\[ \cstore_\sigma \subset_m \mathcal{T} \times \mathcal{T} \]

\[ \cstoreinit = \emptyset \]

\[ \csadd{\cstore_\sigma}{t_1}{t_2} = \cstore_\sigma \cup \{(t_1, t_2)\} \]

\[ \csupdate{\cstore_\sigma}{\delta} = \cstore_\sigma \]

The interpretation of such constraint store is the set of all representing functions that does not equate terms in any pair:

\[ \sembr{\cstore_\sigma} = \{\reprfun \colon \mathcal{A}\mapsto\mathcal{D} \mid \forall (t_1, t_2) \in \cstore_\sigma, \; \overline{\reprfun}\,(t_1) \neq \overline{\reprfun}\,(t_2)\} \]

This is a correct implementation (although for the full implementation a programmer should find a way to present restrictions stored this way in answers adequately) and it satisfies sufficient conditions for completeness trivially, but it is not very practical. In particular, it does not use information acquired from disequalities to halt the search in case of contradiction and it can return meaningless answers with empty interpretations (such as $([\alpha_0 \mapsto 5], [\alpha_0 \neq 5], 1)$).

We certified this implementation in Coq, which allowed us to extract a correct-by-construction interpreter for \textsc{miniKanren} with disequality constraints.

\subsection{Implementation B}

This implementation is more similar to those in real miniKanren versions and takes an approach that is close to one described is~\cite{CKanren}.

In this version, every constraint is represented as a substitution containing variable bindings that should not be satisfied.

\[ \cstore_\sigma \subset_m \Sigma \]

So if a constraint store $\cstore_\sigma$ contains a substitution $\omega$ the set of representing functions prohibited by it is $\sembr{\sigma \omega}$
which provides the following denotational interpretation for a constraint store:

\[ \sembr{\cstore_\sigma} = \bigcap_{\omega \in \cstore_\sigma} \overline{ \sembr{\sigma \omega} } \]

We start with the empty store

\[ \cstoreinit = \emptyset \]

When encounter disequality for two terms we try to unify them and update constrain depending on the result of unification.

\[
\csadd{\cstore_\sigma}{t_1}{t_2} =
    \begin{cases}
       \cstore_\sigma                                & \not\exists mgu(t_1 \sigma, t_2 \sigma) \\
       \bot                                                 & mgu(t_1 \sigma, t_2 \sigma) = \epsilon \\
       \cstore_\sigma \cup \{\omega\}      & mgu(t_1 \sigma, t_2 \sigma) = \omega \neq \epsilon
    \end{cases}
\]

If terms are not unifiable, there is no need to change the constraint store.
If they are unified by the current substitution the constraint is already violated and we signal fail.
Otherwise, the most general unifier is an appropriate representation for this constraint.

When updating constraint store with an additional substitution $\delta$ we try to update each individual constraint substitution by treating it
as a list of pairs of terms that should not be unified (first element of each pair is variable),
applying $\delta$ to these terms and trying to unify all pairs simultaniously:

\[ \cupdate{[x_1 \mapsto t_1, \dots, x_k \mapsto t_k]}{\delta} = mgu([\delta(x_1), \dots, \delta(x_k)],[t_1 \delta, \dots, t_k \delta]) \]

We construct the updated constraint store from the results of all constraint updates.

\[
\csupdate{\cstore_\sigma}{\delta} =
\begin{cases}
  \bot                                                 & \exists \omega \in \cstore_\sigma: \cupdate{\omega}{\delta} = \epsilon \\
  \{ \omega' \mid \cupdate{\omega}{\delta} = \omega' \neq \bot, \; \omega \in \cstore_\sigma \}   & \textit{otherwise}
\end{cases}
\]

If any constraint is violated by the additional substitution we signal fail, otherwise we take in the store updated constraints
(and some constraints are thrown away as they are no longer possible to be satisfied).

The sufficient conditions for completeness can be proved for this implementation too,
but it requires rigorous case analysis and relies significantly on certain properties of substitution constructed by the miniKanren search,
so we only have a proof on paper for now.

\section{Application}

\subsection{SLD Semantics}

Interlving to SLD --- changing order in disjunctions:

\[
  \begin{array}{cr}
    \dfrac{s_1 \xrightarrow{\circ} s'_1}{(s_1 \oplus s_2) \xrightarrow{\circ} (s'_1 \oplus s_2)} &\ruleno{DisjStep}\\[5mm]
    \dfrac{s_1 \xrightarrow{r} s'_1}{(s_1 \oplus s_2) \xrightarrow{r} (s'_1 \oplus s_2)} &\ruleno{DisjStepAns}\\[5mm]
  \end{array}
\]

\subsection{Cut}

Now, cuts.

New states:

\[
S = \mathcal{G}\times\Sigma\times\mathbb{N}\mid S\oplus S \mid  S \circledast S \mid S \otimes \mathcal{G}
\]


Changing \textsc{ConjStepAns} to

  \[
  \begin{array}{cr}
    \dfrac{s \xrightarrow{(\sigma, n)} s'}{(s \otimes g) \xrightarrow{\circ} (\inbr{g, \sigma, n} \circledast (s' \otimes g))} & \ruleno{ConjStepAns} 
  \end{array}
  \]
  
Adding simple rules for asterisk (the same as those for plus):

  \[
  \begin{array}{cr}
    \dfrac{s_1 \xrightarrow{\circ} \Diamond}{(s_1 \circledast s_2) \xrightarrow{\circ} s_2} & \ruleno{AstStop}\\[5mm]
    \dfrac{s_1 \xrightarrow{r} \Diamond}{(s_1 \circledast s_2) \xrightarrow{r} s_2} & \ruleno{AstStopAns}\\[5mm]
    \dfrac{s_1 \xrightarrow{\circ} s'_1}{(s_1 \circledast s_2) \xrightarrow{\circ} (s'_1 \circledast s_2)} &\ruleno{AstStep}\\[5mm]
    \dfrac{s_1 \xrightarrow{r} s'_1}{(s_1 \circledast s_2) \xrightarrow{r} (s'_1 \circledast s_2)} &\ruleno{AstStepAns}\\[5mm]
  \end{array}
\]
  
Cut signal creation:

  \[
  \begin{array}{cr}
    \inbr{!, \sigma, n} \xrightarrow{(\sigma, n)}_c \Diamond &\ruleno{UnifyFail} \\[2mm]
  \end{array}
\]

Cut signal for pluses and asterisks:

  \[
  \begin{array}{cr}
    \dfrac{s_1 \xrightarrow{\circ}_c \Diamond}{(s_1 \oplus s_2) \xrightarrow{\circ} \Diamond} & \ruleno{PlusStopC}\\[5mm]
    \dfrac{s_1 \xrightarrow{r}_c \Diamond}{(s_1 \oplus s_2) \xrightarrow{r} \Diamond} & \ruleno{PlusStopAnsC}\\[5mm]
    \dfrac{s_1 \xrightarrow{\circ}_c s'_1}{(s_1 \oplus s_2) \xrightarrow{\circ} s'_1} &\ruleno{PlusStepC}\\[5mm]
    \dfrac{s_1 \xrightarrow{r}_c s'_1}{(s_1 \oplus s_2) \xrightarrow{r} s'_1} &\ruleno{PlusStepAnsC}\\[5mm]
    \dfrac{s_1 \xrightarrow{\circ}_c \Diamond}{(s_1 \circledast s_2) \xrightarrow{\circ}_c \Diamond} & \ruleno{AstStopC}\\[5mm]
    \dfrac{s_1 \xrightarrow{r}_c \Diamond}{(s_1 \circledast s_2) \xrightarrow{r}_c \Diamond} & \ruleno{AstStopAnsC}\\[5mm]
    \dfrac{s_1 \xrightarrow{\circ}_c s'_1}{(s_1 \circledast s_2) \xrightarrow{\circ}_c s'_1} &\ruleno{AstStepC}\\[5mm]
    \dfrac{s_1 \xrightarrow{r}_c s'_1}{(s_1 \circledast s_2) \xrightarrow{r}_c s'_1} &\ruleno{AstStepAnsC}\\[5mm]
  \end{array}
  \]
  
And cut signal propogation in crosses (just adding $c$ to all arrows)

\section{Conclusion}

We presented a strongly-typed implementation of \miniKanren for OCaml. Our implementation
passes all tests written for \miniKanren (including those for disequality constraints);
in addition we implemented many interesting relational programs known from
the literature. We claim that our implementation can be used both as a convenient
relational DSL for OCaml and an experimental framework for future research in the area of
relational programming.

%We also want to express our gratitude to William Byrd, who infected us with relational programming,
%and for the extra time he sacrificed as both our tutor and friend.


%% Acknowledgments
%\begin{acks}                            %% acks environment is optional
                                        %% contents suppressed with 'anonymous'
  %% Commands \grantsponsor{<sponsorID>}{<name>}{<url>} and
  %% \grantnum[<url>]{<sponsorID>}{<number>} should be used to
  %% acknowledge financial support and will be used by metadata
  %% extraction tools.
 % This material is based upon work supported by the
  %\grantsponsor{GS100000001}{National Science
   % Foundation}{http://dx.doi.org/10.13039/100000001} under Grant
  %No.~\grantnum{GS100000001}{nnnnnnn} and Grant
  %No.~\grantnum{GS100000001}{mmmmmmm}.  Any opinions, findings, and
  %conclusions or recommendations expressed in this material are those
  %of the author and do not necessarily reflect the views of the
  %National Science Foundation.
%\end{acks}


%% Bibliography
\bibliography{main}


%% Appendix
%\appendix
%\section{Appendix}

%Text of appendix \ldots

\end{document}
