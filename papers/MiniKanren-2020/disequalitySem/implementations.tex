\section{Concrete Implementations}
\label{sec:implementations}

In this section, we define two concrete implementations of constraint stores which can be incorporated in operational semantics: the trivial one and the one, which is close to existing real implementation in a certain version of \textsc{miniKanren}~\cite{CKanren}. We prove that they satisfy the sufficient condintion for search completeness from the previous sectiob
n. Both implementations are certified in \textsc{Coq}, which allowed us to extract two correct-by-construction interpreters for \textsc{miniKanren} with
disequality constraints.

\subsection{Trivial Implementation}

This trivial implementation simply stores all pairs of terms, which the search encounters, in a multiset and never uses them:

\[ \cstore_\sigma \subset_m \mathcal{T} \times \mathcal{T} \]

\[ \cstoreinit = \emptyset \]

\[ \csadd{\cstore_\sigma}{t_1}{t_2} = \cstore_\sigma \cup \{(t_1, t_2)\} \]

\[ \csupdate{\cstore_\sigma}{\delta} = \cstore_\sigma \]

The interpretation of such constraint store is the set of all representing functions that does not equate terms in any pair:

\[ \sembr{\cstore_\sigma} = \{\reprfun \colon \mathcal{A}\mapsto\mathcal{D} \mid \forall (t_1, t_2) \in \cstore_\sigma, \; \overline{\reprfun}\,(t_1) \neq \overline{\reprfun}\,(t_2)\} \]

This is a correct implementation (although for the full implementation we should find a way to present restrictions stored this way
in answers adequately) and it satisfies the sufficient conditions for completeness trivially, but it is not very practical.
In particular, it does not use information acquired from disequalities to halt the search in case of contradiction and it can return contradictory answers with the final disequality constraint violated by the final substitution (such as $([\alpha_0 \mapsto 5], [\alpha_0 \neq 5], 1)$): since such answers have empty interpretations, their presence does not affect search completeness.

\subsection{ReaIistic mplementation}

This implementation is more similar to those in existing \textsc{miniKanren} implementations and takes an approach that is close to one described is~\cite{CKanren}.

In this version, every constraint is represented as a substitution containing variable bindings which should \emph{not} be satisfied.

\[ \cstore_\sigma \subset_m \Sigma \]

So if a constraint store $\cstore_\sigma$ contains a substitution $\omega$ the set of representing functions prohibited by it is $\sembr{\sigma \omega}$,
which provides the following denotational interpretation for a constraint store:

\[ \sembr{\cstore_\sigma} = \bigcap_{\omega \in \cstore_\sigma} \overline{ \sembr{\sigma \omega} } \]

We start with the empty store

\[ \cstoreinit = \emptyset \]

When we encounter a disequality for two terms we try to unify them and update constraint store depending on the result of unification:

\[
\csadd{\cstore_\sigma}{t_1}{t_2} =
    \begin{cases}
       \cstore_\sigma                                & \not\exists mgu(t_1 \sigma, t_2 \sigma) \\
       \bot                                                 & mgu(t_1 \sigma, t_2 \sigma) = \epsilon \\
       \cstore_\sigma \cup \{\omega\}      & mgu(t_1 \sigma, t_2 \sigma) = \omega \neq \epsilon
    \end{cases}
\]

If terms are not unifiable, there is no need to change the constraint store. If they are unified by current substitution the constraint is already violated and we signal fail.
Otherwise, the most general unifier is an appropriate representation of this constraint.

When updating constraint store with an additional substitution $\delta$ we try to update each individual constraint substitution by treating it
as a list of pairs of terms that should not be unified (the first element of each pair is a variable), applying $\delta$ to these terms and trying to
unify all pairs simultaniously:

\[ \cupdate{[x_1 \mapsto t_1, \dots, x_k \mapsto t_k]}{\delta} = mgu([\delta(x_1), \dots, \delta(x_k)],[t_1 \delta, \dots, t_k \delta]) \]

We construct the updated constraint store from the results of all constraint updates:

\[
\csupdate{\cstore_\sigma}{\delta} =
\begin{cases}
  \bot                                                 & \exists \omega \in \cstore_\sigma: \cupdate{\omega}{\delta} = \epsilon \\
  \{ \omega' \mid \cupdate{\omega}{\delta} = \omega' \neq \bot, \; \omega \in \cstore_\sigma \}   & \textit{otherwise}
\end{cases}
\]

If any constraint is violated by the additional substitution we signal fail, otherwise we take in the store updated constraints
(and some constraints are thrown away as they can no longer be violated).

We proved the sufficient conditions for completeness for this implementation too, but it required us to prove first that all substitutions constructed by miniKanren search have a specific form. Namely, a current subsitution $\sigma$ at any point of the search (started from the initial state) is always \emph{narrowing} --- which means that $\ran{\sigma} \cap \dom{\sigma} = \emptyset$ --- and every time a current substitution $\sigma$ is updated by composing with some substitution $\delta$ (in rule $\ruleno{UnifySuccess}$) this substitution is \emph{extending} --- which means that $\dom{\delta} \cap \dom{\sigma} = \emptyset \land \ran{\delta} \cap \dom{\sigma} = \emptyset$.
