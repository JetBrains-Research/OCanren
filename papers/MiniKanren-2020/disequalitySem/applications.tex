\section{Applications}

In addition to verification of correctness of different implementations of disequality constraints we can use the extended framework to formally
state and prove some of its other important properties. Thanks to the completeness result, we can do it in the denotational context,
where the reasoning is much easier.

For example, we can define meaningless answers with empty interpretation, which we pointed out for Implementation A from the previous section,
and prove their absence for Implementation B.

So, for implementation B the following holds:

\begin{lemma}
If all free variables in a goal $g$ belong to the set $\{\alpha_1,\dots,\alpha_n\}$, then

\[ \forall (\sigma, \cstore_\sigma, n_r) \in Tr_{\inbr{g, \epsilon, \cstoreinit, n}}, \quad \sembr{\sigma} \cap \sembr{\cstore_\sigma} \neq \emptyset \]
\end{lemma}

It is based on the following lemma about combining constraints, which we can prove only when there are infinitely many constructors in the language (otherwise it is not true):

\begin{lemma}
If for a finite constraint store $\cstore_\sigma$
\[ \forall \omega \in \cstore_\sigma,  \sembr{\sigma} \cap \sembr{\omega} \neq \emptyset, \]
then
\[ \sembr{\sigma} \cap \sembr{\cstore_\sigma} \neq \emptyset. \]
\end{lemma}

Another example is the justification of optimizations in constraint store implementation. For example, the following obvious (in denotational context) statement
allows deleting subsumed constraints in Implementation B:

\begin{lemma}
For any constraint store $\cstore_\sigma$ and two constraint substitutions $\omega$ and $\omega'$, if

\[ \exists \tau, \omega' = \omega \tau \]

then

\[ \sembr{\cstore_\sigma \cup \{\omega, \omega'\}} = \sembr{\cstore_\sigma \cup \{\omega\}}. \]
\end{lemma}
