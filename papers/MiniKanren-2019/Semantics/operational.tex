\section{Operational Semantics}

In this section we describe operational semantics of \textsc{miniKanren}, which corresponds to the known
implementations with interleaving search. The semantics will be given in the form of labeled transition system (LTS). From now on we
assume the set of semantic variables to be linearly ordered ($\mathcal{A}=\{\alpha_1,\alpha_2,\dots\}$).

We introduce the notion of substitution

\[
  \sigma : \mathcal{A}\to\mathcal{T_A}
\]

as a (partial) mapping from semantic variables to terms over the set of semantic variables. We denote $\Sigma$ the
set of all substitutions, $\dom{\sigma}$~--- the domain for a substitution $\sigma$,
$\ran{\sigma}=\bigcup_{\alpha\in\mathcal{D}om\,(\sigma)}\fv{\sigma\,(\alpha)}$~--- its range (the set of all free variables in the image).

The states in the transition system have the following shape

\[
S = \mathcal{G}\times\Sigma\times\mathbb{N}\mid S\oplus S \mid S \otimes \mathcal{G}
\]

As we will see later, an evaluation of a goal is separated into elementary steps, and these steps are performed interchangeably for different subgoals. 
Thus, a state has a tree-like structure with intermediate nodes corresponding to partially-evaluated conjunctions (``$\otimes$'') or
disjunctions (``$\oplus$''). A leaf in the form $\inbr{g, \sigma, n}$ determines a goal in a context, where $g$~--- a goal, $\sigma$~--- a substitution accumulated so far,
and $n$~--- a natural number, which corresponds to a number of semantic variables used to this point. For a conjunction node its right child is always a goal since
it cannot be evaluated unless some result is provided by the left conjunct.

We also need extended states

\[
\overline{S} = \diamond \mid S
\]

where $\diamond$ symbolises the end of evaluation, and the following well-formedness condition:

\begin{definition}
  Well-formedness condition for extended states:
  
  \begin{itemize}
  \item $\diamond$ is well-formed;
  \item $\inbr{g, \sigma, n}$ is well-formed iff $\fv{g}\cup\dom{\sigma}\cup\ran{\sigma}\subset\{\alpha_1,\dots,\alpha_n\}$;
  \item $s_1\oplus s_2$ is well-formed iff $s_1$ and $s_2$ well-formed;
  \item $s\otimes g$ is well-formed iff $s$ is well-formed and for all leaf triplets $\inbr{\_,\_,n}$ in $s$ $\fv{g}\subseteq\{\alpha_1,\dots,\alpha_n\}$.
  \end{itemize}
  
\end{definition}

Informally the well-formedness restricts the set of states to those in which all goals use only allocated variables.

Finally, we define the set of labels:

\[
L = \circ \mid \Sigma\times \mathbb{N}
\]

The label ``$\circ$'' is used to mark those steps which do not provide an answer; otherwise a transition is labeled by a pair of a substitution and a number of allocated
variables. The substitution is one of the answers, and the number is threaded through the derivation to keep track of allocated variables; we ignore it in further explanations.

The transition rules are shown on Figure~\ref{lts}. The first two rules specify the semantics of unification. If two terms are not unifiable under the current substitution
$\sigma$ then the evaluation stops with no answer; otherwise it stops with the answer equal to the most general unifier.

The next two rules describe the steps performed when disjunction (conjunction) is encountered on the top level of the current goal. For disjunction it schedules both goals (using ``$\oplus$'') for
evaluating in the same context as the parent state, for conjunction~--- schedules the left goal and postpones the right one (using ``$\otimes$'').

The rule for ``\lstinline|fresh|'' substitutes bound syntactic variable with a newly allocated semantic one and proceeds with the goal; no answer provided at this step.

The rule for relation invocation finds a corresponding definition, substitutes its formal parameters with the actual ones, and proceeds with the body.

The rest of the rules specify the steps performed during the evaluation of two remaining types of the states~--- conjunction and disjunction. In all cases the left state
is evaluated first. If its evaluation stops with a result then the right state (or goal) is scheduled for evaluation, and the label is propagated. If there is no result then
the conjunction evaluation stops with no result (\textsc{ConjStop}) as well while the disjunction evaluation proceeds with the right state (\textsc{DisjStop}).

The last four rules describe \emph{interleaving}, which occurs when the evaluation of the left state suspends with some residual state (with or without an answer). In the case of disjunction
the answer (if any) is propagated, and the constituents of the disjunction are swapped (\textsc{DisjStep}, \textsc{DisjStepAns}). In case of conjunction, if the evaluation step in
the left conjunct did not provide any answer, the evaluation is continued in the same order since there is still no information to proceed with the evaluation of the right
conjunct (\textsc{ConjStep}); if there is some answer, then the disjunction of the right conjunct in the context of the answer and the remaining conjunction is
scheduled for evaluation (\textsc{ConjStepAns}).

\begin{figure}
  \[
  \begin{array}{cr}
    \inbr{t_1 \equiv t_2, \sigma, n} \xrightarrow{\circ} \Diamond , \, \, \nexists\; mgu\,(t_1, t_2, \sigma) &\ruleno{UnifyFail} \\[2mm]
    \inbr{t_1 \equiv t_2, \sigma, n} \xrightarrow{(mgu\,(t_1, t_2, \sigma),\, n)} \Diamond & \ruleno{UnifySuccess} \\[2mm]
    \inbr{g_1 \lor g_2, \sigma, n} \xrightarrow{\circ} \inbr{g_1, \sigma, n} \oplus \inbr{g_2, \sigma, n} & \ruleno{Disj} \\[2mm]
    \inbr{g_1 \land g_2, \sigma, n} \xrightarrow{\circ} \inbr{ g_1, \sigma, n} \otimes g_2 & \ruleno{Conj} \\[2mm]
    \inbr{\mbox{\lstinline|fresh|}\, x\, .\, g, \sigma, n} \xrightarrow{\circ} \inbr{g\,[\bigslant{\alpha_{n}}{x}], \sigma, n + 1} & \ruleno{Fresh}\\[2mm]
    \dfrac{R_i^{k_i}=\lambda\,x_1\dots x_{k_i}\,.\,g}{\inbr{R_i^{k_i}\,(t_1,\dots,t_{k_i}),\sigma,n} \xrightarrow{\circ} \inbr{g\,[\bigslant{t_1}{x_1}\dots\bigslant{t_{k_i}}{x_{k_i}}], \sigma, n}} & \ruleno{Invoke}\\[5mm]
    \dfrac{s_1 \xrightarrow{\circ} \Diamond}{(s_1 \oplus s_2) \xrightarrow{\circ} s_2} & \ruleno{DisjStop}\\[5mm]
    \dfrac{s_1 \xrightarrow{r} \Diamond}{(s_1 \oplus s_2) \xrightarrow{r} s_2} & \ruleno{DisjStopAns}\\[5mm]
    \dfrac{s \xrightarrow{\circ} \Diamond}{(s \otimes g) \xrightarrow{\circ} \Diamond} &\ruleno{ConjStop}\\[5mm]
    \dfrac{s \xrightarrow{(\sigma, n)} \Diamond}{(s \otimes g) \xrightarrow{\circ} \inbr{g, \sigma, n}}  & \ruleno{ConjStopAns}\\[5mm]
    \dfrac{s_1 \xrightarrow{\circ} s'_1}{(s_1 \oplus s_2) \xrightarrow{\circ} (s_2 \oplus s'_1)} &\ruleno{DisjStep}\\[5mm]
    \dfrac{s_1 \xrightarrow{r} s'_1}{(s_1 \oplus s_2) \xrightarrow{r} (s_2 \oplus s'_1)} &\ruleno{DisjStepAns}\\[5mm]
    \dfrac{s \xrightarrow{\circ} s'}{(s \otimes g) \xrightarrow{\circ} (s' \otimes g)} &\ruleno{ConjStep}\\[5mm]
    \dfrac{s \xrightarrow{(\sigma, n)} s'}{(s \otimes g) \xrightarrow{\circ} (\inbr{g, \sigma, n} \oplus (s' \otimes g))} & \ruleno{ConjStepAns} 
  \end{array}
  \]
  \caption{Operational semantics of interleaving search}
  \label{lts}
\end{figure}
