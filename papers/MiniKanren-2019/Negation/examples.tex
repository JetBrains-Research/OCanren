\section{Motivating Examples}

\label{sec:motivation}

In this section we give further motivation 
for adding the negation into relational programming.
We present several examples of how the negation can be used.

\subsection{Relational If-then-else}

\label{sec:ifte}

In the Prolog one can simulate the conditional if-then-else
operator using so-called \emph{soft cut}~\cite{naish1995pruning}.
The behavior of the soft cut \lstinline{c $\rightarrow^*$ t ; e}
can be described as follows:

\begin{itemize}
  \item if the goal \lstinline{c} succeeds 
        (i.e. produces at least one answer) then
        the result of \lstinline{c $\rightarrow^*$ t ; e}
        is equivalent to \lstinline{c /\ t};
  \item if the goal \lstinline{c} fails 
        (i.e. produces no answers at all) then
        the result of \lstinline{c $\rightarrow^*$ t ; e}
        is equivalent to \lstinline{e}.
\end{itemize}

The soft cut is an example of a non-relational feature.
Such features usually do not compose well
in the sense that they are sensitive to the 
order in which they appear in a program.
For example, consider the goal 
\lstinline{(c $\rightarrow^*$ t ; e) /\ g},
and assume that \lstinline{c /\ g} always fails
regardless of the order of conjuncts.
Then if \lstinline{c} succeeds the result 
of the above goal will be equivalent to 
\lstinline{c /\ t /\ g} and thus it will fail.
Suppose we reorder the conjuncts as follows:
\lstinline{g /\ (c $\rightarrow^*$ t ; e)}.
Now the goal \lstinline{c}, when executed after \lstinline{g}, 
certainly fails, and thus the result 
of the whole goal will be equivalent to
\lstinline{g /\ e} which does not fails necessarily.
One can see that the simple reordering 
of subgoals in the program can lead to the different results.

With the help of constructive negation
if-then-else can be simply expressed as follows:

\begin{minipage}[h]{\textwidth}
\begin{lstlisting}[
  % caption={Logical if-then-else},
  label={lst:ifte},
  escapeinside={(*}{*)},
]
let ifte c t e = 
  (c /\ t) \/ (~ c /\ e)
\end{lstlisting}
\end{minipage}

The behavior of if-then-else defined in this way
subsumes the behavior of the soft cut.
That is, every answer to the query \lstinline{run (c $\rightarrow^*$ t ; e)}
is also an answer to the query \lstinline{run (ifte c t e)}.
Moreover, \lstinline{ifte} is not sensitive to the order
of subgoals in the program.

\subsection{Classical Implication and Universal Quantification}

\label{sec:impl-univ}

With the negation added to the language, one can easily express other 
connectives of the classical first-order logic,
namely the implication and universal quantification\footnote{
It is easy to see that \lstinline[mathescape]{c => t} 
is equivalent to \lstinline{ifte c t succ}},
using the well-known equivalences.

\begin{minipage}[h]{\textwidth}
\begin{lstlisting}[
  % caption={Implication and universal quantification},
  label={lst:impl-univ},
  escapeinside={(*}{*)},
]
let (=>) : goal -> goal -> goal = 
  fun g$_1$ g$_2$ ->  ~ g$_1$ \/ (g$_1$ /\ g$_2$)

let forall : ('a -> goal) -> goal = 
  fun g -> ~ fresh (x) (~ g x)
\end{lstlisting}
\end{minipage}

% There are at least two common interpretations of universal quantification.
% Under the \emph{intentional} interpretation we assume that 
% $\forall x. P(X)$ is true if $P(c)$ is true for 
% an arbitrary choice of some constant $c$.
% Under the \emph{extensional} interpretation 
% $\forall x. P(X)$ is true if $P(t)$ is true for every $t$.
% The \lstinline{forall} presented above implements 
% the extensional universal quantification.

However, we should make a few remarks here.
It is well known that the search implemented 
in conventional \textsc{MiniKanren} is complete, 
meaning that every answer to an arbitrary query 
will be found eventually. 
In the presence of constructive negation 
(and thus implication and universal quantification defined through negation)
the search becomes incomplete as we will later see.
Moreover, constructive negation is computationally heavy
and thus the double usage of it,
as in the definition of \lstinline{forall},
can be inefficient in some cases.

Despite all this trouble we have found that 
the above definitions are still useful.
Some of the previous \textsc{MiniKanren} implementations 
introduced \emph{eigen} variables,
adopted from $\lambda$Prolog~\cite{miller2012programming}.
Eigen variables act as a universally quantified variables.
Yet, to the best of our knowledge,
there is no sound implementation of eigen variables
with the support of disequality constraints.
We observed that our implementation 
of universal quantifications through double negation 
works nicely with disequalities 
(we give some examples in the Section~\ref{sec:evaluation}).
 
\subsection{Graph Unreachability Problem}

One of the classical examples of negation application 
in logic programming is a problem of checking whether 
one node of the graph is unreachable 
from the another~\cite{przymusinski1989constructive}.
The code on Listing~\ref{lst:unreach}
defines binary relation \lstinline{edge},
which binds pairs of nodes in graph, 
connected by some edge,
and binary relation \lstinline{reachable},
which is nothing more than a 
transitive closure of the \lstinline{edge} relation.
Then the relation \lstinline{unreachable}
is simply negation of \lstinline{reachable}.

\begin{minipage}[h]{\textwidth}
\begin{lstlisting}[
  caption={Unreachability in a graph},
  label={lst:unreach},
  escapeinside={(*}{*)},
]
let edge x y = 
  (x, y) === ('a', 'b') \/
  (x, y) === ('b', 'a') \/
  (x, y) === ('b', 'c') \/
  (x, y) === ('c', 'd') 

let reachable x y = 
  x === y \/
  fresh (z) (
    edge x z /\ reachable z y
  )

let unreachable x y = 
  ~(reachable x y)
\end{lstlisting}
\end{minipage}

Given this definition the query \lstinline{run unreachable 'c' 'a'} will succeed.
A knowledgeable reader might notice that 
constructive negation is not necessary in this case 
because negation as failure will deliver the same result.
But the query \lstinline{run unreachable 'c' q} will fail under
negation as failure because of the free variable \lstinline{q} which 
will appear under the negation.
However constructive negation will succeed 
delivering the constraint \lstinline{q =/= 'd'}.

\subsection{Unreachability in Labeled Transition Systems}

One can consider a special kind of graphs --- 
\emph{labeled transition systems}~\cite{baier2008principles}.
Labeled transition system is defined by a set of states $S$,
a set of labels $L$ and a ternary transition relation $R \subseteq S \times L \times S$.
By existential quantification over labels one can then obtain a binary relation. 
Taking its transitive closure gives the reachability relation.
The negation of the reachability relation can be used 
to check that some state $s'$ is not reachable from the initial state $s$.
The Listing~\ref{lst:lts} shows an encoding of an
abstract labeled transition system in \textsc{OCanren}.

Labeled transition systems are often used to 
describe the behavior of imperative languages.
Although the naive encoding of transition relation in \textsc{OCanren}
with simple enumeration of reachable states is often
not tractable for checking reachability (or unreachability) in 
huge state spaces arising in practical imperative programs,
it still can be used, for example, for the task of 
prototyping the semantics of such languages.

\begin{minipage}[htp]{\textwidth}
\begin{lstlisting}[
  caption={Unreachability in a labeled transition system},
  label={lst:lts},
  escapeinside={(*}{*)},
]
module type LTS = sig
  type state
  type label

  val transition : state -> label -> state -> goal
end

module LTSExt (T : LTS) = struct

  let reachable : T.state -> T.state -> goal = 
    fun s s'' -> 
      s === s'' 
      \/
      fresh (l s') (
        T.transition s l s' /\
        reachable s' s''
      )

  let unreachable : T.state -> T.state -> goal = 
    fun s s' -> 
      ~(reachable s s')
end

\end{lstlisting}
\end{minipage}