\section{Searching for Paths in a Graph with a Relational Verifier}
\label{sec:example}

In this section we demonstrate how to solve a concrete problem of searching for paths in a directed graph with a relational verifier. 
A directed graph is a tuple $(N, E, start, end)$, where $N$ is a finite set of \emph{nodes}, $E$ is a finite set of \emph{edges}, functions $start, end : E \rightarrow N$ return a start and an end nodes for a given edge respectively.
A path in a directed graph is a sequence:
\[
\langle n_0, e_0, n_1, e_1, \dots, n_k, e_k, n_{k+1} \rangle
\]

such that 
\[
\forall i \in \{ 0 \dots k \}\; :\; n_i = start\,(e_i) \text{ and } n_{i+1} = end\,(e_i).
\]

The problem of searching for paths in a graph is to find a set $\{ p \mid p \text{ is a path in } g\}$, where $g$ is a graph. 
There~are many concrete algorithms which search for paths in a graph. 
Implementing any of them involves determining in which way to traverse the graph, how to ensure one does not get stuck exploring a cycle in the graph (a cycle is a path in the graph of form $\langle n_0, e_0, \dots, n_k, e_k, n_0 \rangle$), how to ensure one path is not processed multiple times, and so~on. 
A much easier task is to implement a simple verifier, which checks if a sequence is indeed a path in a graph, and generate the path searching routine from it by the relational conversion.

Below is the implementation of the verifier ``\lstinline{isPath}''. 
This function takes as an input a list of nodes ``\lstinline{ns}'' and a graph ``\lstinline{g}''. 
We represent the graph as a list of edges, stipulating there are no parallel edges. 
Each edge $e$ is represented as a pair of nodes $(n, m)$, where $n = start(e)$, $m = end(e)$.
Given $ns = [n_0, \dots, n_{k+1}]$ and a graph $g = [e_0, \dots, e_l]$, the function returns true, if $\exists i_0 \dots i_k \text{ such that } \langle n_0, e_{i_0}, n_1, e_{i_1}, \dots, e_{i_k}, n_{k+1} \rangle$ is a path in $g$.

\begin{lstlisting}[numbers=left,numberstyle=\small,escapeinside={@}{@}]
let rec isPath ns g =
  match ns with
  @\label{lst:isPath_5}@| x$_1$ :: x$_2$ :: xs -> elem (x$_1$, x$_2$) g && isPath (x$_2$ :: xs) g 
  @\label{lst:isPath_4}@| [_]            -> true
\end{lstlisting}

The function ``\lstinline{elem}'' checks if an edge ``\lstinline{e}''  exists in the graph ``\lstinline{g}''. 
We omit the definition of equality check for edges ``\lstinline{eq}'', since it is trivial to implement and is not relevant for the example.

\begin{lstlisting}
let rec elem e g =
  match g with
  | []      -> false
  | x :: xs -> if eq e x then true else elem e xs
\end{lstlisting}

We stipulate that a path must include at least two nodes, since searching for shorter paths is trivial. 
Line~\ref{lst:isPath_5} of the ``\lstinline{isPath}'' definition checks that the first two nodes of the list form an edge of the graph. 
Then it checks that what is left after deleting the first node from the list is still a path in the graph.
Line~\ref{lst:isPath_4} may come off a little counterintuitive, since it states that a path which includes a single arbitrary node is in the input graph.
However we only execute this branch by a recursive call of \lstquot{isPath}, which only happens after we have already ensured with the call to the ``\lstinline{elem}'' function that the said node is in the graph. 

The relational conversion of the verifier function ``\lstinline{isPath}'' generates a relation ``\lstinline{isPath$^o$}'' defined for a path \lstquot{ns}, a graph \lstquot{g} and a boolean value \lstquot{res}, which is true if ``\lstinline{ns}'' is a path in the graph ``\lstinline{g}'' and false otherwise. 
The function ``\lstinline{elem}'' is transformed into a relation ``\lstinline{elem$^o$}'' defined for an edge ``\lstinline{e}'', a graph ``\lstinline{g}'' and a boolean value ``\lstinline{res}'', which is true if ``\lstinline{e}'' is an edge in the graph ``\lstinline{g}'' and false otherwise.
The result of the relational conversion of the functions ``\lstinline{isPath}'' and ``\lstinline{elem}'' is presented below.

\begin{lstlisting}[firstnumber=5, numbers=left,numberstyle=\small,escapeinside={@}{@}]
let rec elem$^o$ e g res = conde [
  (g === nil () /\ res === ^false);
  (fresh (x xs resEq) (
    (g === x % xs) /\ 
    (eq$^o$ e x resEq) /\ 
    (conde [
      (resEq === ^true  /\ res === ^true); 
      (resEq === ^false /\ elem$^o$ e xs res)])))]

let rec isPath$^o$ ns g res = conde [
  (fresh (el) (
    (ns === el % nil ()) /\ 
    (res === ^true));
 @\label{isPatho:fst}@(fresh (x$_1$ x$_2$ xs resElem resIsPath) (
    (ns === x$_1$ % (x$_2$ % xs)) /\ 
    (elem$^o$ (pair x$_1$ x$_2$) g resElem) /\
    (isPath$^o$ (x$_2$ % xs) g resIsPath) /\ 
    (conde [
 @\label{isPatho:die}@     (resElem === ^false /\ res === ^false); 
 @\label{isPatho:lst}@     (resElem === ^true  /\ res === resIsPath)])))]
\end{lstlisting}

Here we use the syntax of \textsc{OCanren}. 
A new relation is defined as a recursive function with the keywords ``\lstinline{let rec}''. 
The body of the relation is a goal created with the following goal constructors. 

\begin{itemize}
    \item Disjunction $g_1 \vee g_2$, where $g_1, g_2$ --- some goals. The two goals are evaluated independently and their results are combined.
    \item Disjunction of goal list \lstinline{conde [$g_1; \ldots; g_n$]}, where $g_1; \ldots; g_n$ --- some goals.
    \item Conjunction $g_1 \wedge g_2$, where $g_1, g_2$ --- some goals. The goal $g_2$ is evaluated only if the evaluation of $g_1$ succeeded; the evaluation of $g_2$ uses the results of $g_1$.
    \item Syntactic unification  $t_1 \equiv t_2$, where $t_1, t_2$ --- some terms. Unification is a basic goal constructor. If $t_1$ and $t_2$ can be unified, the goal is considered successful and failed otherwise. 
    \item Relation call $r^n t_1 \dots t_n$ where $r^n$ is a name of some $n$-ary relation, and $t_i$ are terms. 
    \item To introduce fresh variables into scope, one should use $\textbf{fresh} \; (\overline{x}) \; g$, where $\overline{x}$ is a list of variable names.
\end{itemize}

Besides goal constructors we use some syntactic sugar for values and lists. 
``\lstinline{^}'' is used to transform a value into a logic value. 
The empty list is represented as ``\lstinline{nil ()}'', and to construct a new list from a value ``\lstinline{h}'' and a list ``\lstinline{t}'' we use ``\lstinline{h % t}''.
A tuple of ``\lstinline{x}'' and ``\lstinline{y}'' is created with ``\lstinline{pair x y}''.

Regrettably, this relational interpreter suffers from poor performance. 
Query ``\lstinline{isPath$^o$ q <graph> true}'' for path searching takes more than ten minutes even for graphs of 5 nodes. 
This is somewhat expected, considering that the relational conversion generates a relation which can be used for many different queries, which is excessive when any particular query is in question. 
This is, of course, not a desirable behaviour. Fortunately, further transformation of the relation can improve the performance. 

For example, if we consider a query ``\lstinline{isPath$^o$ q <graph> ^true}'', we can simplify lines~\ref{isPatho:fst} through~\ref{isPatho:lst} of its definition. 
First, we notice that, having ``\lstinline{res}'' be equal to ``\lstinline{^true}'', we can safely remove the disjunct in line~\ref{isPatho:die}, after what the whole ``\lstinline{conde}'' becomes unnecessary and can be removed. 
After moving the unifications for ``\lstinline{resElem}'' and ``\lstinline{resIsPath}'' to the top level, we get the following equivalent definition of the ``\lstinline{isPath$^o$}'' relation. 
Note, that the call to the ``\lstinline{elem$^o$}'' relation is done with the last argument being unified with ``\lstinline{^true}'', so further specialization is still possible. 

\begin{lstlisting}[firstnumber=25, numbers=left,numberstyle=\small,escapeinside={@}{@}]
let rec isPath$^o$ ns g res = conde [
  (fresh (el) (
    (ns === el % nil ()) /\ 
    (res === ^true)));
  (fresh (x$_1$ x$_2$ xs resElem resIsPath) (
    (resElem === ^true) /\
    (resIsPath === ^true) /\
    (ns === x$_1$ % (x$_2$ % xs)) /\ 
    (elem$^o$ (pair x$_1$ x$_2$) g resElem) /\
    (isPath$^o$ (x$_2$ % xs) g resIsPath)))]
\end{lstlisting}

The specialized version of the relation is much more performant than the original one.
Before, searching paths of length 5 took more than 10 minutes while the specialized version finds paths of length 10 in the graph with 100 edges in a few seconds. 

This transformation can be performed automatically with conjunctive partial deduction. 
The result of partially deducing the ``\lstinline{isPath$^o$ q p ^true}'', where ``\lstinline{p}'' and ``\lstinline{q}'' are fresh variables is about 40 lines of code long and it has the same performance as the manually transformed relation. 
We omit the transformed program because of the space concerns, but it can be found in the repository\footnote{https://github.com/Lozov-Petr/miniKanren-2019-Relational-Interpreters-for-Search-Problems}.
